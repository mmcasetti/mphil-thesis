\section{The Lemke-Howson Algorithm}
\label{lh-sect}

Theorem \ref{nash-ppad-complete-thm} makes the study of solutions of
$n$-Nash as endpoints of paths particularly interesting. In this
section we will see an algorithm that describes exactly this idea.

Let $P$ be a simple $d$-polytope with $n$ facets.
We {\em pivot on the vertices} of $P$ by moving from a
vertex $x$ to another vertex $y$ connected to $x$ by an edge.
Note that, since $P$ is simple, there are exactly
$d$ possible choices for $y$.
Analogously, we {\em pivot on the facets} of a simplicial
polytope $P^\Delta$ in
dimension $d$ by moving from a facet $F$ to a facet $G$ that
shares all vertices but one with~$F$. Since $P^\Delta$ is
simplicial, there are $d$ possible choices for $G$.

Suppose now that there is a labeling $l_f:[n]\to [d]$ of the facets of the
simple polytope $P$.
If we pivot from vertex $x$ to vertex $x'$ we ``leave behind'' a facet $F$
to which $x$, but not $x'$, belong; let $F$ have label $k$. At the same
time, we ``reach'' a facet $F'$, to which $x$ does not belong, but $x'$
does; let $F'$ have label $h$.
Therefore, if $x$ has labels $(l_1,\ldots,k,\ldots,l_d)$, then
$x'$ has labels $(l_1,\ldots,h,\ldots,l_d)$. We call this
{\em dropping label $k$ and picking up label $h$}, or
{\em pivoting on label $k$}; see Figure
\ref{pivot-vertex-fig} for illustration.
Analogously, if there is a labeling $l_v:[n]\to [d]$ of the vertices
of the simplicial poytope $P^\Delta$
and we pivot from a facet $F$ with labels $(l_1,\ldots,k,\ldots,l_d)$
to a facet $F'$ with labels $(l_1,\ldots,h,\ldots,l_d)$, we say that we
{\em drop label $k$ and pick up label $h$}, or that we
{\em pivot on label $k$}; see Figure \ref{pivot-facet-fig}.

\begin{figure}[hp]
\strut\hfill
\includegraphics[width=45ex]{chapter-3/fig-lh/cube-pivot.pdf}%
\hfill\strut
\caption[A pivot on the vertices of the cube]{%
A pivot on label $k$, dropping vertex $x$ with labels $(l_1,l_2,k)$ and
picking up vertex $x'$ with labels $(l_1,l_2,h)$.
}
\label{pivot-vertex-fig}
\end{figure}

\begin{figure}[hp]
\strut\hfill
\includegraphics[width=55ex]{chapter-3/fig-lh/octahedron-pivot.pdf}%
\hfill\strut
\caption[A pivot on the facets of the octahedron]{%
A pivot on label $k$, dropping facet $F$ with labels $(l_1,l_2,k)$ and
picking up facet $F'$ with labels $(l_1,l_2,h)$.
}
\label{pivot-facet-fig}
\end{figure}

\clearpage

% again, I don't think what you do here makes sense
%Let $m,n\in \naturals$ with $m\leq n$; consider a set $X$
%and a labeling $l:X\to [m]$. The $n$-uple $x=(x_1,\ldots,x_n)\in X^n$
Consider a labeling function $l:[n]\to[d]$, and a subset $S$
of $[n]$ with $|S|=d$.
Then $S$ is called {\em almost completely labeled} if
\begin{equation}
\label{acl-equation}
l(S)~=~\{\,l(s)\mid s\in S\}~=~[d]\setminus\{k\}\,,
\end{equation}
that is, all labels appear once in $S$ except for one {\em
missing label} $k\in [d]$.
Because $|S|=d$, in that case there is one
{\em duplicate label} $h\in [d]$ that appears twice in~$S$.

The set $S$ will be the set of labels of a vertex in a
simple polytope (as the set of labels of the facets containing
that vertex), or the set of labels of a facet in a
simplicial polytope (as the set of labels of the vertices on
that facet).
Correspondingly, we call such a vertex (or facet) ``almost
completely labeled''.
It is easy to see that if we pivot from an almost completely
labeled vertex (or facet) on the duplicate label, or from a
completely labeled vertex (or facet) by on any label (which
will become the missing label~$k$), then we reach either an
almost completely labeled or a completely labeled vertex
(or facet).

The algorithm by Lemke and Howson \cite{lh} finds one Nash
equilibrium of a bimatrix game.
In a modern description (e.g., Savani and von Stengel \cite{svs}),
it employs pivoting on the vertices of a simple polytope,
moving through a succession of almost completely labeled
vertices with missing label~$k$, where this polytope is
the product $P\times Q$ of the best-response polytopes.
This can be abstracted slightly further by considering only
a single polytope $P$ in dimension $d$ with facets labels
from $[d]$.
The resulting algorithm is also known as computing a ``Lemke
path'', in the terminology used by Morris~\cite{morris}, and
shown in Algorithm~\ref{lh-alg}.
We will use Lemke paths to prove some fundamental properties
of both the Lemke-Howson Algorithm and the problem {\sc
Another Completely Labeled} {\sc Vertex}.

\begin{algorithm}[htb]
\SetKwInOut{Input}{input}
\SetKwInOut{Output}{output}
\Input{%
A simple $d$-polytope $P$ with $n$ facets and a
labeling $l_f:[n]\to [d]$ of the facets of $P$.
A vertex $x_0$ of $P$ that is completely labeled by $l_f$.
}
\Output{%
A vertex $x\neq x_0$ of $P$ that is completely labeled by $l_f$.
}
\BlankLine
choose any label $k\in [d]$ as missing label \\
pivot on label $k$ from $x_0$ to $x$ reaching a new facet
with label $h$\\
\While{ $h\ne k$, so $x$ is not completely labeled ~}
{
% you don't use x_0 in the while loop
pivot away from the other facet with label $h$ from $x$ to $x'$  \\
let $h$ be the label of the new facet of $x'$ and
set $x = x'$
}
\Return $x$
\caption{Lemke Path}
\label{lh-alg}
\end{algorithm}

\begin{proposition}\label{lh-works-ppa-thm}
The Lemke Path Algorithm \ref{lh-alg} returns a solution to
the {\bf PPA} problem {\sc Another Completely Labeled Vertex}.
Furthermore, the number of completely labeled vertices in a simple
polytope with labeled facets is even.
\end{proposition}

\begin{proof}
We first show that the Lemke Path Algorithm works.
From the completely labeled vertex $x_0$, there is a unique
edge that leaves the facet with label~$k$ which leads to a
new vertex $x$ as in step~2 of the algorithm.
If $x$ is completely labeled, then the algorithm terminates
with output $x$ where clearly $x\ne x_0$.
Otherwise, $x$ is an almost completely labeled vertex with
duplicate label~$h$, where one of the facets that contain $x$
and have label~$h$ is a ``new'' facet that did not contain
the preceding vertex on the Lemke path.
Because $S$ is simple, $x$ is always on exactly $d$ facets
and the duplicate label is unique.
Hence no vertex (including $x_0$) can ever be re-visited
on the path because it would otherwise offer an alternative
way to proceed when the vertex was encountered for the first
time.

The parity is proven by the following argument: each Lemke path is
uniquely determined by its missing label and its starting point, so the
Lemke path from the endpoint with the same missing label will lead back
to the starting point. Since the endpoint and the starting point are
different, the Lemke paths must connect an even number of points.

Finally, for each label $k\in [d]$ chosen in line 1 of
Algorithm~\ref{lh-alg}, the Lemke paths are disjoint paths connecting
all the completely labeled vertices of $P$, with a standard
starting point $x_0$.
The problem {\sc Another Completely Labeled Vertex} correspond to finding
a non-standard endpoints of this graph, which is a {\bf PPA} problem.
\end{proof}

Applying the parity result of Proposition \ref{lh-works-ppa-thm} to the
case of a bimatrix game (not necessarily a unit vector game), and
remembering that the point $(\0,\0)$ corresponds to the
``artificial'' equilibrium, we have the following result, due to Lemke and
Howson \cite{lh}.

\begin{theorem}{\rm (Lemke-Howson \cite{lh}.)}
Every non-degenerate bimatrix game has an odd number of Nash equilibria.
\end{theorem}

There are two ways of using the Lemke-Howson Algorithm to find a Nash
equilibrium of a bimatrix game $(A,B)$.
The first one is to ``symmetrize'' the game as in Proposition
\ref{symmetrize-c}. Let
$R = \{ z\in\reals^{m+n}\ |\ z\geq\0,\ Cz\leq\1 \}$ be the
polytope associated to the game $(C,C\T)$, where
\[
C=\binom{~0~~~ A\,}{B\T~ 0\,}.
\]
The facets of $C$ correspond to $2(m+n)$ inequalities. We label both
the $i$-th and the $(m+n+i)$-th inequality as $i\in [m+n]$ and we apply the
Lemke Path algorithm starting from the vertex $\0$. This returns a Nash
equilibrium $(z,z)$ of $C$, which corresponds to a Nash equilibrium
$(x,y)=z$ of $(A,B)$.

We can also follow the ``traditional'' exposition of the Lemke-Howson
Algorithm by alternating a move on the best response polytopes $P$ and a move
on the best response polytope $Q$ of (\ref{br-polytopes}).
Since the polytopes $P$ and $Q$ are in $\reals^m$ and $\reals^n$, whereas $R$
is a polytope in $\reals^{m+n}$, this second version is much easier to
visualize.

\begin{example}({\rm Savani and von Stengel \cite{uvg}}.)
Consider the $3\times 3$ game $(A,B)$ of Example \ref{br-game-ex}.
\begin{equation*}
A = \left(\begin{matrix}1&0&0\\ 0&1&0\\
0&0&1\end{matrix}\right),
\qquad
B = \left(\begin{matrix}0&2&4\\ 3&2&0\\
0&2&0\end{matrix}\right).
\end{equation*}
The best response polytopes can be represented as the best response regions
of Figure \ref{br-regions-fig} extended to the origin $\0$,
as in Figure~\ref{lh-path-fig}, which is in fact the
original exposition by Shapley~\cite{shapley}.
The path starts from $(\0,\0)$.
We choose the missing label~2 and move in the polytope $P$.
Then label~6 is duplicate; so we drop it and we make the next move on the
polytope $Q$, and so on until we reach the point $x$ in $P$
and $y$ in~$Q$, which gives here the only Nash equilibrium $(x,y)$ of $(A,B)$.
\begin{figure}[hbt]
\strut\hfill
\includegraphics[width=75ex]{chapter-3/fig-lh/lemke-path.pdf}%
\hfill\strut
\caption[A Lemke path for a bimatrix game]{%
Lemke path for missing label 2 on the best response polytopes $P$ (left)
and $Q$ (right) of game (\ref{AB}).
}
\label{lh-path-fig}
\end{figure}
\end{example}

%\newpage

It is possible to have an equilibrium that cannot be reached applying the
Lemke-Howson Algorithm from the artificial equilibrium, or even from
the endpoint of a Lemke path from the artificial equilibrium. This is
can be seen in the next example. Notice that, by the parity result in
Proposition \ref{lh-works-ppa-thm}, there must be at least another equilibrium
that is ``disconnected'' from the artificial one in this way.

\begin{example}(Shapley \cite{shapley}.)
Consider the symmetric game $(C,C\T)$ with
\begin{equation}
\label{disj-lp-game}
C = \left(
    \begin{matrix}
        0&3&0 \\
        2&2&0 \\
        3&0&1
    \end{matrix}
    \right).
\end{equation}
There are three equilibria of $(C,C\T)$, all of them symmetric,
at $(x_i,x_i)$ with
$x_1=(0,0,1)$,
$x_2=(1/6,1/3,1/2)$ and
$x_3=(1/3,2/3,0)$. All Lemke paths from the
artificial equilibrium $(0,0)$ end at $(x_1,x_1)$, and consequently all
other Lemke paths connect $(x_2,x_2)$ and $(x_3,x_3)$; see
Figure \ref{disj-lp-br-poly}.

\begin{figure}[hbt]
\strut\hfill
\includegraphics[width=80ex]{chapter-3/fig-lh/shapley-game.pdf}%
\hfill\strut
\caption[A game with disjoint Lemke paths]{%
Lemke paths for missing label 1 (blue), 2 (green) and 3 (pink) on the best
response polytopes of game (\ref{disj-lp-game}). The paths for missing label
4, 5 and 6 on the best response polytope of player 1 and 2 are the same
as the paths of 1, 2 and 3 on the best response polytope of player 2 and~1,
after the corresponding change of labels.
}
\label{disj-lp-br-poly}
\end{figure}
\end{example}

\clearpage

The dual version of the Lemke Path Algorithm \ref{lh-alg} and of
Proposition \ref{lh-works-ppa-thm} is straightforward.

\begin{algorithm}
\SetKwInOut{Input}{input}
\SetKwInOut{Output}{output}
\Input{%
A simplicial $m$-polytope $P^\Delta$ with $n$ vertices and a
labeling $l_v:[n]\to [d]$ of the vertices of $P^\Delta$.
A facet $F_0$ of $P^\Delta$ that is completely labeled by $l_v$.
}
\Output{%
A facet $F\neq F_0$ of $P^\Delta$ that is completely labeled by $l_v$.
}
\BlankLine
choose any label $k\in [d]$ as missing label \\
pivot on label $k$ from $F_0$ to $F$ which has a new verte
with label $h$ \\
\While{ $h\ne k$, so $F$ is not completely labeled ~}
{
pivot away from the other vertex with label $h$ from $F$ to $F'$  \\
let $h$ be the label of the new vertex of $F'$ and
set $F = F'$
}
\Return $F$
\caption{Dual Lemke Path}
\label{lh-dual-alg}
\end{algorithm}

\begin{proposition}\label{lh-dual-works-ppa-thm}
The Dual Lemke Path Algorithm \ref{lh-dual-alg} returns
a solution to the {\bf PPA} problem {\sc Another Completely Labeled Facet}.
Furthermore, the number of completely labeled facets
in a simplicial polytope with labeled vertices is even.
\end{proposition}

By Theorem \ref{unit-vector-thm} and Theorem \ref{unit-vector-dual-thm},
in the case of unit vector games it is enough to apply the
Lemke Path Algorithm~\ref{lh-alg} to the polytope $P^l$ in
(\ref{p-l}), or the Dual Lemke Path Algorithm
(\ref{lh-dual-alg}) to the polytope $P^\Delta$ in
(\ref{p-l-dual}).
The following theorem by Savani and von Stengel \cite{uvg} guarantees that
not only does this yield a Nash equilibrium, but no potential solutions
are ``lost'' considering the polytope $P^l$ with $m$ labels
instead of the product of polytopes $P\times Q$ with $m + n$ labels;
an analogous result holds for the dual case.

\begin{theorem}\label{unit-paths}
Let $(U,B)$ be a unit vector game, with
$U=[e_{l(1)}\cdots e_{l(n)}]$ for a labeling $l:[n]\to [m]$.
Let
\[
\arraycolsep.3em
\begin{array}{rcll}
P&=&\{ x\in\reals^m&\mid\ x\geq\0,\ B\T x\leq\1 \}, \\
Q&=&\{ y\in\reals^n&\mid\ A y\leq\1,~y\geq\0 \}, \\
\end{array}
\]
as in $(\ref{br-polytopes})$, and let
\[
P^l=\{ x\in\reals^m \mid\ x\geq\0,\ B\T x\leq\1 \}
\quad \hbox{with labels in $[m]$ as in $(\ref{facet-labeling-unitv})$}
\]
as in $(\ref{p-l})$.
Then for the missing label $k\in [m]$
the Lemke path on $P\times Q$ projects to a path on $P$ that corresponds
to the Lemke path on $P^l$ for the missing label~$k$. For the missing label
$k=m+j$, where $j\in [n]$, the Lemke path on $P\times Q$ projects to a
path on $Q$ that corresponds to the Lemke path on $P^l$ for the
missing label~$l(j)$.
\end{theorem}

We finally focus on the case of Gale games.
We will consider the strings $s\in G(d,n)$, with $d$ even,
as ``wrapped-around strings''.
Let $s(i)=1$ for an index $i\in [n]$. Then, by the Gale evenness condition,
there is an odd run of \1's either on the left or on the right
of position $i$ in $s$. Let $j$ be the first index after this run.
A {\em pivot from $s$ to $s'$} is given by setting $s'(i)=0$
and $s'(j)=1$ to yield the new string~$s'$ which otherwise
agrees with~$s$.
If there is a labeling $l_s:[n]\to [d]$, we say that we
{\em pivot on label $l_s(i)$} (where we have to specify $i$
%% something you missed in your Algorithms
when the label $l_s(i)$ does not uniquely identify~$i$),
{\em dropping label $l_s(i)$} and
{\em picking up label $l_s(j)$}.
The {\em Lemke Path for Gale Algorithm} is given
in Algorithm~\ref{lhg-alg}.

\clearpage

\begin{algorithm}%\label{lhg-alg}
\SetKwInOut{Input}{input}
\SetKwInOut{Output}{output}
\Input{%
A labeling $l_s:[n]\to [d]$, where $d$ is even, such that there is a
completely labeled Gale string $s_0 \in G(d,n)$.
}
\Output{%
A Gale string $s\in G(d,n)$ that $s$ is completely labeled by $l_s$,
such that $s\neq s_0$.
}
\BlankLine
choose a missing label $k\in [d]$ \\
pivot on label $k$ from $s_0$ to $s$ reaching a new \1 bit
with label $h$\\
\While{ $h\ne k$, so $s$ is not completely labeled ~}
{
pivot away from the other \1 bit in $s$ with label $h$ from $s$ to $s'$  \\
let $h$ be the label of the new \1 bit in $s'$ and
set $s = s'$
}
\Return $s$
\caption{Lemke Path for Gale}
\label{lhg-alg}
\end{algorithm}

The next example illustrates the correspondence between the
Lemke Path Algorithm and the Lemke Path for Gale Algorithm.

\begin{example}
Figure \ref{lhg-123432-fig} shows the cyclic polytope $C_4(6)$
with the labeling
\[
\begin{array}{rll}
l_v(i)= & i\quad & \text{ for }i\in [4], \\
l_v(5)= & 3\,,  \\
l_v(6)= & 2\,.
\end{array}
\]
This corresponds to the labeling $l_s$ for $G(4,6)$ given in
Example \ref{c46-123432-ex}, which gives the four completely labeled Gale
strings
$s_A=\1\1\1\100$, $s_B=\1\10\1\10$, $s_C=\100\1\1\1$ and
$s_D=\10\1\10\1$. These correspond to the facets $A$, $B$, $C$ and
$D$ of $C_4(6)$.
Pivoting from $s_A=\1\1\1\100$ with missing label 3 returns $s_B=\1\10\1\10$.
Analogously, pivoting from facet $A$ with missing label 3 returns facet $B$.

\clearpage

\begin{figure}[ht]
\strut\hfill
\includegraphics[width=40ex]{chapter-3/fig-lh/123432-lh-pivot.pdf}%
\hfill
\small
\begin{tabular}{c | c @{ } c @{ } c @{ } c @{ } c @{ } c @{ } c }
facet & {\bf 1} & {\bf 2} & {\bf 3} & {\bf 4} & {\bf 3} & {\bf 2}\\
\hline
{\bf A} & \1 & \1 & $\underline{\1}$ & \1 & 0               & 0 \\
{\bf B} & \1 & \1 & 0                & \1 & $\overline{\1}$ & 0 \\
\hline
facet & {\bf 1} & {\bf 2} & {\bf 3} & {\bf 4} & {\bf 3} & {\bf 2}
\end{tabular}
\hfill\strut
\caption[A pivot on $G(4,6)$ and on $C_4(6)$]{%
The pivoting to $s_A=\1\1\1\1 00$ to $s_B=\1\10\1\10$ in the
Lemke Path
for Gale Algorithm corresponds to the pivoting from facet $A$ to facet $B$
in the Dual Lemke Path Algorithm.%
}
\label{lhg-123432-fig}
\end{figure}
\end{example}

The membership of \anothergale\ in the complexity class
{\bf PPA} follows from an argument similar to Proposition
\ref{lh-works-ppa-thm}.
We extend the result to prove its membership in {\bf PPAD}.

A {\em permutation} of elements of an ordered set $S$ is a sequence
without repetition. This gives a rearrangement of the elements of $S$.
A {\em transposition} is a permutation of exactly two elements.
The {\em sign of a permutation} is
$\sign(\sigma)=(-1)^m$, where $m$ is the number of transpositions needed
to get the {\em natural order} $\sigma_0=1\ldots n$ from $\sigma$.
It is immediate to see that any two permutations that differ in only
one transposition have opposite sign.
We define the {\em sign of a completely labeled Gale string} $s\in G(d,n)$
%% what is "relative labeling"?
% as follows: let $l:[n]\to [d]$ be the relative labeling of $G(d,n)$, and
as follows: let $l:[n]\to [d]$ be the labeling of $G(d,n)$, and
let $l_0$ be the string of labels $l(i)$ such that $s(i)=\1$ and that
two labels corresponding to a run in $l$ are adjacent in $l_0$. Then we
define $\sign(s)=\sign(l_0)$.
Notice that if $l(i)=i$ for $i\in [d]$ then the sign of the completely
labeled Gale string $\1^d 0^{(n-d)}$ is always positive.
The {\em sign of an almost completely labeled Gale string} $s\in G(d,n)$ with
missing label~$k$ and duplicate label $h$ is defined on two different strings.
Let $i_1$ be the index of $h$ reached by the last pivot
(the ``new'' position of the \1) and let $i_2$ be the index of $h$ such that
$s(i_2)=\1$ before the last pivot (the ``old'' position of the \1).
Let $l_1$ be the string obtained as $l_0$ substituting $k$ to
$h$ at index $i_1$, and let $l_2$ be the string obtained as $l_0$
substituting $k$ to $h$ at index $i_2$. Notice that $\sign(l_1)=-\sign(l_2)$,
since they can be obtained from each other applying the transposition $(kh)$.

Consider now the steps of the Lemke paths in the Lemke Path for Gale
Algorithm in the case where $\sign(s_0)=+1$.
The negative case is analogous, with opposite signs.
If the first pivot returns another completely labeled Gale
string $s$, this must have negative sign because it has been obtained
``jumping'' over an odd number of \1's.
For the same reason, if the pivoting returns an almost completely labeled
Gale string, we have that $\sign(l_1)=-1$, which implies $\sign(l_2)=+1$.
The next pivoting step drops the label $h$ from index $i_2$, so again we
change sign. This shows that
the Lemke Path for Gale Algorithm results in the sign
of the completely and almost completely labeled Gale strings
``swinging'' as in Table \ref{lhg-sign-figure}.
Notice that all this construction can be done in polynomial time.
Orienting all Lemke paths from positive to negative reduces the problem
\anothergale\ to {\sc End Of The Line}.
\begin{table}[hbt]
$\xymatrix@C=30mm@R-=5mm@M=3mm{
\text{\footnotesize completely labeled string} \ar[r]^{pivot}
    & \text{\footnotesize missing label in new position}  \ar[dl]|{=}  \\
\text{\footnotesize missing label in old position} \ar[r]^{pivot}
    & \qquad\qquad\cdots\qquad\qquad \ar[dl]|{=} \\
\qquad\qquad\cdots\qquad\qquad \ar[r]^{pivot}
    & \text{\footnotesize completely labeled string}
}$
\caption[Sign switching on the Lemke Path for Gale Algorithm]
{Sign switching on the Lemke Path for Gale Algorithm.}
\label{lhg-sign-figure}
\end{table}

\begin{proposition}\label{lhg-works-ppad-thm}
The Lemke Path for Gale Algorithm~\ref{lhg-alg} returns a
solution to the {\bf PPAD} problem \anothergale.
% d even assumed in Algo
Furthermore, the number of completely labeled Gale strings
$s\in G(d,n)$ is even, and the two completely labeled Gale
strings at opposite ends of any Lemke path have opposite
sign.
\end{proposition}

\begin{example}
Let $l_s=123432$. Consider the Lemke path from the completely labeled
Gale string $s=\1\1\1\100$ with missing label $1$.
Figure \ref{lhg-sign-ex-figure} shows the graph of Table
\ref{lhg-sign-figure}. Notice that
$\sign(\10\1\10\1)=\sign(l(6)l(1)l(3)l(4))=\sign((2134))$,
since $s(6)=s(1)=1$ and therefore the indices $6$ and $1$ are consecutive
in the same run.
\begin{figure}[h]
\strut\hfill
\includegraphics[width=60ex]{chapter-3/fig-lh/lhg-sign.pdf}%
\hfill\strut
\caption[Pivoting with sign]
{Pivoting with sign on $123432$.}
\label{lhg-sign-ex-figure}
\end{figure}
\end{example}

A generalization of Proposition \ref{lhg-works-ppad-thm}
is given in Shapley \cite{shapley}: two Nash equilibria at the ends of a
Lemke path have opposite {\em index}, a concept analogous to the sign but
defined using determinants on the payoff matrices for the equilibrium
support.
The index of a Nash equilibrium is usually normalized (by
multiplication with $-1$ of all signs if necessary) so that
the artificial equilibrium has index $-1$.
Then a nondegenerate game with $n$ Nash equilibria
with index $+1$ has $n-1$ Nash equilibria with index $-1$.

\clearpage

Morris \cite{morris} gave an example of a labeling
$l:[2d]\to [d]$ where the length of the Lemke paths on the
cyclic polytope $C_d(2d)$ for the Lemke Path
Algorithm~\ref{lh-alg}
grows exponentially in $d$ for every missing label.
These paths are therefore also exponential on $G(d,2d)$ for
the Lemke Path for Gale Algorithm~\ref{lhg-alg}.
The labeling by Morris is given for $d$ both
even and odd as follows:
\[
\arraycolsep.2em
\begin{array}{rcll}
l(k)&=&k          & \quad\quad\text{for }k\in [d],    \\
l(d+1)&=&d,       & \\
l(d+k)&=&d-k      & \quad\quad\text{for }2\leq k< d,\ k\text{ even },   \\
l(d+k)&=&d-k+2    & \quad\quad\text{for }2\leq k\leq d,\ k\text{ odd }, \\
l(2d)&=&1         & \quad\quad\text{for }k\text{ even}.
\end{array}
\]
Without repetitions of labels in consecutive positions,
and restricted to the case of even~$d$,
Morris's labeling is simplified to $l:[2d-2]\to [d]$ as follows:
\[
\arraycolsep.2em
\begin{array}{rcll}
l(k)&=&k          & \quad\quad\text{for }k\in [d],    \\
l(d+k)&=&d-k+1    & \quad\quad\text{for }k\in [d],\ k\text{ even},   \\
l(d+k)&=&d-k-1    & \quad\quad\text{for }k\in [d],\ k\text{ odd}.
\end{array}
\]
Savani and von Stengel \cite{svs} \cite{uvg} extend Morris's
example in order to construct different ``hard to solve'' games.
These results give a strong motivation to study the
complexity of \anothergale.
Our main result, in the next section, will give a
{\bf FP} algorithm that circumvents the problem of any exponential running
time of the Lemke Path for Gale Algorithm by using a
different algorithm.

\clearpage

\begin{example}
\label{morris-ex}
Consider the labeling $l=1234564523$ for $G(6,10)$. The only two completely
labeled Gale string  are $s=\1\1\1\1\1\100$ and $s'=\1 00000\1\1\1\1\1$.
Table \ref{morris-6-fig} shows the Lemke path for missing label~1.
\begin{table}[hbt]
\begin{center}
\large
\begin{tabular}{c @{ } c @{ } c @{ } c @{ } c @{ } c @{ } c @{ } c @{ } c @{ } c @{ } c }
{\bf 1} & {\bf 2} & {\bf 3} & {\bf 4} & {\bf 5} & {\bf 6} & {\bf 4} & {\bf 5} & {\bf 2} & {\bf 3} \\
\hline
\d1 & \1 & \1 & \1 & \1 & \1 & 0 & 0 & 0 & 0 \\
0 & \1 & \1 & \d1 & \1 & \1 & \u1 & 0 & 0 & 0 \\
0 & \1 & \1 & 0 & \d1 & \1 & \1 & \u1 & 0 & 0 \\
0 & \d1 & \1 & 0 & 0 & \1 & \1 & \1 & \u1 & 0 \\
0 & 0 & \1 & \u1 & 0 & \1 & \d1 & \1 & \1 & 0 \\
0 & 0 & \1 & \1 & \u1 & \1 & 0 & \d1 & \1 & 0 \\
0 & 0 & \d1 & \1 & \1 & \1 & 0 & 0 & \1 & \u1 \\
0 & 0 & 0 & \d1 & \1 & \1 & \u1 & 0 & \1 & \1 \\
0 & 0 & 0 & 0 & \d1 & \1 & \1 & \u1 & \1 & \1 \\
\u1 & 0 & 0 & 0 & 0 & \1 & \1 & \1 & \1 & \1 \\
\hline
{\bf 1} & {\bf 2} & {\bf 3} & {\bf 4} & {\bf 5} & {\bf 6} & {\bf 4} & {\bf 5} & {\bf 2} & {\bf 3} \\
\end{tabular}
\normalsize
\end{center}
\caption[A Lemke path for the Morris labeling]
{The Lemke path for the Morris labeling on $G(4,6)$ with missing label 1.
The bit position \1 that is dropped in the next pivoting step
is underlined, the bit \1 that has just been picked up is
overlined.}
\label{morris-6-fig}
\end{table}
\end{example}
