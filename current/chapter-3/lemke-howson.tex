\section{The Lemke-Howson Algorithm}
\label{lh-section}

Let $P$ be a simple $d$-polytope with $n$ facets.
We {\em pivot on the vertices} of $P$ by moving from a
vertex $x$ to another vertex $y$ connected to $x$ by an edge.
Note that, since $P$ is simple, there are exactly
$d$ possible choices for $y$.
Analogously, we {\em pivot on the facets} of a simplicial polytope $Q$ in
dimension $d$ by moving from a facet $F$ to a facet $G$ that share
all vertices but one; since $Q$ is simplicial, there are $d$ possible
choices for $G$.

Suppose now that there is a labeling $l_f:[n]\to [d]$ of the facets of the
simple polytope $P$.
If we pivot from a vertex $x$ to a vertex $x'$ we ``leave behind'' a facet $F$
with label $k$; so, if $x$ has labels $(l_1,\ldots,k,\ldots,l_d)$, then
$x'$ has labels $(l_1,\ldots,h,\ldots,l_d)$, where $h$ is the label of the
facet $F'$ that does not have $x$ as its vertex. We call this
{\em dropping label $k$ and picking up label $h$}, or
{\em pivoting on label $k$}; see Figure \ref{pivot-poly} left for an example.
Analogously, if there is a labeling $l_v:[n]\to [d]$ of the vertices
of the simplicial poytope $Q$
and we pivot from a facet $F$ with labels $(l_1,\ldots,k,\ldots,l_d)$
to a facet $F'$ with labels $(l_1,\ldots,h,\ldots,l_d)$, we say that we
{\em drop label $k$ and pick up label $h$}, or that we
{\em pivot on label $k$}; see Figure \ref{pivot-poly-fig} right.

\begin{figure}[hbt]
\strut\hfill
\includegraphics[width=35ex]{chapter-3/fig-lh/cube-1-pivot.pdf}%
\hfill
\includegraphics[width=35ex]{chapter-3/fig-lh/octahedron-1-pivot.pdf}%
\hfill\strut
\caption[Two examples of pivoting]{%
Left: A pivot on the vertices of the cube.\\
Right: A pivot on the facets of the octahedron.
}
\label{pivot-poly-fig}
\end{figure}

Let $m,n\in \naturals$ with $m\leq n$; consider a set $X$
and a labeling $l:X\to [m]$. The $m$-uple $x=(x_1,\ldots,x_m)\in X^m$
is {\em almost completely labeled} if
$\{ j\in [n]\ |\ x_i=j\text{ for some }i\in [m] \}= [m]\setminus \{ k \}$
for exactly one $k\in [m]$. That is, all labels appear once in $x$,
except for the {\em missing label} $k$ and a {\em duplicate label}
$h\in [m]$ that appears twice.
It's easy to see that if we pivot from an almost completely labeled facet
(or vertex) on the duplicate label, or from a completely labeled facet (or
vertex) on any label, we reach either an almost completely labeled or a
completely labeled facet (or vertex).

The classic Lemke-Howson Algorithm, see Algorithm \ref{lh-alg}, relies on
pivoting on the vertices of a simple polytope. It was first introduced by
Lemke and Howson \cite{lh}; we follow the very clear exposition given by
Shapley \cite{shapley}.

\begin{algorithm}
\SetKwInOut{Input}{input}
\SetKwInOut{Output}{output}
\Input{
A simple $d$-polytope $P$ with $n$ facets.
A labeling $l_f:[n]\to [d]$ of the facets of $P$.
A vertex $x_0$ of $P$, completely labeled for $l$.
}
\Output{
A completely labeled vertex $x\neq x_0$ of $P$.
}
\BlankLine
choose any label $k\in [d]$ \\
pivot on label $k$ from $x_0$ to $x$ \\
\While{ $x$ is not completely labeled }
{
pivot on the duplicate label $h$ from $x$ to $x'\neq x_0$  \\
set $x_0 = x$, $x = x'$
}
\Return $x$
\caption{Lemke-Howson Algorithm}
\label{lh-alg}
\end{algorithm}

The Lemke-Howson Algorithm goes through a series of almost completely
labeled vertices with missing label $k$; these vertices can be seen as
steps of a path, called {\em Lemke path}. We will use Lemke paths to prove
some fundamental properties of both the Lemke-Howson Algorithm and
the problem {\sc Another Completely Labeled Vertex}.

\begin{proposition}\label{lh-works-ppa-thm}
The Lemke-Howson Algorithm, see Algorithm \ref{lh-alg}, returns a solution to the
{\bf PPA} problem {\sc Another Completely Labeled Vertex}.

Furthermore, the number of completely labeled vertices in a simple
polytope with labeled facets is even.

\begin{proof}
We first prove that the Lemke-Howson Algorithm works.
At each almost completely labeled vertex $x'$ of the Lemke path with
duplicate label $h$ there are only two facets with duplicate label $h$,
each one of which corresponds to exactly an edge, since $P$ is simple.
One of these is the edge that has been traversed to get to $x'$, the
other one will be traversed to leave it in the next step; therefore,
there are no ``loops'' where a vertex is visited more than once. (The
Lemke paths are {\em simple}.)

The parity is proven by the following argument. Each Lemke path is
uniquely determined by its missing label and its starting point, so the
Lemke path from the endpoint with the same missing label will lead back
to the starting point. Since the endpoint and the starting point are
different, the Lemke paths must connect an even number of points.

Finally, for each label $k\in [d]$ chosen in line 1 of
Algorithm \ref{lh-alg}, the Lemke paths are disjoint paths connecting
all the completely labeled vertices of $P$, with a standard endpoint $x_0$.
The problem {\sc Another Completely Labeled Vertex} correspond to finding
a non-standard endpoints of this graph; this is clearly a {\bf PPA} problem.
\end{proof}
\end{proposition}

Applying the parity result of Proposition \ref{lh-works-ppa-thm} to the
case of a bimatrix game (not necessarily a unit vector game), and
remembering that the point $(\0,\0)$ corresponds to the
``artificial'' equilibrium, we have the following result, due to Lemke and
Howson \cite{lh}.

\begin{theorem}{\rm (Lemke-Howson \cite{lh})}
Every non-degenerate bimatrix game has an odd number of Nash equilibria.
\end{theorem}

There are two ways of using the Lemke-Howson Algorithm to find a Nash
equilibrium of a bimatrix game $(A,B)$.
The first one is to ``symmetrize'' the game as in Proposition
\ref{symmetrize-c}. Let
$S = \{ z\in\reals^{m+n}\ |\ z\geq\0,\ Cz\leq\1 \}$ be the
polytope associated to the game $(C,C\T)$, where $C=\binom{0\ A}{B\T\ 0}$.
The facets of $C$ correspond to $2(m+n)$ inequalities; we label both
the $i$-th and the $m+n+i$-th inequality as $i\in [m+n]$ and we apply the
Lemke-Howson Algorithm starting from the vertex $\0$. This returns a Nash
equilibrium $(z,z)$ of $C$, which corresponds to a Nash equilibrium
$(x,y)=z$ of $(A,B)$.
We can also follow the ``traditional'' version of the Lemke-Howson
Algorithm, starting from the couple of vertices $(\0,\0)$ and alternating
moves on the best response polytopes $P$ and $Q$ of (\ref{br-polytopes}).
Since we move in $\reals^m$ and $\reals^n$ instead of $\reals^{m+n}$,
this version is much easier to visualize.

\begin{example}
Consider the $3\times 3$ game $(A,B)$ of Example \ref{br-game-ex}.
\begin{equation*}
A = \left(\begin{matrix}1&0&0\\ 0&1&0\\
0&0&1\end{matrix}\right),
\qquad
B = \left(\begin{matrix}0&2&4\\ 3&2&0\\
0&2&0\end{matrix}\right).
\end{equation*}
The best response polytopes can be represented as the best response regions
(see figure \ref{br-regions-fig}) extended to the origin $\0$, as in figure
\ref{lh-path-fig}; the label ``outside'' refers to the ``back'' of the
polytope.
\begin{figure}[hbt]
\strut\hfill
\includegraphics[width=65ex]{chapter-3/fig-lh/lh.pdf}%
\hfill\strut
\caption[A Lemke path for a bimatrix game]{%
Lemke path for missing label 2 on the best response polytopes of
game (\ref{AB}).
}
\label{lh-path-fig}
\end{figure}
The path starts in $(\0,\0)$; we drop the label 2 and we move on the polytope
$P$. The label 6 is duplicate; so we drop the label 6 and we move on the
polytope $Q$; an so on until we reach the point $x$, that is a Nash
equilibrium of $(A,B)$.
\end{example}

The dual version of the Lemke-Howson Algorithm \ref{lh-alg}
and of proposition \ref{lh-works-ppa-thm} is quite straightforward.
\begin{algorithm}
\SetKwInOut{Input}{input}
\SetKwInOut{Output}{output}
\Input{
A simplicial $m$-polytope $Q$ with $n$ vertices.
A labeling $l_v:[n]\to [d]$ of the vertices of $P$.
A vertex $F_0$ of $Q$, completely labeled for $l$.
}
\Output{
A completely labeled facet $F\neq F_0$ of $Q$.
}
\BlankLine
choose any label $k\in [d]$ \\
pivot on label $k$ from $F_0$ to $F$ \\
\While{ $x$ is not completely labeled }
{
pivot on the duplicate label $h$ from $F$ to $F'\neq x_0$  \\
set $F_0 = x$, $F = F'$
}
\Return $x$
\caption{Dual Lemke-Howson Algorithm}
\label{lh-dual-alg}
\end{algorithm}

\begin{proposition}\label{lh-dual-works-ppa-thm}
The Dual Lemke-Howson Algorithm, see Algorithm \ref{lh-dual-alg}, returns
a solution to the {\bf PPA} problem {\sc Another Completely Labeled Facet}.

Furthermore, the number of completely labeled facets
in a simplicial polytope with labeled vertices is even.
\end{proposition}

By Theorem \ref{unit-vector-thm} and Theorem \ref{unit-vector-dual-thm},
in the case of unit vector games it is enough to apply the Lemke-Howson
Algorithm to the polytope
$P^l= \{ x\in\reals^m\ |\ x\geq\0,\ B\T x\leq\1 \}$ in (\ref{p-l}),
or the Dual Lemke-Howson Algorithm to the polytope
$Q=\conv(\{ e_1,\ldots,e_m \})\ \cup\ \{ c_1,\ldots,c_n \})$ in
(\ref{p-l-dual}).
The following theorem by Savani and von Stengel \cite{uvg} guarantees that
not only this yield a Nash equilibrium, but no potential solutions
are ``lost'' considering the polytope $P^l$ with $m$ labels
instead of the product of polytopes $P\times Q$ with $m + n$ labels,
as stated in t; an analogous result holds for the dual case.

\begin{theorem}\label{unit-paths}
Let $(U,B)$ be a unit vector game, with
$U=(e_{l(1)}\cdots e_{l(n)})$ for a labeling $l:[n]\to [m]$.
Let $P = \{ x\in\reals^m | x\geq\0,\ B\T x\leq\1 \}$ and
$Q = \{ y\in\reals^n | y\geq\0,\ A y\leq\1 \}$ as in (\ref{br-polytopes});
let $P^l= \{ x\in\reals^m\ |\ x\geq\0,\ B\T x\leq\1 \}$
as in (\ref{p-l}). Then the Lemke path on $P\times Q$ for the missing
label $k\in [m]$ projects to a path on $P$ that is the Lemke path on $P^l$
for missing label $k$, and the Lemke path on $P\times Q$ for missing label
$k=m+j$, where $j\in [n]$, projects on the Lemke path on $P^l$ for
missing label $l(j)$.
\end{theorem}

We finally focus on the case of unit vector games
where the simplicial polytope $Q$ is cyclic; that is,
the case that we can study from the point of view of Gale strings.
We will consider $s\in G(m,n)$ with $d$ even, and we will see them as
``loops''.
Let $s(i)=1$ for an index $i\in [n]$; then, by Gale evenness condition,
there is an odd run of \1 's in $s$ either on the left or on the right
of position $i$; let $j$ be the first index after this run.
A {\em pivot on $s$} is then defined as setting $s(i)=0$ and $s(j)=1$.
Given a labeling $l_s:[n]\to [m]$, we say that we
{\em pivot on label $l(i)$}, {\em dropping label $l(i)$} and
{\em picking up label $l(j)$}.
The {\em Lemke-Howson for Gale Algorithm} is defined as follows.

\begin{algorithm}\label{lhg-alg}
\SetKwInOut{Input}{input}
\SetKwInOut{Output}{output}
\Input{
A labeling $l_s:[n]\to [d]$, where $d$ is even, such that there is a
completely labeled Gale string $s_0 \in G(d,n)$.
}
\Output{
A completely labeled Gale string $s\in G(d,n)$ such that $s\neq s_0$.
}
\BlankLine
choose a label $k\in [d]$ \\
pivot on label $k$ from $s_0$ to $s$ \\
\While{ $s$ is not completely labeled }
{
pivot on the duplicate label $h$ from $s$ to $s'\neq s_0$ \\
rename $s_0=s$, $s=s'$
}
\Return $s$
\caption{Lemke-Howson for Gale Algorithm}
\label{lhg-alg}
\end{algorithm}

We see the correspondence between the Lemke-Howson Algorithm and the
Lemke-Howson for Gale Algorithm in the next example.

\begin{example}
Figure \ref{lhg-123432-fig} shows the cyclic polytope $C_4(6)$
with the labeling
\begin{eqnarray*}
l_v(i)= & i\quad\text{ for }i\in [4]; \\
l_v(5)= & 3; \\
l_v(6)= & 2.
\end{eqnarray*}
This corresponds to the labeling $l_s$ for $G(4,6)$ given in
example \ref{c46-123432-ex}, for which there are
four completely labeled Gale strings:
$s_A=\1\1\1\100$, $s_B=\1\10\1\10$, $s_C=\100\1\1\1$ and
$s_D=\10\1\10\1$, which in turn correspond to the facets $A$, $B$, $C$ and
$D$. Pivoting from facet $A$ dropping label 3 yields facet $B$, just as
pivoting from $s_A=\1\1\1\100$ dropping label 3 yields $s_B=\1\10\1\10$.
\begin{figure}[hbt]
\strut\hfill
\includegraphics[width=60ex]{chapter-3/fig-lh/123432-lh.pdf}%
\hfill\strut
\caption[A pivot on $G(4,6)$ and on $C_4(6)$]{%
The pivoting to $s_A=\1\1\1\1 00$ to $s_B=\1\10\1\10$ in the Lemke-Howson
for Gale Algorithm corresponds to the pivoting from facet $A$ to facet $B$
in the Dual Lemke-Howson Algorithm.%
}
\label{lhg-123432-fig}
\end{figure}
\end{example}

Once again, we have the analogous of Proposition \ref{lh-works-ppa-thm}.

\begin{proposition}\label{lhg-works-thm}
The Lemke-Howson for Gale Algorithm, see Algorithm \ref{lhg-alg}, returns a
solution to the {\bf PPA} problem \anothergale.

Furthermore, the number of completely labeled Gale strings
$s\in G(d,n)$, where $d$ is even, is even.
\end{proposition}

In the case of Gale strings, it is quite easy to extend the proof of
Proposition \ref{lh-works-ppa-thm} and prove a stronger result of
{\bf PPAD} rather than just {\bf PPA} complexity.
We will do so by giving a {\em sign}, positive or negative,
to the completely labeled Gale strings, then proving that all the
endpoints of the Lemke paths have different sign. If the sign of the string
$s_0$ is positive, we can then orient the Lemke paths from positive to
negative, so that $s_0$ is a standard source; if the sign of the string $s_0$
is negative, we will give the opposite orientation.

A {\em permutation} of the elements of an ordered set $S$ is a sequence
without repetition of elements of $S$.
The {\em sign} of a permutation $\sigma$ is defined as
$\sign(\sigma)=(-1)^m$, where $m$ is the number of the exchanges of
exactly two elements of $\sigma$ (called {\em transpositions}) needed
to get $\sigma'=1\ldots n$ from $\sigma$. Note that any two permutations
that differ in only one transposition have opposite sign.
The {\em sign of a completely labeled Gale string} $s\in G(d,n)$ is defined
as follows. Let $l:[n]\to [d]$ be a labeling that completely labels $s$, and
let $l_0$ be the string of labels $l(i)$ such that $s(i)=1$ and that
two labels corresponding to a run in $l$ are adjacent in $l_0$. Then we
define $\sign(s)=\sign(l_0)$.
Note that if $l(i)=i$ for $i\in [d]$ then the sign of the completely
labeled Gale string $1^d 0^{(n-d)}$ is always positive.
Let $s\in G(d,n)$ be an almost completely labeled string for $l$ with
missing label $k$ and duplicate label $h$; let $i_1$ be the index of $h$
reached by the last pivot (the ``new'' position of $h$) and let $i_2$ be
the index of $h$ such that $s(i_2)=1$ before the last pivot (the ``old''
position). Let $l_1$ be the string obtained as $l_0$ substituting $k$ to
$h$ at index $i_1$, and let $l_2$ be the string obtained as $l_0$
substituting $k$ to $h$ at index $i_2$. Then $\sign(l_1)=-\sign(l_2)$,
since they can be obtained from each other by the transposition $(kh)$.
Consider now the steps of the Lemke-Howson for Gale Algorithm;
assume that $\sign(s_0)=+1$ (without loss of generality: the negative case
is symmetric). If the first pivot returns another completely labeled Gale
string $s'$, this must have negative sign because it has been obtained
``jumping'' over an odd number of \1's.
For the same reason, if the pivoting returns an almost completely labeled
Gale string, we have that $\sign(l_1)=-1$, so $\sign(l_2)=+1$.
The next pivoting step drops the label $h$ from index $i_2$, so again we
change sign.
Running the Lemke-Howson for Gale Algorithm will therefore result in sign
switching back and forth as in Table \ref{lhg-sign-figure}.
\begin{table}[hbt]
$\xymatrix@C=30mm@R-=5mm@M=3mm{
\text{\footnotesize completely labeled string} \ar[r]^{pivot}
    & \text{\footnotesize missing label in new position}  \ar[dl]|{=}  \\
\text{\footnotesize missing label in old position} \ar[r]^{pivot}
    & \qquad\qquad\cdots\qquad\qquad \ar[dl]|{=} \\
\qquad\qquad\cdots\qquad\qquad \ar[r]^{pivot}
    & \text{\footnotesize completely labeled string}
}$
\caption[Sign switching on the Lemke-Howson for Gale Algorithm]
{Sign switching on the Lemke-Howson for Gale Algorithm.}
\label{lhg-sign-figure}
\end{table}
Orienting all Lemke paths from positive to negative gives the {\bf PPAD}
complexity of \anothergale.

\begin{proposition}\label{gale-ppad-thm}
\anothergale\ is in {\bf PPAD}.
\end{proposition}

\begin{example}
Let $l_s=123432$; consider the Lemke path from the completely labeled
Gale string $s=\1\1\1\100$ dropping label $1$.
Figure \ref{lhg-sign-ex-figure} shows the graph of Table
\ref{lhg-sign-figure}. Note that $\sign(\10\1\10\1)=\sign((21)(34))$,
since $s(6)=s(1)=1$ and therefore the indices $6$ and $1$ are consecutive
in the same run.
\begin{figure}[hbt]
\strut\hfill
\includegraphics[width=75ex]{chapter-3/fig-lh/pivoting-sign-example.pdf}%
\hfill\strut
\caption[An example of pivoting with sign]
{Pivoting with sign on $123432$.}
\label{lhg-sign-ex-figure}
\end{figure}
\end{example}

A result similar to Proposition \ref{gale-ppad-thm}, but more general, is given
in Shapley \cite{shapley}; it shows that two equilibria at the ends of a
Lemke path have opposite {\em index}. The index is defined in terms of the
signs of the determinants of the square submatries of the payoff matrices
for the equilibrium support; the artificial equilibrium is assigned
index $+1$. The main result of Shapley's article is that if a
nondegenerate game has $n$ Nash equilibria with index $+1$, then the game has
$n+1$ Nash equilibria with index $-1$. The article also gives an interesting
example of a game for which the graph of all Lemke paths is disjoint.
\begin{equation}
\label{disj-lp-game}
A = \left(
    \begin{matrix}
        2&2&0 \\
        0&3&0 \\
        3&0&1
    \end{matrix}
    \right),
\qquad
B = \left(
    \begin{matrix}
        3&0&2 \\
        0&3&2 \\
        0&0&1
    \end{matrix}
    \right),
\end{equation}
The equilibria of $(A,B)$ are $(x_1,y_1)=((0,0,1), (0,0,1))$,
$(x_2,y_2)=((0,1/2,1/2),(0,3/4,1/4))$ and
$(x_3,y_3)=((0,1,0),(0,1,0))$. All Lemke paths from the
artificial equilibrium $(0,0)$ end at $(x_1,y_1)$, and consequently all
other Lemke paths connect $(x_2,y_2)$ and $(x_3,y_3)$; see
Figure \ref{disj-lp-br-poly}.
\begin{figure}[hbt]
\strut\hfill
\includegraphics[width=38ex]{chapter-3/fig-lh/shapley-game-1.pdf}%
\hfill
\includegraphics[width=38ex]{chapter-3/fig-lh/shapley-game-2.pdf}%
\hfill\strut
\caption[Best response polytopes of a game with disjoint Lemke paths]{%
Best response polytopes of game \ref{disj-lp-game}.
}
\label{disj-lp-br-poly}
\end{figure}


An interesting example of the Lemke-Howson for Gale Algorithm is the
following, due to Morris \cite{morris}.

\begin{example}
Consider the labeling $l=1234564523$ for $G(6,10)$ and the completely
labeled Gale string $s=\1\1\1\1\1\100$. Dropping the label 1, the
Lemke-Howson for Gale algorithm will run as in figure \ref{morris-6-fig}.

\begin{figure}[hbt]
\begin{center}
\large
\begin{tabular}{c @{ } c @{ } c @{ } c @{ } c @{ } c @{ } c @{ } c @{ } c @{ } c @{ } c }
{\bf 1} & {\bf 2} & {\bf 3} & {\bf 4} & {\bf 5} & {\bf 6} & {\bf 4} & {\bf 5} & {\bf 2} & {\bf 3} \\
\hline
\d1 & \1 & \1 & \1 & \1 & \1 & \tdot & \tdot & \tdot & \tdot \\
\tdot & \1 & \1 & \d1 & \1 & \1 & \u1 & \tdot & \tdot & \tdot \\
\tdot & \1 & \1 & \tdot & \d1 & \1 & \1 & \u1 & \tdot & \tdot \\
\tdot & \d1 & \1 & \tdot & \tdot & \1 & \1 & \1 & \u1 & \tdot \\
\tdot & \tdot & \1 & \u1 & \tdot & \1 & \d1 & \1 & \1 & \tdot \\
\tdot & \tdot & \1 & \1 & \u1 & \1 & \tdot & \d1 & \1 & \tdot \\
\tdot & \tdot & \d1 & \1 & \1 & \1 & \tdot & \tdot & \1 & \u1 \\
\tdot & \tdot & \tdot & \d1 & \1 & \1 & \u1 & \tdot & \1 & \1 \\
\tdot & \tdot & \tdot & \tdot & \d1 & \1 & \1 & \u1 & \1 & \1 \\
\u1 & \tdot & \tdot & \tdot & \tdot & \1 & \1 & \1 & \1 & \1 \\
\hline
{\bf 1} & {\bf 2} & {\bf 3} & {\bf 4} & {\bf 5} & {\bf 6} & {\bf 4} & {\bf 5} & {\bf 2} & {\bf 3} \\
\end{tabular}
\normalsize
\end{center}
\caption{Morris path on $C(6,10)$}
\label{morris-6-fig}
\end{figure}
\end{example}



\todo[inline]{Morris paths are exponentially long


Savani and von Stengel (2006) construct bimatrix games where both players
have dual cyclic polytopes as their best response polytopes. They call
these games m ⇥ n-double cyclic polytope games. For square games,
where m = n = d and where the labels are derived from
Morris’s construction (see Morris (1994) and Section 4.4 for the
introduction of these labels), the LH-algorithm takes exponentially
many steps to find an equilibrium.

However, square games are not âĂIJhard to solveâĂİ by the sup-
port enumeration algorithm; see Savani (2006). The support enumeration
algorithm considers strategies of the players with equal support size and
checks whether they are best replies to each other. Since square games
have a unique completely mixed equilibrium, where both players play d
pure strategies with positive probability, the support enumeration algo-
rithm terminates quickly.
Games where both solution concepts take exponentially long are then
called hard- to-solve bimatrix games One class of hard-to-solve games is
constructed from 3d âĞě d games with one cyclic polytope of dimension d
and one simplotope, which is a product of simplices, here d tetrahedra (Sa-
vani (2006)). He calls these triple imitation games, due to the connection
to imitation games. Nash equilibria of triple imitation games are fully de-
scribed by completely labeled Gale strings in G(d,4d). They are thus Gale
games and an equilibrium can be found via the problem ANOTHER COM-
PLETELY LABELED GALE STRING.

}

This gives a strong motivation to study the complexity of \anothergale,
since it seems that it relates to games that are hard to solve. Our
main result, in the next section, will give a {\bf FP} algorithm to solve it.
