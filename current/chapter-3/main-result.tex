\section{The Complexity of \anothergale}

% proposition "enough to solve anothergale for d even": \label{d-even-anothergale}

A {\em matching} of a graph $G=(V,E)$ is a set $M\subseteq E$ such that every
vertex $v\in V$ is the endpoint of at most one edge $m\in M$. A
{\em perfect matching} is a matching such that there is an edge $m\in M$
incident to every vertex $v\in V$.
\linebreak[5]
Edmonds \cite{edm} proved the {\bf FP}
complexity of finding a perfect matching. The result can be easily extended
to multigraphs which have parallel edges, since a perfect matching of a multigraph $G$ corresponds to
a perfect matching of the graph $G'$ obtained by ignoring all parallel edges
of $G$.

\begin{theorem}{\rm (Edmonds \cite{edm}.)}
\label{edm-thm}
It takes polynomial time to find a perfect matching of a
graph or to decide that no such matching exists.
\end{theorem}

\begin{problem}
{Perfect Matching}
{A multigraph $G = (V,E)$.}
{A perfect matching for $G$, or {\sc No} if there is no possible perfect
matching for $G$.}
\label{pm-pbl}
\end{problem}

\begin{proposition}\label{pm-thm}
{\sc Perfect Matching} is in {\bf FP}.
\end{proposition}

To prove our main result on \anothergale, we will prove that the
%\linebreak[4]
related problem \gale\ can be solved in polynomial time.

\begin{problem}
{\gale}
{A labeling $l:[n]\to[d]$, where $d$ is even and $n>d$.}
{A Gale string $s\in G(d,n)$ that is completely labeled by $l$ or {\sc No}
if no such string exists.}
\end{problem}

\begin{theorem}\label{gale-thm}
\gale\ is in {\bf FP}.
\end{theorem}

\begin{proof}
Recall that we consider the Gale strings $s\in G(d,n)$ as
``wrapped-around strings'', that is,
$s(n+i)=s(i)$ and $l(n+i)=l(i)$ for all~$i\in[n]$.

Let $G=(V,E)$ be the multigraph with $V=[d]$, so that the vertices of $G$
correspond to the labels $l(i)\in [d]$, and
\begin{equation}
\label{edges-graph}
E=\{\,(l(i),l(i+1))\mid i\in[n]\, \},
\end{equation}
so that there is an edge
between two vertices if and only if the corresponding labels are
next to each other at some index $i$.

Let $s\in G(d,n)$ be a completely labeled Gale string for $l$.
By the Gale evenness
condition, every run of length $k$ in $s$ corresponds uniquely to $k/2$
disjoint pairs of indices $(i,i+1)$ with $s(i)=s(i+1)=1$.
Let $M$ be the set of corresponding edges $(l(i),l(i+1))$ in~$E$.
%% does not define M clearly
% \begin{equation}
% \label{matching-graph}
% M=\{ (l(i),l(i+1))\ \mid\ s(i)=s(i+1)=1\text{ without repetitions of } i \}.
% \end{equation}
Since $s$ is completely labeled, every label $l(i)\in [d]$
occurs at exactly one of the endpoints of an edge in~$M$.
Since the there are no repetitions of $i$'s in~$M$, there is only one such
edge. Hence, $M$ is a perfect matching of~$G$.

Conversely, let $M$ be a perfect matching for $G=(V,E)$.
% as in (\ref{matching-graph}).
Let $s$ be the bitstring with $s(i)=s(i+1)$ for every
$(l(i),l(i+1))\in M$ and $s(i)=0$ otherwise.
Since $M$ is a matching, all the $(l(i),l(i+1))\in M$ are disjoint, so
every run of $s$ is of even length. Therefore $s$ satisfies the Gale
evenness condition. Furthermore, since $M$ is perfect, every
vertex $l(i)\in [d]$ is the endpoint of an edge $(l(i),l(i+1))\in M$,
therefore $s$ is completely labeled.

This proves that \gale\ reduces to {\sc Perfect Matching} in polynomial time.
By Proposition \ref{pm-thm}, Algorithm~\ref{gale-alg} takes
polynomial time to find a solution of an instance of \gale\
or to decide that no such solution exists.
\end{proof}

It is interesting to notice that the graph $G=(V,E)$ is an Euler graph,
since its edges have been defined following an Euler tour starting in $l(1)$.

\begin{algorithm}
\SetKwInOut{Input}{input}
\SetKwInOut{Output}{output}
\Input{%
A labeling $l:[n]\to [d]$.
}
\Output{%
A Gale string $s\in G(d,n)$ that is completely labeled by $l$ or {\sc No} if
no such string exists.
}
\BlankLine
Set $G=(V,E)$, with $V=[d]$ and $E=\varnothing$ \\
\For{ $i\in [n-1]$ }
    {
    $E=E\cup\{ (l(i),l(i+1)) \}$
    }
$E=E\cup\{ (l(n),l(1)) \}$ \\
Run Edmonds's Algorithm on $G$ \\
\If{ Edmonds's Algorithm returns a perfect matching $M$ of $G$ }
    {
    set $s=0^n$ \\
    \For{ $(l(i),l(j))\in M$ }
        {
        set $s(i)=s(j)=\1$
        }
    Return $s$
    }
\Else{
    Return {\sc No}
}
\caption{Find a Completely Labeled Gale String}
\label{gale-alg}
\end{algorithm}

\clearpage

\begin{example}
\label{gs-pm-ex}
Figure \ref{perfect-matching-fig} shows the graph for the Morris labeling
\linebreak[5]
$l_s=1234564523$, and its two matchings
$M=\{ e_1,e_3,e_5 \}$ and $M'=\{ e_8,e_6,e_{10} \}$.
These, in turn, correspond to the completely labeled Gale strings
\linebreak[5]
$s=\1\1\1\1\1\1 0000$ and $s=\1 0000\1\1\1\1\1$.

\begin{figure}[h]
\strut\hfill
\includegraphics[width=35ex]{chapter-3/fig-main/morris-6-graph.pdf}%
\hfill\strut
\caption[The Morris graph for $G(6,4)$]{%
The perfect matchings for the Morris labeling $l=1234564523$, corresponding
to the Gale strings $s=\1\1\1\1\1\1 0000$ (in blue) and
$s=\1 0000\1\1\1\1\1$ (in red).
}
\label{perfect-matching-fig}
\end{figure}
\end{example}

Notice that the existence of a completely labeled Gale string is not
gua\-ran\-teed.
%\mbox{guaranteed.}

\begin{example}
Consider the labeling $l_s=121314$. Figure \ref{no-matching} shows the
corresponding graph. It is easy to see that a perfect matching is not
possible.
Analogously, as we have seen in Example \ref{no-clgs},
there isn't any possible completely labeled Gale string for $l_s$.
\begin{figure}[h]
\strut\hfill
\includegraphics[width=20ex]{chapter-3/fig-main/no-matching.pdf}%
\hfill\strut
\caption[A graph without a perfect matching]{%
The graph for the labeling $l=121314$
}
\label{no-matching}
\end{figure}
\end{example}

\clearpage

We finally extend the proof of Theorem \ref{gale-thm} to give the complexity
\linebreak[5]
of \anothergale.

\begin{theorem}
\label{anothergale-thm}
\anothergale\ is in {\bf FP}.
\end{theorem}

\begin{proof}
Let $G=(V,E)$ be the graph corresponding to the labeling
\linebreak[5]
$l:[n]\to [d]$ with $V=[d]$ and $E$ as in (\ref{edges-graph}). Let
$M_0$ be the perfect
\linebreak[5]
matching of $G$ corresponding to the completely
labeled Gale string $s_0\in G(d,n)$. % as in (\ref{matching-graph}).

Theorem \ref{lhg-works-ppad-thm} guarantees the existence of another
completely labeled Gale string $s\neq s_0$, therefore of a corresponding
perfect matching $M\neq M_0$.
Since $M\neq M_0$, there is at least one edge $e\in M_0$ such that
$e\notin M$. Consider the $d/2$ graphs $G_i=(V,E_i)$, where
$E_i=E\setminus\{ e_i \}$ for $e_i\in M_0$. Since $V(G)=V(G_i)$ and
$E(G_i)\subset E(G)$, a perfect matching $M_i$ of $G_i$
is a perfect matching for $G$ as well. We have that
$e_i\notin M_i\subset E_i$ but $e_i\in M_0$. Therefore $M_i\neq M_0$.

There are exactly $d/2$ possible $G_i$'s. By Proposition \ref{edm-thm},
finding a perfect matching or deciding that there is none takes polynomial
time for each $G_i$. Therefore, by applying Algorithm (\ref{anothergale-alg})
it is possible to find a perfect matching $M_i\neq M_0$
and the corresponding completely labeled Gale string $s\neq s_0$ in
polynomial time.
\end{proof}

\begin{algorithm}
\SetKwInOut{Input}{input}
\SetKwInOut{Output}{output}
\Input{%
A labeling $l:[n]\to [d]$ and a Gale string $s_0\in G(d,n)$ that is
completely labeled by $l$.
}
\Output{%
A Gale string $s\in G(d,n)$, completely labeled by $l$, such that
$s\neq s_0$.
}
\BlankLine
set $G=(V,E)$ as in Algorithm \ref{gale-alg} \\
set $M_0$ as the matching corresponding to $s_0$\\
% \\
% Set $i_0$ as the first $i$ such that $s(i)=1$ after a run of $0$'s \\
% \For{ $i\in [n-1]$, $i\geq i_0$ }
%     {
%     \If{ $s_0(i)=s_0(i+1)=1$ }
%         {
%         $M_0=M_0\cup\{ (l(i),l(i+1)) \}$
%         }
%     }
% \If{ $(l(n-1),l(n))\notin M_0$ and  $s_0(n)=s_0(1)=1$ }
%     {
%     $M_0=M_0\cup\{ (l(n),l(1)) \}$ \\
%     Set $i=1$
%     }
% \Else
%     {
%     Set $i=0$
%     }
% \For{ $i\in [i_0 - 1]$ }
%     {
%     \If{ $s_0(i) = s_0(i+1)=1$ }
%         {
%         $M_0=M_0\cup\{ (l(i),l(i+1)) \}$
%         }
%     }
\For{ $m\in M_0$ }
    {
    $E_m=E\setminus\{ m \}$ \\
    Run Edmonds's Algorithm on $G_m=(V,E_m)$ \\
    \If{ Edmonds's Algorithm returns a perfect matching $M_m$ of $G_m$ }
        {
        set $s=0^n$ \\
        \For{ $(l(i),l(j))\in M_m$ }
            {
            set $s(i)=s(j)=\1$
            }
        \Return $s$
        }
    }
\caption{Find Another Gale String}
\label{anothergale-alg}
\end{algorithm}


Applying Theorem \ref{galenash-to-another-gale} to
Theorem \ref{anothergale-thm} we have our main result: {\sc Gale Nash}
is in {\bf FP}, under the assumption that the construction
of the game from a cyclic polytope is given.

\begin{theorem}
\label{main-thm}
Finding a Nash equilibrium of a Gale game takes polynomial time.
\end{theorem}


\clearpage

\begin{example}
The labeling $l=123432$ gives the graph $G$ in Figure \ref{matching-2-edges}.
\linebreak[5]
Suppose that Edmonds's algorithm returns the matching $M_0=\{ e_1,e_3 \}$ of
\linebreak[5]
Figure \ref{matching-2-edges} (left), corresponding to the completely labeled
Gale string
\linebreak[5]
$s_0=\1\1\1\1 00$. A second perfect matching $M=\{ e_3,e_6 \}$,
corresponding to the Gale string $s=\1 0\1\1 0\1$, can be found in the
graph $G_1=(V(G),E(G)\setminus\{e_1 \})$, see Figure \ref{matching-2-edges}
(right).
\begin{figure}[h]
\strut\hfill
\includegraphics[width=25ex]{chapter-3/fig-main/matching-123432-1st.pdf}%
\hfill
\includegraphics[width=25ex]{chapter-3/fig-main/matching-123432-2nd.pdf}%
\hfill\strut
\caption[Finding a second perfect matching]{%
Left: The graph for the labeling $l=123432$ with the perfect matchings
corresponding to the completely labeled Gale strings $s_0=\1\1\1\1 00$.\\
Right: The graph $G_1=(V(G),E(G)\setminus\{e_1 \})$ and the perfect
matching corresponding to the Gale string $s=\1 0\1\1 0\1$.
}
\label{matching-2-edges}
\end{figure}
\end{example}

\clearpage

\begin{example}
Consider the Morris graph of Example \ref{gs-pm-ex}. Suppose that
\linebreak[5]
Edmonds's algorithm returns the perfect matching $M_0=\{ e_1,e_3,e_5 \}$,
as in
\linebreak[5]
Figure \ref{morris-matching} (left), corresponding
to the completely labeled Gale string
\linebreak[5]
$s_0=\1\1\1\1\1\1 0000$.
We can then delete the edge $e_1$ to obtain the graph
$G_1$, as in Figure \ref{morris-matching} (center). The graph $G_1$
has a perfect matching $M=\{ e_6,e_8,e_{10} \}$, shown in
Figure \ref{morris-matching} (right). This is also a perfect
matching of $G$. It corresponds to the only other completely labeled Gale
string $s=\1 0000\1\1\1\1\1$.

\begin{figure}[h]
\strut\hfill
\includegraphics[width=26ex]{chapter-3/fig-main/morris-6-graph-1st-match.pdf}%
\hfill
\includegraphics[width=26ex]{chapter-3/fig-main/morris-6-graph-inter.pdf}%
\hfill
\includegraphics[width=26ex]{chapter-3/fig-main/morris-6-graph-2nd-match.pdf}%
\hfill\strut
\caption[The second matching of the Morris graph]{%
Left: The Morris graph $G=(V,E)$ and its matching
$M=\{ e_1,e_3,e_5 \}$. \\
Center: The graph $G_1=(V(G),E(G)\setminus\{e_1 \})$. \\
Right: The set $M_1=\{ e_6,e_8,e_{10} \}$ is a perfect matching of both
\linebreak[4]
$G_1$ and $G$.
}
\label{morris-matching}
\end{figure}
\end{example}
