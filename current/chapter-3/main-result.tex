\section{The Complexity of \anothergale}

A {\em matching} of a graph $G=(V,E)$ is a set $M\subseteq E$ such that every
vertex $v\in V$ is the endpoint of at most one edge $m\in M$. A
{\em perfect matching} is a matching such that there is an edge $m\in M$
incident to every vertex $v\in V$. Edmonds \cite{edm} proved the {\bf FP}
complexity of finding a perfect matching. The result can be easily extended
to multigraphs.

\begin{problem}
{Perfect Matching}
{A multigraph $G = (V,E)$.}
{A perfect matching for $G$, or {\sc No} if there is no possible perfect
matching for $G$.}
\label{pm-pbl}
\end{problem}

\begin{theorem}{\rm (Edmonds \cite{edm})}\label{edm-thm}
It takes polynomial time to find the perfect matching of a graph or decide
that no such matching exists.
\end{theorem}

\begin{proposition}\label{pm-thm}
{\sc Perfect Matching} is in {\bf PF}.
\end{proposition}

To prove our main result on \anothergale, we will prove that
the accessory problem \gale\ can be solved in polynomial time.

\begin{problem}
{\gale}
{A labeling $l:[n]\to[d]$, where $d$ is even and $d<n$.}
{A Gale string $s\in G(d,n)$ that is completely labeled by $l$}
\end{problem}

\begin{theorem}\label{gale-thm}
\gale\ is in {\bf FP}.

\begin{proof}
Consider the multigraph $G=(V,E)$ with $V=[d]$, so that the vertices of $G$
correspond to the labels $l(i)\in [d]$, and
$E=\{(l(i),l(i+1))\text{ for }i\in[n] \}$, so that there is an edge
between two vertices if and only if the corresponding labels are
next to each other at some index $i$.

Let $s\in G(d,n)$ be a completely labeled Gale string. By the Gale evenness
condition, every run of length $k$ in $s$ corresponds uniquely to $k/2$
pairs of indices $(i,i+1)$ with $s(i)=s(i+1)=1$. Since $s$ is completely
labeled, all labels $l(i)\in [d]$ occur at exactly one of these indices and
the edges $(l(i),l(i+1))$ form a perfect matching of $G$.

Conversely, let $M$ be a perfect matching for $G=(V,E)$, defined as above.
Let $s$ be a bitstring with $s(i)=s(i+1)$ for every
$(l(i),l(i+1))\in M$ and $s(i)=0$ otherwise.
Since $M$ is a matching, all the $(l(i),l(i+1))\in M$ are disjoint, so
every run of $s$ is of even length; therefore $s$ satisfies the Gale evenness
condition. Furthermore, since $M$ is perfect, $s$ is completely labeled.

This proves that \gale\ reduces to {\sc Perfect Matching} in polynomial time;
therefore, by Proposition \ref{pm-thm}, Algorithm \ref{gale-alg} takes
polynomial time to find a solution of an instance of \gale\ or deciding that
no such solution exists.
\end{proof}
\end{theorem}

\newpage

\begin{algorithm}
\SetKwInOut{Input}{input}
\SetKwInOut{Output}{output}
\Input{
A labeling $l:[n]\to [d]$.
}
\Output{
A Gale string $s\in G(d,n)$ completely labeled by $l$.
}
\BlankLine
Set $G=(V,E)$, with $V=[d]$ and $E=\varnothing$ \\
Set $s=0^n$ \\
\For{ $i\in [n-1]$ }
    {
    $E=E\cup\{ (l(i),l(i+1)) \}$
    }
$E=E\cup\{ (l(n),l(1)) \}$ \\
Run the Edmonds' Algorithm on $G$ \\
\If{ The Edmonds' Algorithm returns a perfect matching $M$ of $G$ }
    {
    \For{ $(l(i),l(j))\in M$ }
        {
        $s(i)=s(j)=\1$
        }
    Return $s$
    }
\Else{
    Return {\sc No}
}
\caption{Completely Labeled Gale String Finder}
\label{gale-alg}
\end{algorithm}

\newpage

We give an example of the construction, using the Morris' labeling
of Example \ref{morris-ex}.

\begin{example}
\label{gs-pm-ex}
Figure \ref{perfect-matching-fig} shows the graph for the Morris
labeling $l_s=1234564523$, and its two matchings
$M=\{ e_1,e_3,e_5 \}$ and $M'=\{ e_8,e_6,e_{10} \}$.
These, in turn, correspond to the completely labeled Gale strings
$s=\1\1\1\1\1\1 0000$ and $s=\1 0000\1\1\1\1\1$.
\begin{figure}[hbt]
\strut\hfill
\includegraphics[width=25ex]{chapter-3/fig-main/morris-6-graph.pdf}%
\hfill\strut
\caption[The Morris graph]{%
The graph $G$ and its matchings for the Morris labeling $l=1234564523$.
}
\label{perfect-matching-fig}
\end{figure}
\end{example}

Notice that the existence of a completely labeled Gale string is not
\mbox{guaranteed.}

\begin{example}
Consider the labeling $l_s=121314$. Figure \ref{no-matching} shows the
corresponding graph, that has no possible perfect matching.
Analogously, as we have seen in Example \ref{no-clgs},
there isn't any possible completely labeled Gale string for $l_s$.
\begin{figure}[hbt]
\strut\hfill
\includegraphics[width=20ex]{chapter-3/fig-main/no-matching.pdf}%
\hfill\strut
\caption[A graph without a perfect matching]{%
The graph for the labeling $l=121314$
}
\label{no-matching}
\end{figure}
\end{example}

\newpage

We finally extend the proof of Theorem \ref{gale-thm} to the complexity
of \anothergale.

\begin{theorem}
\label{anothergale-thm}
\anothergale\ is in {\bf FP}.

\begin{proof}
Let $G=(V,E)$ be the graph corresponding to the labeling $l:[n]\to [d]$ as
in the proof of Theorem \ref{gale-thm} and let $M_0$ be the perfect matching
of $G$ corresponding to the completely labeled Gale string $s_0\in G(d,n)$.

Theorem \ref{lhg-works-thm} guarantees the existence of another
completely labeled Gale string $s\neq s_0$, therefore of the corresponding
perfect matching $M\neq M_0$.
Since $M\neq M_0$, there is at least one edge $e\in M_0$ such that
$e\notin M$. Consider the $d/2$ graphs $G_i=(V,E_i)$, where
$E_i=E\setminus\{ e_i \}$ for $e_i\in M_0$. Since $V(G)=V(G_i)$ and
$E(G_i)\subset E(G)$, if $G_i$ has a perfect matching $M_i$, then this
is a perfect matching for $G$ as well. Since $e_i\notin E_i$, then
$e_i\notin M_i$, and $M_i\neq M$.
Since the number of $G_i$ is exactly $d/2$, by Proposition \ref{edm-thm}
it takes at most polynomial time to find a perfect matching $M\neq M_0$
in one of the $G_i$'s, and therefore the corresponding completely labeled
Gale string $s\neq s_0$, applying Algorithm \ref{anothergale-alg}.
\end{proof}
\end{theorem}

\clearpage

\begin{algorithm}
\SetKwInOut{Input}{input}
\SetKwInOut{Output}{output}
\Input{
A labeling $l:[n]\to [d]$ and a Gale string $s_0\in G(d,n)$ completely
labeled by $l$.
}
\Output{
A Gale string $s\in G(d,n)$ completely labeled by $l$, such that
$s\neq s_0$.
}
\BlankLine
Set $G=(V,E)$, with $V=[d]$ and $E=M_0=\varnothing$ \\
Set $s=0^n$ \\
\For{ $i\in [n-1]$ }
    {
    $E=E\cup\{ (l(i),l(i+1)) \}$ \\
    \If{ $s_0(i)=s_0(i+1)=1$ }
        {
        $M_0=M_0\cup\{ (l(i),l(i+1)) \}$
        }
    }
$E=E\cup\{ (l(n),l(1)) \}$ \\
\If{ $s_0(n)=s_0(1)=1$ and $(s_0(1),s_0(2))\notin M_0$}
    {
    $M_0=M_0\cup\{ (l(n),l(1)) \}$
    }
\For{ $m\in M_0$ }
    {
    $E_m=E\setminus\{ m \}$ \\
    Run the Edmonds' Algorithm on $G_m=(V,E_m)$ \\
    \If{ The Edmonds' Algorithm returns a perfect matching $M_m$ of $G_m$ }
        {
        \For{ $(l(i),l(j))\in M_m$ }
            {
            $s(i)=s(j)=\1$
            }
        Return $s$
        }
    }
\caption{Another Gale Finder}
\label{anothergale-alg}
\end{algorithm}

\clearpage

We give two examples of the construction of theorem \ref{anothergale-thm}.

\begin{example}
The labeling $l=123432$ gives the graph $G$ in Figure \ref{matching-2-edges}.
Suppose that Edmonds' algorithm returns the matching $M_0=\{ e_1,e_3 \}$
of Figure \ref{matching-2-edges} left, corresponding to the completely labeled
Gale string $s_0=\1\1\1\1 00$; a second perfect matching $M=\{ e_3,e_6 \}$,
corresponding to the Gale string $s=\1 0\1\1 0\1$, can be found in the
graph $G_1=(V(G),E(G)\setminus\{e_1 \})$, see Figure \ref{matching-2-edges}
right.
\begin{figure}[hbt]
\strut\hfill
\includegraphics[width=35ex]{chapter-3/fig-main/matching-123432-1st.pdf}%
\hfill
\includegraphics[width=35ex]{chapter-3/fig-main/matching-123432-2nd.pdf}%
\hfill\strut
\caption[Two perfect matchings for a graph]{%
Left: The graph for the labeling $l=123432$ with the perfect matchings
corresponding to the completely labeled Gale strings $s_0=\1\1\1\1 00$.\\
Right: The graph $G_1=(V(G),E(G)\setminus\{e_1 \})$ and the perfect
matching corresponding to the Gale string $s=\1 0\1\1 0\1$.
}
\label{matching-2-edges}
\end{figure}
\end{example}

\clearpage

\begin{example}
Consider the Morris graph of Example \ref{gs-pm-ex}; suppose that
Edmonds' algorithm returns the perfect matching $M_0=\{ e_1,e_3,e_5 \}$,
as in Figure \ref{morris-matching} left, corresponding
to the completely labeled Gale string $s_0=\1\1\1\1\1\1 0000$.
We can then delete the edge $e_1$ to obtain the graph
$G_1$, as in Figure \ref{morris-matching} center. The graph $G_1$
has a perfect matching $M=\{ e_6,e_8,e_{10} \}$, shown in
Figure \ref{morris-matching} right; this is also a perfect
matching of $G$, corresponding to the only other completely labeled Gale
string $s=\1 0000\1\1\1\1\1$.
\begin{figure}[hbt]
\strut\hfill
\includegraphics[width=26ex]{chapter-3/fig-main/morris-6-1st-match.pdf}%
\hfill
\includegraphics[width=26ex]{chapter-3/fig-main/morris-6-inter.pdf}%
\hfill
\includegraphics[width=26ex]{chapter-3/fig-main/morris-6-2nd-match.pdf}%
\hfill\strut
\caption[The second matching of the Morris graph]{%
Left: The Morris graph $G=(V,E)$ with the matching $M=\{ e_1,e_3,e_5 \}$. \\
Centre: The graph $G_1=(V(G),E(G)\setminus\{e_1 \})$. \\
Right: The set $M'=\{ e_6,e_8,e_{10} \}$ is a perfect matching of bot $G_1$
and $G$.
}
\label{morris-matching}
\end{figure}
\end{example}

Applying Theorem \ref{galenash-to-another-gale} to
Theorem \ref{anothergale-thm} we have our main result: {\sc Gale Nash}
is in {\bf FP}.

\begin{theorem}
\label{main-thm}
Finding a Nash equilibrium of a Gale game takes polynomial time.
\end{theorem}
