\section{Proofs by Parity}

Megiddo and Papadimitriou \cite{megiddo-papad} introduced the classes
$\mathrm{\mathbf{FNP}}$, {\em function non-deterministic polynomial}, and
$\mathrm{\mathbf{TFNP}}$, {\em total function non-deterministic polynomial};
these can be seen as the equivalent of $\mathrm{\mathbf{NP}}$ for
(respectively) function and total function problems.
Formally, {\bf FNP} is defined as the class of binary relations
$R(x,y)$ such that there is a polynomial-time algorithm that decides $R(x,y)$
for $x,y$ such that $|y|\leq p(|x|)$, where $p$ is a polynomial, whereas
{\bf TFNP} is the class of all such problems for which $y$ is guaranteed to
exist. Megiddo and Papadimitriou \cite{megiddo-papad} also proved that
$\mathrm{\mathbf{TFNP}}$ is a {\em semantic} class, that is,
a class without complete problems, unless
$\mathrm{\mathbf{NP}}=\mathrm{\mathbf{co-NP}}$.
To circumvent this limitation, Papadimitriou \cite{ppad} focused
on the arguments that proves the solution of the problems in {\bf TFNP},
introducing the classes
$\mathrm{\mathbf{PPA}}$ ({\em Proof by Parity Argument}) and
$\mathrm{\mathbf{PPAD}}$ ({\em Proof by Parity Argument, Directed version}).

The existence of a solution for a problem in {\bf PPA} can be proved
using the argument ``in any undirected graph with one odd-degree node
there must be another odd-degree node.'' It is
interesting to notice that {\bf PPA}-complete problems are yet to be found.
Problems in {\bf PPAD}, analogously, are guaranteed to have a
solution by a proof employing the argument ``in any directed graph in
which all vertices have indegree and outdegree at most one where there is
a {\em source} (a node with indegree zero) there must be a {\em sink}
(a node with outdegree zero).''
Formally: a {\em circuit} with $n$ {\em input bits} and
$m$ {\em output bits} is a function $C:\{ 0,1 \}^n\to \{ 0,1 \}^m$;
we define {\bf PPAD} as the class of problems reducible to
the problem {\sc End Of The Line}, see Table \ref{eotl}.
This is the definition given in Daskalakis, Goldberg and Papadimitriou
\cite{dgp}; the original definition in Papadimitriou \cite{ppad} is given
in terms of Turing machines.

\begin{problem}
{End Of The Line}
{Two circuits $S$ and $P$ with $n$ input bits and $n$ output bits such that
$P(0^n)=0^n\neq S(0^n)$.}
{An input $x\in \{ 0,1 \}^n$ such that $P(S(x))\neq x$ or
$S(P(x))\neq x\neq 0^n$}
\label{eotl}
\end{problem}

The problems in {\bf PPAD} can be seen as a circuit $S$ (``successor''),
and a circuit $P$ (``predecessor'') that are used to build a directed
graph with an edge $(x,y)$ if and only if $S(x)=y$ and $P(y)=x$; furthermore,
a {\em standard source} $0^n$ is given, guaranteeing the existence of the
output, which consists in either a sink or a non-standard source.
Figure \ref{ppad-graph} presents an example of a graph implicit in a
{\bf PPAD} problem; a graph for a {\bf PPA} problem is analogous, but
it is undirected and instead of sources and sinks there are
generic endpoints. Another class relying on proofs by parity argument is
{\bf PPADS}, defined by Daskalakis, Goldberg and Papadimitriou \cite{dgp};
its definition is analogous to {\bf PPAD}, but the output of the problem is
required to be a sink of the {\sc End Of The Line} graph. We have that
$\mathbf{PPADS}\subseteq\mathbf{PPAD}\subseteq\mathbf{PPA}$; it is
an open problem whether the inclusion is strict.

\clearpage

\begin{figure}[hbtp]
\strut\hfill
\includegraphics[width=80ex]{chapter-3/fig-ppad/PPAD.pdf}%
\hfill\strut
\caption[A PPAD problem]{%
A {\bf PPAD} problem as a directed graph with maximal indegree and
outdegree 1; it consists of paths (in black), cycles (in blue) and
isolated points (in purple).
The input is given by the circuits $S$ (in green) and $P$ (in red) and
the standard source (the black node). The output can be either a sink
(a red node) or a nonstandard source (a green node).
}
\label{ppad-graph}
\end{figure}


As we have already noticed, the problem $n$-{\sc Nash}, see
Table \ref{n-nash}, is a total function problem.
Megiddo and Papadimitriou \cite{megiddo-papad} have proven that it
is in {\bf TFNP}; Daskalakis, Goldberg and Papadimitriou \cite{dgp} and Chen
and Deng \cite{cd} have later proven its {\bf PPAD}-completeness, the
former for $n\geq 3$ and the latter for $n\geq 2$.
A small amendment of the proof in \cite{dgp} can be found in Casetti
\cite{msc-diss}.

\begin{theorem}{\rm (Daskalakis, Goldberg and Papadimitriou \cite{dgp};
Chen and
\linebreak[5]
Deng \cite{cd})}\label{nash-ppad-complete-thm}
For $n\geq 2$, the problem {\sc $n$-Nash} is {\bf PPAD}-complete.
\end{theorem}
