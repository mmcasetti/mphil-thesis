In the previous chapter we have defined some problems of the form
``find
another completely labeled'' vertex, facet or
Gale string; in this chapter we finally study the complexity
of these problems.
A solution of {\sc Another Completely Labeled Vertex}
is given by the classic algorithm first
\linebreak
introduced by
Lemke and Howson \cite {lh}. In turn, the Lemke-Howson algorithm prompted
the definition of the classes {\bf PPAD} and {\bf PPA} by
Papadimitriou~\cite{ppad}; the definition of these classes is given
in the first section of this chapter.

The Lemke-Howson algorithm can be used for
to prove that the ``another completely labeled'' problems are in the
complexity class {\bf PPAD}. It is
\linebreak
interesting to notice that all this
proof can be seen as ultimately relying on Shapley's \cite{shapley}
work discussed in the previous chapter, see Savani and von
Stengel \cite{svs}, Merschen \cite{jm}, and
V\'{e}gh and von Stengel \cite{vvs}.
In the second section we relate the different versions of the
Lemke-Howson algorithm and the consequent proof that
the ``another completely labeled'' problems are in the class {\bf PPA}.
We first use it to solve {\sc Another Completely Labeled Vertex},
and therefore 2-{\sc Nash}, its original motivation. The version for
{\sc Another Completely Labeled Vertex} follows immediately by a
duality argument. Finally, we focus
on \anothergale\ and give the full
proof that it belongs to the {\bf PPAD} complexity class, following
he clear exposition in Merschen \cite{jm}.
We close the section
with an example of a labeling, due to Morris \cite{morris}, for which
the Lemke-Howson-Gale Algorithm has exponential running time. This
is the labeling that Savani and von Stengel \cite{svs} have used
to construct ``hard to solve'' games.

The third and last section presents our original result: a polynomial
time algorithm for \anothergale, that is, a proof that {\sc Gale Nash} is in
{\bf FP}. Unless {\bf PPAD = P}, this goes
in the opposite direction of our first conjecture of {\bf PPAD}-completeness
suggested by the ``hard to solve'' games by Savani and von
Stengel \cite{svs}. Our proof relies on
a theorem by Edmonds \cite{edm} that gives a polynomial-time algorithm
to find a perfect matching of a graph or decide that it is not possible
to find one. The key of the proof is the construction of a graph from any
string of labels such that the perfect matchings of the graph correspond to
the completely labeled Gale strings for the labels. We first prove
the {\bf FP} complexity of finding one of these completely labeled Gale
strings, then we extend the proof to find the second string
required by \anothergale.
