In the previous chapter we have defined some problems of the form
``find
\linebreak[3]
another completely labeled\ldots'' for vertices, facets and
Gale strings. In this chapter we will finally study the complexity
of these problems.
In the first section we will give the definition of the classes {\bf PPAD}
and {\bf PPA}, first introduced by Papadimitriou \cite{ppad}. In the second
section we will present different versions of the standard algorithm
introduced by Lemke and Howson \cite{lh}; we will apply it to
{\sc Another Completely Labeled Vertex} and
to find a solution for 2-{\sc Nash}, its original motivation.
We will then study the Lemke-Howson for Gale Algorithm,
which returns a solution to \anothergale.
We will use these algorithms to prove the {\bf PPA} complexity of
{\sc Another Completely Labeled Vertex}
and {\sc Another Completely Labeled Facet}; in the case of \anothergale\ we
will extend the proof to {\bf PPAD} complexity. We will close the section
with an example of labeling, due to Morris \cite{morris}, for which
the Lemke-Howson for Gale Algorithm presents exponential running time; this
is the labeling that Savani and von Stengel \cite{svs} \cite{uvg} have used
to build ``hard to solve'' games.

The third and last section will presents our original result: a polynomial
time algorithm for \anothergale, a proof that {\sc Gale Nash} is in
{\bf FP}. Unless {\bf PPAD = P}, this goes
in the opposite direction of our first conjecture of {\bf PPAD}-completeness
suggested by the ``hard to solve'' games by Savani and von
Stengel \cite{svs}. Our proof relies on
a theorem by Edmonds \cite{edm} that gives a polyomial-time algorithm
to find a perfect matching of a graph or deciding that it is not possible
to find one. The key of the proof is the construction of a graph from any
string of labels; the perfect matchings of the graph will correspond to
the completely labeled Gale strings for the labels. We will first prove
the {\bf FP} complexity of finding one of these completely labeled Gale
strings; we will then extend the proof to find the second string
required by \anothergale.
