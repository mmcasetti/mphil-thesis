In the previous chapter we have defined some problems of the form
``find another completely labeled\ldots'' for vertices, facets and
Gale strings. In this chapter we will finally study the complexity
of these problems.

The first section will present different versions of a standard algorithm,
first introduced by Lemke and Howson in \cite{lh}, that relies on the
pivoting routine. We will see the Lemke-Howson algorithm for
the problem {\sc Another Completely Labeled Vertex}, in particular as
a tool to find a solution for 2-{\sc Nash}, its original application;
we will then touch on the dual version for
{\sc Another Completely Labeled Facet}; finally, we will describe the
Lemke-Howson for Gale algorithm, that returns a solution to \anothergale.
Each one of these algorithms leads to a very straightforward proof of
{\bf PPA} complexity for its respective problem; in the case of \anothergale,
we will also prove its {\bf PPAD} complexity. We will close the section
with an example of labeling, due to Morris \cite{morris}, for which
the Lemke-Howson for Gale algorithm presents exponential running time.
Savani and von Stengel \cite{svs} \cite{uvg} have used this labeling to
build Gale games that are ``hard to solve''; this, in turn, had given the
main motivation for our study, since it had led to conjecturing that
Gale games could be used to give a proof of {\bf PPAD}-completeness
of 2-{\sc Nash}.

The second section presents our original result: a polynomial time
algorithm for \anothergale, which proves that {\sc Gale 2-Nash} is in
{\bf FP}. Since it seems unlikely that {\bf PPAD}={\bf P}, this goes
in the opposite direction of our previous conjecture. The proof relies on
a theorem by Edmonds \cite{edm} that gives a polyomial-time algorithm
to find a perfect matching of a graph, or deciding that it is not possible
to find one. The key of the proof is the construction of a graph from any
string of labels; the perfect matchings of the graph will correspond to
the completely labeled Gale strings for the labels. We will first prove
the {\bf FP} complexity of finding one of these completely labeled Gale
strings; we will then extend the proof to find the second string, as
required by \anothergale.
