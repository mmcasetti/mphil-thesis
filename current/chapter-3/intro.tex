In the previous chapter we have defined some problems of the form
``find
\linebreak[4]
another completely labeled'' vertex, facet or
Gale string; in this chapter we finally study the complexity
of these problems.
In the first section we give the definition of the classes {\bf PPAD}
and {\bf PPA}, first introduced by
\linebreak[5]
Papadimitriou~\cite{ppad}.
In the second section we prove that the ``another
\linebreak[4]
completely labeled''
problems are in the complexity class {\bf PPA}; in the case of
\anothergale\ we extend the proof to the {\bf PPAD} complexity class.
We do so by applying
different versions of the classic algorithm first
introduced by Lemke and
Howson \cite {lh}. Its original motivation was to
find a solution of {\sc Another Completely Labeled Vertex}, and therefore
of 2-{\sc Nash}; we will give different examples of this application.
We will then move on to the study of the Lemke-Howson-Gale Algorithm, that
uses an analogous technique to return a solution of \anothergale.
We close the section
with an example of a labeling, due to Morris \cite{morris}, for which
the Lemke-Howson-Gale Algorithm has exponential running time. This
is the labeling that Savani and von Stengel \cite{svs} have used
to construct ``hard to solve'' games.

The third and last section presents our original result: a polynomial
time algorithm for \anothergale, that is, a proof that {\sc Gale Nash} is in
{\bf FP}. Unless {\bf PPAD = P}, this goes
in the opposite direction of our first conjecture of {\bf PPAD}-completeness
suggested by the ``hard to solve'' games by Savani and von
Stengel \cite{svs}. Our proof relies on
a theorem by Edmonds \cite{edm} that gives a polynomial-time algorithm
to find a perfect matching of a graph or decide that it is not possible
to find one. The key of the proof is the construction of a graph from any
string of labels such that the perfect matchings of the graph correspond to
the completely labeled Gale strings for the labels. We first prove
the {\bf FP} complexity of finding one of these completely labeled Gale
strings, then we extend the proof to find the second string
required by \anothergale.
