In the previous chapter we have defined some problems of the form
``find another completely labeled\ldots'' for vertices, facets and
Gale strings. In this chapter we will finally study the complexity
of these problems.
The first section will present different versions of the standard algorithm
introduced by Lemke and Howson \cite{lh}. We will first apply it to
the problem {\sc Another Completely Labeled Vertex}, in particular as
a tool to find a solution for 2-{\sc Nash}, its original motivation;
we will then touch on its dual version for the problem
{\sc Another Completely Labeled Facet}; finally, we will describe the
Lemke-Howson for Gale Algorithm, which returns a solution to \anothergale.
We will follow each one of these algorithms with a very straightforward
proof of the {\bf PPA} complexity of {\sc Another Completely Labeled Vertex}
and {\sc Another Completely Labeled Facet}; in the case of \anothergale
we will give a result of {\bf PPAD} complexity. We will close the section
with an example of labeling, due to Morris \cite{morris}, for which
the Lemke-Howson for Gale algorithm presents exponential running time; this
is the labeling that Savani and von Stengel \cite{svs} \cite{uvg} have used
to build Gale games that are ``hard to solve''.

The second section presents our original result: a polynomial time
algorithm for \anothergale, therefore proving that {\sc Gale Nash} is in
{\bf FP}. Unless {\bf PPAD}={\bf P}, this goes
in the opposite direction of our first conjecture of {\sc PPAD}-completeness
suggested by the ``hard to solve'' games. Our proof relies on
a theorem by Edmonds \cite{edm} that gives a polyomial-time algorithm
to find a perfect matching of a graph, or deciding that it is not possible
to find one. The key of the proof is the construction of a graph from any
string of labels; the perfect matchings of the graph will correspond to
the completely labeled Gale strings for the labels. We will first prove
the {\bf FP} complexity of finding one of these completely labeled Gale
strings; we will then extend the proof to find the second string
required by \anothergale.
