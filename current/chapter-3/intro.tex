In the previous chapter we have defined some problems of the form
``find
\linebreak[3]
another completely labeled'' vertex, facet or
Gale string. In this chapter we finally study the complexity
of these problems.
In the first section we give the definition of the classes {\bf PPAD}
and {\bf PPA}, first introduced by Papadimitriou \cite{ppad}. In the second
section we present different versions of the standard algorithm
introduced by Lemke and Howson \cite{lh}. We apply it to
{\sc Another Completely Labeled Vertex} and
to find a solution for 2-{\sc Nash}, its original motivation.
We then study the Lemke Path for Gale Algorithm,
which returns a solution to \anothergale.
We use these algorithms to prove the {\bf PPA} complexity of
{\sc Another Completely Labeled Vertex}
and {\sc Another Completely Labeled Facet}. In the case of \anothergale\ we
extend the proof to {\bf PPAD} complexity. We close the section
with an example of a labeling, due to Morris \cite{morris}, for which
the Lemke Path for Gale Algorithm has exponential running time. This
is the labeling that Savani and von Stengel \cite{svs} \cite{uvg} have used
to construct ``hard to solve'' games.

The third and last section presents our original result: a polynomial
time algorithm for \anothergale, a proof that {\sc Gale Nash} is in
{\bf FP}. Unless {\bf PPAD = P}, this goes
in the opposite direction of our first conjecture of {\bf PPAD}-completeness
suggested by the ``hard to solve'' games by Savani and von
Stengel \cite{svs}. Our proof relies on
a theorem by Edmonds \cite{edm} that gives a polynomial-time algorithm
to find a perfect matching of a graph or deciding that it is not possible
to find one. The key of the proof is the construction of a graph from any
string of labels. The perfect matchings of the graph correspond to
the completely labeled Gale strings for the labels. We first prove
the {\bf FP} complexity of finding one of these completely labeled Gale
strings. We then extend the proof to find the second string
required by \anothergale.
