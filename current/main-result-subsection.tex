\documentclass[11pt, draft]{article}

%% TYPESETTING

% draft:
\linespread{1.3}
\usepackage{todonotes}

% final (?) TODO: check with LSE requirements
% page measurements
% \setlength{\hoffset}{0mm}
% \setlength{\oddsidemargin}{25mm}
% \setlength{\textwidth}{130mm}

%% PACKAGES

\usepackage{amssymb}
\usepackage{amsmath}
\usepackage{amsthm}

\usepackage[all]{xy}

\usepackage[ruled,vlined,linesnumbered]{algorithm2e}

%% THEOREMS and ENVIRONMENTS

% theorems

\newtheorem{theorem}{Theorem}
\newtheorem{property}{Property}[section]
\theoremstyle{definition}\newtheorem{definition}{Definition}
\theoremstyle{remark}\newtheorem{example}{Example}[section]

% environments

% computational problem
% TODO check p{textwidth} in final version

\newenvironment{problem}[3]
{
\noindent
\begin{tabular}{p{13mm} @{\textbf{:} } p{105mm}}
\hline
\multicolumn{2}{l}{\noindent \textsc{#1}} \\
\hline
\textbf{input} & #2 \\
\textbf{output} & #3 \\
\hline
\end{tabular}
}
{\\}

% use:
% \begin{problem}
% {name of problem}
% {input of problem}
% {output of problem}
% \end{problem}


%% SHORTCUTS

% definitions

\def\reals{{\mathbb R}}
\def\naturals{{\mathbb N}}
\def\conv{{\rm conv}}
\def\0{{\bf0}}
\def\1{{\bf1}}
\def\T{^{\top}}
\def\rone{{\1\T}}

% to choose the name of the problem - GALE or COMPLETELY LABELED GALE STRING

\def\gale{{\sc{Gale}}}
\def\anothergale{{\sc{Another Gale}}}


%%% END PREAMBLE

\begin{document}

\subsection{The Complexity of \gale\ and \anothergale}
\label{main-thm-subsection}

We will now give our main result: \anothergale\ can be solved in polynomial
time. Therefore, it takes polynomial time to find a Nash Equilibrium of a
bimatrix game for which the best response polytope is cyclic.

Our proof will be based on a simple graph construction, and it will exploit
the following result by Edmonds \ref{edm-flowers} on the problem
\textsc{Perfect Matching}, defined as follows.\\

\begin{problem}
{\textsc{Perfect Matching}}
{A graph $G = (V,E)$.}
{Whether there is a set $M\subseteq E$ of pairwise non-adjacent edges so that
every vertex $v \in V$ is incident to exactly one edge in~$M$.}
\end{problem}

\begin{theorem}[\ref{edm-flowers}]\label{pm-poly}
The problem \textsc{Perfect Matching} is solvable in polynomial time.
\end{theorem}

We begin by considering the accessory problem \gale, and proving that it is
solvable in polynomial time.\\

\begin{problem}
{\gale}
{A labeling $l:[n]\to[d]$, where $d$ is even and $d<n$.}
{Whether there is a completely labeled Gale string~$s$ in~$G(d,n)$
associated with $l$.}
\end{problem}

\begin{theorem}\label{gale-p-thm}
The problem \gale\ is solvable in polynomial time.
\end{theorem}
\begin{proof}

\todo[inline]{proof from old draft: edit!}

We give a rather simple reduction to \textsc{Perfect Matching}.
Given the labeling $l:[n]\to[d]$, construct the
(multi-)graph $G$ with vertex set $V=[d]$ and up to $n$
(possibly parallel) edges with endpoints $l(i),l(i+1)$ for
$i\in [n]$ whenever these endpoints are distinct (so $G$ has
no loops); here we let $n+1=1$ (``modulo~$n$'') so that
$n,n+1$ is to be understood as $n,1$.
Then a completely labeled Gale string $s$ in $G(d,n)$ splits
into a number of runs which are uniquely split into $d/2$
pairs $i,i+1$ so that the labels $l(i)$ and $l(i+1)$ are
distinct, and all labels $1,\ldots,n$ occur among them.
So this defines a perfect matching for~$G$.

Conversely, a perfect matching $M$ of $G$ defines a Gale
string $s$ where $s(i)=s({i+1})=1$ if the edge that joins
$l(i)$ and $l(i+1)$ is in $M$ and $s(i)=0$ otherwise, so $s$
is completely labeled.
This shows how \textsc{Completely labeled Gale string}
reduces to \textsc{Perfect Matching}.
Finding a perfect matching, or deciding that $G$ has none,
can be done in polynomial time~\cite{edm-flowers}.
\end{proof}

Two examples of the construction used in theorem \ref{gale-p-thm} follows.
In the first one, there is a perfect matching and a corresponding Gale
string.

\begin{example}
Let $l=12343122$ be a string of labels. Then we have the edges $e_i$ as
follows:

\[
\underbrace{
1\overbrace{\phantom{\scriptscriptstyle 1}}^{e_1}
2\overbrace{\phantom{\scriptscriptstyle 1}}^{e_2}
3\overbrace{\phantom{\scriptscriptstyle 1}}^{e_3}
4\overbrace{\phantom{\scriptscriptstyle 1}}^{e_4}
3\overbrace{\phantom{\scriptscriptstyle 1}}^{e_5}
1\overbrace{\phantom{\scriptscriptstyle 1}}^{e_6}
2\overbrace{\phantom{\scriptscriptstyle 1}}^{\scriptscriptstyle cycle}
2
}_{e_7}
\]

So the graph $G$ will be:

\begin{displaymath}
\xymatrix
@M=5pt
{
1
\ar@/^1pc/@{-}[rr] |{e_1}
\ar@{-}[rr] |{e_6}
\ar@/_1pc/@{-}[rr] |{e_7}
\ar@{-}[ddrr] |{e_5}
& & 2
\ar@{-}[dd] |{e_2}
\\
\\
3
\ar@{-}[rr] |{e_3}
\ar@/_1pc/@{-}[rr] |{e_4}
& & 4
}
\end{displaymath}

A possible completely labeled Gale string for $l$ is $11011000$, which
corresponds to $e_1, e_4$ in $G$:

\[
\underbrace{
\mathbf{1}\overbrace{\phantom{\scriptscriptstyle 1}}^{e_1}
\mathbf{1}\overbrace{\phantom{\scriptscriptstyle 1}}
0\overbrace{\phantom{\scriptscriptstyle 1}}
\mathbf{1}\overbrace{\phantom{\scriptscriptstyle 1}}^{e_4}
\mathbf{1}\overbrace{\phantom{\scriptscriptstyle 1}}
0\overbrace{\phantom{\scriptscriptstyle 1}}
0\overbrace{\phantom{\scriptscriptstyle 1}}
0
}
\]


\end{example}

An example where there isn't a perfect matching, and therefore there isn't
any possible Gale string for the labeling is the following.

\begin{example}
Let us consider the labeling $l=121314$. The associated graph $G$ will be

\begin{displaymath}
\xymatrix
@M=5pt
{
1
\ar@/^1pc/@{-}[rr] |{e_1}
\ar@{-}[rr] |{e_2}
\ar@/^1pc/@{-}[ddrr] |{e_3}
\ar@{-}[ddrr] |{e_4}
\ar@{-}[dd] |{e_5}
\ar@/_1pc/@{-}[dd] |{e_6}
& & 2
\\
\\
3
& & 4
}
\end{displaymath}

It's trivial to see that it's not possible to find a perfect matching for
$G$.
\end{example}

We finally extend the proof of theorem \ref{gale-p-thm} to show that
\anothergale\ is polynomial-time solvable.

\begin{theorem}\label{gale-p-thm}
The problem \anothergale\ is solvable in polynomial time.
\end{theorem}
\begin{proof}

\todo[inline]{proof from old draft: to edit!}

The reduction for \anothergale\
is an extension of this.
Consider the given completely labeled Gale string $s$ and
the matching $M$ for it.
If $G$ has multiple edges between two nodes and one of them
is in $M$, simply replace that edge by a parallel edge to
obtain another completely labeled Gale string~$s'$.
Hence, we can assume that $M$ has no edges that have a
parallel edge.
Another completely labeled Gale string $s'$ exists by
Theorem~\ref{t-even}.
The corresponding matching $M'$ does not use at least one
edge in $M$.
Hence, at least one of the $d/2$ graphs $G$ which have one
of the edges of $M$ removed has a perfect matching $M'$,
which is a perfect matching of $G$, and which defines
a completely labeled Gale string $s'$ different from~$s$.
The search for $M'$ takes again polynomial time.
\end{proof}

\todo[inline]{examples for second PM (one w/ double edges, one without)}

\newpage

\begin{thebibliography}{00}

\frenchspacing\parskip-1ex
\small

\bibitem{main} M. M. Casetti, J. Merschen, B. von Stengel (2010).
Finding Gale Strings.
\emph{Electronic Notes in Discrete Mathematics}
\todo[inline]{issue, pp. n--m.}

\bibitem{cd} X. Chen, X. Deng (2006).
Settling the complexity of two-player Nash equilibrium.
\emph{Proc. 47th FOCS}, pp. 261--272.

\bibitem{dgp} C. Daskalakis, P. W. Goldberg, C. H. Papadimitriou (2006).
The complexity of computing a Nash equilibrium.
\emph{Proc. Ann. 38th STOC}, pp. 71--78.
\todo[inline]{change ref to econometrica(?)}

\bibitem{edm} J. Edmonds (1965).
Paths, trees, and flowers.
\emph{Canad. J. Math.} 17, pp. 449--467.

\bibitem{gale} D. Gale (1963),
Neighborly and cyclic polytopes.
\emph{Convexity, Proc. Symposia in Pure Math.}, Vol. 7, ed. V. Klee, American Math. Soc., Providence, Rhode Island, pp. 225--232.
\todo[inline]{check if right typography}

\bibitem{jm} J. Merschen (2012).
\todo[inline]{thesis}

\bibitem{lh} C. E. Lemke, J. T. Howson, Jr. (1964).
Equilibrium points of bimatrix games.
\emph{J.  Soc. Indust. Appl. Mathematics} 12, pp.  413--423.

\bibitem{ppad} C. H. Papadimitriou (1994).
On the complexity of the parity argument and other inefficient proofs of existence.
\emph{J. Comput. System Sci.} 48, pp. 498--532.

\bibitem{svs} R. Savani, B. von Stengel (2006).
Hard-to-solve bimatrix games.
\emph{Econometrica} 74, pp. 397--429.

\bibitem{vvs} L. V\'{egh}, B. von Stengel
\todo[inline]{ref}


\end{thebibliography}

\end{document}
