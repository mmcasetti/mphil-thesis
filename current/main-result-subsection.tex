\documentclass[11pt, draft]{article}

%% TYPESETTING

% draft:
\linespread{1.3}
\usepackage{todonotes}

% final (?) TODO: check with LSE requirements
% page measurements
% \setlength{\hoffset}{0mm}
% \setlength{\oddsidemargin}{25mm}
% \setlength{\textwidth}{130mm}

%% PACKAGES

\usepackage{amssymb}
\usepackage{amsmath}
\usepackage{amsthm}

\usepackage[all]{xy}

\usepackage[ruled,vlined,linesnumbered]{algorithm2e}

%% THEOREMS and ENVIRONMENTS

% theorems

\newtheorem{theorem}{Theorem}
\newtheorem{property}{Property}[section]
\theoremstyle{definition}\newtheorem{definition}{Definition}
\theoremstyle{remark}\newtheorem{example}{Example}[section]

% environments

% computational problem
% TODO check p{textwidth} in final version

\newenvironment{problem}[3]
{
\vspace{2.5ex}
\noindent
\begin{tabular}{p{13mm} @{\textbf{:} } p{105mm}}
\hline
\multicolumn{2}{l}{\noindent {\sc #1}} \\
\hline
\textbf{input} & #2 \\
\textbf{output} & #3 \\
\hline
\end{tabular}
}
{
\vspace{1.5ex}
}

% use:
% \begin{problem}
% {name of problem}
% {input of problem}
% {output of problem}
% \end{problem}


%% SHORTCUTS

% definitions

\def\reals{{\mathbb R}}
\def\naturals{{\mathbb N}}
\def\conv{{\rm conv}}
\def\0{{\bf0}}
\def\1{{\bf1}}
\def\T{^{\top}}
\def\rone{{\1\T}}

% to choose the name of the problem - GALE or COMPLETELY LABELED GALE STRING

\def\gale{{\sc{Gale}}}
\def\anothergale{{\sc{Another Gale}}}


%%% END PREAMBLE

\begin{document}

\begin{verbatim}

\subsection{Cyclic Polytopes and Gale Strings}\label{gs-ssect}

\begin{example}\label{no-clgs}
For $l = 121314$, there are no completely labeled Gale strings.
\end{example}

\end{verbatim}

\subsection{The Complexity of \gale\ and \anothergale}
\label{main-result-subsection}

We will now give our main result: \anothergale\ can be solved in polynomial
time. Therefore, it takes polynomial time to find a Nash Equilibrium of a
bimatrix game for which the best response polytope is cyclic.

Our proof will be based on a simple graph construction.

\begin{definition}

A {\em perfect matching} for a graph $G=(V,E)$ is a set $M\subseteq E$ of
pairwise non-adjacent edges so that every vertex $v \in V$ is incident to
exactly one edge in~$M$.

We define the problem {\sc Perfect Matching} as follows:

\begin{problem}
{Perfect Matching}
{A graph $G = (V,E)$.}
{A perfect matching for $G$.}
\end{problem}

\end{definition}

The complexity of {\sc Perfect Matching} has been proven to be in P by
Edmonds \cite{edm}.

\begin{theorem}[\cite{edm}]\label{pm-thm}
The problem {\sc Perfect Matching} is solvable in polynomial time.
\end{theorem}

We will first consider the accessory problem \gale, and we will show that
it is solvable in polynomial time by using theorem \ref{pm-thm}.\\

\begin{problem}
{\gale}
{A labeling $l:[n]\to[d]$, where $d$ is even and $d<n$.}
{A completely labeled Gale string~$s$ in~$G(d,n)$ associated with $l$.}
\end{problem}

\begin{theorem}\label{gale-thm}
The problem \gale\ is solvable in polynomial time.
\begin{proof}

We give a reduction of \gale\ to {\sc Perfect Matching}.

In the following, we will consider every Gale string as a ``loop,'' as seen
in section \ref{gs-ssect}, so $n+1=1$.

Given the labeling $l:[n]\to[d]$, let $V=[d]$, let $E=\{(l(i),l(i+1))
\mbox{ for }i\in[n]\mbox{ for every }i\mbox{ such that }l(j)\neq l(i+1)\}$,
and consider the multigraph $G=(V,E)$.

Let $s\in G(d,n)$ be a completely labeled Gale string. Then every run of $s$
splits uniquely into $d/2$ pairs $(i,i+1)$ such that the labels $l(i)$
satisfy the condition $l(i)\neq l(i+1)$, and all the labels
$l(i)\in [d]$ occur. Then the labels will correspond to all the vertices of
$G$, and the pairs will correspond to the edges of a perfect matching for
$G$.

Conversely, let $l:[n]\to [d]$ be a labeling, and let $M$ be a perfect
matching for $G$ as above. We can construct a string $s$ such that
$s(i)=s(i+1)$ for every $(l(i),l(i+1))\in M$ and $s(i)=0$
otherwise. Since $M$ is a matching, all the $(l(i),l(i+1))\in M$ are
disjoint, so, considering $s$ as a ``loop,'' every run is of even
length. Furthermore, since $M$ is a perfect matching, every vertex
$v\in [d]$ is the endpoint of an edge $(l(i),l(i+1))$, so $s$ is
completely labeled.

We have a reduction from \gale\ to the problem {\sc Perfect Matching},
which is polynomial-time solvable by theorem \ref{pm-thm}.
Finding a Gale string for a given labeling, or deciding that there isn't
one, can therefore be done in polynomial time.
\end{proof}
\end{theorem}

We give two examples of the construction used in theorem \ref{gale-thm}.

\begin{example}
Let $l=12343122$ be a string of labels. Then the egdes $e_i$ of the graph
$G$ obtained from the construction in the proof of theorem  \ref{gale-thm}
will be as follow:

\[
\underbrace{
1\overbrace{\phantom{\scriptscriptstyle 1}}^{e_1}
2\overbrace{\phantom{\scriptscriptstyle 1}}^{e_2}
3\overbrace{\phantom{\scriptscriptstyle 1}}^{e_3}
4\overbrace{\phantom{\scriptscriptstyle 1}}^{e_4}
3\overbrace{\phantom{\scriptscriptstyle 1}}^{e_5}
1\overbrace{\phantom{\scriptscriptstyle 1}}^{e_6}
2\overbrace{\phantom{\scriptscriptstyle 1}}^{\scriptscriptstyle cycle}
2
}_{e_7}
\]

Given the vertices $v\in [4]$, the graph $G$ will be:

\begin{displaymath}
\xymatrix
@M=5pt
{
1
\ar@/^1pc/@{-}[rr] |{e_1}
\ar@{-}[rr] |{e_6}
\ar@/_1pc/@{-}[rr] |{e_7}
\ar@{-}[ddrr] |{e_5}
& & 2
\ar@{-}[dd] |{e_2}
\\
\\
3
\ar@{-}[rr] |{e_3}
\ar@/_1pc/@{-}[rr] |{e_4}
& & 4
}
\end{displaymath}

The perfect matching for $G$ given by $M=\{e_1, e_4\}$ will then correspond
to the completely labeled Gale string $11011000$.

\[
\underbrace{
\mathbf{1}\overbrace{\phantom{\scriptscriptstyle 1}}^{e_1}
\mathbf{1}\overbrace{\phantom{\scriptscriptstyle 1}}
0\overbrace{\phantom{\scriptscriptstyle 1}}
\mathbf{1}\overbrace{\phantom{\scriptscriptstyle 1}}^{e_4}
\mathbf{1}\overbrace{\phantom{\scriptscriptstyle 1}}
0\overbrace{\phantom{\scriptscriptstyle 1}}
0\overbrace{\phantom{\scriptscriptstyle 1}}
0
}
\]
\end{example}

A perfect matching for a graph, and therefore a Gale string for a labeling,
is not always possible, as shown in the next example.

\begin{example}
Let us consider the labeling $l=121314$. The associated graph $G$ will be

\begin{displaymath}
\xymatrix
@M=5pt
{
1
\ar@/^1pc/@{-}[rr] |{e_1}
\ar@{-}[rr] |{e_2}
\ar@/^1pc/@{-}[ddrr] |{e_3}
\ar@{-}[ddrr] |{e_4}
\ar@{-}[dd] |{e_5}
\ar@/_1pc/@{-}[dd] |{e_6}
& & 2
\\
\\
3
& & 4
}
\end{displaymath}

Since there aren't any disjoint edges, it's not possible to find a perfect
matching for $G$. Analogously, we have seen in example \ref{no-clgs} that
there isn't any possible completely labeled Gale string for the labeling
$l$.
\end{example}

We finally extend the proof of theorem \ref{gale-thm} to show that
\anothergale\ is polynomial-time solvable.

\begin{theorem}\label{gale-p-thm}
The problem \anothergale\ is solvable in polynomial time.
\end{theorem}
\begin{proof}

\todo[inline]{proof from old draft: to edit!}

The reduction for \anothergale\
is an extension of this.
Consider the given completely labeled Gale string $s$ and
the matching $M$ for it.
If $G$ has multiple edges between two nodes and one of them
is in $M$, simply replace that edge by a parallel edge to
obtain another completely labeled Gale string~$s'$.
Hence, we can assume that $M$ has no edges that have a
parallel edge.
Another completely labeled Gale string $s'$ exists by
Theorem~\ref{t-even}.
The corresponding matching $M'$ does not use at least one
edge in $M$.
Hence, at least one of the $d/2$ graphs $G$ which have one
of the edges of $M$ removed has a perfect matching $M'$,
which is a perfect matching of $G$, and which defines
a completely labeled Gale string $s'$ different from~$s$.
The search for $M'$ takes again polynomial time.
\end{proof}

\todo[inline]{examples for second PM (one w/ double edges, one without)}

\newpage

\begin{thebibliography}{00}

\frenchspacing\parskip-1ex
\small

\bibitem{main} M. M. Casetti, J. Merschen, B. von Stengel (2010).
Finding Gale Strings.
\emph{Electronic Notes in Discrete Mathematics}
\todo[inline]{issue, pp. n--m.}

\bibitem{cd} X. Chen, X. Deng (2006).
Settling the complexity of two-player Nash equilibrium.
\emph{Proc. 47th FOCS}, pp. 261--272.

\bibitem{dgp} C. Daskalakis, P. W. Goldberg, C. H. Papadimitriou (2006).
The complexity of computing a Nash equilibrium.
\emph{Proc. Ann. 38th STOC}, pp. 71--78.
\todo[inline]{change ref to econometrica(?)}

\bibitem{edm} J. Edmonds (1965).
Paths, trees, and flowers.
\emph{Canad. J. Math.} 17, pp. 449--467.

\bibitem{gale} D. Gale (1963),
Neighborly and cyclic polytopes.
\emph{Convexity, Proc. Symposia in Pure Math.}, Vol. 7, ed. V. Klee, American Math. Soc., Providence, Rhode Island, pp. 225--232.
\todo[inline]{check if right typography}

\bibitem{jm} J. Merschen (2012).
\todo[inline]{thesis}

\bibitem{lh} C. E. Lemke, J. T. Howson, Jr. (1964).
Equilibrium points of bimatrix games.
\emph{J.  Soc. Indust. Appl. Mathematics} 12, pp.  413--423.

\bibitem{ppad} C. H. Papadimitriou (1994).
On the complexity of the parity argument and other inefficient proofs of existence.
\emph{J. Comput. System Sci.} 48, pp. 498--532.

\bibitem{svs} R. Savani, B. von Stengel (2006).
Hard-to-solve bimatrix games.
\emph{Econometrica} 74, pp. 397--429.

\bibitem{vvs} L. V\'{egh}, B. von Stengel
\todo[inline]{ref}


\end{thebibliography}

\end{document}
