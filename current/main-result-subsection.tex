\documentclass[11pt, draft]{article}

%% TYPESETTING

% draft:
\linespread{1.3}
\usepackage{todonotes}

% final (?) TODO: check with LSE requirements
% page measurements
% \setlength{\hoffset}{0mm}
% \setlength{\oddsidemargin}{25mm}
% \setlength{\textwidth}{130mm}

%% PACKAGES

\usepackage{amssymb}
\usepackage{amsmath}
\usepackage{amsthm}

\usepackage[all]{xy}

\usepackage[ruled,vlined,linesnumbered]{algorithm2e}

%% THEOREMS and ENVIRONMENTS

% theorems

\newtheorem{theorem}{Theorem}
\newtheorem{property}{Property}[section]
\theoremstyle{definition}\newtheorem{definition}{Definition}
\theoremstyle{remark}\newtheorem{example}{Example}[section]

% environments

% decision problem
% TODO check p{textwidth} in final version

\newenvironment{decproblem}[3]
{
\vspace{2.5ex}
\noindent
\begin{tabular}{p{18mm} @{\textbf{:} } p{100mm}}
\hline
\multicolumn{2}{l}{\noindent {\sc #1}} \\
\hline
\textbf{input} & #2 \\
\textbf{question} & #3 \\
\hline
\end{tabular}
}
{
\vspace{1.5ex}
}

% function problem

\newenvironment{fctproblem}[3]
{
\vspace{2.5ex}
\noindent
\begin{tabular}{p{15mm} @{\textbf{:} } p{102mm}}
\hline
\multicolumn{2}{l}{\noindent {\sc #1}} \\
\hline
\textbf{input} & #2 \\
\textbf{output} & #3 \\
\hline
\end{tabular}
}
{
\vspace{1.5ex}
}


% use:
% \begin{problem}
% {name of problem}
% {input of problem}
% {output of problem}
% \end{problem}


%% SHORTCUTS

% definitions

\def\reals{{\mathbb R}}
\def\naturals{{\mathbb N}}
\def\conv{{\rm conv}}
\def\0{{\bf0}}
\def\1{{\bf1}}
\def\T{^{\top}}
\def\rone{{\1\T}}

% to choose the name of the problem - GALE or COMPLETELY LABELED GALE STRING

\def\gale{{\sc{Gale}}}
\def\anothergale{{\sc{Another Gale}}}


%%% END PREAMBLE

\begin{document}

\begin{verbatim}

\subsection{Cyclic Polytopes and Gale Strings}\label{gs-ssect}

\begin{example}\label{no-clgs}
For $l = 121314$, there are no completely labeled Gale strings.
\end{example}

\begin{theorem}\label{even-number-gale}
For any labeling $l:[n]\to[d]$, where $d$ is even and $d<n$,
the number of completely labeled Gale strings associated with $l$ is even.
\end{theorem}

\bibitem{edm} J. Edmonds (1965).
Paths, trees, and flowers.
\emph{Canad. J. Math.} 17, pp. 449--467.

\end{verbatim}

\subsection{The Complexity of \gale\ and \anothergale}
\label{main-result-subsection}

We will now give our main result: \anothergale\ can be solved in polynomial
time. Therefore, it takes polynomial time to find a Nash Equilibrium of a
bimatrix game for which the best response polytope is cyclic.

Our proof will be based on a simple graph construction.

\begin{definition}

A {\em perfect matching} for a graph $G=(V,E)$ is a set $M\subseteq E$ of
pairwise non-adjacent edges so that every vertex $v \in V$ is incident to
exactly one edge in~$M$.

We define the problem {\sc Perfect Matching} as follows:

\begin{decproblem}
{Perfect Matching}
{A graph $G = (V,E)$.}
{Is there a perfect matching for $G$?}
\end{decproblem}

\end{definition}

The complexity of {\sc Perfect Matching} has been proven to be in P by
Edmonds \cite{edm}.

\begin{theorem}[\cite{edm}]\label{pm-thm}
The problem {\sc Perfect Matching} is solvable in polynomial time.
\end{theorem}

We will first consider the accessory problem \gale, and we will show that
it is solvable in polynomial time by using theorem \ref{pm-thm}.\\

\begin{decproblem}
{\gale}
{A labeling $l:[n]\to[d]$, where $d$ is even and $d<n$.}
{Does there exists a completely labeled Gale string~$s$ in~$G(d,n)$ associated
with $l$?}
\end{decproblem}

\begin{theorem}\label{gale-thm}
The problem \gale\ is solvable in polynomial time.
\begin{proof}

We give a reduction of \gale\ to {\sc Perfect Matching}.

In the following, we will consider every Gale string as a ``loop,'' as seen
in section \ref{gs-ssect}, so $n+1=1$.

Given the labeling $l:[n]\to[d]$, let $V=[d]$, let $E=\{(l(i),l(i+1))
\mbox{ for }i\in[n]\mbox{ for every }i\mbox{ such that }l(j)\neq l(i+1)\}$,
and consider the multigraph $G=(V,E)$.

Let $s\in G(d,n)$ be a completely labeled Gale string. Then every run of $s$
splits uniquely into $d/2$ pairs $(i,i+1)$ such that the labels $l(i)$
satisfy the condition $l(i)\neq l(i+1)$, and all the labels
$l(i)\in [d]$ occur. Then the labels will correspond to all the vertices of
$G$, and the pairs will correspond to the edges of a perfect matching for
$G$.

Conversely, let $l:[n]\to [d]$ be a labeling, and let $M$ be a perfect
matching for $G$ as above. We can construct a string $s$ such that
$s(i)=s(i+1)$ for every $(l(i),l(i+1))\in M$ and $s(i)=0$
otherwise. Since $M$ is a matching, all the $(l(i),l(i+1))\in M$ are
disjoint, so, considering $s$ as a ``loop,'' every run is of even
length. Furthermore, since $M$ is a perfect matching, every vertex
$v\in [d]$ is the endpoint of an edge $(l(i),l(i+1))$, so $s$ is
completely labeled.

We have a reduction from \gale\ to the problem {\sc Perfect Matching},
which is polynomial-time solvable by theorem \ref{pm-thm}.
Finding a Gale string for a given labeling, or deciding that there isn't
one, can therefore be done in polynomial time.
\end{proof}
\end{theorem}

We give two examples of the construction used in theorem \ref{gale-thm}.

\begin{example}\label{gs-pm-ex}
Let $l=12343122$ be a string of labels. Then the egdes $e_i$ of the graph
$G$ obtained from the construction in the proof of theorem  \ref{gale-thm}
will be as follow:

\begin{displaymath}
\xymatrix
@M=5pt
{
*+[o][F-]{1} \ar@/^1pc/@{-}[r] |{e_1} &
*+[o][F-]{2} \ar@/^1pc/@{-}[r] |{e_2} &
*+[o][F-]{3} \ar@/^1pc/@{-}[r] |{e_3} &
*+[o][F-]{4} \ar@/^1pc/@{-}[r] |{e_4} &
*+[o][F-]{3} \ar@/^1pc/@{-}[r] |{e_5} &
*+[o][F-]{1} \ar@/^1pc/@{-}[r] |{e_6} &
*+[o][F-]{2} \ar@/^1pc/@{-}[r] |{cycle} &
*+[o][F-]{2} \ar@/^2pc/@{-}[lllllll] |{e_7} \\
}
\end{displaymath}

Given the vertices $v\in [4]$, the graph $G$ will be:

\begin{displaymath}
\xymatrix
@M=5pt
{
*+[o][F-]{1}
\ar@/^1pc/@{-}[rr] |{e_1}
\ar@{-}[rr] |{e_6}
\ar@/_1pc/@{-}[rr] |{e_7}
\ar@{-}[ddrr] |{e_5}
& &
*+[o][F-]{2}
\ar@{-}[dd] |{e_2}
\\
\\
*+[o][F-]{3}
\ar@{-}[rr] |{e_3}
\ar@/_1pc/@{-}[rr] |{e_4}
& &
*+[o][F-]{4}
}
\end{displaymath}

A perfect matching for $G$ is given by $M=\{e_1, e_4\}$.

\begin{displaymath}
\xymatrix
@M=5pt
{
*+[o][F-]{1}
\ar@/^1pc/@{-}[rr] |{e_1}
\ar@{.}[rr] |{e_6}
\ar@/_1pc/@{.}[rr] |{e_7}
\ar@{.}[ddrr] |{e_5}
& &
*+[o][F-]{2}
\ar@{.}[dd] |{e_2}
\\
&
&
\\
*+[o][F-]{3}
\ar@{.}[rr] |{e_3}
\ar@/_1pc/@{-}[rr] |{e_4}
& &
*+[o][F-]{4}
}
\end{displaymath}


In turn, this corresponds to the completely labeled Gale string $11011000$.

\begin{displaymath}
\xymatrix
@M=5pt
{
*+<1em>[F-:<12pt>]{\txt{1 \\ {\bf 1}}} \ar@/^1pc/@{-}[r] |{{\bf e_1}} &
*+<1em>[F-:<12pt>]{\txt{2 \\ {\bf 1}}} \ar@/^1pc/@{-}[r] |{e_2} &
*+<1em>[F-:<12pt>]{\txt{3 \\ 0}} \ar@/^1pc/@{-}[r] |{e_3} &
*+<1em>[F-:<12pt>]{\txt{4 \\ {\bf 1}}} \ar@/^1pc/@{-}[r] |{{\bf e_4}} &
*+<1em>[F-:<12pt>]{\txt{3 \\ {\bf 1}}} \ar@/^1pc/@{-}[r] |{e_5} &
*+<1em>[F-:<12pt>]{\txt{1 \\ 0}} \ar@/^1pc/@{-}[r] |{e_6} &
*+<1em>[F-:<12pt>]{\txt{2 \\ 0}} \ar@/^1pc/@{-}[r] |{cycle} &
*+<1em>[F-:<12pt>]{\txt{2 \\ 0}} \ar@/^3pc/@{-}[lllllll] |{e_7} \\
}
\end{displaymath}

\end{example}

A perfect matching for a graph, and therefore a Gale string for a labeling,
is not always possible, as shown in the next example.

\begin{example}
Let us consider the labeling $l=121314$. The associated graph $G$ will be

\begin{displaymath}
\xymatrix
@M=5pt
{
1
\ar@/^1pc/@{-}[rr] |{e_1}
\ar@{-}[rr] |{e_2}
\ar@/^1pc/@{-}[ddrr] |{e_3}
\ar@{-}[ddrr] |{e_4}
\ar@{-}[dd] |{e_5}
\ar@/_1pc/@{-}[dd] |{e_6}
& & 2
\\
\\
3
& & 4
}
\end{displaymath}

Since there aren't any disjoint edges, it's not possible to find a perfect
matching for $G$. Analogously, we have seen in example \ref{no-clgs} that
there isn't any possible completely labeled Gale string for the labeling
$l$.
\end{example}

We finally extend the proof of theorem \ref{gale-thm} to show that
\anothergale\ is polynomial-time solvable.

\begin{theorem}\label{anothergale-thm}
The problem \anothergale\ is solvable in polynomial time.

\begin{proof}

Let $l:[n]\to [d]$ be a labeling, and let $s\in G(d,n)$ be a completely
labeled Gale string for $l$. Let $G=(V,E)$ be the graph constructed from $l$
as in the proof of theorem \ref{gale-thm}, and let $M$ be its perfect
matching for $G$ corresponding to $s$.

If there is an edge $e=(l(i),l(i+1))\in M$ and there is an edge
$e'\neq e$ in $G$ such that $e'=(l(i),l(i+1))$ (recall that $G$ can be a
multigraph), we simply consider the matching
$M'=M\setminus \{e\} \cup \{e'\}$. Let $s'$ be the completely labeled
Gale string corresponding to $M'$; the 1's corresponding to the labels
$l(i),l(i+1)$, that in $s$ were in the position given by the edge $e$, for
$s'$ are in the position given by $e'\neq e$. Therefore, we have
a completely labeled Gale string that is different from the one in the
input of the problem.

We now assume that all the edges in every perfect matching $M$ for $G$
don't have a parallel edge. Note that this condition is only on the
edges in the matching; $G$ can still be a multigraph.

Theorem \ref{even-number-gale} guarantees the existence of a completely
labeled Gale string $s'\neq s$; since the two strings are different, the
perfect matching $M'\neq M$ corresponding to one of these $s'$ does not
use at least one edge $e\in M$. There are $d/2$ possible graphs
$G_i'=(V,E_i')$, where $E_i'=E\setminus \{e_i\}$ for each $e_i\in M$; since
$V(G)=V(G')$ and $E(G)\subset E(G')$, every perfect matching for $G'$
is a perfect matching for $G$ as well.
The existence of $s'$ implies that there is at least one graph $G'$
with a perfect matching $M'\neq M$.
With a brute force
approach, the time to find this $G'$ and the corresponding $M'$ will
be given by the time to find a perfect matching multiplied by a factor
$O(d)$. Therefore, searching for a completely labeled Gale string $s'\neq s$
takes again polynomial time.
\end{proof}
\end{theorem}

We give two examples of the construction of theorem \ref{anothergale-thm}.

\begin{example}
We consider the labeling the string of labels $l=1234312$. We have found in
example \ref{gs-pm-ex} the completely labeled Gale string $1101100$,
corresponding to the perfect matching $M=\{e_1, e_4\}$ in the graph $G$.

\begin{displaymath}
\xymatrix
@M=5pt
{
*+[o][F-]{1}
\ar@/^1pc/@{-}[rr] |{e_1}
\ar@{--}[rr] |{e_6}
\ar@/_1pc/@{.}[rr] |{e_7}
\ar@{.}[ddrr] |{e_5}
& &
*+[o][F-]{2}
\ar@{.}[dd] |{e_2}
\\
&
&
\\
*+[o][F-]{3}
\ar@{.}[rr] |{e_3}
\ar@/_1pc/@{-}[rr] |{e_4}
& &
*+[o][F-]{4}
}
\end{displaymath}

If instead of $e_1$ we take the parallel edge $e_6$, the resulting matching
is still perfect.
\end{example}

A case in which all the edges in every perfect matching don't have a
parallel edge is the following; note that $G$ is a multigraph.

\begin{example}
We consider the labeling $l=123142$. There are only two possible perfect
matchings for the corresponding graph: $M=\{e_2,e_4\}$, that corresponds to
the completely labeled Gale string $s=011110$, and $M'=\{e_3,e_5\}$, that
corresponds to $s'=001111$.

\begin{displaymath}
\xymatrix
@M=5pt
{
*+[o][F-]{1}
\ar@/^1pc/@{.}[rr] |{e_6}
\ar@{.}[rr] |{e_1}
\ar@{--}[dd] |{e_4}
\ar@{-}[ddrr] |>>>>>>{e_3}
& &
*+[o][F-]{2}
\ar@{--}[dd] |{e_2}
\ar@{-}[ddll] |>>>>>>{e_5}
\\
&
&
\\
*+[o][F-]{3}
& &
*+[o][F-]{4}
}
\end{displaymath}
\end{example}

\end{document}
