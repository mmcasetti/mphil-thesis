\subsection{Lemke Paths and the Lemke-Howson for Gale Algorithm}

\todo[inline]{
We have NEs $\Leftrightarrow$ completely labeled things (facets, vertices, GS)
We give now different versions of fundam algorithm to deal with labeling
looking for compl.label. - in particular in these cases

first in version on simple polytopes with labeled facets,
(name: Lemke-Howson;
Lemke-Howson 1964, Shapley 1974 beautiful exposition)

then in its corresponding version on Gale strings (name: Lemke-Howson for Gale,
where???).

We will also mention dual version on simplicial polytopes with labeled vertices
(name: exchange algorithm, Edmonds - Sanit\`{a}). (This case: index more
problematic - see )

Sth about why (exp lemke paths; conjecture)
}


Consider a labeling $l:[n]\to [m]$, for a set $X$ with $|X|=n$.
Then $x=(x_1,\ldots,x_m)\in X^m$ is {\em almost completely labeled} if
$|\{ j\in [n]\ |\ x_i=j\text{ for some }i\in [m]  \}|
= [m]\setminus \{ k \}$ for exactly one $k\in [m]$.
This mean that all labels appear once in $x$,
except for the {\em missing label} $k$,
and a {\em duplicate label} $\bar{k}\in [m]$ that appears twice.

\todo[inline]{now we see in poly with labeled facets, cl vertices}

Let $P$ be a simple polytope in dimension $m$ with $n$ facets.
We define the operation of {\em pivoting on vertices} as moving from a
vertex $x$ of $P$ to another vertex $y$ such that there is an edge
between $x$ and $y$. Note that, since $P$ is simple, there are exactly
$m$ possible choices for $y$.

Now let $l_f:[n]\to [m]$ be a labeling of the facets of $P$ such that there
is at least one completely labeled vertex $x_0$ of $P$.
Note that if we pivot from vertex $v$ we ``leave behind''
a facet $F$, that has label $k$; we call this {\em dropping label $k$}.
We will then reach a vertex $w$ that shares with $v$ all facets except
$F$ (that contains $v$ but not $w$) and another facet $G$
(that contains $w$ but not $v$) that has label $j$;
we will call this {\em picking up label $j$}.
We give the {\em Lemke-Howson algorithm} \todo{reference!}
as in \ref{lh-alg}.

\begin{algorithm}\label{lh-alg}
\SetKwInOut{Input}{input}
\SetKwInOut{Output}{output}
\Input{
A simple $m$-polytope $P$ with $n$ facets.
A labeling $l_f:[n]\to [m]$ of the facets of $P$.
A vertex $x_0$ of $P$, completely labeled for $l$.
}
\Output{
A completely labeled vertex $y\neq x_0$ of $P$.
}
\BlankLine
choose a label $k\in [n]$ \\
pivot from $x_0$ to $y$ dropping label $k$ \\
\While{ $y$ is not completely labeled }
{
pivot from $y$ to $y'$ dropping duplicate label $j$, moving away from $x_0$ \\
rename $x_0 = y$, $y = y'$
}
\Return $y$
\caption{Lemke-Howson algorithm}
\end{algorithm}

The steps of the Lemke-Howson algorithm result in a {\em Lemke path}
that connects two completely labeled vertices through
{\em $k$-almost complementary} vertices and edges, that is,
almost completely labeled vertices and edges where the only missing label
is $k$.
It remains to show that $y\neq x_0$. This comes from the fact that the Lemke
paths are {\em simple paths}, that is, there are no ``loops'' where a vertex
is visited more than once.
This is not possible because at each vertex there are
only two edges corresponding to the missing label $k$, since $P$ is not
degenerate; one is the edge that is traversed to get to the vertex, one is
the one that is traversed to leave it in the next step.
This proves the following.

\begin{theorem}\label{lh-works-thm}
The Lemke-Howson algorithm \ref{lh-alg} returns a solution to the problem
{\sc Another Completely Labeled Vertex}.
\end{theorem}

In the context of finding the Nash equilibrium of a bimatrix game $(A,B)$,
there are two equivalent implementations of the Lemke-Howson algorithm.

We can consider the game $C$ as in proposition \ref{symmetric-eq-thm},
and the associated polytope
$S = \{ z\in\reals^{m+n}\ |\ z\geq\0,\ Cz\leq\1 \}$,
labeling the $2(m+n)$ inequalities defining the facets of of $S$
as $1,\ldots,m+n,1,\ldots,m+n$.
Then applying the Lemke-Howson algorithm starting from
vertex $\0$ returns a Nash equilibrium $(z,z)$ of $C$ and a corresponding
$(x,y)=z$ a Nash equilibrium of $(A,B)$.

We can also follow the ``traditional'' version of the Lemke-Howson algorithm;
a very clear exposition of this can be found in Shapley \cite{shapley}.
Let $P$ and $Q$ be the best response polytopes of $(A,B)$ as in
\ref{br-polytopes}. We then move alternately on $P$ and $Q$, starting from
the couple of vertices $(\0,\0)$.
Since we move in $\reals^m$ and $\reals^n$ instead of $\reals^{m+n}$,
this version is more practical to visualize,
as shown in the following example.

\begin{example}

\todo[inline]{ex Savani - von Stengel, pag. 11; fig 8 are Schegel diagrams of
BR polytopes.}

\end{example}

Theorem \ref{lh-works-thm} has a straightforward dual version.
Let $Q$ be a simplicial polytope in dimension $m$ with $n$ vertices.
We {\em pivot on facets} by moving from facet $F$ to facet $G$
that shares an edge with $F$. Since $P$ is simplicial,
there are exactly $m$ possible choices for $G$.
Let $l_v:[n]\to [m]$ be a labeling of the vertices of $P$ such that there
is at least one completely labeled facet $F_0$.
We {\em drop label $k$} and {\em pick up label $j$} when pivoting
from a facet $F$ to a facet $G$ that shares with $F$ all vertices
except a vertex $v$ with label $k$ that belongs to $F$ but not $G$,
and another vertex $w$ with label $j$ that belongs to $G$ but not $F$.
The Lemke-Howson algorithm then becomes the {\em exchange algorithm}
\todo{reference!} as in \ref{exchange-alg}.

\begin{algorithm}\label{exchange-alg}
\SetKwInOut{Input}{input}
\SetKwInOut{Output}{output}
\Input{
A simplicial $m$-polytope $Q$ with $n$ vertices.
A labeling $l_v:[n]\to [m]$ of the vertices of $P$.
A vertex $F_0$ of $Q$, completely labeled for $l$.
}
\Output{
A completely labeled facet $F\neq F_0$ of $Q$.
}
\BlankLine
choose a label $k\in [n]$ \\
pivot from $F_0$ to $F$ dropping label $k$ \\
\While{ $F$ is not completely labeled }
{
pivot from $F$ to $F'$ dropping duplicate label $j$, moving away from $F_0$ \\
rename $F_0 = F$, $F = F'$
}
\Return $F$
\caption{Exchange algorithm}
\end{algorithm}

Considering the dual of Lemke paths on (almost) completely labeled facets,
we get the dual result to theorem \ref{lh-works-thm}.

\begin{theorem}\label{exchange-works-thm}
The exchange algorithm \ref{exchange-alg} returns a solution to the problem
{\sc Another Completely Labeled Facet}.
\end{theorem}

As before, we now consider the case where the simplicial polytope $Q$
is cyclic, that is, the case that we can study from the point of view of
Gale strings.




\todo[inline]{
sp case cycl poly leads to gs

\ref{vert-labeling-unitv} in thm \ref{unit-vector-dual-thm}

then gs:

first we note a result on generic unit vector games (SvS 15)...

(here? before? - it's on all games! - nowhere?)
thm Savani '06 - pg 12 SvS-15:
not only LH paths on $P\times Q$ simple
path, but projection to simple paths on both $P$ and $Q$.
}

In the case of a unit vector game $(U,B)$, where
$U=(e_{l(1)}\cdots e_{l(n)})$ for a labeling $l:[n]\to [m]$, let
as in \ref{p-l-unitv}; theorem \ref{unit-vector-thm} then shows that
$P^l$ encodes all the Nash equilibria of $(U,B)$ as completely labeled
vertices. We can apply the Lemke-Howson algorithm to both $P^l$ and
to the product of the best response polytopes $P$ and $Q$.
The projection of a Lemke path for a missing label $i\in [m]$
on $P\times Q$ to $P$ defines a Lemke path in $P^l$. \todo{how?}
However, $P\times Q$ has $m+n$ labels, therefore there could be
Lemke paths for a missing label $m+j$ with $j\in [n]$
on $P\times Q$ that get lost in the projection on $P^l$.
The following theorem, proved in Savani and von Stengel \cite{uvg},
% theorem 5, page 12
shows that there is no loss of generality in studying Lemke paths on
$P^l$.

\begin{theorem}
Let $(U,B)$ be a unit vector game, with
$U=(e_{l(1)}\cdots e_{l(n)})$ for a labeling $l:[n]\to [m]$;
let $P = \{ x\in\reals^m | x\geq\0,\ B\T x\leq\1 \}$ and
$Q = \{ y\in\reals^n | y\geq\0,\ A y\leq\1 \}$, as in \ref{br-polytopes};
and let $P^l= \{ x\in\reals^m\ |\ x\geq\0,\ B\T x\leq\1 \}$
as in \ref{p-l-unitv}. Then the Lemke path on $P\times Q$ for the missing
label $k$ projects to a path on $P$ that is the Lemke path on $P^l$
for missing label $k$ if $k\in [m]$,
and for missing label $l(j)$ if $k=m+j$ with $j\in [n]$.
\end{theorem}

\todo[inline]{
so if poly dual cyclic, when cons from pow gale strings
we could take the str of labels $l(n+j)\cdots l(n+m)$
instead of $l(1)\cdots l(n+m)$,
that is, we could cut the ``artificial'' first labels $12...n$.



After all, in main we're studying ANOTHER GALE
in general, not nec starting from $12...n$; and we're interested in finding
{\em one} eq that's not the one we started from
(and is at other end of LPath, since index and so on),
{\em not all equilibria};
but the eq we started from is not nec the artificial one - actually,
if we go with this we can take any NE to start looking for another,
and we're sure to find a ``non-artificial'' one.
Note: if we were looking for all NE, LH doesn't work anyway - see ex by
Wilson in Shapley, where ``disconnected'' paths between equilibria.

----

complexity considerations, PPAD

Needed for why result interesting;
it can be done at the end, with Morris paths
}
