\subsection{Lemke Paths and the Lemke-Howson for Gale Algorithm}

\todo[inline]{
We have NEs $\Leftrightarrow$ completely labeled things (facets, vertices, GS)
We give now different versions of fundam algorithm to deal with labeling
looking for compl.label. - in particular in these cases

first in version on simple polytopes with labeled facets,
(name: Lemke-Howson;
Lemke-Howson 1964, Shapley 1974 beautiful exposition)

then in its corresponding version on Gale strings (name: Lemke-Howson for Gale,
where???).

We will also mention dual version on simplicial polytopes with labeled vertices
(name: exchange algorithm,
Edmonds - Sanit\`{a}).

Sth about why (exp lemke paths; conjecture)
}


Consider a labeling $l:[n]\to [m]$, where $n\geq m$;
then $x = \{ l(i)\in [n]\ |\ i\in [m] \}$ is {\em almost completely labeled}
if $x = [m]\setminus \{ k \}$
for exactly one {\em missing label} $k\in [m]$.
Since $|x|=m$, this mean that all other labels appear once in $x$ except for
one {\em duplicate label} $j$ that appears twice.

Let $P$ be a simple polytope in dimension $m$ with $n$ facets.
We define the operation of {\em pivoting on vertices} as moving from a
vertex $x$ of $P$ to another vertex $y$ such that there is an edge
between $x$ and $y$. Note that, since $P$ is simple, there are exactly
$m$ possible choices for $y$.

Now let $l:[n]\to [m]$ be a labeling of the facets of $P$ such that there
is at least one completely labeled vertex $x_0$ of $P$.
Note that if we pivot from a vertex we ``leave behind''
a facet, that has label $k$; we call this {\em dropping label $k$}.
We will then reach a vertex that lies on a new facet, that has label $j$;
we will call this {\em picking up label $j$}
We give the {\em Lemke-Howson algorithm} \todo{reference!} as follows:

\begin{algorithm}\label{lh-one-polytope}
\SetKwInOut{Input}{input}
\SetKwInOut{Output}{output}
\Input{
A simple $m$-polytope $P$ with $n$ facets.
A labeling $l:[n]\to [m]$ of the facets of $P$.
A vertex $x$ of $P$, completely labeled for $l$.
}
\Output{
A completely labeled vertex $y\neq x_0$ of $P$.
}
\BlankLine
choose a label $k\in [n]$ \\
pivot from $x$ to $y$ dropping label $k$ \\
\While{ $y$ is not completely labeled }
{
let $j$ be the duplicate label of $y$ \\
pivot from $y$ to $y'$ dropping label $j$ {\em on the facet shared with $x$} \\
rename $y = y'$
}
\Return $y$
\caption{Lemke-Howson algorithm}
\end{algorithm}

\todo[inline]{
no cycles (ref? from Lemke? LH? Of course well explained in Shapley...)
which is condition for the alg to effectively return $y\neq x$

\begin{theorem}\label{no-cycles}
LH returns sol of ANOTHER CL VERTEX
\end{theorem}

complexity considerations, PPAD
(check def PPAD: do we need next step in P-time?)

simple paths - Lemke paths: def so by Morris 94

[later: extend term ``Lemke paths'' to paths of every LH-style algorithms we
see.
...
In this case, Lemke paths ``translate'' as / correspond to...
]
}

In the context of finding the Nash equilibrium of a bimatrix game $(A,B)$,
there are two equivalent implementations of the Lemke-Howson algorithm.

We can consider the game $C$ as in proposition \ref{symmetric-eq-thm},
and the associated polytope
$S = \{ z\in\reals^{m+n}\ |\ z\geq\0,\ Cz\leq\1 \}$,
labeling the $2(m+n)$ inequalities defining the facets of of $S$
as $1,\ldots,m+n,1,\ldots,m+n$.
Then applying the Lemke-Howson algorithm starting from
vertex $\0$ returns a Nash equilibrium $(z,z)$ of $C$ and a corresponding
$(x,y)=z$ a Nash equilibrium of $(A,B)$.

We can also follow the ``traditional'' version of the Lemke-Howson algorithm;
a very clear exposition of this can be found in Shapley \cite{shapley}.
Let $P$ and $Q$ be the best response polytopes of $(A,B)$ as in
\ref{br-polytopes}. We then move alternately on $P$ and $Q$, starting from
the couple of vertices $(\0,\0)$.
Since we move in $\reals^m$ and $\reals^n$ instead of $\reals^{m+n}$,
this version is more practical to visualize,
as shown in the following example.

\begin{example}

\todo[inline]{ex Savani - von Stengel, pag. 11; fig 8 are Schegel diagrams of
BR polytopes.}

\end{example}

In the case of unit vector games $(U,B)$,

\todo[inline]{
(?) thm Savani: not only simple path, but projection to simple paths.

note, thm SvS-15: $P^l$, dim $m$, is enough to study all.
(we could take the str of labels for the gale pow
$l(n+j)\cdots l(n+m)$ instead of $l(1)\cdots l(n+m)$,
that is, we could cut the ``artificial'' first labels $12...n$.

Do we go straight for this, or?
YES!

After all, in main we're studying ANOTHER GALE
in general, not nec starting from $12...n$; and we're interested in finding
{\em one} eq that's not the one we started from
(and is at other end of LPath, since index and so on),
{\em not all equilibria};
but the eq we started from is not nec the artificial one - actually,
if we go with this we can take any NE to start looking for another,
and we're sure to find a ``non-artificial'' one.
Note: if we were looking for all NE, LH doesn't work anyway - see ex by
Wilson in Shapley, where ``disconnected'' paths between equilibria.
}
