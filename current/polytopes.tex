
\subsection{Some Geometrical Notation}

\todo[inline]{so results in labels section - after this - don't get lost in
boredom, and each section in background is about 150-200 lines (see: getting
lost).

Maybe turn this into an appendix? It would make sense if something more about
proof of Nash}

We denote the transpose of a matrix $A$ as $A\T$.
We consider vectors $u,v\in\reals^d$ as column vectors, so $u\T v$ is
their scalar product. A vector in $\reals^d$ for which all components
are $0$'s will be denoted as $\0$; similarly, a vectors for which all
components are $1$'s will be denoted as $\1$.
The {\em unit vector} $e_i$ is the vector that has $i$-th component
\smash{${e_i}_i = 1$} and \smash{${e_i}_j=0$} for all other components.
When writing an inequality of the form $u\geq v$ (and analogous), we mean
that it holds for every component; that is, $u_i\geq v_i$ for all
$i\in [d]$.

An {\em affine combination} of points in an Euclidean space $z_1,\ldots,z_n$
is
\[
\sum_{i=1}^n \lambda_i z_i \quad \text{where }\lambda_i\in\reals
\text{ such that }\sum_{i=1}^n \lambda_i = 1
\]

The points $z_1,\ldots,z_n$ are {\em affinely independent} if none of them
is an affine combination of the others.

A {\em convex combination} of points $z_1,\ldots,z_n$ is an affine
combination where $\lambda_i\geq 0$ for all $i\in [n]$.
Note that such $\lambda_i$'s can be seen as a probability distribution over
the $z_i$'s.

\todo[inline]{def simplex: here?}

A set of point $Z$ is {\em convex} if it is closed under forming convex
combinations, that is, if $\bar{z}=\sum_{i=1}^n \lambda_i z_i$,
where $z_i\in Z$, $\lambda_i\geq 0$ and $\sum_{i=1}^n \lambda_i = 1$,
then $\bar{z}\in Z$. A convex set has {\em dimension} $d$ if it has exactly
$d + 1$ affinely independent points.


\todo[inline]{def simplex: here?}


\todo[inline]{convex hull (needed for def cyclic poly);

pow hyperplanes;

polyhedron, polytopes}



\todo[inline]{from here: notes - copy-paste}

A ($d$-dimensional) {\em simplicial polytope} $P$ is the convex hull of a set
of at least $d+1$ points $v$ in $\reals^d$ in general position, that is, no
$d+1$ of them are on a common hyperplane.

If a point $v$ cannot be omitted from these points without changing $P$ then
$v$ is called a {\em vertex} of $P$. A {\em facet} of $P$ is the convex hull
$\conv\,F$ of a set $F$ of $d$ vertices of $P$ that lie on a hyperplane
$\{ x\in \reals^d\mid a^T x=a_0\}$ so that $a^T u<a_0$ for all other vertices
$u$ of $P$; the vector $a$ (unique up to a scalar multiple) is called the
{\em normal vector} of the facet. We often identify the facet with its set of
vertices~$F$.

\newpage

\subsection{Bimatrix Games, Labels and Polytopes}

In the rest of this thesis we will focus on two-player normal-form games.
For sake of readability, we will use feminine pronouns when referring to
player 1 and masculine pronouns when referring to player 2.

Two-player normal-form games are
also called {\em bimatrix games}, since they can be characterized by the
$m \times n$ payoff matrices $A$ and $B$, where $a_{ij}$ and $b_{ij}$ are
the payoffs of player 1 and 2 when she plays her $i$th pure strategy
and he plays his $j$th pure strategy.
We will assume that $(A,B)$ are non-negative, and that $A$ and $B\T$ have
no zero column. This can be easily obtained without loss of generality via
an affine transformation that will not affect the equilibria of the game.

The Nash equilibria of bimatrix games can be analysed from a combinatorial
point of view using {\em labels}. This method is due to Shapley
\cite{shapley}, in a study building on ideas introduced in a
paper by Lemke and Howson \cite{lh}.


Let $(A,B)$ be bimatrix game. The mixed-strategy simplices of player 1 and 2
are, respectively

\begin{equation}
X = \{ x\in\reals^m | x\geq\0,\ \1\T x = 1 \},\quad
Y = \{ y\in\reals^n | y\geq\0,\ \1\T y = 1 \}
\end{equation}

A {\em labeling} of the game is then given as follows:

\begin{enumerate}
\item the $m$ pure strategies of player 1 are identified by $1,\ldots,m$;
\item the $n$ pure strategies of player 2 are identified by $m+1,\ldots,m+n$;
\item each mixed strategy $x\in X$ of player 1 has
    \begin{itemize}
    \item label $i$ for each $i\in [m]$ such that $x_i = 0$, that is if in
    $x$ player 1 does not play her $i$th pure strategy;
    \item label $m + j$ for each $j\in [n]$ such that the $j$th pure strategy
    of player 2 is a best response to $x$;
    \end{itemize}
\item each mixed strategy $y\in Y$ of player 2 has
    \begin{itemize}
    \item label $m + j$ for each $j\in [n]$ such that $y_j = 0$, that is if in
    $y$ player 2 does not play his $j$th pure strategy;
    \item label $i$ for each $i\in [m]$ such that the $i$th pure strategy
    of player 1 is a best response to $y$;
    \end{itemize}
\end{enumerate}

A strategy profile $(x,y)\in X\times Y$ is {\em completely labeled} if every
label $1,\ldots,m+n$ is a label of either $x$ or $y$. We have the following
theorem (Theorem 1 in \cite{shapley}):

\begin{theorem}\label{comp-label-bimatrix-thm}
Let $(x,y)\in X\times Y$; then $(x,y)$ is a Nash equilibrium of the bimatrix
game $(A,B)$ if and only if $(x,y)$ is completely labeled.

\begin{proof}
The mixed strategy $x\in X$ has label $m + j$ for some $j\in [n]$ if and
only if the $j$th pure strategy of player 2 is a best response to $x$; this,
in turn, is a necessary and sufficient condition for player 2 to play his
$j$th strategy at an equilibrium against $x$. Therefore, at an equilibrium
$(x,y)$ all labels $m + 1,\ldots,m + n$ will appear either as labels of
$x$ or of $y$. The analogous holds for the labels $i\in [n]$.
\end{proof}
\end{theorem}

An useful geometrical representation of labels can be given on the mixed
strategies simplices $X$ and $Y$. The outside of each simplex is
labeled according to the player's own pure strategies that are {\em not}
played; so, for instance, the outside of $X$ will have labels
$1,\ldots,n$. The interior of each simplex is subdivided in closed polyhedral
sets, called {\em best-response regions}. These are labeled according to
the other player's pure strategy that is a best response in that set;
so, for instance, the inside of $X$ will have labels $m + 1,\ldots,m + n$.

We give an example of this construction.

\begin{example}

\todo[inline]{page 3--4 of Savani, von Stengel, Unit Vector Games.

With graphics.}

\end{example}

We will now give a description of labeling on polytopes equivalent to
the construction based on best-response regions.

We begin by noticing that the best-response regions can be obtained as
projections on $X$ and $Y$ of the {\em best-response facets} of
the polyhedra

\begin{equation}\label{br-polyhedron}
\bar{P} = \{ (x,v)\in X\times\reals | B\T x\leq\1 v \},\quad
\bar{Q} = \{ (y,u)\in Y\times\reals | A y\leq\1 u \}.
\end{equation}

These facets in $\bar{P}$ are defined as the points $(x,v)\in X\times\reals$
such that $(B\T x)_j = v$. These points represent the strategies $x\in X$ of
player 1 that give exactly payoff $v$ to player 2 when he plays strategy $j$.
The projection of the facet defined by $(B\T x)_j = v$ to $X$ will have
label $j$. Analogously, in $\bar{Q}$, the facets are the points
$(y,u)\in Y\times\reals$ such that $a_i y = u$, and their projection to $Y$
will be the best-response region with label $i$.

\begin{example}

\todo[inline]{cont of ex above, page 4--5, image on page 5 left}

\end{example}

Given the assumptions on non-negativity of $A$ and $B\T$, we can give a
change coordinates to $x_i / v$ and $y_j / u$ and replace $\bar{P}$ and
$\bar{Q}$ with the {\em best-response polytopes}

\begin{equation}\label{br-polytopes}
P = \{ x\in\reals^m | x\geq\0,\ B\T x\leq\1 \},\quad
Q = \{ y\in\reals^n | y\geq\0,\ A y\leq\1 \},\quad
\end{equation}

Each one of these polytope is defined by half spaces corresponding to
either the player's own strategy that is not being played or the other
player's best response; each one of the facets of the polytope is labeled
by the strategy corresponding to the relative half-space.

This means that a point in $P$ has label $k$ if and only if either
$x_k = 0$ for $k\in \{ 1,\ldots,m \}$ or $(B\T x)_{k - m} = 0$ for
$k\in \{ m+1,\ldots,m+n \}$;
analogously, a point in $Q$ has label $k$ if and only if either
$y_{k - m} = 0$ for $k\in \{ m+1,\ldots,m+n \}$ or $(A y)_{k}$ for
$k\in \{ m+1,\ldots,m+n \}$. A point $(x,y)\in P\times Q$ is
{\em completely labeled} if every $k\in [m + n]$ is a label of $x$ or $y$.
Note that the point $(\0,\0)$ is completely labeled. Rescaling back to
$\bar{P}$ and $\bar{Q}$, all the non-zero completely labeled points give
exactly all the equilibria of $(A,B)$. In this construction, we will
call $(\0,\0)$ {\em artificial equilibrium}.

\begin{example}

\todo[inline]{ex in Savani, von Stengel, image on page 5 right}

\end{example}

A characterization of the completely labeled pairs in $P\times Q$ can be
given as follows.

\begin{proposition}\label{compl-orth-cond}
The pair $(x,y)\in P\times Q$ is completely labeled if and only if one of
the following condition holds:
\begin{itemize}
\item {\em (Complementarity condition)}

\begin{equation}
x_i = 0\text{ or }(Ay)_i = 1\text{ for all }i\in [m],\quad
y_j = 0\text{ or }(B\T x)_j\text{ for all }j\in [n]
\end{equation}

\item {\em (Orthogonality condition)}

\begin{equation}
x\T (\1 - Ay) = 0,\quad
y\T (\1 - B\T x) = 0
\end{equation}
\end{itemize}
\end{proposition}

Proposition \ref{compl-orth-cond} can be used to prove a useful property:
{\em symmetric games}, that is, games that have payoff matrix of the form
$(C,C\T)$ for some matrix $C$, can be used to study generic bimatrix games
without loss of generality. The result is due to Gale, Kuhn and Tucker
\cite{gale-kuhn-tucker} for zero-sum games; its extension to non-zero-sum
games is a folklore result.

\begin{proposition}
Let $(A,B)$ be a bimatrix game and $(x,y)$ be one of its Nash equilibria.
Then $(z,z)$, where $z=(x,y)$, is a Nash equilibrium of the symmetric game
$(C,C\T)$, where

\[
C = \left(
    \begin{array}{cc}
    0 & A \\
    B\T & 0
    \end{array}
    \right).
\]
\end{proposition}

The converse has been proved by McLennan and Tourky \cite{mclennan-tourky} in
their study of {\em imitiation games}, that is, bimatrix games of the form
$(I,B)$.

\begin{proposition}\label{imitation-thm}
The pair $(x,x)$ is a symmetric Nash equilibrium of the symmetric bimatrix
game $(C,C\T)$ if and only if there is some $y$ such that $(x,y)$ is a
Nash equilibrium of the imitation game $(I,C\T)$.
\end{proposition}

\begin{example}
Consider the symmetric game $(C,C\T)$, where $C\T = B$ in the previous
examples.

\todo[inline]{ex Savani, von Stengel, pg 8}

\end{example}

Balthasar \cite{balthasar} and V\'{e}gh and von Stengel \cite{vvs} extended
proposition \ref{imitation-thm} to {\em unit vector games}, that is, games
of the form $(U,B)$, where the columns of the matrix $U$ are unit vectors.
The form of the theorem that we give here follows the version in Savani
and von Stengel \cite{svs-15}, dual to the one in Balthasar \cite{balthasar}.

\begin{theorem}\label{unit-vector-thm}
Let $l:[n]\to [m]$, and let $(U,B)$ be the unit vector game where
$U=(e_{l(1)}\ \cdots\ e_{l(n)})$. Consider the polytopes $P^l$ and $Q^l$
where

\begin{equation}
P^l = \{ x\in\reals^m | x\geq\0,\ B\T x\leq\1 \}
\end{equation}

\begin{equation}
Q^l = \{ y\in\reals^n | y\geq\0,\
\sum_{\substack{j\in N_i \\ i\in [m]}} y_j\leq 1 \}
\end{equation}

where $N_i = \{ j\in [n] | l(j)=i \}$ for $i\in [m]$.

Label every facet of $P^l$ according to the inequality defining it,
as follows:

\begin{itemize}
\item $x_i\geq 0$ has label $i$, for $i\in [m]$
\item $(B\T x)_j \leq 1$ has label $l(j)$, for $j\in [n]$
\end{itemize}

Then $x\in P^l$ is a completely labeled point of $P^l\setminus\{\0\}$
if and only if there is some $y\in Q^l$ such that, after scaling,
the pair $(x,y)$ is a Nash equilibrium of $(U,B)$

\begin{proof}
Let $P,Q$ be the polytopes associated to the game $(U,B)$ as before.

Let $(x,y)\in P\times Q\setminus\{ \0,\0 \}$ be a Nash equilibrium of
$(U,B)$, therefore completely labeled in $[m + n]$.
Then, if $x_i=0$, then $x$ has label $i\in m$.
If $x_i > 0$ instead, then $y$ has label $i$, therefore $(Uy)_i = 1$,
therefore for some $j\in [n]$ we have $y_j > 0$ and $U_j = e_i$, so $l(j)=i$.
Since $y_j > 0$ and $(x,y)$ is completely labeled, $x\in P$ has label $m+j$,
that is, $(B\T x)_j = 1$, therefore $x\in P^l$ has label $l(j) = i$.
Hence, $x$ is a completely labeled point of $P^l$.

Conversely, let $x\in P^l\setminus \{ \0 \}$ be completely labeled.
If $x_i > 0$, then there is $j\in [m]$ such that $(B\T x) = j$ and
$l(j) = i$, that is, $j\in N_i$. For all $i$ such that $x_i >0 $,
define $y$ as follows: $y_h = 0$ for all $h\in N_i\setminus \{ j \}$,
$y_j = 1$. Then $(x,y)\in P\times Q$ is completely labeled.
\end{proof}
\end{theorem}


\todo[inline]{
nondegeneracy; made nondegenerate by ``lexicographic'' perurbation
(what does it mean?);

ex pg 9; odd no eq, mention homotopy method (find ref)
(tie with Nash, again?)
}
