
\subsection{Bimatrix Games, Labels and Polytopes}

In the rest of this thesis we will focus on two-player normal-form games,
also called {\em bimatrix games}, since they can be characterized by the
payoff matrices by the two players. The Nash equilibria of these games can be
analysed from a combinatorial point of view using {\em labels}. This method is
due to Shapley \cite{shapley}, in a study building on ideas introduced in a
paper by Lemke and Howson \cite{lh}.

From here on, we will assume that the payoff matrices $(A,B)$ of both
players are non-negative, and that $A$ and $B\T$ have no zero column. This
can be easily obtained without loss of generality via an affine
transformation that will not affect the equilibria of the game.

Let $(A,B)$ be bimatrix game. The mixed-strategy simplices of player 1 and 2
are, respectively

\begin{equation}
X = \{ x\in\reals^m | x\geq\0,\ \1\T x = 1 \},\quad
Y = \{ y\in\reals^n | y\geq\0,\ \1\T y = 1 \}
\end{equation}

A {\em labeling} of the game is then given as follows:

\begin{enumerate}
\item the $m$ pure strategies of player 1 are identified by $1,\ldots,m$;
\item the $n$ pure strategies of player 2 are identified by $m+1,\ldots,m+n$;
\item each mixed strategy $x\in X$ of player 1 has
    \begin{itemize}
    \item label $i$ for each $i\in [m]$ such that $x_i = 0$, that is if in
    $x$ player 1 does not play her $i$th pure strategy;
    \item label $m + j$ for each $j\in [n]$ such that the $j$th pure strategy
    of player 2 is a best response to $x$;
    \end{itemize}
\item each mixed strategy $y\in Y$ of player 2 has
    \begin{itemize}
    \item label $m + j$ for each $j\in [n]$ such that $y_j = 0$, that is if in
    $y$ player 2 does not play his $j$th pure strategy;
    \item label $i$ for each $i\in [m]$ such that the $i$th pure strategy
    of player 1 is a best response to $y$;
    \end{itemize}
\end{enumerate}

A strategy profile $(x,y)\in X\times Y$ is {\em completely labeled} if every
label $1,\ldots,m+n$ is a label of either $x$ or $y$. We have the following
theorem (Theorem 1 in \cite{shapley}):

\begin{theorem}\label{comp-label-bimatrix-thm}
Let $(x,y)\in X\times Y$; then $(x,y)$ is a Nash equilibrium of the bimatrix
game $(A,B)$ if and only if $(x,y)$ is completely labeled.

\begin{proof}
The mixed strategy $x\in X$ has label $m + j$ for some $j\in [n]$ if and
only if the $j$th pure strategy of player 2 is a best response to $x$; this,
in turn, is a necessary and sufficient condition for player 2 to play his
$j$th strategy at an equilibrium against $x$. Therefore, at an equilibrium
$(x,y)$ all labels $m + 1,\ldots,m + n$ will appear either as labels of
$x$ or of $y$. The analogous holds for the labels $i\in [n]$.
\end{proof}
\end{theorem}

An useful geometrical representation of labels can be given on the mixed
strategies simplices $X$ and $Y$. The outside of each simplex is
labeled according to the player's own pure strategies that are {\em not}
played; so, for instance, the outside of $X$ will have labels
$1,\ldots,n$. The interior of each simplex is subdivided in closed polyhedral
sets, called {\em best-response regions}. These are labeled according to
the other player's pure strategy that is a best response in that set;
so, for instance, the inside of $X$ will have labels $m + 1,\ldots,m + n$.

We give an example of this construction.

\begin{example}

\todo[inline]{page 3--4 of Savani, von Stengel, Unit Vector Games.

With graphics.}

\end{example}

We will now give a description of labeling on polytopes equivalent to
the construction based on best-response regions.

We begin by noticing that the best-response regions can be obtained as
projections on $X$ and $Y$ of the {\em best-response facets} of
the polyhedra

\begin{equation}\label{br-polyhedron}
\bar{P} = \{ (x,v)\in X\times\reals | B\T x\leq\1 v \},\quad
\bar{Q} = \{ (y,u)\in Y\times\reals | A y\leq\1 u \}.
\end{equation}

These facets are defined as the points $(x,v)$ such that $b_i\T x = v$ in
$\bar{P}$; these correspond to the strategies $x\in X$ of player 1 that
give exactly payoff $v$ to player 2. Analogously, in $\bar{Q}$, the facets
are the points $(y,u)$ such that $a_j y = u$.


\todo[inline]{labels on these facets - that then are on projection = brreg}


\begin{example}

\todo[inline]{cont of ex above, page 4--5}

\end{example}

Given the assumptions on non-negativity of $A$ and $B\T$, we can give a
change coordinates to $x_i / v$ and $y_j / u$ and replace $\bar{P}$ and
$\bar{Q}$ with the {\em best-response polytopes}

\begin{equation}\label{br-polytopes}
P = \{ x\in\reals^m | x\geq\0 B\T x\leq\1 \},\quad
Q = \{ y\in\reals^n | y\geq\0 A y\leq\1 \},\quad
\end{equation}

\todo[inline]{labels on facets, ex cont'd}

\todo[inline]{compl label + 0 = equil

compl and orth conditions

symmetric games; vice versa, imitation games; ex pg 8

nondegeneracy; ex pg 9; odd no eq, mention homotopy method (find ref)
(tie with Nash, again?)}


\newpage

\subsection{Some Geometrical Notation}

\todo[inline]{so results in labels section - after this - don't get lost in
boredom, and each section in background is about 150-200 lines (see: getting
lost).

Maybe turn this into an appendix? It would make sense if something more about
proof of Nash}

We denote the transpose of a matrix $A$ as $A\T$.
We consider vectors $u,v\in\reals^d$ as column vectors, so $u\T v$ is
their scalar product. A vector in $\reals^d$ for which all components
are $0$'s will be denoted as $\0$; similarly, a vectors for which all
components are $1$'s will be denoted as $\1$.
The {\em unit vector} $e_i$ is the vector that has $i$-th component
\smash{${e_i}_i = 1$} and \smash{${e_i}_j=0$} for all other components.
When writing an inequality of the form $u\geq v$ (and analogous), we mean
that it holds for every component; that is, $u_i\geq v_i$ for all
$i\in [d]$.

An {\em affine combination} of points in an Euclidean space $z_1,\ldots,z_n$
is
\[
\sum_{i=1}^n \lambda_i z_i \quad \text{where }\lambda_i\in\reals
\text{ such that }\sum_{i=1}^n \lambda_i = 1
\]

The points $z_1,\ldots,z_n$ are {\em affinely independent} if none of them
is an affine combination of the others.

A {\em convex combination} of points $z_1,\ldots,z_n$ is an affine
combination where $\lambda_i\geq 0$ for all $i\in [n]$.
Note that such $\lambda_i$'s can be seen as a probability distribution over
the $z_i$'s.

\todo[inline]{def simplex: here?}

A set of point $Z$ is {\em convex} if it is closed under forming convex
combinations, that is, if $\bar{z}=\sum_{i=1}^n \lambda_i z_i$,
where $z_i\in Z$, $\lambda_i\geq 0$ and $\sum_{i=1}^n \lambda_i = 1$,
then $\bar{z}\in Z$. A convex set has {\em dimension} $d$ if it has exactly
$d + 1$ affinely independent points.


\todo[inline]{def simplex: here?}


\todo[inline]{convex hull; pow hyperplanes; polyhedron, polyopes; needed?
yes, for cyclic poly!}



\todo[inline]{from here: notes - copy-paste}

A ($d$-dimensional) {\em simplicial polytope} $P$ is the convex hull of a set
of at least $d+1$ points $v$ in $\reals^d$ in general position, that is, no
$d+1$ of them are on a common hyperplane.

If a point $v$ cannot be omitted from these points without changing $P$ then
$v$ is called a {\em vertex} of $P$. A {\em facet} of $P$ is the convex hull
$\conv\,F$ of a set $F$ of $d$ vertices of $P$ that lie on a hyperplane
$\{ x\in \reals^d\mid a^T x=a_0\}$ so that $a^T u<a_0$ for all other vertices
$u$ of $P$; the vector $a$ (unique up to a scalar multiple) is called the
{\em normal vector} of the facet. We often identify the facet with its set of
vertices~$F$.


\newpage

\todo[inline]{this from VvS}

The following theorem, due to Balthasar and von Stengel
\cite{B09,BvS10}, establishes a connection between general
labeled polytopes and equilibria of certain $d\times n$
bimatrix games $(U,B)$.

\begin{theorem}\label{t-unitv}
Consider a labeled  $d$-dimensional simplicial polytope $Q$ with ${\bf 0}$ in
its interior, with vertices $-e_1,\ldots,-e_d,c_1,\ldots,c_n$,
% We need to assume that the c_i are pairwise distinct, otherwise
% a vertex can have several labels.
% We need to assume that c_i vertex of Q for the following reason:
% the definition of completely labeled facet is difficult if
% c_i is in a facet but not vertex of the facet.
so that % $F_0$ in {\rm(\ref{F0})}
$F_0=\conv\{-e_1,\ldots,-e_d\}$
is a facet of $Q$.
Let $-e_i$ have label $i$ for $i\in[d]$, and let $c_j$
have label $l(j)\in[d]$ for $j\in[n]$.
Let $(U,B)$ be the $d\times n$ bimatrix game with
$U=[e_{l(1)}\cdots\,e_{l(n)}]$ and $B=[b_1\,\cdots\,b_n]$,
where $b_j=c_j/(1+{\bf 1}^\top c_j)$ for $j\in[n]$.
Then the completely labeled facets $F$ of $Q$, with the
exception of~$F_0$, are in one-to-one correspondence to the
Nash equilibria $(x,y)$ of the game $(U,B)$ as follows:
if $v$ is the normal vector of $F$, then
$x=(v+{\bf 1}){\bf 1}^\top(v+{\bf 1})$,
and $x_i=0$ if and only if $-e_i\in F$ for $i\in[d]$;
any other label~$j$ of $F$, so that $c_j$ is a
vertex of~$F$, represents a pure best reply to~$x$.
The mixed strategy $y$ is the uniform distribution on
the set of pure best replies to~$x$.
\end{theorem}

In the preceding theorem, any simplicial polytope can take
the role of $Q$ as long as it has one completely labeled
facet~$F_0$.
Then an affine transformation, which does not change the
incidences of the facets of $Q$, can be used to map $F_0$ to
the negative unit vectors $-e_1,\ldots,-e_d$ as described,
with $Q$ if necessary expanded in the direction ${\bf 1}$ so that
${\bf 0}$ is in its interior.

A $d\times n$ bimatrix game $(U,B)$ is a {\em unit vector game}
if all columns of $U$ are unit vectors.
For such a game $B$ with $B=[b_1\cdots b_n]$, the columns
$b_j$ for $j\in[n]$ can be obtained from $c_j$ as in
Theorem~\ref{t-unitv} if $b_j>{\bf 0}$ and ${\bf 1}^T b_j<1$.
This is always possible via a positive-affine transformation
of the payoffs in~$B$, which does not change the game.
The unit vectors $e_{l(j)}$ that constitute the columns of
$U$ define the labels of the vertices $c_j$.
The corresponding polytope with these vertices is simplicial
if the game $(U,B)$ is nondegenerate \cite{vS02}, which here
means that no mixed strategy $x$ of the row player has more
than $|\{i\in[d]\mid x_i>0\}|$ pure best replies.
Any game can be made nondegenerate by a suitable
``lexicographic'' perturbation of $B$, which can be
implemented symbolically.

Unit vector games encode arbitrary bimatrix games:
An $m\times n$ bimatrix game $(A,B)$ with (w.l.o.g.{})
positive payoff matrices $A,B$ can be symmetrized so
that its Nash equilibria are in one-to-correspondence to the
symmetric equilibria of the $(m+n)\times(m+n)$
symmetric game $(C^T,C)$ where
\[
\label{symmetrize}
C=\biggl(\begin{matrix} 0 & B\cr A^\top & 0\cr\end{matrix}\biggr).
\]
In turn, as shown by McLennan and Tourky \cite{mt},
the symmetric equilibria $(x,x)$ of any symmetric game
$(C^T,C)$ are in one-to-one correspondence to the Nash
equilibria $(x,y)$ of the ``imitation game'' $(I,C)$ where
$I$ is the identity matrix; the mixed strategy $y$ of the
second player is simply the uniform distribution on the
set $\{i\mid x_i>0\}$.
Clearly, $I$ is a matrix of unit vectors, so $(I,C)$ is a
special unit vector game.
