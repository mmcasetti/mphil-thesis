\documentclass[11pt, draft]{article}

%% TYPESETTING

% draft:
\linespread{1.3}
\usepackage{todonotes}

% final (?) TODO: check with LSE requirements
% page measurements
% \setlength{\hoffset}{0mm}
% \setlength{\oddsidemargin}{25mm}
% \setlength{\textwidth}{130mm}

%% PACKAGES

\usepackage{amssymb}
\usepackage{amsmath}
\usepackage{amsthm}

\usepackage[all]{xy}

\usepackage[ruled,vlined,linesnumbered]{algorithm2e}

%% THEOREMS and ENVIRONMENTS

% theorems

\newtheorem{theorem}{Theorem}
\newtheorem{property}{Property}[section]
\theoremstyle{definition}\newtheorem{definition}{Definition}
\theoremstyle{remark}\newtheorem{example}{Example}[section]

% environments

% computational problem
% TODO check p{textwidth} in final version

\newenvironment{problem}[3]
{
\vspace{2.5ex}
\noindent
\begin{tabular}{p{13mm} @{\textbf{:} } p{105mm}}
\hline
\multicolumn{2}{l}{\noindent {\sc #1}} \\
\hline
\textbf{input} & #2 \\
\textbf{output} & #3 \\
\hline
\end{tabular}
}
{
\vspace{1.5ex}
}

% use:
% \begin{problem}
% {name of problem}
% {input of problem}
% {output of problem}
% \end{problem}


%% SHORTCUTS

% definitions

\def\reals{{\mathbb R}}
\def\naturals{{\mathbb N}}
\def\conv{{\rm conv}}
\def\0{{\bf0}}
\def\1{{\bf1}}
\def\T{^{\top}}
\def\rone{{\1\T}}

% to choose the name of the problem - GALE or COMPLETELY LABELED GALE STRING

\def\gale{{\sc{Gale}}}
\def\anothergale{{\sc{Another Gale}}}


%%% END PREAMBLE

\begin{document}

\subsection{Bimatrix Games and Best Response Polytopes}

\todo[inline]{What do I give for granted? subsection 1 on games, subsection 2
on games and polytopes?}
\todo[inline]{read: unit games article on vS website}


A ($d$-dimensional) {\em simplicial polytope} $P$ is the convex hull of a set of at least $d+1$ points $v$ in $\reals^d$ in general position, that is, no $d+1$ of them are on a common hyperplane.

If a point $v$ cannot be omitted from these points without changing $P$ then $v$ is called a {\em vertex} of $P$. A {\em facet} of $P$ is the convex hull $\conv\,F$ of a set $F$ of $d$ vertices of $P$ that lie on a hyperplane $\{ x\in \reals^d\mid a^T x=a_0\}$ so that $a^T u<a_0$ for all other vertices $u$ of $P$; the vector $a$ (unique up to a scalar multiple) is called the {\em normal vector} of the facet. We often identify the facet with its set of vertices~$F$.


The following theorem, due to Balthasar and von Stengel
\cite{B09,BvS10}, establishes a connection between general
labeled polytopes and equilibria of certain $d\times n$
bimatrix games $(U,B)$.

THIS FROM VvS

\begin{theorem}\label{t-unitv}
Consider a labeled  $d$-dimensional simplicial polytope $Q$ with $\0$ in
its interior, with vertices $-e_1,\ldots,-e_d,c_1,\ldots,c_n$,
% We need to assume that the c_i are pairwise distinct, otherwise
% a vertex can have several labels.
% We need to assume that c_i vertex of Q for the following reason:
% the definition of completely labeled facet is difficult if
% c_i is in a facet but not vertex of the facet.
so that % $F_0$ in {\rm(\ref{F0})}
$F_0=\conv\{-e_1,\ldots,-e_d\}$
is a facet of $Q$.
Let $-e_i$ have label $i$ for $i\in[d]$, and let $c_j$
have label $l(j)\in[d]$ for $j\in[n]$.
Let $(U,B)$ be the $d\times n$ bimatrix game with
$U=[e_{l(1)}\cdots\,e_{l(n)}]$ and $B=[b_1\,\cdots\,b_n]$,
where $b_j=c_j/(1+\rone c_j)$ for $j\in[n]$.
Then the completely labeled facets $F$ of $Q$, with the
exception of~$F_0$, are in one-to-one correspondence to the
Nash equilibria $(x,y)$ of the game $(U,B)$ as follows:
if $v$ is the normal vector of $F$, then
$x=(v+\1)/\rone (v+\1)$,
and $x_i=0$ if and only if $-e_i\in F$ for $i\in[d]$;
any other label~$j$ of $F$, so that $c_j$ is a
vertex of~$F$, represents a pure best reply to~$x$.
The mixed strategy $y$ is the uniform distribution on
the set of pure best replies to~$x$.
\end{theorem}

In the preceding theorem, any simplicial polytope can take
the role of $Q$ as long as it has one completely labeled
facet~$F_0$.
Then an affine transformation, which does not change the
incidences of the facets of $Q$, can be used to map $F_0$ to
the negative unit vectors $-e_1,\ldots,-e_d$ as described,
with $Q$ if necessary expanded in the direction $\1$ so that
$\0$ is in its interior.

A $d\times n$ bimatrix game $(U,B)$ is a {\em unit vector game}
if all columns of $U$ are unit vectors.
For such a game $B$ with $B=[b_1\cdots b_n]$, the columns
$b_j$ for $j\in[n]$ can be obtained from $c_j$ as in
Theorem~\ref{t-unitv} if $b_j>\0$ and $\1^T b_j<1$.
This is always possible via a positive-affine transformation
of the payoffs in~$B$, which does not change the game.
The unit vectors $e_{l(j)}$ that constitute the columns of
$U$ define the labels of the vertices $c_j$.
The corresponding polytope with these vertices is simplicial
if the game $(U,B)$ is nondegenerate \cite{vS02}, which here
means that no mixed strategy $x$ of the row player has more
than $|\{i\in[d]\mid x_i>0\}|$ pure best replies.
Any game can be made nondegenerate by a suitable
``lexicographic'' perturbation of $B$, which can be
implemented symbolically.

Unit vector games encode arbitrary bimatrix games:
An $m\times n$ bimatrix game $(A,B)$ with (w.l.o.g.{})
positive payoff matrices $A,B$ can be symmetrized so
that its Nash equilibria are in one-to-correspondence to the
symmetric equilibria of the $(m+n)\times(m+n)$
symmetric game $(C^T,C)$ where
\[
\label{symmetrize}
C=\biggl(\begin{matrix} 0 & B\cr A^\top & 0\cr\end{matrix}\biggr).
\]
In turn, as shown by McLennan and Tourky \cite{mt},
the symmetric equilibria $(x,x)$ of any symmetric game
$(C^T,C)$ are in one-to-one correspondence to the Nash
equilibria $(x,y)$ of the ``imitation game'' $(I,C)$ where
$I$ is the identity matrix; the mixed strategy $y$ of the
second player is simply the uniform distribution on the
set $\{i\mid x_i>0\}$.
Clearly, $I$ is a matrix of unit vectors, so $(I,C)$ is a
special unit vector game.

\end{document}
