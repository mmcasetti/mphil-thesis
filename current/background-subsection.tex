\documentclass[11pt, draft]{article}

%% TYPESETTING

% draft:
\linespread{1.3}
\usepackage{todonotes}

% final (?) TODO: check with LSE requirements
% page measurements
% \setlength{\hoffset}{0mm}
% \setlength{\oddsidemargin}{25mm}
% \setlength{\textwidth}{130mm}

%% PACKAGES

\usepackage{amssymb}
\usepackage{amsmath}
\usepackage{amsthm}

\usepackage[all]{xy}

\usepackage[ruled,vlined,linesnumbered]{algorithm2e}

%% THEOREMS and ENVIRONMENTS

% theorems

\newtheorem{theorem}{Theorem}
\newtheorem{property}{Property}[section]
% \theoremstyle{definition}\newtheorem{definition}{Definition}
\theoremstyle{remark}\newtheorem{example}{Example}[section]

% environments

% computational problems
% TODO check p{textwidth} in final version

% decision problem

\newenvironment{decproblem}[3]
{
\vspace{2.5ex}
\noindent
\begin{tabular}{p{18mm} @{\textbf{:} } p{100mm}}
\hline
\multicolumn{2}{l}{\noindent {\sc #1}} \\
\hline
\textbf{input} & #2 \\
\textbf{question} & #3 \\
\hline
\end{tabular}
}
{
\vspace{1.5ex}
}

% function problem

\newenvironment{fctproblem}[3]
{
\vspace{2.5ex}
\noindent
\begin{tabular}{p{15mm} @{\textbf{:} } p{102mm}}
\hline
\multicolumn{2}{l}{\noindent {\sc #1}} \\
\hline
\textbf{input} & #2 \\
\textbf{output} & #3 \\
\hline
\end{tabular}
}
{
\vspace{1.5ex}
}

% use:
% \begin{problem}
% {name of problem}
% {input of problem}
% {output of problem}
% \end{problem}


%% SHORTCUTS

% definitions

\def\reals{{\mathbb R}}
\def\naturals{{\mathbb N}}
\def\conv{{\rm conv}}
\def\0{{\bf0}}
\def\1{{\bf1}}
\def\T{^{\top}}
\def\rone{{\1\T}}

% to choose the name of the problem - GALE or COMPLETELY LABELED GALE STRING

\def\gale{{\sc{Gale}}}
\def\anothergale{{\sc{Another Gale}}}


%%% END PREAMBLE

\begin{document}

\section{Complexity, Games, Polytopes and Gale Strings}

\subsection{Some Complexity Classes}

A {\em computational problem} is given by the combination of an {\em input}
and a related {\em output}. A specific input gives an {\em instance} of the
problem.

Computational problems can be classified
according to the form of their output: for instance, the output of
{\em decision problems} is either ``YES'' or ``NO''. A
{\em function problem}, on the other hand, has a more generic output $y$
related to the input $x$ by a $R(x,y)$ given by the problem.

An example of decision problem could be ``(input) given a graph, (question)
is it possible to find an {\em Euler tour}, that is, a path through the
input graph that starts and end at the same vertex that traverses each
edge exactly once?'' On the other hand, a function problem could be
``(input) given a graph, (output) return an Euler tour.''

\todo[inline]{check: search problems - def}

{\em Search problems} return either an output $y$ satisfying a given relation
$R(x,y)$, where $x$ is the input of the problem, or ``NO'', if it's not
possible to find any such $y$. If $y$ is guaranteed to exist, the problem
is called a {\em total function problem}. {\em Counting problems}, finally,
return the {\em number} of $y$'s that satisfy $R(x,y)$; given a problem $R$
we denote the associated counting problem $\# R$.

Computational problems are also classified according to their
{\em computational complexity}, given by the {\em reducibility} from each
other.

\todo[inline]{Turing machines: here}

Let $P_1$ be a computational problem. For an instance $x$ of $P_1$, let
$|x|$ be the the number of bits needed to encode $x$. $P_1$ {\em reduces to
the problem $P_2$ in polynomial time}, denoted $P_1\leq_P P_2$, if there
exists a {\em polynomial-time reduction}, that is, a function
$f: \{0,1\}^\ast \to \{0,1\}^\ast$ and a {\em Turing machine} $\mathcal{M}$
such that for all $x\in\{0,1\}^\ast$

\begin{itemize}
\item $x\in P_1\quad\iff\quad f(x)\in P_2$
\item $\mathcal{M}$ computes $f(x)$
\item $\mathcal{M}$ stops after $p(|x|)$ steps, where $p$ is a polynomial
\end{itemize}

The complexity class $\mathrm{P}$ contains all the polynomially decidable
problems, that is, all problems $P$ such that there exists a Turing machine
$\mathcal{M}$ that outputs either ``YES'' or ``NO''  for all inputs
$x\in\{0,1\}^\ast$ of $P$ after $p(|x|)$ steps, where $p$ is a polynomial.
problems in $P$ are often described as {\em efficient}. The class
$\mathrm{FP}$ of all the function problems that can be solved in polynomial
time is analogously defined.

\todo[inline]{
NP - finding certificates

counting certificates of NP: \# P

FNP / TFNP $\Rightarrow$ PPA(D)
}

\subsection{Normal Form Games and Nash Equilibria}

\todo[inline]{until here}

\subsection{Bimatrix Games and Best Response Polytopes}

\todo[inline]{file: polytopes-subsection}

\subsection{Cyclic Polytopes and Gale Strings}

\subsection{Labeling and the Problem \anothergale}

\todo[inline]{file: gale-def-subsection}

\newpage

\begin{thebibliography}{00}

\frenchspacing\parskip-1ex
\small

\bibitem{main} M. M. Casetti, J. Merschen, B. von Stengel (2010).
Finding Gale Strings.
\emph{Electronic Notes in Discrete Mathematics}
\todo[inline]{issue, pp. n--m.}

\bibitem{cd} X. Chen, X. Deng (2006).
Settling the complexity of two-player Nash equilibrium.
\emph{Proc. 47th FOCS}, pp. 261--272.

\bibitem{dgp} C. Daskalakis, P. W. Goldberg, C. H. Papadimitriou (2006).
The complexity of computing a Nash equilibrium.
\emph{Proc. Ann. 38th STOC}, pp. 71--78.
\todo[inline]{change ref to econometrica(?)}

\bibitem{edm} J. Edmonds (1965).
Paths, trees, and flowers.
\emph{Canad. J. Math.} 17, pp. 449--467.

\bibitem{gale} D. Gale (1963),
Neighborly and cyclic polytopes.
\emph{Convexity, Proc. Symposia in Pure Math.}, Vol. 7, ed. V. Klee, American Math. Soc., Providence, Rhode Island, pp. 225--232.
\todo[inline]{check if right typography}

\bibitem{jm} J. Merschen (2012).
\todo[inline]{thesis}

\bibitem{lh} C. E. Lemke, J. T. Howson, Jr. (1964).
Equilibrium points of bimatrix games.
\emph{J.  Soc. Indust. Appl. Mathematics} 12, pp.  413--423.

\bibitem{ppad} C. H. Papadimitriou (1994).
On the complexity of the parity argument and other inefficient proofs of existence.
\emph{J. Comput. System Sci.} 48, pp. 498--532.

\bibitem{svs} R. Savani, B. von Stengel (2006).
Hard-to-solve bimatrix games.
\emph{Econometrica} 74, pp. 397--429.

\bibitem{vvs} L. V\'{egh}, B. von Stengel
\todo[inline]{ref}


\end{thebibliography}

\end{document}
