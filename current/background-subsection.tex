\documentclass[11pt, draft]{article}

%% TYPESETTING

% draft:
\linespread{1.3}
\usepackage{todonotes}

% final (?) TODO: check with LSE requirements
% page measurements
% \setlength{\hoffset}{0mm}
% \setlength{\oddsidemargin}{25mm}
% \setlength{\textwidth}{130mm}

%% PACKAGES

\usepackage{amssymb}
\usepackage{amsmath}
\usepackage{amsthm}

\usepackage[all]{xy}

\usepackage[ruled,vlined,linesnumbered]{algorithm2e}

%% THEOREMS and ENVIRONMENTS

% theorems

\newtheorem{theorem}{Theorem}
\newtheorem{property}{Property}[section]
\theoremstyle{definition}\newtheorem{definition}{Definition}
\theoremstyle{remark}\newtheorem{example}{Example}[section]

% environments

% computational problems
% TODO check p{textwidth} in final version

% decision problem

\newenvironment{decproblem}[3]
{
\vspace{2.5ex}
\noindent
\begin{tabular}{p{18mm} @{\textbf{:} } p{100mm}}
\hline
\multicolumn{2}{l}{\noindent {\sc #1}} \\
\hline
\textbf{input} & #2 \\
\textbf{question} & #3 \\
\hline
\end{tabular}
}
{
\vspace{1.5ex}
}

% function problem

\newenvironment{fctproblem}[3]
{
\vspace{2.5ex}
\noindent
\begin{tabular}{p{15mm} @{\textbf{:} } p{102mm}}
\hline
\multicolumn{2}{l}{\noindent {\sc #1}} \\
\hline
\textbf{input} & #2 \\
\textbf{output} & #3 \\
\hline
\end{tabular}
}
{
\vspace{1.5ex}
}

% use:
% \begin{problem}
% {name of problem}
% {input of problem}
% {output of problem}
% \end{problem}


%% SHORTCUTS

% definitions

\def\reals{{\mathbb R}}
\def\naturals{{\mathbb N}}
\def\conv{{\rm conv}}
\def\0{{\bf0}}
\def\1{{\bf1}}
\def\T{^{\top}}
\def\rone{{\1\T}}

% to choose the name of the problem - GALE or COMPLETELY LABELED GALE STRING

\def\gale{{\sc{Gale}}}
\def\anothergale{{\sc{Another Gale}}}


%%% END PREAMBLE

\begin{document}

\section{Complexity, Games, Polytopes and Gale Strings}

\subsection{The Complexity Classes P and PPAD}



\subsection{Normal Form Games and Nash Equilibria}

\todo[inline]{until here}

\subsection{Bimatrix Games and Best Response Polytopes}

\todo[inline]{file: polytopes-subsection}

\subsection{Cyclic Polytopes and Gale Strings}

\subsection{Labeling and the Problem \anothergale}

\todo[inline]{file: gale-def-subsection}

\newpage

\begin{thebibliography}{00}

\frenchspacing\parskip-1ex
\small

\bibitem{main} M. M. Casetti, J. Merschen, B. von Stengel (2010).
Finding Gale Strings.
\emph{Electronic Notes in Discrete Mathematics}
\todo[inline]{issue, pp. n--m.}

\bibitem{cd} X. Chen, X. Deng (2006).
Settling the complexity of two-player Nash equilibrium.
\emph{Proc. 47th FOCS}, pp. 261--272.

\bibitem{dgp} C. Daskalakis, P. W. Goldberg, C. H. Papadimitriou (2006).
The complexity of computing a Nash equilibrium.
\emph{Proc. Ann. 38th STOC}, pp. 71--78.
\todo[inline]{change ref to econometrica(?)}

\bibitem{edm} J. Edmonds (1965).
Paths, trees, and flowers.
\emph{Canad. J. Math.} 17, pp. 449--467.

\bibitem{gale} D. Gale (1963),
Neighborly and cyclic polytopes.
\emph{Convexity, Proc. Symposia in Pure Math.}, Vol. 7, ed. V. Klee, American Math. Soc., Providence, Rhode Island, pp. 225--232.
\todo[inline]{check if right typography}

\bibitem{jm} J. Merschen (2012).
\todo[inline]{thesis}

\bibitem{lh} C. E. Lemke, J. T. Howson, Jr. (1964).
Equilibrium points of bimatrix games.
\emph{J.  Soc. Indust. Appl. Mathematics} 12, pp.  413--423.

\bibitem{ppad} C. H. Papadimitriou (1994).
On the complexity of the parity argument and other inefficient proofs of existence.
\emph{J. Comput. System Sci.} 48, pp. 498--532.

\bibitem{svs} R. Savani, B. von Stengel (2006).
Hard-to-solve bimatrix games.
\emph{Econometrica} 74, pp. 397--429.

\bibitem{vvs} L. V\'{egh}, B. von Stengel
\todo[inline]{ref}


\end{thebibliography}

\end{document}
