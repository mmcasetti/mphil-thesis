\documentclass[11pt, draft]{article}

%% TYPESETTING

% draft:
\linespread{1.3}
\usepackage{todonotes}

% final (?) TODO: check with LSE requirements
% page measurements
% \setlength{\hoffset}{0mm}
% \setlength{\oddsidemargin}{25mm}
% \setlength{\textwidth}{130mm}

%% PACKAGES

\usepackage{amssymb}
\usepackage{amsmath}
\usepackage{amsthm}

\usepackage[all]{xy}

\usepackage[ruled,vlined,linesnumbered]{algorithm2e}

%% THEOREMS and ENVIRONMENTS

% theorems

\newtheorem{theorem}{Theorem}
\newtheorem{property}{Property}[section]
\theoremstyle{remark}\newtheorem{example}{Example}[section]

% environments

% computational problems
% TODO check p{textwidth} in final version

% decision problem

\newenvironment{decproblem}[3]
{
\vspace{2.5ex}
\noindent
\begin{tabular}{p{18mm} @{\textbf{:} } p{100mm}}
\hline
\multicolumn{2}{l}{\noindent {\sc #1}} \\
\hline
\textbf{input} & #2 \\
\textbf{question} & #3 \\
\hline
\end{tabular}
}
{
\vspace{1.5ex}
}

% function problem

\newenvironment{fctproblem}[3]
{
\vspace{2.5ex}
\noindent
\begin{tabular}{p{15mm} @{\textbf{:} } p{102mm}}
\hline
\multicolumn{2}{l}{\noindent {\sc #1}} \\
\hline
\textbf{input} & #2 \\
\textbf{output} & #3 \\
\hline
\end{tabular}
}
{
\vspace{1.5ex}
}

% use:
% \begin{problem}
% {name of problem}
% {input of problem}
% {output of problem}
% \end{problem}


%% SHORTCUTS

% definitions

\def\reals{{\mathbb R}}
\def\naturals{{\mathbb N}}
\def\conv{{\rm conv}}
\def\0{{\bf0}}
\def\1{{\bf1}}
\def\T{^{\top}}
\def\rone{{\1\T}}

% to choose the name of the problem - GALE or COMPLETELY LABELED GALE STRING

\def\gale{{\sc{Gale}}}
\def\anothergale{{\sc{Another Gale}}}


%%% END PREAMBLE

\begin{document}

\section{Complexity, Games, Polytopes and Gale Strings}

\subsection{Some Complexity Classes}

\todo[inline]{reference - take Papadimitriou (book) for general, then article
Megiddo and Papadimitriou (1991) for (T)FNP and Papadimitriou 1994 for PPAD}

A {\em computational problem} is given by the combination of an {\em input}
and a related {\em output}. A specific input gives an {\em instance} of the
problem.

\todo[inline]{complement of a problem - needed for co-NP}

Computational problems can be classified according to the form of their
output. A {\em decision problems} outputs either ``YES'' or ``NO''. An
instance $x$ {\em function problem}, on the other hand, returns a more
generic output $y$ that satisfies a given relation $R(x,y)$.

{\em Search problems} are function problems that return either an output
$y$ satisfying a given relation $R(x,y)$ or ``NO'', if it's not possible to
find any such $y$. If $y$ is guaranteed to exist, the problem is called a
{\em total function problem}. {\em Counting problems}, finally, return the
{\em number} of $y$'s that satisfy $R(x,y)$; given a problem $R$ we denote
the associated counting problem $\# R$.

An example of decision problem is: ``(input) given a graph, (question)
is it possible to find an {\em Euler tour} for the graph?'' A search
problem is: ``(input) given a graph, (output) return one
Euler tour of the graph, or ``NO'' if no such tour exists.'' A total
function problem is: ``(input) given an Euler graph, (output) return
one of its Euler tours.'' A counting problem is ``(input) given a graph,
(output) return the number of its Euler tours.''

Computational problems are also classified according to their
{\em computational complexity}, given by the {\em reducibility} from each
other.

\todo[inline]{Turing machines: here - not that in the following deterministic
TM}

Let $P_1$ be a computational problem. For an instance $x$ of $P_1$, let
$|x|$ be the the number of bits needed to encode $x$. $P_1$ {\em reduces to
the problem $P_2$ in polynomial time}, denoted $P_1\leq_P P_2$, if there
exists a {\em polynomial-time reduction}, that is, a function
$f: \{0,1\}^\ast \to \{0,1\}^\ast$ and a Turing machine $\mathcal{M}$
such that for all $x\in\{0,1\}^\ast$

\begin{enumerate}
\item $x\in P_1\quad\iff\quad f(x)\in P_2$;
\item $\mathcal{M}$ computes $f(x)$;
\item $\mathcal{M}$ stops after $p(|x|)$ steps, where $p$ is a polynomial.
\end{enumerate}

The complexity class $\mathrm{\mathbf{P}}$ contains all the
{\em polynomially decidable problems}, that is, all problems $P$ such that
there exists a Turing machine $\mathcal{M}$ that outputs either ``YES'' or
``NO''  for all inputs $x\in\{0,1\}^\ast$ of $P$ after $p(|x|)$ steps, where
$p$ is a polynomial. Intuitively, a decision problem is in
$\mathrm{\mathbf{P}}$ if the answer to its question can be found in a number
of steps that is polynomial in the input of the problem.

A problem $P$ belongs to the class $\mathrm{\mathbf{NP}}$,
{\em non-deterministic polynomial-time problems}, if there exists a
Turing machine $\mathcal{M}$ and polynomials $p_1,p_2$ such that

\begin{enumerate}
\item for all $x\in P$ there exists a {\em certificate} $y\in \{0,1\}^\ast$
which satisfies $|y|\leq p_1(|x|)$;
\item $\mathcal{M}$ accepts the combined input $xy$, stopping after at most
$p_2(|x| + |y|)$ steps;
\item for all $x\notin P$ there does not exist $y\in \{0,1\}^\ast$ such
that $\mathcal{M}$ accepts the combined input $xy$.
\end{enumerate}

Intuitively: a decision problem is in $\mathrm{\mathbf{NP}}$ if it takes
polynomial time to verify whether the ``certificate solution'' $y$ is,
indeed, a correct answer to the question posed by the problem. A problem is
in the class $\mathrm{\mathbf{co-NP}}$ if its em complement is in
$\mathrm{\mathbf{NP}}$.
The class $\mathrm{\mathbf{\# P}}$ captures the problem of counting the
number of possible certificates for a problem in $\mathrm{\mathbf{NP}}$.

\todo[inline]{Formally, $\mathrm{\mathbf{\# P}}$ is defined as...}

In \cite{megiddo-papad}

The class $\mathrm{\mathbf{FNP}}$, {\em function non-deterministic
polynomial}, is defined as the class of binary
relations $R(x,y)$ such that there is a polynomial-time algorithm that
decides whether $R(x,y)$ holds for given $x,y$ satisfying $|y|\leq p(|x|)$,
where $p$ is a polynomial. If a $y$ as above is guaranteed to exist, the
problem belongs to the class $\mathrm{\mathbf{TFNP}}$,
{\em total function non-deterministic polynomial}.
That is, $\mathrm{\mathbf{FNP}}$ and  $\mathrm{\mathbf{TFNP}}$ are analogous
to $\mathrm{\mathbf{NP}}$, but they allow for problems of (respectively)
function and total function form.


\todo[inline]{
Cite: Papadimitriou / Megiddo 1991 - def of (T)FNP

More on TFNP: no complete pbls unless NP=co-NP (def co-NP)

$\Rightarrow$

definition of PPA(D)
}

\subsection{Normal Form Games and Nash Equilibria}

\todo[inline]{until here}

\subsection{Best Response Polytopes}

% and Completely Labeled Vertices

\todo[inline]{file: polytopes-subsection}

\subsection{Cyclic Polytopes and Gale Strings}

\subsection{The Problem \anothergale}

% Completely Labeled Gale Strings and

\todo[inline]{file: gale-def-subsection

merge in one section "Gale strings" or "CP and GS"?}

\newpage


\begin{thebibliography}{00}

\frenchspacing\parskip0.3ex
\small

\bibitem{balthasar} A. V. Balthasar (2009).
``Geometry and equilibria in bimatrix games.''
PhD Thesis, London School of Economics and Political Science.

\bibitem{brightwell} G. R. Brightwell, P. Winkler (2004).
``Note on Counting Eulerian Circuits.''
CDAM Research Report LSE-CDAM-2004-12.

\bibitem{msc-diss} M. M. Casetti (2008).
``PPAD Completeness of Equilibrium Computation.''
MSc Thesis, London School of Economics and Political Science.

\bibitem{main} M. M. Casetti, J. Merschen, B. von Stengel (2010).
``Finding Gale Strings.''
\emph{Electronic Notes in Discrete Mathematics} 36, pp. 1065--1082.

\bibitem{cd} X. Chen, X. Deng (2006).
``Settling the Complexity of 2-Player Nash Equilibrium.''
\emph{Proc. 47th Annual IEEE Symposium on Foundations of
Computer Science (FOCS)}, pp. 261--272.

\bibitem{dgp} C. Daskalakis, P. W. Goldberg, C. H. Papadimitriou (2009).
``The Complexity of Computing a Nash Equilibrium.''
\emph{SIAM Journal on Computing} 39, pp. 195--259.

\bibitem{edm} J. Edmonds (1965).
``Paths, Trees, and Flowers.''
\emph{Canad. J. Math.} 17, pp. 449--467.

\bibitem{edm-oiks} J. Edmonds (2009).
``Euler complexes.''
In: \emph{Research Trends in Combinatorial Optimization}, eds. W.
Cook, L. Lovasz, and J. Vygen, Springer, Berlin, pp. 65--68.

\bibitem{edm-sanita} J. Edmonds, L. Sanit\`a (2010).
``On finding another room-partitioning of the vertices.''
\emph{Electronic Notes in Discrete Mathematics} 36, pp. 1257--1264.

\bibitem{gale} D. Gale (1963).
``Neighborly and Cyclic Polytopes.''
In: \emph{Convexity, Proc. Symposia in Pure Math.}, Vol. 7,
ed. V. Klee, American Math. Soc., Providence, Rhode Island, pp. 225--232.

\bibitem{gale-kuhn-tucker} D. Gale, H. W. Kuhn, A. W. Tucker (1950).
``On Symmetric Games.''
In: \emph{Contributions to the Theory of Games}~I, eds. H. W. Kuhn and A. W.
Tucker, \emph{Annals of Mathematics Studies} 24, Princeton University Press,
Princeton, pp. 81--87.

\bibitem{gilboa-zemel} I. Gilboa, E. Zemel (1989).
``Nash and correlated equilibria: some complexity considerations.''
\emph{Games and Economic Behavior} 1, pp. 80--93.

\bibitem{wicdiv} K. Gillen, J. McKelvie, M. Wilson (2015).
``Fear and Loathing in Eternity.''
{\em The Wicked + The Divine}, issue 9, ed. Image Comics.

\bibitem{gps} P. W. Goldberg, C. H. Papadimitriou, R. Savani (2011).
``The Complexity of the Homotopy Method, Equilibrium Selection, and
Lemke-Howson solutions.''
\emph{Proc. 52nd Annual IEEE Symposium on Foundations of Computer Science (FOCS)}, pp. 67--76.

\bibitem{lh} C. E. Lemke, J. T. Howson, Jr. (1964).
``Equilibrium Points of Bimatrix Games.''
\emph{J.  Soc. Indust. Appl. Mathematics} 12, pp.  413--423.

\bibitem{mclennan-tourky} A. McLennan, R. Tourky (2010).
``Imitation Games and Computation.''
\emph{Games and Economic Behavior} 70, pp. 4--11.

\bibitem{megiddo-papad} N. Megiddo, C. H. Papadimitriou (1991).
``On Total Functions, Existence Theorems and Computational Complexity.''
\emph{Theoretical Computer Science} 81, pp. 317--324.

\bibitem{jm} J. Merschen (2012).
``Nash Equilibria, Gale Strings, and Perfect Matchings.''
PhD Thesis, London School of Economics and Political Science.

\bibitem{morris} W. D. Morris Jr. (1994).
``Lemke Paths on Simple Polytopes.''
\emph{Math. Oper. Res.} 19, pp. 780--789.

\bibitem{nash} J. F. Nash (1951).
``Noncooperative games.''
\emph{Annals of Mathematics}, 54, pp. 289--295.

\bibitem{gth} M. J. Osborne, A. Rubinstein (1994).
{\em A Course in Game Theory.}
The MIT Press, Cambridge, Massachusetts.

\bibitem{papad-cc} C. H. Papadimitriou (1994).
{\em Computational Complexity.}
Addison-Wesley, Reading, MA.

\bibitem{ppad} C. H. Papadimitriou (1994).
``On the Complexity of the Parity Argument and Other Inefficient Proofs of
Existence.''
\emph{J. Comput. System Sci.} 48, pp. 498--532.

\bibitem{svs} R. Savani, B. von Stengel (2006).
``Hard-to-solve Bimatrix Games.''
\emph{Econometrica} 74, pp. 397--429.

\bibitem{uvg} R. Savani, B. von Stengel (2015).
``Unit Vector Games.''
arXiv:1501.02243v1 [cs.GT]

\bibitem{shapley} L. S. Shapley (1974).
``A Note on the Lemke-Howson Algorithm.''
\emph{Mathematical Programming Study 1: Pivoting and Extensions}, pp. 175--189

\bibitem{vvs} L. A. V\'{e}gh, B. von Stengel (2015),
``Oriented Euler Complexes and Signed Perfect Matchings.''
\emph{Mathematical Programming Series B} 150, pp. 153--178.

\bibitem{vn28} J. von Neumann (1928).
``Zur Theorie der Gesellschaftspiele.''
\emph{Mathematische Annalen} 100, pp. 295--320.

\bibitem{vs-agt} B. von Stengel (2007).
``Equilibrium computation for two-player games in strategic and extensive
form.'' Chapter 3, ``Algorithmic Game Theory,''
eds. N. Nisan, T. Roughgarden, E. Tardos, V. Vazirani.
Cambridge Univ. Press, Cambridge, pp. 53--78.

\bibitem{vs-noclf} B. von Stengel (2012).
``Completely Labeled Facet is NP-Complete.''
Manuscript, 6 pp.

\bibitem{ziegler} G. M. Ziegler (1995).
{\em Lectures on Polytopes.}
Springer, New York.

\end{thebibliography}


\end{document}
