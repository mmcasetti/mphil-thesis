\documentclass[11pt, draft]{article}

%% TYPESETTING

% draft:
\linespread{1.3}
\usepackage{todonotes}

% final (?) TODO: check with LSE requirements
% page measurements
% \setlength{\hoffset}{0mm}
% \setlength{\oddsidemargin}{25mm}
% \setlength{\textwidth}{130mm}

%% PACKAGES

\usepackage{amssymb}
\usepackage{amsmath}
\usepackage{amsthm}

\usepackage[all]{xy}

\usepackage[ruled,vlined,linesnumbered]{algorithm2e}

%% THEOREMS and ENVIRONMENTS

% theorems

\newtheorem{theorem}{Theorem}
\newtheorem{property}{Property}[section]
\theoremstyle{definition}\newtheorem{definition}{Definition}
\theoremstyle{remark}\newtheorem{example}{Example}[section]

% environments

% computational problems
% TODO check p{textwidth} in final version

% decision problem

\newenvironment{decproblem}[3]
{
\vspace{2.5ex}
\noindent
\begin{tabular}{p{18mm} @{\textbf{:} } p{100mm}}
\hline
\multicolumn{2}{l}{\noindent {\sc #1}} \\
\hline
\textbf{input} & #2 \\
\textbf{question} & #3 \\
\hline
\end{tabular}
}
{
\vspace{1.5ex}
}

% function problem

\newenvironment{fctproblem}[3]
{
\vspace{2.5ex}
\noindent
\begin{tabular}{p{15mm} @{\textbf{:} } p{102mm}}
\hline
\multicolumn{2}{l}{\noindent {\sc #1}} \\
\hline
\textbf{input} & #2 \\
\textbf{output} & #3 \\
\hline
\end{tabular}
}
{
\vspace{1.5ex}
}

% use:
% \begin{problem}
% {name of problem}
% {input of problem}
% {output of problem}
% \end{problem}


%% SHORTCUTS

% definitions

\def\reals{{\mathbb R}}
\def\naturals{{\mathbb N}}
\def\conv{{\rm conv}}
\def\0{{\bf0}}
\def\1{{\bf1}}
\def\T{^{\top}}
\def\rone{{\1\T}}

% to choose the name of the problem - GALE or COMPLETELY LABELED GALE STRING

\def\gale{{\sc{Gale}}}
\def\anothergale{{\sc{Another Gale}}}


%%% END PREAMBLE

\begin{document}

\section{Introduction}

\section{Complexity, Games, Polytopes and Gale Strings}

\subsection{The Complexity Classes P and PPAD}

\subsection{Normal Form Games and Nash Equilibria}

\subsection{Bimatrix Games and Best Response Polytopes}

\subsection{Cyclic Polytopes and Gale Strings}\label{gs-ssect}

A special case of games is obtained by taking a particular case of best
response polytope in theorem \ref{t-unitv}.

\begin{definition}\label{cyclic-polytope}
A {\em cyclic polytope} $P$ in dimension~$d$ with $n$ vertices is the
convex hull of distinct points $\mu(t_j)$, where $j\in [n]$ and $\mu$ is the
{\em moment curve}
\[
\mu\colon t\mapsto(t,t^2,\ldots,t^d)^\top
\]
\end{definition}

Cyclic polytopes can be represented through a combinatorial structure, the
{\em Gale strings}. This makes their study particularly interesting, and, as
we will see, it can be used to obtain very elegant proofs.

\begin{definition}\label{bitstring-def}
For any integer $k$ and any set $S$, we can represent the function
$f_s:[k]\to S$ as the string $s = s(1)s(2)\cdots s(k)$. If $S=\{0,1\}$ we
denote
\begin{align*}
\mathnormal{1}(s) & = s^{-1}(1) \\
     & = \{j\in [k]\mid s(j)=1\}
\end{align*}

The indicator function of $\mathnormal{1}(s)$ will then correspond to a
{\em bitstring} $s$, a sequence of $0$'s and $1$'s.

A maximal substring of consecutive $1$'s in a bitstring is called a
{\em run}.
\end{definition}

\begin{example}
Let $k = 6$, and let $f_s(j)=0$ if $j$ is even and $f_s(j)=1$ if $j$ is odd.
Then $s = 101010$ and $1(s) = {1,3,5}$.
\end{example}

We can now give the definition of {\em Gale string}.

\begin{definition}\label{gs-def}
We denote as $G(d,n)$ the set of all bitstrings $s$ of length $n$ such that

\begin{enumerate}
\item exactly $d$ bits in $s$ are $1$ and
\item $s$ fulfills the {\em Gale evenness condition}:
\[
01^k0\hbox{ is a substring of }s\quad{\Rightarrow}\quad k\hbox{ is even.}
\]
\end{enumerate}

An element of of $G(d,n)$ is called a {\em Gale string of dimension~$d$ and
length~$n$}.
\end{definition}

Definition \ref{gs-def} characterises Gale strings as bitstrings of length~$n$ with exactly~$d$ elements equal to $1$, such that {\em interior}
runs (that is, runs bounded on both sides by~$0$s) must be of even length.
Note that this condition allows Gale strings to start or end with an
odd-length run.

This leads to an important consequence when~$d$ is even.

\begin{property}\label{even-d-gs}
Let $d$ be even, and let $s$ in $G(d,n)$. Then if $s$ starts with an odd run
it will also end with an odd run, and if $s$ starts with an even run it will
end with an even run.

\todo[inline]{use "modulo" - even more than "cyclic shift"}
That is, the set of Gale strings of even dimension is therefore invariant
under a cyclic shift of the strings.
\end{property}

We can then consider the Gale strings in $G(d,n)$ with even $d$ as a ``loop''
obtained by ``glueing together'' the extremes of the string to form an even
run.

\begin{example}\label{gs-example}
We consider $G(4,6)$. We have
\begin{align*}
G(4,6) = \{ & 111100, \\
        & 111001, \\
        & 110011, \\
        & 100111, \\
        & 001111, \\
        & 011110, \\
        & 110110, \\
        & 101101, \\
        & 011011\}
\end{align*}

\todo[inline]{change mention of "cyclic shift" if not used before}
The strings $111100$, $111001$, $110011$, $100111$, $001111$ and $011110$ are
equivalent under a cyclic shift, as are the strings $110110$, $101101$ and
$011011$.
\end{example}

\todo[inline]{here to end subsect: polytopes}

The relation between cyclic polytopes and Gale strings is given by the
following theorem by Gale \cite{gale-cyclicpoly}.

\begin{theorem}[\cite{gale-cyclicpoly}]\label{cp-gs-gale}
For any positive integer $n$, assume that $t_1 < t_2 < \cdots < t_n$ and let
P be the cyclic polytope obtained by taking $t_j$, where $j \in [n]$, in
definition \ref{cyclic-polytope}.

Then the facets of $P$ are encoded by $G(d,n)$; that is, $F$ is a facet of
$P$ if and only if
\[
F = \conv\{\mu(t_i)\mid i\in 1(s)\} \qquad \hbox{ for some }s\in G(d,n)
\]
\end{theorem}

\todo[inline]{sketch of pf if not too long and it uses relevant techniques}

\todo[inline]{graphics of cyclic polytope - cfr vS articles and talks}

From this point forward, we will assume that $d$ is even.

\todo[inline]{give something to generalise to odd case}

\subsection{Labeling and the Problem \anothergale}

\begin{definition}
Given a set $G$ of bitstrings of length $n$ and a parameter $d$, a
{\em labeling} is a function $l:[n]\to[d]$. A string $s$ in $G(d,n)$ is
{\em completely labeled} if $l(\mathnormal{1}(s))=[d]$. Any $l(i)\in [d]$
is called a {\em label}
\end{definition}

If $s \in G(d,n)$ is completely labeled for the labeling $l:[n]\to[d]$, then
for each label $l(i)$ there is a bit $s(i)=1$. We therefore have exactly
$d$ positions $i$ for which $s(i)=1$; hence, $|l(\mathnormal{1}(s))|=d$.

\begin{example}
Given the string of labels $l=123432$, there are four associated completely
labeled Gale strings: $111100$, $110110$, $100111$ and $101101$.\\

\begin{center}
{\renewcommand{\tabcolsep}{2ex}
\begin{tabular}{|c|c|c|c|}
\hline
\textbf{1234}32 &
\textbf{12}3\textbf{43}2 &
\textbf{1}23\textbf{432} &
\textbf{1}2\textbf{34}3\textbf{2} \\
\textbf{1111}00 &
\textbf{11}0\textbf{11}0 &
\textbf{1}00\textbf{111} &
\textbf{1}0\textbf{11}0\textbf{1} \\
\hline
\end{tabular}
}\\
\end{center}

\end{example}

Sometimes there aren't any completely labeled Gale strings that are
associated with a given labeling.

\begin{example}
For $l = 121314$, there are no completely labeled Gale strings.
\end{example}

\todo[inline]{here to end subsect: polytopes}
\todo[inline]{graphics of labeled cyclic polytope}

% Essentially, this holds because any set $S\subset [n]$
% the moment curve defines a unique hyperplane which is crossed
% (and not just touched) by the moment curve; if the bitstring
% $s$ that encodes $F$ as $1(s)$ has a substring $01^k0$
For this cyclic polytope $P$, a labeling $l:[n]\to[d]$ can
be understood as a label $l(j)$ for each vertex $\mu(t_j)$
for $j\in [n]$.
A completely labeled Gale string $s$ therefore represents a
facet $F$ of $P$ that is completely labeled.

Special games are obtained by using cyclic polytopes in
Theorem~\ref{t-unitv}, suitably affinely transformed with
a completely labeled facet $F_0$.
When $Q$ is a cyclic polytope in dimension $d$ with $d+n$
vertices, then the string of labels $l(1)\cdots l(n)$ in
Theorem~\ref{t-unitv} defines a labeling $l':[d+n]\to [d]$
where $l'(i)=i$ for $i\in [d]$ and
$l'(d+j)=l(j)$ for $j\in [n]$.
In other words, the string of labels $l(1)\cdots l(n)$ is
just prefixed with the string $1\,2\cdots d$ to give $l'$.
Then $l'$ has a trivial completely labeled Gale string
$1^d0^n$ which defines the facet $F_0$.
Then the problem \anothergale\ defines exactly the problem of finding a Nash
equilibrium of the unit vector game $(I,B)$.
Note again that $B$ is here not a general matrix (which would
define a general game) but obtained from the last $n$ of
$d+n$ vertices of a cyclic polytope in dimension~$d$.

\begin{fctproblem}
{\anothergale}
{A labeling $l:[n]\to[d]$, where $d$ is even and $d<n$, and an associated
completely labeled Gale string $s$ in $G(d,n)$.}
{A completely labeled Gale string $s'$ in $G(d,n)$ associated with $l$, such
that $s' \neq s$.}
\end{fctproblem}

\newpage


\begin{thebibliography}{00}

\frenchspacing\parskip0.3ex
\small

\bibitem{balthasar} A. V. Balthasar (2009).
``Geometry and equilibria in bimatrix games.''
PhD Thesis, London School of Economics and Political Science.

\bibitem{brightwell} G. R. Brightwell, P. Winkler (2004).
``Note on Counting Eulerian Circuits.''
CDAM Research Report LSE-CDAM-2004-12.

\bibitem{msc-diss} M. M. Casetti (2008).
``PPAD Completeness of Equilibrium Computation.''
MSc Thesis, London School of Economics and Political Science.

\bibitem{main} M. M. Casetti, J. Merschen, B. von Stengel (2010).
``Finding Gale Strings.''
\emph{Electronic Notes in Discrete Mathematics} 36, pp. 1065--1082.

\bibitem{cd} X. Chen, X. Deng (2006).
``Settling the Complexity of 2-Player Nash Equilibrium.''
\emph{Proc. 47th Annual IEEE Symposium on Foundations of
Computer Science (FOCS)}, pp. 261--272.

\bibitem{dgp} C. Daskalakis, P. W. Goldberg, C. H. Papadimitriou (2009).
``The Complexity of Computing a Nash Equilibrium.''
\emph{SIAM Journal on Computing} 39, pp. 195--259.

\bibitem{edm} J. Edmonds (1965).
``Paths, Trees, and Flowers.''
\emph{Canad. J. Math.} 17, pp. 449--467.

\bibitem{edm-oiks} J. Edmonds (2009).
``Euler complexes.''
In: \emph{Research Trends in Combinatorial Optimization}, eds. W.
Cook, L. Lovasz, and J. Vygen, Springer, Berlin, pp. 65--68.

\bibitem{edm-sanita} J. Edmonds, L. Sanit\`a (2010).
``On finding another room-partitioning of the vertices.''
\emph{Electronic Notes in Discrete Mathematics} 36, pp. 1257--1264.

\bibitem{gale} D. Gale (1963).
``Neighborly and Cyclic Polytopes.''
In: \emph{Convexity, Proc. Symposia in Pure Math.}, Vol. 7,
ed. V. Klee, American Math. Soc., Providence, Rhode Island, pp. 225--232.

\bibitem{gale-kuhn-tucker} D. Gale, H. W. Kuhn, A. W. Tucker (1950).
``On Symmetric Games.''
In: \emph{Contributions to the Theory of Games}~I, eds. H. W. Kuhn and A. W.
Tucker, \emph{Annals of Mathematics Studies} 24, Princeton University Press,
Princeton, pp. 81--87.

\bibitem{gilboa-zemel} I. Gilboa, E. Zemel (1989).
``Nash and correlated equilibria: some complexity considerations.''
\emph{Games and Economic Behavior} 1, pp. 80--93.

\bibitem{wicdiv} K. Gillen, J. McKelvie, M. Wilson (2015).
``Fear and Loathing in Eternity.''
{\em The Wicked + The Divine}, issue 9, ed. Image Comics.

\bibitem{gps} P. W. Goldberg, C. H. Papadimitriou, R. Savani (2011).
``The Complexity of the Homotopy Method, Equilibrium Selection, and
Lemke-Howson solutions.''
\emph{Proc. 52nd Annual IEEE Symposium on Foundations of Computer Science (FOCS)}, pp. 67--76.

\bibitem{lh} C. E. Lemke, J. T. Howson, Jr. (1964).
``Equilibrium Points of Bimatrix Games.''
\emph{J.  Soc. Indust. Appl. Mathematics} 12, pp.  413--423.

\bibitem{mclennan-tourky} A. McLennan, R. Tourky (2010).
``Imitation Games and Computation.''
\emph{Games and Economic Behavior} 70, pp. 4--11.

\bibitem{megiddo-papad} N. Megiddo, C. H. Papadimitriou (1991).
``On Total Functions, Existence Theorems and Computational Complexity.''
\emph{Theoretical Computer Science} 81, pp. 317--324.

\bibitem{jm} J. Merschen (2012).
``Nash Equilibria, Gale Strings, and Perfect Matchings.''
PhD Thesis, London School of Economics and Political Science.

\bibitem{morris} W. D. Morris Jr. (1994).
``Lemke Paths on Simple Polytopes.''
\emph{Math. Oper. Res.} 19, pp. 780--789.

\bibitem{nash} J. F. Nash (1951).
``Noncooperative games.''
\emph{Annals of Mathematics}, 54, pp. 289--295.

\bibitem{gth} M. J. Osborne, A. Rubinstein (1994).
{\em A Course in Game Theory.}
The MIT Press, Cambridge, Massachusetts.

\bibitem{papad-cc} C. H. Papadimitriou (1994).
{\em Computational Complexity.}
Addison-Wesley, Reading, MA.

\bibitem{ppad} C. H. Papadimitriou (1994).
``On the Complexity of the Parity Argument and Other Inefficient Proofs of
Existence.''
\emph{J. Comput. System Sci.} 48, pp. 498--532.

\bibitem{svs} R. Savani, B. von Stengel (2006).
``Hard-to-solve Bimatrix Games.''
\emph{Econometrica} 74, pp. 397--429.

\bibitem{uvg} R. Savani, B. von Stengel (2015).
``Unit Vector Games.''
arXiv:1501.02243v1 [cs.GT]

\bibitem{shapley} L. S. Shapley (1974).
``A Note on the Lemke-Howson Algorithm.''
\emph{Mathematical Programming Study 1: Pivoting and Extensions}, pp. 175--189

\bibitem{vvs} L. A. V\'{e}gh, B. von Stengel (2015),
``Oriented Euler Complexes and Signed Perfect Matchings.''
\emph{Mathematical Programming Series B} 150, pp. 153--178.

\bibitem{vn28} J. von Neumann (1928).
``Zur Theorie der Gesellschaftspiele.''
\emph{Mathematische Annalen} 100, pp. 295--320.

\bibitem{vs-agt} B. von Stengel (2007).
``Equilibrium computation for two-player games in strategic and extensive
form.'' Chapter 3, ``Algorithmic Game Theory,''
eds. N. Nisan, T. Roughgarden, E. Tardos, V. Vazirani.
Cambridge Univ. Press, Cambridge, pp. 53--78.

\bibitem{vs-noclf} B. von Stengel (2012).
``Completely Labeled Facet is NP-Complete.''
Manuscript, 6 pp.

\bibitem{ziegler} G. M. Ziegler (1995).
{\em Lectures on Polytopes.}
Springer, New York.

\end{thebibliography}


\end{document}
