In the remainder of this thesis we will focus bimatrix games.
We will identify the game with its payoff matrices $(A,B)$, and
we will assume that both $A$ and $B$ are non-negative, and that neither $A$
nor $B\T$ has a zero column; this can be done without loss of generality,
applying an affine transformation to $A$ and $B$ if necessary.

We will start by showing a construction to translate a bimatrix game to
two simplices subdivided in regions which are labeled so that the Nash
equilibria of the game correspond to the ``completely labeled'' points of
the simplices. This idea is due to Lemke and Howson
\cite{lh}; we will follow the very clear approach given by Shapley
\cite{shapley}. From labeled regions we will then move on to an equivalent
formulation in terms of polytopes; a result by McLennan and Tourky
\cite{mclennan-tourky} on a special case of games, called imitation games,
and its extension to the more general unit vector games, due to
Balthasar \cite{balthasar}, will allow us to simplify the labeling
construction on polytopes; a result by Balthasar \cite{balthasar} gives
a condition on facets of a simplicial polytope, whereas Savani and von
Stengel \cite{uvg} give an equivalent condition on vertices of a
simple polytope. The problem 2-{\sc Nash} will then be shown to have a
polynomial-time reduction to a problem on facets or vertices of a
labeled polytope, under certain simple conditions.

In the second section of the chapter we will restrict our scope to
Gale games, a case of unit vector games for which the facets of the
polytopes in the previous section can be represented as an elegant
combinatorial structure, called Gale strings. We will show this
representation, then we will define a labeling for Gale strings that
will lead to a reduction from {\sc 2-Nash} to a problem on Gale strings.
