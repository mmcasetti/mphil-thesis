% use present not future tense - more direct
In the remainder of this thesis we focus bimatrix games.
We identify the game with its payoff matrices $(A,B)$, and
we assume that both $A$ and $B$ are non-negative, and that neither $A$
nor $B\T$ has a zero column; this can be done without loss of generality,
by adding a constant to all entries in $A$ and $B$ if necessary.

We start by showing a construction to represent a bimatrix game to
two simplices subdivided in labeled regions such that the Nash
equilibria of the game correspond to ``completely labeled'' points of
the simplices. This idea is due to Lemke and Howson
\cite{lh}. We follow the very clear approach given by Shapley
\cite{shapley}. From labeled regions we then move on to an equivalent
formulation in terms of polytopes; a result by McLennan and Tourky
\cite{mclennan-tourky} on a special case of games, called imitation games,
and its extension to the more general unit vector games, due to
Balthasar \cite{balthasar} and Savani and von
Stengel \cite{uvg}, allows us to simplify this
construction. The problem 2-{\sc Nash} can then be reduced to a problem
on the facets or vertices of a labeled polytope, under some simple
conditions.

In the second section of the chapter we restrict our scope to
Gale games, a case of unit vector games for which the
polytopes in the previous section can be
represented by combinatorial structures called Gale strings.
We first show the
correspondence between polytopes and Gale strings, then we define
a labeling for the latter that
leads to a reduction from 2-{\sc Nash} to a problem on Gale strings.
