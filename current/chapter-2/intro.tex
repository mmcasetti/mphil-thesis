In the remainder of this thesis we focus on bimatrix games.
We identify the game with its payoff matrices $(A,B)$, and
we assume that both $A$ and $B$ are non-negative, and that neither $A$
nor $B\T$ has a zero column; this can be done without loss of generality,
by adding a constant to all entries in $A$ and $B$ if necessary.

We start by showing a construction to represent a bimatrix game as
two simplices subdivided in labeled regions such that the Nash
equilibria of the game correspond to ``completely labeled'' points of
the simplices. This idea is due to Lemke and Howson
\cite{lh}.
As in Balthasar \cite{balthasar} and Savani and von
Stengel \cite{uvg}, that constitute the main source of material for this
chapter as a whole, we follow the very clear approach given by Shapley
\cite{shapley}.

From labeled regions we then move on to an equivalent
formulation in terms of ``best response'' polytopes; a result by McLennan
and Tourky
\cite{mclennan-tourky} on a special case of games, called imitation games,
and its extension to the more general unit vector games, due to
Balthasar \cite{balthasar} and Savani and von~Stengel \cite{uvg},
allows us to simplify this construction. The problem 2-{\sc Nash}
can then be reduced to a problem
on the facets or vertices of a labeled polytope, under some simple
conditions.

We further restrict our scope to Gale games, that is,
unit vector games for which the
polytopes in the previous section can be
represented by a combinatorial structure called Gale strings.
In the second section, we give the basic definitions and we show how
Gale strings correspond to a special case of polytopes.
The third section, finally, will be dedicated to show that the problem
of finding a Nash equilibrium for Gale games reduces
to a problem on Gale strings.
