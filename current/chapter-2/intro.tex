In the rest of this thesis we will focus bimatrix games, analizing their
Nash equilibria with the use of labels and combinatorial structures.

We will start by showing a construction to translate a bimatrix game to
labeled regions of two simplices so that its Nash equilibria correspond
to ``completely labeled'' points; this idea is due to Lemke and Howson
\cite{lh}, and we will follow the very clear approach given by Shapley
\cite{shapley}. From labeled regions we will then move on to an equivalent
formulation on polytopes; a result by McLennan and Tourky
\cite{mclennan-tourky} on a special case of games, called imitation games,
and its extension to the more general unit vector games, due to
Balthasar \cite{balthasar}, will allow us to simplify the labeling
construction on polytopes; a result by Balthasar \cite{balthasar} gives
a condition on facets of a simplicial polytope, whereas Savani and von
Stengel \cite{uvg} give an equivalent condition on vertices of a
simple polytope. The problem 2-{\sc Nash} will then be shown to have a
polynomial-time reduction to a problem on facets, or vertices, of a
labeled polytope, under certain simple conditions.

In the second section of the chapter we will restrict our scope to
Gale games, a case of unit vector games for which the labeling problem
can be translated to an elegant combinatorial structure, called Gale
strings. These are bitstrings satisfying some simple conditions, and
they can be used to represent the facets of a particular kind of
polytopes, called cyclic polytopes. We will show this representation, then
we will define a labeling for Gale strings that will lead to a reduction
from 2-{\sc Nash} to the problem \anothergale.

We will identify the bimatrix game with its payoff matrices $(A,B)$, and
we will assume that both $A$ and $B$ are non-negative, and that neither $A$
nor $B\T$ has a zero column; this can be done without loss of generality,
applying an affine transformation to $A$ and $B$ if necessary.
