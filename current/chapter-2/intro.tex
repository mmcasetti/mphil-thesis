In the rest of this thesis we will focus bimatrix games, analizing their
Nash equilibria with the use of labels and combinatorial structures.
This idea is due to Lemke and Howson \cite{lh}; we will follow the
approach given by Shapley \cite{shapley}.
We will first relate the labeling construction that allows to identify
Nash equilibria with ``completely labeled'' vertices or facets of certain
polytopes; we will then move on to unit vector games, a special class of
games that allows to simplify the labeling construction; see
Balthasar \cite{balthasar} and Savani and von Stengel \cite{uvg}.
In the second section of the chapter we will restrict our scope to
Gale games, a further case of unit vector games. The labeling problems for
these games are equivalent to an elegant combinatorial problem on
a case of bitstrings satisfying some simple conditions, called Gale strings.
We will define Gale strings, see their correspondence to the structures
given in the first section, and finally get a reduction from 2-{\sc Nash}
to the Gale string problem \anothergale.

We will identify the bimatrix game with its payoff matrices $(A,B)$, and
we will assume that both $A$ and $B$ are non-negative, and that neither $A$
nor $B\T$ has a zero column; this can be done without loss of generality,
applying an affine transformation to $A$ and $B$ if necessary.
