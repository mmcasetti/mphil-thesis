In the remainder of this thesis we will focus bimatrix games.
We will identify the game with its payoff matrices $(A,B)$, and
we will assume that both $A$ and $B$ are non-negative, and that neither $A$
nor $B\T$ has a zero column; this can be done without loss of generality,
applying an affine transformation to $A$ and $B$ if necessary.

We will start by showing a construction to represent a bimatrix game to
two simplices subdivided in labeled regions such that the Nash
equilibria of the game correspond to ``completely labeled'' points of
the simplices. This idea is due to Lemke and Howson
\cite{lh}; we will follow the very clear approach given by Shapley
\cite{shapley}. From labeled regions we will then move on to an equivalent
formulation in terms of polytopes; a result by McLennan and Tourky
\cite{mclennan-tourky} on a special case of games, called imitation games,
and its extension to the more general unit vector games, due to
Balthasar \cite{balthasar} and Savani and von
Stengel \cite{uvg}, will allow us to simplify this
construction. The problem 2-{\sc Nash} will then be reduced to a problem
on the facets or vertices of a labeled polytope, under some simple
conditions.

In the second section of the chapter we will restrict our scope to
Gale games, a case of unit vector games for which the
polytopes in the previous section can be
represented through a combinatorial structures called Gale strings.
We will first show the
correspondence between polytopes and Gale strings, then we will define
a labeling for the latter that
will lead to a reduction from 2-{\sc Nash} to a problem on Gale strings.
