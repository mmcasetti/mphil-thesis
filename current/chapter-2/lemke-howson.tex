\section{The Lemke-Howson Algorithm}

In the previous chapter we have defined some problems
of the form ``find another completely labeled\ldots'' for
vertices, facets and Gale strings. In this section we
will focus on different versions of a standard algorithm, first introduced
by Lemke and Howson in \cite{lh}, that yields a solution to these problems
using two main concepts: the {\em pivoting} routine and
{\em almost complete labeled} vertices, facets and Gale strings.

Let $P$ be a simple polytope in dimension $d$ with $n$ facets.
We {\em pivot on the vertices} of $P$ by moving from a
vertex $x$ to another vertex $y$ connected to $x$ by an edge.
Note that, since $P$ is simple, there are exactly
$d$ possible choices for $y$.
Analogously, we {\em pivot on the facets} of a simplicial polytope $Q$ in
dimension $d$ by moving from a facet $F$ to a facet $G$ that share
all vertices but one; and since $Q$ is simplicial, there are $d$ possible
choices for $G$.

Suppose now that there is a labeling $l_f:[n]\to [d]$ of the facets of the
simple polytope $P$.
If we pivot from a vertex $x$ to a vertex $x'$ we ``leave behind'' a facet $F$
with label $k$; so, if $x$ has labels $(l_1,\ldots,k,\ldots,l_d)$, then
$x'$ has labels $(l_1,\ldots,h,\ldots,l_d)$, where $h$ is the label of the
facet $F'$ that does not have $x$ as its vertex. We call this
{\em dropping label $k$ and picking up label $h$}, or
{\em pivoting on label $k$}.
Analogously, if there is a labeling $l_v:[n]\to [d]$ of the vertices
of the simplicial poytope $Q$
and we pivot from a facet $F$ with labels $(l_1,\ldots,k,\ldots,l_d)$
to a facet $F'$ with labels $(l_1,\ldots,h,\ldots,l_d)$, we say that we
{\em drop label $k$ and pick up label $h$}, or that we
{\em pivot on label $k$}.

Let $m,n\in \naturals$ with $m\leq n$; consider a set $X$ with $|X|=n$
and a labeling $l:X\to [m]$. The $m$-uple $x=(x_1,\ldots,x_m)\in X^m$
is {\em almost completely labeled} if
$\{ j\in [n]\ |\ x_i=j\text{ for some }i\in [m] \}= [m]\setminus \{ k \}$
for exactly one $k\in [m]$. That is, all labels appear once in $x$,
except for the {\em missing label} $k$ and a {\em duplicate label}
$h\in [m]$ that appears twice.

It's easy to see that if we pivot from an almost completely labeled facet
(or vertex) on the duplicate label, or from a completely labeled facet (or
vertex) on any label, we reach either an almost completely labeled or a
completely labeled facet (or vertex).

We now focus on the case of pivoting on vertices of simple polytopes with
labeled facets, that is, the case of the classic Lemke-Howson algorithm,
first given by Lemke and Howson in \cite{lh}; we follow the very clear
exposition given by Shapley in \cite{shapley}.

\begin{algorithm}\label{lh-alg}
\SetKwInOut{Input}{input}
\SetKwInOut{Output}{output}
\Input{
A simple $d$-polytope $P$ with $n$ facets.
A labeling $l_f:[n]\to [d]$ of the facets of $P$.
A vertex $x_0$ of $P$, completely labeled for $l$.
}
\Output{
A completely labeled vertex $x\neq x_0$ of $P$.
}
\BlankLine
choose any label $k\in [d]$ \\
pivot on label $k$ from $x_0$ to $x$ \\
\While{ $x$ is not completely labeled }
{
pivot on the duplicate label $h$ from $x$ to $x'\neq x_0$  \\
set $x_0 = x$, $x = x'$
}
\Return $x$
\caption{Lemke-Howson algorithm}
\label{lh-alg}
\end{algorithm}

Running the Lemke-Howson algorithm defines a {\em Lemke path}
that connects two different completely labeled vertices through
almost completely labeled vertices and edges where the only missing label
is $k$.

\begin{proposition}\label{lh-works-thm}
The Lemke-Howson algorithm \ref{lh-alg} returns a solution to the problem
{\sc Another Completely Labeled Vertex}.

Furthermore, the number of completely labeled vertices
in a simple polytope with labeled facets is even.

\begin{proof}

The fact that $x$ is completely labeled is trivial;
we must show that $x\neq x_0$.
At each vertex $x'$ of the Lemke path there are only two edges corresponding to
the missing label $k$, since $P$ is simple; one is the edge that has been
traversed to get to $x'$, the other one will be traversed to leave it
in the next step. Therefore, there are no ``loops'' where a vertex
is visited more than once; Lemke paths are {\em simple paths}.

Each Lemke path is uniquely determined by its missing label and its
starting point; furthermore, the Lemke path from the endpoint with
the same missing label will lead back to the starting point. Since the
endpoint and the starting point are different, the Lemke paths must connect
an even number of points.
\end{proof}
\end{proposition}

Applying the parity result in proposition \ref{lh-works-thm} to the case of
a bimatrix game (not necessarily a unit vector game), and remembering that
the point $(\0,\0)$ corresponds to an
``artificial'' equilibrium, we have the following result, due to Lemke and
Howson \cite{lh}.

\begin{theorem}{\rm (Lemke-Howson \cite{lh})}
Every non-degenerate bimatrix game has an odd number of Nash equilibria.
\end{theorem}

There are two ways of using the Lemke-Howson algorithm to find a Nash
equilibrium of a bimatrix game $(A,B)$.

The first way is to ``symmetrize'' the game as in proposition
\ref{symmetrize-c}. Let $C=\binom{0\ A}{B\T\ 0}$ and let
$S = \{ z\in\reals^{m+n}\ |\ z\geq\0,\ Cz\leq\1 \}$ be the
polytope associated to the game $(C,C\T)$. We can label the
$2(m+n)$ inequalities defining the facets of of $S$ as
$1,\ldots,m+n,1,\ldots,m+n$ and apply the Lemke-Howson algorithm
starting from the vertex $\0$; this will return a Nash equilibrium $(z,z)$ of
$C$, and the corresponding Nash equilibrium $(x,y)=z$ of $(A,B)$.

We can also follow the ``traditional'' version of the Lemke-Howson algorithm,
alternating moves on the best response polytopes $P$ and $Q$ as defined in
\ref{br-polytopes}, starting from the couple of vertices $(\0,\0)$.
Since we move in $\reals^m$ and $\reals^n$ instead of $\reals^{m+n}$,
this version is much easier to visualize.

\begin{example}
Consider the $3\times 3$ game $(A,B)$ of example \ref{br-game-ex}.
\begin{equation*}
A = \left(\begin{matrix}1&0&0\\ 0&1&0\\
0&0&1\end{matrix}\right),
\qquad
B = \left(\begin{matrix}0&2&4\\ 3&2&0\\
0&2&0\end{matrix}\right).
\end{equation*}

The best response polytopes can be represented as the best response regions
(see figure \ref{br-regions-fig}) extended to the origin $\0$, as in figure
\ref{lh-path-fig}; the label ``outside'' refers to the ``back'' of the
polytope.

\begin{figure}[hbt]
\strut\hfill
\includegraphics[width=65ex]{chapter-2/fig/lh.pdf}%
\hfill\strut
\caption[Lemke-Howson path for a bimatrix game.]{%
Lemke path for missing label 2 on the best response polytopes $P$ and $Q$ of
game (\ref{AB}).
}
\label{lh-path-fig}
\end{figure}

The path starts in $(\0,\0)$; we drop the label 2 and we move on the polytope
$P$. The label 6 is duplicate; so we drop the label 6 and we move on the
polytope $Q$; an so on until we reach the point $x$, that is a Nash
equilibrium of $(A,B)$.
\end{example}

The dual version of the Lemke-Howson algorithm \ref{lh-alg}
and of proposition \ref{lh-works-thm} is quite straightforward.

\begin{algorithm}\label{lh-dual-alg}
\SetKwInOut{Input}{input}
\SetKwInOut{Output}{output}
\Input{
A simplicial $m$-polytope $Q$ with $n$ vertices.
A labeling $l_v:[n]\to [d]$ of the vertices of $P$.
A vertex $F_0$ of $Q$, completely labeled for $l$.
}
\Output{
A completely labeled facet $F\neq F_0$ of $Q$.
}
\BlankLine
choose any label $k\in [d]$ \\
pivot on label $k$ from $F_0$ to $F$ \\
\While{ $x$ is not completely labeled }
{
pivot on the duplicate label $h$ from $F$ to $F'\neq x_0$  \\
set $F_0 = x$, $F = F'$
}
\Return $x$
\caption{Dual Lemke-Howson algorithm}
\label{lh-dual-alg}
\end{algorithm}

\begin{proposition}\label{lh-dual-works-thm}
The dual Lemke-Howson algorithm \ref{lh-dual-alg} returns a solution
to the problem {\sc Another Completely Labeled Facet}.

Furthermore, the number of completely labeled facets
in a simplicial polytope with labeled vertices is even.
\end{proposition}

\begin{example}
Consider the octahedron with vertices labeled as in figure
\ref{octahedron-fig}. The facet $F$ is completely labeled; dropping the
vertex with label 1 we pivot to the completely labeled facet $F'$.

\begin{figure}[hbt]
\strut\hfill
\includegraphics[width=55ex]{chapter-2/fig/octahedron-lhdual.pdf}%
\hfill\strut
\caption{%
Pivoting on the facets of the octahedron.%
}
\label{lh-path-fig}
\end{figure}
\end{example}

By theorems \ref{unit-vector-thm} and \ref{unit-vector-dual-thm},
in the case of unit vector games is enough to apply the Lemke-Howson
algorithm to the polytope
$P^l= \{ x\in\reals^m\ |\ x\geq\0,\ B\T x\leq\1 \}$ in (\ref{p-l}),
or the dual Lemke-Howson algorithm to the polytope
$Q=\conv(\{ e_1,\ldots,e_m \})\ \cup\ \{ c_1,\ldots,c_n \})$ in
(\ref{p-l-dual}).
Not only this yield a Nash equilibrium, but no potential solutions
are ``lost'' considering the polytope $P^l$ with $m$ labels
instead of the product of polytopes $P\times Q$ with $m + n$ labels,
as stated in the following theorem by Savani and von Stengel \cite{uvg};
an analogous result holds for the dual case.

\begin{theorem}\label{unit-paths}
Let $(U,B)$ be a unit vector game, with
$U=(e_{l(1)}\cdots e_{l(n)})$ for a labeling $l:[n]\to [m]$;
let $P = \{ x\in\reals^m | x\geq\0,\ B\T x\leq\1 \}$ and
$Q = \{ y\in\reals^n | y\geq\0,\ A y\leq\1 \}$, as in \ref{br-polytopes};
and let $P^l= \{ x\in\reals^m\ |\ x\geq\0,\ B\T x\leq\1 \}$
as in \ref{p-l-unitv}. Then the Lemke path on $P\times Q$ for the missing
label $k$ projects to a path on $P$ that is the Lemke path on $P^l$
for missing label $k$ if $k\in [m]$,
and for missing label $l(j)$ if $k=m+j$ with $j\in [n]$.
\end{theorem}

We finally focus on the case of unit vector games
where the simplicial polytope $Q$ is cyclic; that is,
the case that we can study from the point of view of Gale strings.
Consider $s\in G(m,n)$ with $d$ even as a ``loop'', and
let $s(i)=1$ for an index $i\in [n]$. Then, by Gale evenness condition,
there is an odd run of \1 's in $s$ either on the left or on the right
of position $i$; let $j$ be the first index after this run.
A {\em pivot on $s$} is then defined as setting $s(i)=0$ and $s(j)=1$.
Given a labeling $l_s:[n]\to [m]$, we say that we
{\em pivot on label $l(i)$}, {\em dropping label $l(i)$} and
{\em picking up label $l(j)$}.
The {\em Lemke-Howson for Gale algorithm} is defined as follows.

\begin{algorithm}\label{lhg-alg}
\SetKwInOut{Input}{input}
\SetKwInOut{Output}{output}
\Input{
A labeling $l_s:[n]\to [d]$ such that there is a completely labeled
Gale string $s_0 \in G(d,n)$.
}
\Output{
A completely labeled Gale string $s\in G(d,n)$ such that $s\neq s_0$.
}
\BlankLine
choose a label $k\in [d]$ \\
pivot on label $k$ from $s_0$ to $s$ \\
\While{ $s$ is not completely labeled }
{
pivot on the duplicate label $h$ from $s$ to $s'\neq s_0$ \\
rename $s_0=s$, $s=s'$
}
\Return $s$
\caption{Lemke-Howson for Gale algorithm}
\label{lhg-alg}
\end{algorithm}

We see the correspondence between the Lemke-Howson and the
Lemke-Howson for Gale algorithms in the next example.

\begin{example}
Figure \ref{lhg-123432-fig} shows the cyclic polytope $C_4(6)$
with the labeling given in example \ref{c46-123432-ex}.
This corresponds to the labeling $l=123432$, for which there are
four completely labeled Gale strings in $G(4,6)$:
$s_A=\1\1\1\100$, $s_B=\1\10\1\10$, $s_C=\100\1\1\1$ and
$s_D=\10\1\10\1$, corresponding to the facets $A$, $B$, $C$ and $D$.

\begin{figure}[hbt]
\strut\hfill
\includegraphics[width=60ex]{chapter-2/fig/123432-lh.pdf}%
\hfill\strut
\caption[Lemke-Howson for Gale algorithm and cyclic polytope]{%
Lemke-Howson for Gale algorithm: the pivoting to $s_A=\1\1\1\1 00$ to
$s_B=\1\10\1\10$ correspond to the pivoting from the facet $A$ to
the facet $B$.%
}
\label{lhg-123432-fig}
\end{figure}

Pivoting from facet $A$ dropping label 3 yields facet $B$; pivoting from
$s_A=\1\1\1\100$ dropping label 1 yields $s_B=\1\10\1\10$.
\end{example}

The analogous of propositions \ref{lh-works-thm} and
\ref{lh-dual-works-thm} is the following.

\begin{proposition}\label{lhg-works-thm}
The Lemke-Howson for Gale algorithm \ref{lhg-alg} returns a
solution to the problem \anothergale.

Furthermore, the number of completely labeled Gale strings
$s\in G(d,n)$, where $d$ is even, is even.
\end{proposition}

An interesting example, as we will see in the next section, is the
following, due to Morris \cite{morris}.

\begin{example}
Consider the labeling $l=1234564523$ for $G(6,10)$ and the completely
labeled Gale string $s=\1\1\1\1\1\100$. Dropping the label 1, the
Lemke-Howson for Gale algorithm will run as in figure \ref{morris-6-fig}.

\begin{figure}[hbt]
\begin{center}
\large
\begin{tabular}{c @{ } c @{ } c @{ } c @{ } c @{ } c @{ } c @{ } c @{ } c @{ } c @{ } c }
{\bf 1} & {\bf 2} & {\bf 3} & {\bf 4} & {\bf 5} & {\bf 6} & {\bf 4} & {\bf 5} & {\bf 2} & {\bf 3} \\
\hline
\d1 & \1 & \1 & \1 & \1 & \1 & \tdot & \tdot & \tdot & \tdot \\
\tdot & \1 & \1 & \d1 & \1 & \1 & \u1 & \tdot & \tdot & \tdot \\
\tdot & \1 & \1 & \tdot & \d1 & \1 & \1 & \u1 & \tdot & \tdot \\
\tdot & \d1 & \1 & \tdot & \tdot & \1 & \1 & \1 & \u1 & \tdot \\
\tdot & \tdot & \1 & \u1 & \tdot & \1 & \d1 & \1 & \1 & \tdot \\
\tdot & \tdot & \1 & \1 & \u1 & \1 & \tdot & \d1 & \1 & \tdot \\
\tdot & \tdot & \d1 & \1 & \1 & \1 & \tdot & \tdot & \1 & \u1 \\
\tdot & \tdot & \tdot & \d1 & \1 & \1 & \u1 & \tdot & \1 & \1 \\
\tdot & \tdot & \tdot & \tdot & \d1 & \1 & \1 & \u1 & \1 & \1 \\
\u1 & \tdot & \tdot & \tdot & \tdot & \1 & \1 & \1 & \1 & \1 \\
\hline
{\bf 1} & {\bf 2} & {\bf 3} & {\bf 4} & {\bf 5} & {\bf 6} & {\bf 4} & {\bf 5} & {\bf 2} & {\bf 3} \\
\end{tabular}
\normalsize
\end{center}
\caption{Morris path on $C(6,10)$}
\label{morris-6-fig}
\end{figure}
\end{example}
