

\section{Lemke Paths and the Lemke-Howson for Gale Algorithm}

In the previous chapter we have defined different, but related, problems
of the form ``find another completely labeled\ldots'' In this section we
will focus on different versions of a standard algorithm, first introduced
by Lemke and Howson in \cite{lh}, that will solve these problems through
an operation called {\em pivoting} and introducing {\em almost complete}
labelings next to complete ones. In the next session we will tackle the
issue of the computational complexity of these algorithms:
{\sc Another Completely Labeled Facet} and
{\sc Another Completely Labeled Vertex} are {\bf PPA},
{\sc Nash} is {\bf PPAD}, as first shown in Papadimitriou \cite{ppad};
furhermore, as shown by Morris \cite{morris} and by
Savani and von Stengel \cite{svs}, there are cases of exponential running
time. This had led us to conjecture that these problems could be exploited
for a proof of {\bf PPAD} completeness, also considering that finding
a completely labeled facet (or vertex, or the existence of a Nash
equilibrium) is {\bf NP} in the case of a general labeled polytope, as
proven by von Stengel \cite{vs-np-facet}
\todo{check proof, citation}
. In the last section we will finally present our original
\todo{original? main?}
result, that goes in the opposite direction: the problem \anothergale
can be solved in polynomial time, that is, it is a problem in {\bf TFP}.
Unit vector games with dual cyclic best response polytope present therefore
a case apart, as expected, but not because they are harder than others, but
because they are easier.

We begin this section with the definition of almost complete labeling; we
then move on to the classic version of the Lemke-Howson algorithm
for the problem {\sc Another Completely Labeled Vertex}, as given
in the beautiful exposition by Shapley \cite{shapley}, and its dual
version for {\sc Another Completely Labeled Facet}. Finally, we present
the Lemke-Howson for Gale algorithm.

Let $m,n\in \naturals$ with $m\leq n$; consider a set $X$ with $|X|=n$
and a labeling $l:X\to [m]$. The $m$-uple $x=(x_1,\ldots,x_m)\in X^m$
is {\em almost completely labeled} if
$\{ j\in [n]\ |\ x_i=j\text{ for some }i\in [m] \}= [m]\setminus \{ k \}$
for exactly one $k\in [m]$. That is, all labels appear once in $x$,
except for the {\em missing label} $k$ and a {\em duplicate label}
$h\in [m]$ that appears twice.

Let $P$ be a simple polytope in dimension $d$ with $n$ facets.
We {\em pivot on the vertices} of $P$ by moving from a
vertex $x$ to another vertex $y$ connected to $x$ by an edge.
Note that, since $P$ is simple, there are exactly
$d$ possible choices for $y$.
Analogously, we {\em pivot on the facets} of a simplicial polytope $Q$ in
dimension $d$ by moving from a facet $F$ to a facet $G$ that share
all vertices but one; and since $Q$ is simplicial, there are $d$ possible
choices for $G$.

Suppose now that there is a labeling $l_f:[n]\to [d]$ of the facets of the
simple polytope $P$.
If we pivot from a vertex $x$ to a vertex $x'$ we ``leave behind'' a facet $F$
with label $k$; so, if $x$ has labels $(l_1,\ldots,k,\ldots,l_d)$, then
$x'$ has labels $(l_1,\ldots,h,\ldots,l_d)$, where $h$ is the label of the
facet $F'$ that does not have $x$ as its vertex. We call this
{\em dropping label $k$ and picking up label $h$}, or
{\em pivoting on label $k$}.
Analogously, if there is a labeling $l_v:[n]\to [d]$ of the vertices
of the simplicial poytope $Q$
and we pivot from a facet $F$ with labels $(l_1,\ldots,k,\ldots,l_d)$
to a facet $F'$ with labels $(l_1,\ldots,h,\ldots,l_d)$, we say that we
{\em drop label $k$ and pick up label $h$}, or that we
{\em pivot on label $k$}.

We consider now the case of a simple polytope with labeled facets.
Suppose that the labeling $l_f$ is such that there is at least one
completely labeled vertex $x_0$ of $P$. Algorithm  \ref{lh-alg} then gives
the {\em Lemke-Howson algorithm} algorithm (see Shapley in \cite{shapley}).

\begin{algorithm}\label{lh-alg}
\SetKwInOut{Input}{input}
\SetKwInOut{Output}{output}
\Input{
A simple $d$-polytope $P$ with $n$ facets.
A labeling $l_f:[n]\to [d]$ of the facets of $P$.
A vertex $x_0$ of $P$, completely labeled for $l$.
}
\Output{
A completely labeled vertex $x\neq x_0$ of $P$.
}
\BlankLine
choose any label $k\in [d]$ \\
pivot on label $k$ from $x_0$ to $x$ \\
\While{ $x$ is not completely labeled }
{
pivot on the duplicate label $h$ from $x$ to $x'\neq x_0$  \\
set $x_0 = x$, $x = x'$
}
\Return $x$
\caption{Lemke-Howson algorithm}
\end{algorithm}

Running the Lemke-Howson algorithm defines a {\em Lemke path}
that connects two completely labeled vertices through
almost completely labeled vertices and edges where the only missing label
is $k$. This is fundamental to show that the Lemke-Howson algorithm does,
indeed, work.

\begin{proposition}\label{lh-works-thm}
The Lemke-Howson algorithm \ref{lh-alg} returns a solution to the problem
{\sc Another Completely Labeled Vertex}.

Furthermore, the number of completely labeled vertices
in a simple polytope with labeled facets is even.

\begin{proof}

The fact that $x$ is completely labeled is trivial;
we must show that $x\neq x_0$.
At each vertex $x'$ of the Lemke path there are only two edges corresponding to
the missing label $k$, since $P$ is simple; one is the edge that has been
traversed to get to $x'$, the other one will be traversed to leave it
in the next step. Therefore, there are no ``loops'' where a vertex
is visited more than once; Lemke paths are {\em simple paths}.

Suppose now that there are an odd number of completely labeled vertices;
that is, that there is an odd number of endpoints of Lemke paths.

\todo[inline]{pf parity via Lemke paths}

\end{proof}
\end{proposition}

Applying the parity result in proposition \ref{lh-works-thm} to the case of
a bimatrix game (not necessarily a unit vector game)
\todo{note: two polytopes, see todo at end section labels}
, and remembering that
the point $(\0,\0)$ corresponds to an
``artificial'' equilibrium, we have the following result, due to Lemke and
Howson \cite{lh}.

\begin{theorem}{\rm (Lemke-Howson \cite{lh})}
Every non-degenerate bimatrix game has an odd number of Nash equilibria.
\end{theorem}

\begin{example}
We now give an example of implementing the Lemke-Howson algorithm to find
a Nash equilibrium of a bimatrix game.

We can consider the game $C$ as in proposition \ref{symmetric-eq-thm},
and the associated polytope
$S = \{ z\in\reals^{m+n}\ |\ z\geq\0,\ Cz\leq\1 \}$,
labeling the $2(m+n)$ inequalities defining the facets of of $S$
as $1,\ldots,m+n,1,\ldots,m+n$.
Then we will apply the Lemke-Howson algorithm starting from
vertex $\0$; this will eventually return a Nash equilibrium $(z,z)$ of
$C$, and the corresponding Nash equilibrium $(x,y)=z$ of $(A,B)$.

We can also follow the ``traditional'' version of the Lemke-Howson algorithm;
a very clear exposition of this can be found (once again) in
Shapley \cite{shapley}.

Let $P$ and $Q$ be the best response polytopes of $(A,B)$ as in
\ref{br-polytopes}. We then move alternately on $P$ and $Q$, starting from
the couple of vertices $(\0,\0)$.
Since we move in $\reals^m$ and $\reals^n$ instead of $\reals^{m+n}$,
this version is much easier to visualize.

\todo[inline]{ex Savani - von Stengel, pag. 11; fig 8 are Schegel diagrams of
BR polytopes.}

\end{example}

The dual version of the Lemke-Howson algorithm \ref{lh-alg}
and of proposition \ref{lh-works-thm} is quite straightforward.

\begin{algorithm}\label{lh-dual-alg}
\SetKwInOut{Input}{input}
\SetKwInOut{Output}{output}
\Input{
A simplicial $m$-polytope $Q$ with $n$ vertices.
A labeling $l_v:[n]\to [d]$ of the vertices of $P$.
A vertex $F_0$ of $Q$, completely labeled for $l$.
}
\Output{
A completely labeled facet $F\neq F_0$ of $Q$.
}
\BlankLine
choose any label $k\in [d]$ \\
pivot on label $k$ from $F_0$ to $F$ \\
\While{ $x$ is not completely labeled }
{
pivot on the duplicate label $h$ from $F$ to $F'\neq x_0$  \\
set $F_0 = x$, $F = F'$
}
\Return $x$
\caption{Lemke-Howson algorithm on facets}
\end{algorithm}

\begin{proposition}\label{lh-dual-works-thm}
Algorithm \ref{lh-dual-alg} returns a solution to the problem
{\sc Another Completely Labeled Facet}.

Furthermore, the number of completely labeled facets
in a simplicial polytope with labeled vertices is even.
\end{proposition}

\begin{example}

\todo[inline]{example: octahedron! so we're ready for index!}

\end{example}

To find a Nash equilibrium of a unit vector game $(U,B)$, where
$U=(e_{l(1)}\cdots e_{l(n)})$ for a labeling $l:[n]\to [m]$,
we can apply theorem \ref{unit-vector-thm} and algorithm \ref{lh-alg},
or we can apply the dual theorem \ref{unit-vector-dual-thm} and
algorithm \ref{lh-dual-alg}.
The first case relies on the polytope
$P^l= \{ x\in\reals^m\ |\ x\geq\0,\ B\T x\leq\1 \}$ defined
in \ref{p-l-unitv}; theorem \ref{unit-vector-thm} shows that
$P^l$ encodes all the Nash equilibria of $(U,B)$ as completely labeled
vertices, with an ``artificial'' equilibrium corresponding to the
vertex $\0$.
The second case relies on the polytope $Q$, defined
as the convex hull of vertices $-e_i$ for $i\in [m]$ and
$c_j=\frac{b_j}{1 - \1\T b_j}$ for $j\in [n]$; analogously, theorem
\ref{unit-vector-dual-thm} shows that $Q$ encodes all the Nash equilibria
of $(U,B)$ as completely labeled facets, with the ``artificial''
equilibrium corresponding to the facet $F_0=\conv(-e_1,\ldots,-e_m)$.

On the other hand, we can consider $(U,B)$ as any bimatrix game, and
apply algorithm \ref{lh-alg} to the product of the best response
polytopes $P$ and $Q$.
The projection of a Lemke path for a missing label $i\in [m]$
on $P\times Q$ to $P$ defines a Lemke path in $P^l$. \todo{how?}
However, $P\times Q$ has $m+n$ labels, therefore there could be
Lemke paths for a missing label $m+j$ with $j\in [n]$
on $P\times Q$ that get lost in the projection on $P^l$.
The following theorem, proved by Savani and von Stengel in \cite{uvg},
shows that there is no loss of generality in studying Lemke paths on
$P^l$; an analogous result holds for the dual case.

\begin{theorem}\label{unit-paths}
Let $(U,B)$ be a unit vector game, with
$U=(e_{l(1)}\cdots e_{l(n)})$ for a labeling $l:[n]\to [m]$;
let $P = \{ x\in\reals^m | x\geq\0,\ B\T x\leq\1 \}$ and
$Q = \{ y\in\reals^n | y\geq\0,\ A y\leq\1 \}$, as in \ref{br-polytopes};
and let $P^l= \{ x\in\reals^m\ |\ x\geq\0,\ B\T x\leq\1 \}$
as in \ref{p-l-unitv}. Then the Lemke path on $P\times Q$ for the missing
label $k$ projects to a path on $P$ that is the Lemke path on $P^l$
for missing label $k$ if $k\in [m]$,
and for missing label $l(j)$ if $k=m+j$ with $j\in [n]$.
\end{theorem}

As before, we now focus on the case of unit vector games
where the simplicial polytope $Q$ is cyclic; that is,
the case that we can study from the point of view of Gale strings.

Consider a Gale string $s\in G(m,n)$ with $d$ even, identifying the
indices modulo $n$ (that is, considering the string as a ``loop'');
let $s(k)=1$ for an index $k\in [n]$. Then, by Gale evenness condition,
there is an odd run of \1 's in $s$ either on the left or on the right
of position $k$; let $j$ be the first index after this run.
A {\em pivoting on $s$} is then defined as setting $s(k)=0$ and $s(j)=1$.
Given a labeling $l_s:[n]\to [m]$, pivoting as above becomes
{\em dropping label $k$} and {\em picking up label $j$}.
The {\em Lemke-Howson for Gale algorithm} is defined as in \ref{lhg-alg}.

\begin{algorithm}\label{lhg-alg}
\SetKwInOut{Input}{input}
\SetKwInOut{Output}{output}
\Input{
A labeling $l_s:[n]\to [d]$ such that there is a completely labeled
Gale string $s_0 \in G(d,n)$.
}
\Output{
A completely labeled Gale string $s\in G(d,n)$ such that $s\neq s_0$.
}
\BlankLine
choose a label $k\in [d]$ \\
pivot on $s_0$ dropping $k$, obtaining $s$ \\
\While{ $s$ is not completely labeled }
{
pivot from $s$ to $s'\neq s_0$ dropping the duplicate label $j$ \\
rename $s_0=s$, $s=s'$
}
\Return $s$
\caption{Lemke-Howson for Gale algorithm}
\end{algorithm}

Considering the Lemke paths defined by algorithm \ref{lhg-alg} we can
prove that the Lemke-Howson for Gale algorithm works and a parity result,
analogously to theorems \label{lh-works-thm} and \label{lh-dual-works-thm}.

\begin{theorem}\label{lhg-works-thm}
The Lemke-Howson for Gale algorithm \ref{lhg-alg} returns a
solution to the problem \anothergale.

Furthermore, the number of completely labeled Gale strings
$s\in G(d,n)$, where $d$ is even, is even.
\end{theorem}

\begin{example}

\todo[inline]{example}

\end{example}

In \ref{gs-labeling-unitv} we have given a labeling $l_s:[n+d]\to [d]$
to study the Nash equilibria (including the ``artificial'' equilibrium)
of the unit vector game $(U,B)$, where $U=(e_{l(1)}\cdots e_{l(n)})$
and $l:[n]\to [d]$, as Gale strings $s\in G(n+d,d)$ that are completely
labeled by $l_s$, with the ``artificial'' equilibrium corresponding to
the string $1^d0^n$.



\todo[inline]{

check!

thanks to dual of theorem SvS-15 \ref{unit-paths}, when doing labeling
we can take the str of labels $l(n+j)\cdots l(n+m)$
instead of $l(1)\cdots l(n+m)$,
that is, we could cut the ``artificial'' first labels $12...n$.

%l_s(i)=i\text{ for }i\in [d] \\
%l_s(d+j)=l(j)\text{ for }j\in [n].

After all, in main we're studying ANOTHER GALE
in general, not nec starting from $12...n$; and we're interested in finding
{\em one} eq that's not the one we started from
(and is at other end of LPath, since index and so on),
{\em not all equilibria};
but the eq we started from is not nec the artificial one - actually,
if we go with this we can take any NE to start looking for another,
and we're sure to find a ``non-artificial'' one.
Note: if we were looking for all NE, LH doesn't work anyway - see ex by
Wilson in Shapley, where ``disconnected'' paths between equilibria.
}


\todo[inline]{

complexity considerations: PPAD complexity of GALE
(also of ANOTHER CL VERTEX/FACET, analogous pf;
for 2-NASH we have completeness by DGP+CD)

Now: it's PPAD; but is it complete?
(Also, PPAD is relatively new, results welcome...)

Interesting case: Morris paths (Morris 1994), translates in terms of games
by SvS 2006. Exp running time!

Could it be used to show a completeness result?
Next section: we show that on the other hand it takes polynomial (!) time
to find a result, so a completeness result would mean that P=PPAD
- very unlikely! (One can dream, though...)

----

Advantage of two dual cyclic polytopes as in SvS 06 over only one
(and without "prefix" in labels for gale string) as seen in SvS 15 is:
good example of exp on supports - see SvS 15
}
