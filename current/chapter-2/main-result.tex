\section{The Complexity of \gale\ and \anothergale}

We will now give our main result: \anothergale\ can be solved in polynomial
time; therefore, it takes polynomial time to find a Nash Equilibrium of a
bimatrix game with dual cyclic best response polytope. Our proof will
rely on the construction of a graph and, if possible, a perfect matching
for it.
A {\em perfect matching} of a multigraph $G=(V,E)$ is a set $M\subseteq E$
of pairwise non-adjacent edges so that every vertex $v \in V$ is incident
to exactly one edge in~$M$. A theorem by Edmonds (\cite{edm}) gives
the complexity of the associated problem {\sc Perfect Matching}.

\begin{fctproblem}
{Perfect Matching}
{A multigraph $G = (V,E)$.}
{A perfect matching for $G$, or {\sc No} if there is no possible perfect
matching for $G$.}
\end{fctproblem}

\begin{theorem}{\rm (Edmonds \cite{edm})}\label{pm-thm}
The problem {\sc Perfect Matching} can be solved in polynomial time.
\end{theorem}

To prove our main result on \anothergale, we will first focus on the
accessory problem \gale, and we will use theorem \ref{pm-thm} to prove
that it is solvable in polynomial time. We will consider every Gale string
as a ``loop.''

\begin{fctproblem}
{\gale}
{A labeling $l:[n]\to[d]$, where $d$ is even and $d<n$.}
{A Gale string $s\in G(d,n)$ that is completely labeled by $l$}
\end{fctproblem}

\begin{theorem}\label{gale-thm}
The problem \gale\ is solvable in polynomial time.

\begin{proof}
We give a reduction of \gale\ to {\sc Perfect Matching}.

Consider the multigraph $G=(V,E)$ with $V=[d]$, so that the vertices of $G$
correspond to the labels $l(i)\in [d]$, and
$E=\{(l(i),l(i+1))\text{ for }i\in[n] \}$, so that there is an edge
between two vertices if and only if the corresponding labels are
next to each other at some index $i$.
Let $s\in G(d,n)$ be a completely labeled Gale string. By Gale evenness
condition, every run of $s$ corresponds uniquely to $d/2$ pairs of indices
$(i,i+1)$ with $s(i)=s(i+1)=1$, and since $s$ is completely labeled, all
labels $l(i)\in [d]$ occur at exactly one of these indices. Then the edges
$(l(i),l(i+1))$ form a perfect matching of $G$.

Conversely, let $l:[n]\to [d]$ be a labeling, and let $M$ be a perfect
matching for $G$. Consider a bitstring $s$ with $s(i)=s(i+1)$ for every
$(l(i),l(i+1))\in M$ and $s(i)=0$ otherwise.
Since $M$ is a matching, all the $(l(i),l(i+1))\in M$ are
disjoint, so, considering $s$ as a ``loop,'' every run of $s$ is of even
length, thus satisfying the Gale evenness condition. Since $M$ is perfect,
every vertex $v\in [d]$ is the endpoint of an edge $(l(i),l(i+1))$, so $s$
has exactly $d$ bits equal to $\1$, so it is completely labeled.

We have therefore reduced the problem \gale to {\sc Perfect Matching}, that
by theorem \ref{gale-thm} can be solved in polynomial time.
\end{proof}
\end{theorem}

We give two examples of the construction used in theorem \ref{gale-thm}.

\begin{example}
\label{gs-pm-ex}
Figure \ref{perfect-matching-fig} shows the graph for the Morris
labeling $l=1234564523$, and its two matchings
$M=\{ e_1,e_3,e_5 \}$ and $M'=\{ e_8,e_6,e_{10} \}$.

\begin{figure}[hbt]
\strut\hfill
\includegraphics[width=40ex]{chapter-2/fig/morris-6-graph.pdf}%
\hfill\strut
\caption[The Morris graph]{%
The graph $G$ and its matchings for the Morris labeling $l=1234564523$.
}
\label{perfect-matching-fig}
\end{figure}

These, in turn, correspond to the completely labeled Gale strings
$s=\1\1\1\1\1\1 0000$ and $s=\1 0000\1\1\1\1\1$.
\end{example}

A perfect matching for a graph, and therefore a Gale string for a labeling,
is not always possible, as shown in the next example.

\begin{example}
Consider the labeling $l=121314$. The graph $G$ is shown in figure
\ref{no-matching}

\begin{figure}[hbt]
\strut\hfill
\includegraphics[width=35ex]{chapter-2/fig/no-matching.pdf}%
\hfill\strut
\caption[A graph without a perfect matching]{%
The graph for the labeling $l=121314$
}
\label{no-matching}
\end{figure}

Since there aren't any disjoint edges, it's not possible to find a perfect
matching for $G$. We have already seen in example \ref{no-clgs} that
there isn't any possible completely labeled Gale string for $l=121314$.
\end{example}

We finally extend the proof of theorem \ref{gale-thm} to \anothergale.

\begin{theorem}\label{anothergale-thm}
The problem \anothergale\ is solvable in polynomial time.

\begin{proof}
Let $G=(V,E)$ be the graph corresponding to the labeling $l:[n]\to [d]$ as
in the proof of theorem \ref{gale-thm} and let $M$ be the perfect matching
of $G$ corresponding to the completely labeled Gale string $s\in G(d,n)$.

If there are two edges $e,e'\in E$ such that $e\in M$, both $e$ and $e'$
have endpoints $l(i),l(i+1)$, but $e\neq e'$ (recall that $G$ can be a
multigraph), the matching $M'=(M\setminus \{ e \})\cup\{ e' \}$
is perfect.
The corresponding completely labeled Gale string
$s'\in G(d,n)$ satisfies $s'\neq s$, since in $s$ the \1's corresponding
to the labels $l(i),l(i+1)$ are in the positions given by the edge
$e$, while in $s'$ they are in the positions given by $e'\neq e$.
It takes time $d/2$ to check all edges of $M$, the time required
is still polynomial.

We now assume that all the edges in every perfect matching $M$ for $G$
don't have a parallel edge.
Since by theorem \ref{lhg-works-thm} there is an even number of
completely labeled Gale strings, the existence of $s$ guarantees the
existence of another completely labeled Gale string $s'\neq s$ and the
corresponding perfect matching $M'\neq M$.
Since $M'\neq M$, there is at least one edge $e'\in M$ such that
$e'\notin M'$. Consider the $d/2$ graphs $G_i=(V,E_i)$, where
$E_i=E\setminus\{ e_i \}$ for $e_i\in M$. Since $V(G)=V(G_i)$ and
$E(G_i)\subset E(G)$, every perfect matching for one of these $G_i$
is a perfect matching for $G$ as well.
With a brute force approach, we look for a perfect matching in each $G_i$;
this will be $M'$. Since there are $i\in [d/2]$, the time to find it
will be still polynomial.
\end{proof}
\end{theorem}

We give two examples of the construction of theorem \ref{anothergale-thm}.

\begin{example}
The labeling $l=123432$ gives the graph $G$ in figure
\ref{matching-2-edges}. Suppose that Edmonds' algorithm returns the matching
$M=\{ e_1,e_3 \}$, associated to the completely labeled Gale string
$s=\1\1\1\1 00$. The edge $e_1$ has a parallel edge, $e_6$; we immediately
have a second perfect matching in $M'=\{ e_3,e_6 \}$, associated to the
Gale string $s'=\1 0\1\1 0\1 $.

\begin{figure}[hbt]
\strut\hfill
\includegraphics[width=40ex]{chapter-2/fig/matching-123432.pdf}%
\hfill\strut
\caption[A matching with a parallel edge]{%
The graph for the labeling $l=123432$.
}
\label{matching-2-edges}
\end{figure}
\end{example}

A case without parallel edges in the matching is the Morris graph.

\begin{example}
Consider the Morris graph of example \ref{gs-pm-ex}; suppose that
Edmonds' algorithm returns the perfect matching $M=\{ e_1,e_3,e_5 \}$,
as in figure \ref{intermediate-matching} right, corresponding
to the completely labeled Gale string $s=\1\1\1\1\1\1 0000$.
We can then delete the edge $e_1$ to obtain the graph
$G_1$, as in figure \ref{intermediate-matching} left. The graph $G_1$
has a perfect matching $M'=\{ e_6,e_8,e_{10} \}$; this is also a perfect
matching of $G$, corresponding to the string $s'=\1 0000\1\1\1\1\1$.

\begin{figure}[hbt]
\strut\hfill
\includegraphics[width=35ex]{chapter-2/fig/morris-6-1-matching.pdf}%
\hfill
\hfill
\includegraphics[width=35ex]{chapter-2/fig/morris-6-intermediate.pdf}%
\hfill\strut
\caption[The second matching of the Morris graph]{%
Left: the Morris graph $G=(V,E)$ with the matching $M=\{ e_1,e_3,e_5 \}$. \\
Right: the graph $G_1=(V,E\setminus\{ e_1 \})$ with the matching
$M'=\{ e_6,e_8,e_{10} \}$.
}
\label{intermediate-matching}
\end{figure}
\end{example}
