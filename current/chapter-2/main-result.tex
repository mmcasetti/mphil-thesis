\section{The Complexity of \gale\ and \anothergale}

We will now give our main result: \anothergale\ can be solved in polynomial
time; therefore, it takes polynomial time to find a Nash Equilibrium of a
bimatrix game with dual cyclic best response polytope. Our proof will
rely on the construction of a graph and, if possible, a perfect matching
for it.
A {\em perfect matching} of a multigraph $G=(V,E)$ is a set $M\subseteq E$
of pairwise non-adjacent edges so that every vertex $v \in V$ is incident
to exactly one edge in~$M$. A theorem by Edmonds (\cite{edm}) gives
the complexity of the associated problem {\sc Perfect Matching}.

\begin{fctproblem}
{Perfect Matching}
{A multigraph $G = (V,E)$.}
{A perfect matching for $G$, or {\sc No} if there is no possible perfect
matching for $G$.}
\end{fctproblem}

\begin{theorem}{\rm (Edmonds \cite{edm})}\label{pm-thm}
The problem {\sc Perfect Matching} can be solved in polynomial time.
\end{theorem}

To prove our main result on \anothergale, we will first focus on the
accessory problem \gale, and we will use theorem \ref{pm-thm} to prove
that it is solvable in polynomial time. We will consider every Gale string
as a ``loop.''

\begin{fctproblem}
{\gale}
{A labeling $l:[n]\to[d]$, where $d$ is even and $d<n$.}
{A Gale string $s\in G(d,n)$ that is completely labeled by $l$}
\end{fctproblem}

\begin{theorem}\label{gale-thm}
The problem \gale\ is solvable in polynomial time.

\begin{proof}
We give a reduction of \gale\ to {\sc Perfect Matching}.

Consider the multigraph $G=(V,E)$ with $V=[d]$, so that the vertices of $G$
correspond to the labels $l(i)\in [d]$, and
$E=\{(l(i),l(i+1))\text{ for }i\in[n] \}$, so that there is an edge
between two vertices if and only if the corresponding labels are
next to each other at some index $i$.
Let $s\in G(d,n)$ be a completely labeled Gale string. By Gale evenness
condition, every run of $s$ corresponds uniquely to $d/2$ pairs of indices
$(i,i+1)$ with $s(i)=s(i+1)=1$, and since $s$ is completely labeled, all
labels $l(i)\in [d]$ occur at exactly one of these indices. Then the edges
$(l(i),l(i+1))$ form a perfect matching of $G$.

Conversely, let $l:[n]\to [d]$ be a labeling, and let $M$ be a perfect
matching for $G$. Consider a bitstring $s$ with $s(i)=s(i+1)$ for every
$(l(i),l(i+1))\in M$ and $s(i)=0$ otherwise.
Since $M$ is a matching, all the $(l(i),l(i+1))\in M$ are
disjoint, so, considering $s$ as a ``loop,'' every run of $s$ is of even
length, thus satisfying the Gale evenness condition. Since $M$ is perfect,
every vertex $v\in [d]$ is the endpoint of an edge $(l(i),l(i+1))$, so $s$
has exactly $d$ bits equal to $\1$, so it is completely labeled.

We have therefore reduced the problem \gale to {\sc Perfect Matching}, that
by theorem \ref{gale-thm} can be solved in polynomial time.
\end{proof}
\end{theorem}

We give two examples of the construction used in theorem \ref{gale-thm}.

\begin{example}\label{gs-pm-ex}

\todo[inline]{example: new, no double labels, graphs!}

Let $l=12343122$ be a string of labels. Then the egdes $e_i$ of the graph
$G$ obtained from the construction in the proof of theorem  \ref{gale-thm}
will be as follow:

\begin{displaymath}
\xymatrix
@M=5pt
{
*+[o][F-]{1} \ar@/^1pc/@{-}[r] |{e_1} &
*+[o][F-]{2} \ar@/^1pc/@{-}[r] |{e_2} &
*+[o][F-]{3} \ar@/^1pc/@{-}[r] |{e_3} &
*+[o][F-]{4} \ar@/^1pc/@{-}[r] |{e_4} &
*+[o][F-]{3} \ar@/^1pc/@{-}[r] |{e_5} &
*+[o][F-]{1} \ar@/^1pc/@{-}[r] |{e_6} &
*+[o][F-]{2} \ar@/^1pc/@{-}[r] |{cycle} &
*+[o][F-]{2} \ar@/^2pc/@{-}[lllllll] |{e_7} \\
}
\end{displaymath}

Given the vertices $v\in [4]$, the graph $G$ will be:

\begin{displaymath}
\xymatrix
@M=5pt
{
*+[o][F-]{1}
\ar@/^1pc/@{-}[rr] |{e_1}
\ar@{-}[rr] |{e_6}
\ar@/_1pc/@{-}[rr] |{e_7}
\ar@{-}[ddrr] |{e_5}
& &
*+[o][F-]{2}
\ar@{-}[dd] |{e_2}
\\
\\
*+[o][F-]{3}
\ar@{-}[rr] |{e_3}
\ar@/_1pc/@{-}[rr] |{e_4}
& &
*+[o][F-]{4}
}
\end{displaymath}

A perfect matching for $G$ is given by $M=\{e_1, e_4\}$.

\begin{displaymath}
\xymatrix
@M=5pt
{
*+[o][F-]{1}
\ar@/^1pc/@{-}[rr] |{e_1}
\ar@{.}[rr] |{e_6}
\ar@/_1pc/@{.}[rr] |{e_7}
\ar@{.}[ddrr] |{e_5}
& &
*+[o][F-]{2}
\ar@{.}[dd] |{e_2}
\\
&
&
\\
*+[o][F-]{3}
\ar@{.}[rr] |{e_3}
\ar@/_1pc/@{-}[rr] |{e_4}
& &
*+[o][F-]{4}
}
\end{displaymath}

In turn, this corresponds to the completely labeled Gale string $11011000$.

\begin{displaymath}
\xymatrix
@M=5pt
{
*+<1em>[F-:<12pt>]{\txt{1 \\ {\bf 1}}} \ar@/^1pc/@{-}[r] |{{\bf e_1}} &
*+<1em>[F-:<12pt>]{\txt{2 \\ {\bf 1}}} \ar@/^1pc/@{-}[r] |{e_2} &
*+<1em>[F-:<12pt>]{\txt{3 \\ 0}} \ar@/^1pc/@{-}[r] |{e_3} &
*+<1em>[F-:<12pt>]{\txt{4 \\ {\bf 1}}} \ar@/^1pc/@{-}[r] |{{\bf e_4}} &
*+<1em>[F-:<12pt>]{\txt{3 \\ {\bf 1}}} \ar@/^1pc/@{-}[r] |{e_5} &
*+<1em>[F-:<12pt>]{\txt{1 \\ 0}} \ar@/^1pc/@{-}[r] |{e_6} &
*+<1em>[F-:<12pt>]{\txt{2 \\ 0}} \ar@/^1pc/@{-}[r] |{cycle} &
*+<1em>[F-:<12pt>]{\txt{2 \\ 0}} \ar@/^3pc/@{-}[lllllll] |{e_7} \\
}
\end{displaymath}

\end{example}

A perfect matching for a graph, and therefore a Gale string for a labeling,
is not always possible, as shown in the next example.

\begin{example}

\todo[inline]{example: new graphs!}

Let us consider the labeling $l=121314$. The associated graph $G$ will be

\begin{displaymath}
\xymatrix
@M=5pt
{
1
\ar@/^1pc/@{-}[rr] |{e_1}
\ar@{-}[rr] |{e_2}
\ar@/^1pc/@{-}[ddrr] |{e_3}
\ar@{-}[ddrr] |{e_4}
\ar@{-}[dd] |{e_5}
\ar@/_1pc/@{-}[dd] |{e_6}
& & 2
\\
\\
3
& & 4
}
\end{displaymath}

Since there aren't any disjoint edges, it's not possible to find a perfect
matching for $G$. Analogously, we have seen in example \ref{no-clgs} that
there isn't any possible completely labeled Gale string for the labeling
$l$.
\end{example}

We finally extend the proof of theorem \ref{gale-thm} to \anothergale.

\begin{theorem}\label{anothergale-thm}
The problem \anothergale\ is solvable in polynomial time.

\begin{proof}
Let $G=(V,E)$ be the graph corresponding to the labeling $l:[n]\to [d]$ as
in the proof of theorem \ref{gale-thm} and let $M$ be the perfect matching
of $G$ corresponding to the completely labeled Gale string $s\in G(d,n)$.

If there are two edges $e,e'\in E$ such that $e\in M$, both $e$ and $e'$
have endpoints $l(i),l(i+1)$, but $e\neq e'$ (recall that $G$ can be a
multigraph), the matching $M'=(M\setminus \{ e \})\cup\{ e' \}$
is perfect.
The corresponding completely labeled Gale string
$s'\in G(d,n)$ satisfies $s'\neq s$, since in $s$ the \1's corresponding
to the labels $l(i),l(i+1)$ are in the positions given by the edge
$e$, while in $s'$ they are in the positions given by $e'\neq e$.
It takes time $d/2$ to check all edges of $M$, the time required
is still polynomial.

We now assume that all the edges in every perfect matching $M$ for $G$
don't have a parallel edge.
Since by theorem \ref{lhg-works-thm} there is an even number of
completely labeled Gale strings, the existence of $s$ guarantees the
existence of another completely labeled Gale string $s'\neq s$ and the
corresponding perfect matching $M'\neq M$.
Since $M'\neq M$, there is at least one edge $e'\in M$ such that
$e'\notin M'$. Consider the $d/2$ graphs $G_i=(V,E_i)$, where
$E_i=E\setminus\{ e_i \}$ for $e_i\in M$. Since $V(G)=V(G')$ and
$E(G)\subset E(G')$, every perfect matching for one of these $G_i$
is a perfect matching for $G$ as well.
With a brute force approach, we look for a perfect matching in each $G_i$;
this will be $M'$. Since there are $i\in [d/2]$, the time to find it
will be still polynomial.
\end{proof}
\end{theorem}

We give two examples of the construction of theorem \ref{anothergale-thm}.

\begin{example}

\todo[inline]{example: new graphs!}

We consider the labeling the string of labels $l=1234312$. We have found in
example \ref{gs-pm-ex} the completely labeled Gale string $1101100$,
corresponding to the perfect matching $M=\{e_1, e_4\}$ in the graph $G$.

\begin{displaymath}
\xymatrix
@M=5pt
{
*+[o][F-]{1}
\ar@/^1pc/@{-}[rr] |{e_1}
\ar@{--}[rr] |{e_6}
\ar@/_1pc/@{.}[rr] |{e_7}
\ar@{.}[ddrr] |{e_5}
& &
*+[o][F-]{2}
\ar@{.}[dd] |{e_2}
\\
&
&
\\
*+[o][F-]{3}
\ar@{.}[rr] |{e_3}
\ar@/_1pc/@{-}[rr] |{e_4}
& &
*+[o][F-]{4}
}
\end{displaymath}

If instead of $e_1$ we take the parallel edge $e_6$, the resulting matching
is still perfect.
\end{example}

We now show a case in which every edge in every perfect matching does
not have a parallel one is the following. Note that $G$ is a multigraph:
we are interested in the edges in the matching, not in all the edges of the
graph.

\begin{example}
\todo[inline]{example: new graphs!}

We consider the labeling $l=123142$. There are only two possible perfect
matchings for the corresponding graph: $M=\{e_2,e_4\}$, that corresponds to
the completely labeled Gale string $s=011110$, and $M'=\{e_3,e_5\}$, that
corresponds to $s'=001111$.

\begin{displaymath}
\xymatrix
@M=5pt
{
*+[o][F-]{1}
\ar@/^1pc/@{.}[rr] |{e_6}
\ar@{.}[rr] |{e_1}
\ar@{--}[dd] |{e_4}
\ar@{-}[ddrr] |>>>>>>{e_3}
& &
*+[o][F-]{2}
\ar@{--}[dd] |{e_2}
\ar@{-}[ddll] |>>>>>>{e_5}
\\
&
&
\\
*+[o][F-]{3}
& &
*+[o][F-]{4}
}
\end{displaymath}
\end{example}
