\section{Gale Games}

We now apply Theorem \ref{origgale-thm} to the study of bimatrix games.
\linebreak[4]
Proposition \ref{nash-to-acl} states that 2-{\sc Nash} can be reduced to
{\sc Another Completely Labeled Facet}.
If the polytope $P^\Delta$ in Theorem \ref{unit-vector-dual-thm} is cyclic and we
define a labeling for Gale strings such that a completely labeled Gale
string corresponds to a completely labeled facet of~$P^\Delta$, then
we can study unit vector games and their dual cyclic best
response polytope as Gale strings.
We say that $s\in G(d,n)$ is a {\em completely labeled Gale string}
for some labeling function $l_s:[n]\to[d]$ if
$\{ l_s(i)\,\mid\,s(i)=\1\text{ for }i\in [n] \}=[d]$.
Since $s\in G(d,n)$ has exacty $d$ bits equal to \1, this means that for
each $j\in [d]$ there is exactly one
$i\in [n]$ such that $s(i)=\1$ and $l_s(i)=j$.
It may not always be possible to find a completely labeled Gale string.

\begin{example}
\label{no-clgs}
For $l_s = 121314$, there are no completely labeled Gale strings.
The labels $l_s(i)=2,3,4$ appear only once in $l_s$, so we must have
$s(2)=s(4)=s(6)=1$. We also must have $l_s(i)=1$ for exactly one $i=1,3,5$.
The candidate strings are then $s=\1\10\10\1$, $s'=0\1\1\10\1$,
$s''=0\10\1\1\1$, but none of these satisfies the Gale evenness condition.
\end{example}

A {\em Gale game} is a unit vector game $(U,B)$ where
% U is a matrix : not commas
$U=[e_{l(1)}\cdots e_{l(d)}]$ for some labeling $l:[n]\to [d]$ and
for which the dual of the best response polytope is a cyclic polytope
$P^\Delta=\conv\{ e_1,\ldots,e_d,c_1,\ldots,c_n \}$. Theorem
\ref{unit-vector-dual-thm} gives the labeling (\ref{vert-labeling-unitv})
for the $d+n$ vertices of $P^\Delta$ as
\[
\arraycolsep.2em
\begin{array}{rcll}
l_v(-e_i)&=&i\quad & \text{ for }i\in [d],\\
l_v(c_j)&=&l(j)\quad & \text{ for }j\in [n].
\end{array}
\]
We define the labeling $l_s:[d+n]\to [d]$ of $G(d,n)$ as
\begin{equation}
\label{gs-labeling-unitv}
\arraycolsep.2em
\begin{array}{rcll}
l_s(i)&=&i\quad & \text{ for }i\in [d],\\
l_s(d+j)&=&l(j)\quad & \text{ for }j\in [n].
\end{array}
\end{equation}
Then the Gale strings $s\in G(d,d+n)$ that are completely labeled by $l_s$
correspond exactly to facets of $P^\Delta$ that are completely labeled by $l_v$,
with the
\linebreak[5]
facet $F_0$ corresponding to the ``trivial'' completely labeled
string $\1^d 0^n$.

From this point forward, we will assume that $d$ is even;
we will also assume that the labeling $l_s:[d+n]\to [d]$ satisfies
$l_s(i)\neq l_s(i+1)$. This can be done without loss of generality,
given the following consideration.
Suppose that $l_s(i)=l_s(i + 1)$ for some index $i$, and let $s$ be a
completely labeled Gale string for $l_s$. Then only one of
$s(i)$ and $s(i+1)$ can be equal to \1 (it is possible that both
are equal to~0), so $s(i)s(i+1)$ will never be part of a run of even length that
``interferes'' with the Gale Evenness Condition. Therefore, we can
identify the indices $i$ and $i + 1$.

\begin{example}
\label{c46-123432-ex}
Given the string of labels $l_s=123432$, there are four associated completely
labeled Gale strings in $G(4,6)$:
$s_A=\1\1\1\100$, $s_B=\1\10\1\10$,
\linebreak[4]
$s_C=\100\1\1\1$ and
$s_D=\10\1\10\1$. These correspond to the completely labeled facets
for the labeling shown in Figure \ref{c46-123432-fig} on the left.
\begin{figure}[hbt]
\strut\hfill
\includegraphics[width=50ex]{chapter-2/fig-gale-def/123432-fac-name.pdf}%
\hfill
\small
\begin{tabular}{c | c @{ } c @{ } c @{ } c @{ } c @{ } c @{ } c }
facet & {\bf 1} & {\bf 2} & {\bf 3} & {\bf 4} & {\bf 3} & {\bf 2}\\
\hline
{\bf A} & \1 & \1 & \1 & \1 & 0 & 0 \\
{\bf B} & \1 & \1 & 0 & \1 & \1 & 0 \\
{\bf C} & \1 & 0 & 0 & \1 & \1 & \1 \\
{\bf D} & \1 & 0 & \1 & \1 & 0 & \1
\end{tabular}
\hfill\strut
\caption[A labeling of $C_4(6)$ and its completely labeled facets]{%
The cyclic polytope $C_4(6)$. The labeling of the vertices corresponds to
the labeling of $G(4,6)$ given by $l_s=123432$.
}
\label{c46-123432-fig}
\end{figure}
\end{example}

\clearpage

We can now define the problem \anothergale\ as in Table
\ref{another-gale}, where for brevity we write $n$ instead
of $d+n$.
% you have to alert the reader to changes in notation

\begin{problem}
{\anothergale}
{A labeling $l:[n]\to[d]$, where $d$ is even and $d<n$.
A Gale string $s\in G(d,n)$, completely labeled by $l$.}
{A Gale string $s'\in G(d,n)$, completely labeled by $l$,
such that $s' \neq s$.}
\label{another-gale}
\end{problem}

It takes polynomial time to translate
the facets of the cyclic polytope
\linebreak[6]
$C_d(d+n)$ % can be translated in polynomial time
into the corresponding Gale strings in $G(d,d+n)$, following
the proof of Theorem~\ref{origgale-thm}.
Furthermore, defining the labeling $l_s$ from the labeling $l_v$ also takes
polynomial time: for the labels $i\in [d]$ it is immediate, for the labels
$d+j$, where $j\in [n]$, we have to check the $d\times n$ matrix $U$
of the imitation game.
Therefore, by Proposition \ref{nash-to-acl}, we have a reduction from
{\sc Gale Nash} to \anothergale.

\begin{proposition}
\label{galenash-to-another-gale}
{\sc Gale Nash} is polynomial-time reducible to \anothergale.
\end{proposition}
