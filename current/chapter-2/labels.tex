\section{Bimatrix Games and Labels}
\label{labels-section}

Let $n,m\in\naturals$ with $m\leq n$.
A {\em labeling} is a function $l:[n]\to [m]$; an $m$-uple
$x=(x_1,\ldots,x_m)\in [n]^m$ is {\em completely labeled} if
each label $j\in [m]$ appears once and only once in $(l(x_1),\ldots,l(x_m))$;
formally, if $\{ i\in [m]\ |\ l(x_j)=i\text{ for some }j\in [m] \}=[m]$.

Let $(A,B)$ be bimatrix game, and let $X$ and $Y$ be the mixed strategy
simplices of respectively player 1 and 2; that is
\begin{equation}\label{mixed-strat-simplices}
X = \{ x\in\reals^m\ |\ x\geq\0,\ \1\T x = 1 \};\quad
Y = \{ y\in\reals^n\ |\ y\geq\0,\ \1\T y = 1 \}.
\end{equation}
A {\em labeling} of the game is then given as follows:
\begin{enumerate}
\item the $m$ pure strategies of player 1 are identified by $1,\ldots,m$;
\item the $n$ pure strategies of player 2 are identified by $m+1,\ldots,m+n$;
\item each mixed strategy $x\in X$ of player 1 has
    \begin{itemize}
    \item label $i$ for each $i\in [m]$ such that $x_i = 0$, that is if in
    $x$ player 1 does not play her $i$-th pure strategy,
    \item label $m + j$ for each $j\in [n]$ such that the $j$-th pure strategy
    of player 2 is a best response to $x$;
    \end{itemize}
\item each mixed strategy $y\in Y$ of player 2 has
    \begin{itemize}
    \item label $m + j$ for each $j\in [n]$ such that $y_j = 0$, that is if in
    $y$ player 2 does not play his $j$-th pure strategy,
    \item label $i$ for each $i\in [m]$ such that the $i$-th pure strategy
    of player 1 is a best response to $y$.
    \end{itemize}
\end{enumerate}

The labeling of mixed strategy profiles can be used to characterize the
Nash equilibria of the game.

\begin{theorem}{\rm (Shapley \cite{shapley})}\label{comp-label-thm}
Let $(x,y)\in X\times Y$; then $(x,y)$ is a Nash equilibrium of the bimatrix
game $(A,B)$ if and only if $(x,y)$ is completely labeled.

\begin{proof}
The mixed strategy $x\in X$ has label $m + j$ for some $j\in [n]$ if and
only if the $j$-th pure strategy of player 2 is a best response to $x$; this,
in turn, is a necessary and sufficient condition for player 2 to play his
$j$-th strategy at an equilibrium against $x$. Therefore, at an equilibrium
$(x,y)$ all labels $m + j$, with $j\in [n]$, will appear either as labels of
$x$ or of $y$, and analogously for the strategies $y\in Y$.
\end{proof}
\end{theorem}

An useful graphical representation of labels on the simplices $X$ and $Y$
is done by labeling the outside of each simplex according to the player's own
pure strategies that are {\em not} played, and by subdividing its interior
in closed polyhedral sets corresponding to the other player's pure
best response strategies, called {\em best response regions}.
We give an example of this construction.

\begin{example}
\label{br-game-ex}
Consider the $3\times 3$ game $(A,B)$ with
\begin{equation}
\label{AB}
A = \left(\begin{matrix}1&0&0\\ 0&1&0\\
0&0&1\end{matrix}\right),
\qquad
B = \left(\begin{matrix}0&2&4\\ 3&2&0\\
0&2&0\end{matrix}\right).
\end{equation}
The pure strategies of player~1 are labeled as $1,2,3$; the pure strategies
of player~2 are labeled as $4,5,6$.
In the following figures the labels of the strategies will be represented
as circled numbers.
Figure~\ref{br-regions-fig} shows $X$ and $Y$: the exterior facets are
labeled with the pure strategy on the opposite vertex, where only that pure
strategy is played; the interior is covered by the best response regions,
by to the other player's pure best response strategies.
For example, the best-response region in $Y$ with label~$1$
is the set of those $(y_1,y_2,y_3)$ so that $y_1\ge y_2$ and
$y_1\ge y_3$.
There is only one pair $(x,y)$
that is completely labeled, namely $x=(\frac13,\frac23,0)$
with labels $3,4,5$ and $y=(\frac12,\frac12,0)$ with labels
$1,2,6$, so this is the only Nash equilibrium of the game.
\begin{figure}[hbt]
\strut\hfill
\includegraphics[width=30ex]{chapter-2/fig-labels/Xnondeg.pdf}%
\hfill
%\vrule height 52mm ~ \vrule height 27ex
\hfill
\includegraphics[width=30ex]{chapter-2/fig-labels/Yimit.pdf}%
\hfill\strut
\caption[Labeled mixed strategy sets]{%
Mixed strategy sets $X$ and $Y$ of game (\ref{AB}) and their labeled
best response regions.
}
\label{br-regions-fig}
\end{figure}
\end{example}

The point of view of best response regions can be translated to an equivalent
construction on polytopes.
The first step is to notice that the best-response regions can be obtained as
projections on $X$ and $Y$ of the {\em best-response facets} of
the polyhedra
\begin{equation}\label{br-polyhedron}
\overline{P} = \{ (x,v)\in X\times\reals\ |\ B\T x\leq\1 v \};\quad
\overline{Q} = \{ (y,u)\in Y\times\reals\ |\ A y\leq\1 u \}.
\end{equation}
In $\overline{P}$, these facets are the points $(x,v)\in X\times\reals$
such that $(B\T x)_j = v$, which in turn correspond to the strategies
$x\in X$ of player 1 that give exactly payoff $v$ to player 2 when
he plays strategy $j$; the projection of the facet defined by
$(B\T x)_j = v$ to $X$ has then label $j$.
Analogously, the facet of $\overline{Q}$ given by the points
$(y,u)\in Y\times\reals$ such that $(A y)_i = u$ will project to the
best-response region of $Y$ with label $i$.

\begin{example}
In Example \ref{br-game-ex}, the inequalities $B\T x\le\1v$ are
\begin{eqnarray*}
3x_2 & \leq v; \\
2x_1+ 2x_2+ 2x_3 & \leq v \\
4x_1 & \leq v.
\end{eqnarray*}
Figure~\ref{br-polyhedron-fig} shows the best-response facets of
$\overline{P}$ and their projection to $X$ by ignoring the payoff
variable~$v$, which gives the subdivision of $X$ into best-response
regions of Figure \ref{br-regions-fig}.
\begin{figure}[hbt]
\strut\hfill
\includegraphics[width=45ex]{chapter-2/fig-labels/upenv.pdf}%
\hfill\strut
\caption[Best response facets]{%
Best response polyhedron $\overline{P}$ of game (\ref{AB}).}
\label{br-polyhedron-fig}
\end{figure}
\end{example}

Given the assumptions on non-negativity of $A$ and $B\T$, we can
change coordinates to $x_i / v$ and $y_j / u$ and replace $\overline{P}$ and
$\overline{Q}$ with the {\em best-response polytopes}
\begin{equation}
\label{br-polytopes}
P = \{ x\in\reals^m\ |\ x\geq\0,\ B\T x\leq\1 \};\quad
Q = \{ y\in\reals^n\ |\ y\geq\0,\ A y\leq\1 \}.
\end{equation}

The polytope $P$ is the intersection the $m+n$ half spaces
corresponding to either player 1 avoiding her $i$-th pure
strategy, with $i\in [m]$, or to a best response of player 2 that gives
non-zero probability to his $j$-th strategy, with $j\in [n]$.
Formally, a point $x\in P$ has label $k$ if and only if either
$x_k = 0$ for $k\in [m]$ or $(B\T x)_k = 0$ for $k\in [n]$, and
a point in $Q$ has label $k$ if and only if either
$y_k = 0$ for $k\in [n]$ or $(A y)_k$ for $k\in [m]$.
Then a point $(x,y)\in P\times Q$ is completely labeled if and only if
it satisfies the {\em complementarity condition}
\begin{eqnarray}
x_i = 0\text{ or }(Ay)_i = 1\quad\text{ for all }i\in [m];\nonumber \\
y_j = 0\text{ or }(B\T x)_j\quad\text{ for all }j\in [n].
\end{eqnarray}
Therefore, if $(x,y)\in P\times Q$ is completely labeled, either the
corresponding point in $\overline{P}\times\overline{Q}$
is a Nash equilibrium, or $(x,y)=(\0,\0)$; we will refer to the latter case
as {\em artificial equilibrium}.

\begin{example}
Keeping on with Example \ref{br-game-ex}, the best response
polyhedron $\overline{P}$ of Figure \ref{br-polyhedron-fig} becomes
the best response polytope of Figure \ref{br-polytopes-fig}.
Note that the vertex $(\0,\0)$ is completely labeled, since it has
labels $1,2,3$ in $P$ and labels $4,5,6$ in $Q$.
\begin{figure}[htb]
\strut\hfill
\includegraphics[width=35ex]{chapter-2/fig-labels/polytope.pdf}%
\hfill\strut
\caption[Best response polytope]{%
Best response polytope of game (\ref{AB}).
}
\label{br-polytopes-fig}
\end{figure}
\end{example}

We will now see some special cases of games connected by
polynomial-time reductions; our goal is to find a class of games that
captures the complexity of 2-{\sc Nash}.
First of all, we note that any bimatrix game can be ``symmetrized'';
the result is due to Gale, Kuhn and Tucker \cite{gale-kuhn-tucker}
for zero-sum games and its extension to non-zero-sum games is a folklore
result.

\begin{proposition}\label{symmetrize-c}
Let $(A,B)$ be a bimatrix game and $(x,y)$ be one of its Nash equilibria.
Then $(z,z)$, where $z=(x,y)$, is a Nash equilibrium of the symmetric game
$(C,C\T)$, where
\begin{equation}
C = \left(
    \begin{array}{cc}
    0 & A \\
    B\T & 0
    \end{array}
    \right).
\end{equation}
\end{proposition}


McLennan and Tourky \cite{mclennan-tourky} have proven a result in the
opposite direction of proposition \ref{symmetrize-c}: any symmetric game
can be translated into a bimatrix game of the form $(I,B)$, called
{\em imitiation game}.
Since it takes polynomial time in the size of a matrix to calculate its
transpose, we have a polynomial-time reduction from 2-{\sc Nash} to
{\sc Imitation Nash}.

\begin{theorem}{\rm (McLennan and Tourky \cite{mclennan-tourky})}
\label{imitation-thm}
The pair $(x,x)$ is a symmetric Nash equilibrium of the symmetric bimatrix
game $(C,C\T)$ if and only if there is some $y$ such that $(x,y)$ is a
Nash equilibrium of the imitation game $(I,B)$ with $B=C\T$.
\end{theorem}

In any Nash equilibrium of $(I,B)$, the mixed strategy $x$
of player 1 corresponds exactly to the symmetric equilibrium $(x,x)$ in
the symmetric game defined by the payoff matrix of player 2.
Note, though, that Theorem \ref{imitation-thm} applies to the symmetric
equilibria of the symmetric game, but not to all its equilibria; there
could be non-symmetric equilibria of $(C,C\T)$ that are not found
through the imitation game.
We see an example illustrating this.

\begin{example}
As an example, consider the symmetric game $(C,C\T)$ with
\begin{equation}
\label{C}
C = \left(\begin{matrix}0&3&0\\ 2&2&2\\
4&0&0\end{matrix}\right),
\qquad
C\T = \left(\begin{matrix}0&2&4\\ 3&2&0\\
0&2&0\end{matrix}\right),
\end{equation}
so that the corresponding imitation game is $(I,C\T)=(A,B)$, as in (\ref{AB}).
Figure~\ref{imit-fig} shows the labeled mixed-strategy simplices $X$ and $Y$
for the game \ref{C}; since the game is symmetric, only the labels
are different.
In addition to the symmetric equilibrium $(x,x)$ where
$x=(\frac13,\frac23,0)$, the game has two non-symmetric equilibria in $(a,b)$
and $(b,a)$ with $a=(\frac12,\frac12,0)$ and $b=(0,\frac23,\frac13)$;
the corresponding imitation game $(A,B)$ has only one
equilibrium $(x,y)$, corresponding to $(x,x)$.
\begin{figure}[hbt]
\strut\hfill
\includegraphics[width=30ex]{chapter-2/fig-labels/Xnondeg.pdf}%
\hfill
\includegraphics[width=30ex]{chapter-2/fig-labels/Ynondeg.pdf}%
\hfill\strut
\caption[Best response regions of a symmetric game]{%
Best response regions of the symmetric game (\ref{C}).
}
\label{imit-fig}
\end{figure}
\end{example}

The characterization of Nash equilibria as completely labeled pairs $(x,y)$
holds for arbitrary bimatrix games. From now on, we will also impose a
further condition: that all points in $P$ have at most $m$ labels,
and all points in $Q$ have at most $n$ labels. These games are called
{\em nondegenerate}; since any game can be made nondegenerate by
lexicographic perturbation (see von Stengel \cite{vs-agt}), we can impose
the nondegeneracy condition without loss of generality.
In an equilibrium $(x,y)$ of a nondegenerate game each label appears
exactly once; this also means that the number of pure best response
strategies against a mixed strategy is never larger than the size of
the support of that mixed strategy.
Geometrically, this means that no point of the best response polytope $P$
lies on more than $m$ facets and no point of the best response polytope $Q$
lies on more than $n$ facets, so both $P$ and $Q$ are simple.
Furthermore, a point of
$P$ has exactly $m$ labels if and only if it is a vertex, and a point
of $Q$ has exactly $n$ labels if and only if it is a vertex; therefore,
all completely labeled points $(x,y)$ are vertices, and Nash
equilibria are isolated points.

\begin{example}
An example of degenerate game is given by $(C,C\T)$ where
\begin{equation}
\label{Cdeg}
C = \left(\begin{matrix}0&4&0\\ 2&2&2\\
4&0&0\end{matrix}\right),
\qquad
C\T = \left(\begin{matrix}0&2&4\\ 4&2&0\\
0&2&0\end{matrix}\right).
\end{equation}
As it is shown in Figure \ref{deg-fig-1}, the mixed strategy
$x=(\frac12,\frac12,0)$, that also defines the
unique symmetric equilibrium $(x,x)$ of the game, has three pure best
responses.
\begin{figure}[hbt]
\strut\hfill
\includegraphics[width=30ex]{chapter-2/fig-labels/Xsymmdeg.pdf}%
\hfill\strut
\caption[A degenerate symmetric game]{%
Best-response regions of the degenerate symmetric game (\ref{Cdeg}).
Label 3 appears twice at the unique symmetric equilibrium.
}
\label{deg-fig-1}
\end{figure}
Note that the Nash equilibria $(x,y)$ of the imitation game
$(I,C\T)$ are not unique, since any convex
combination of $(\frac12,\frac12,0)$ and $(\frac13,\frac13,\frac13)$
can be chosen for~$y$, as shown in Figure~\ref{deg-fig-2}.
\begin{figure}[hbt]
\strut\hfill
\hfill
\includegraphics[width=30ex]{chapter-2/fig-labels/Yimitdeg.pdf}%
\hfill\strut
\caption[A degenerate imitation game]{%
Labeled mixed-strategy sets for the imitation game $(I,C\T)$. The
equilibria $(x,y)$ corresponding to the unique symmetric equilibrium of
the symmetric game (\ref{Cdeg}) are not unique.
}
\label{deg-fig-2}
\end{figure}
\end{example}

A generalization of imitation games is the class of
{\em unit vector games}, introcuced by Balthasar \cite{balthasar};
they are defined as bimatrix games of the form $(U,B)$ where the
columns of the matrix $U$ are unit vectors.
By the results above, finding a Nash equilibrium of a bimatrix game is
at least as hard as finding a Nash equilibrium of a unit vector game;
that is, 2-{\sc Nash} reduces to {\sc Unit Vector Nash}.
In unit vector games, the problem of finding a completely labeled vertex
of the polytope $P\times Q$ can be translated to the problem of finding a
completely labeled vertex of one simple polytope that encodes all the game.
We first give this result as in Savani and von Stengel \cite{uvg};
we will later see the version in Balthasar \cite{balthasar}.

\begin{theorem}{\rm (Savani and von Stengel \cite{uvg})}
\label{unit-vector-thm}
Let $l:[n]\to [m]$, and let $(U,B)$ be the unit vector game with
$U=(e_{l(1)}\ \cdots\ e_{l(n)})$. Let $N_i = \{ j\in [n]\ |\ l(j)=i \}$ for
$i\in [m]$, and consider the polytopes $P^l$ and $Q^l$
\begin{eqnarray}
P^l = \{ x\in\reals^m\ |\ x\geq\0,\ B\T x\leq\1 \}; \label{p-l} \\
Q^l = \{ y\in\reals^n\ |\ y\geq\0,\
\sum_{\substack{j\in N_i \\ i\in [m]}} y_j\leq 1 \}. \label{q-l} \nonumber
\end{eqnarray}
Let $l_f$ be the labeling of the facets of $P^l$ defined as follows:
\begin{eqnarray}\label{facet-labeling-unitv}
x_i\geq 0\text{ has label }i\text{ for }i\in [m]; \nonumber \\
(B\T x)_j \leq 1\text{ has label }l(j)\text{ for }j\in [n].
\end{eqnarray}

Then $x\in P^l$ is a completely labeled vertex of $P^l\setminus\{\0\}$
if and only if there is some $y\in Q^l$ such that, after scaling,
the pair $(x,y)$ is a Nash equilibrium of $(U,B)$

\begin{proof}
Let $P,Q$ be the best response polytopes of $(U,B)$ as in \ref{br-polytopes},
and let $(x,y)\in P\times Q\setminus\{ \0,\0 \}$ be a Nash equilibrium of
$(U,B)$. Then $(x,y)$ is completely labeled in $[m + n]$; so if
$x_i=0$, then $x$ has label $i\in m$.
If $x_i > 0$, then $y$ has label $i$, so $(Uy)_i = 1$;
then for some $j\in [n]$ we have $y_j > 0$ and $U_j = e_i$; that is,
we have $y_j > 0$ and $l(j)=i$ for some $j\in [n]$.
Since $y_j > 0$, $x\in P$ has label $m+j$; then, $(B\T x)_j = 1$;
therefore $x\in P^l$ has label $l(j) = i$.
Hence, $x$ is a completely labeled vertex of $P^l$.

Conversely, let $x\in P^l\setminus \{ \0 \}$ be completely labeled.
If $x_i > 0$, then there is $j\in [m]$ such that $(B\T x) = j$ and
$l(j) = i$; that is, $j\in N_i$. For all $i$ such that $x_i > 0$,
define $y$ as follows: $y_j = 1$;
$y_h = 0$ for all $h\in N_i\setminus \{ j \}$.
Then $(x,y)\in P\times Q$ is completely labeled.
\end{proof}
\end{theorem}

\begin{example}
The game in Example \ref{br-game-ex} is a unit vector game, with $l(i)=i$.
In the polytope $P^l$ of Figure~\ref{p-l-fig} the labels 4, 5 and 6 of the
best response polytope are replaced by 1, 2 and 3, since
the corresponding columns of $A$ are the unit vectors $e_1,e_2,e_3$.
Apart from the origin $\0$, the only completely labeled
point of $P^l$ is $x$, corresponding to the unique equilibrium of game
(\ref{AB}).
\begin{figure}[hbt]
\strut\hfill
\includegraphics[width=30ex]{chapter-2/fig-labels/p-l-color.pdf}%
\hfill\strut
\caption[The polytope $P^l$ of a unit vector game]{%
The polytope $P^l$ of the unit vector game (\ref{AB}).
Label~$1$ refers to the hidden facet.
}
\label{p-l-fig}
\end{figure}
\end{example}

We will now move on the dual version of Theorem \ref{unit-vector-thm},
as given in Balthasar \cite{balthasar}.
We traslate the polytope $P^l$ of (\ref{p-l}) to
$P = \{ x - \1\ |\ x\in P^l \}$, multiplying all payoffs in $B$ by a
constant, if necessary, so that $\0$ is in the interior of $P$.
We have that
\begin{align*}
P & = \{ x + \1\geq\0,\ (x + \1)\T B\leq\1 \} = \\
  & = \{ x\in\reals^m\ |\ -x_i\leq 1\text{ for }i\in [m],\
    x\T (b_j/(1 - \1\T b_j)) \leq 1\text{ for }j\in [n] \}.
\end{align*}
The polar of $P$ is then
\begin{equation}\label{p-l-dual}
P^\Delta = \conv(\{−e_i\ |\ i\in [m] \}\ \cup\ \{ c_j\ |\ j\in [n] \})
\end{equation}
where $c_j=b_j/(1 - \1\T b_j)$.
Since $P$ is a simple polytope with $\0$ in its interior,
$P^{\Delta\Delta}=P$. Furthermore, $P^\Delta$ is
simplicial, therefore the facets of $P^\Delta$ correspond to
the vertices of $P$ and vice versa.
We label the vertices
of $P^\Delta$ as the corresponding facets in $P^l$, so the completely
labeled facets of $P^\Delta$ correspond to the completely
labeled vertices of $P^l$.
In particular, the facet corresponding to $\0$ is
\begin{equation}
\label{f-0}
F_0 = \{ x\in P^\Delta\ |\ -\1\T x = 1 \} = \conv\{ e_i\ |\ i\in [m] \}.
\end{equation}
Theorem \ref{unit-vector-thm} then translates into the following.

\begin{theorem}{\rm (Balthasar \cite{balthasar})}
\label{unit-vector-dual-thm}
Let $Q$ be a labeled $m$-dimensional simplicial polytope with \0 in
its interior and vertices $e_1,\ldots,e_m,c_1,\ldots,c_n$ such that
(\ref{f-0}) is a facet of $Q$.
Let $(U,B)$ be a unit vector game, with $U=(e_{l(1)}\ \cdots\ e_{l(n)})$
for a labeling $l:[n]\to [m]$ and $B = (b_1\ \cdots\ b_n)$, where
$b_j = c_j/(1 + \1\T c_j)$ for $j\in [n]$. Let $l_v$ be the labeling
of the vertices of $Q$ given by
\begin{eqnarray}\label{vert-labeling-unitv}
l_v(-e_i)=i\text{ for }i\in [m]; \nonumber \\
l_v(c_j)=l(j)\text{ for }j\in [n].
\end{eqnarray}

Then a facet $F\neq F_0$ of $Q$ with normal vector $v$ is completely
labeled if and only if $(x,y)$ is a Nash equilibrium of $(U,B)$, where
$x = (v + \1) / (\1\T (v + \1))$,
so $x_i = 0$ if and only if $−e_i\in F$ for $i\in [m]$
and the mixed strategy $y$ is the uniform distribution on the set of
the pure best replies to $x$, which in turn correspond to $j\in [n]$
such that $c_j$ is a vertex of $F$.
\end{theorem}

As in Theorem \ref{unit-vector-thm} we have a correspondence between
completely labeled vertices of $P^l$ and equilibria of the unit vector game
$(U,B)$ with the ``artificial'' equilibrium corresponding to the vertex $\0$,
in Theorem \ref{unit-vector-dual-thm} we have a correspondence between
the completely labeled facets of the polytope $Q$
and equilibria of $(U,B)$ with the ``artificial'' equilibrium
corresponding to the facet $F_0$ in (\ref{f-0}).

Given a bimatrix game $(A,B)$, it takes polynomial time
to write and solve the linear equations defining its best response polyhedra
$\overline{P},\overline{Q}$ and its best response polytopes $P,Q$.
It also take polynomial time to label $\overline{P},\overline{Q}$ and $P,Q$.
Analogously, given a unit vector game $(U,B)$, it takes polynomial time
to construct and label the polytope $P^l$.
Therefore, Theorem \ref{unit-vector-thm} implies a polynomial time
reduction from the problem 2-{\sc Nash} to the
problem {\sc Another Completely Labeled Vertex}.

\begin{fctproblem}
{Another Completely Labeled Vertex}
{A simple $m$-dimensional polytope $P$ with $m+n$ facets;
a labeling $l_f:[m+n]\to [n]$;
a facet $F_0$ of $P$, completely labeled by $l_f$.}
{A facet $F\neq F_0$ of $S$, completely labeled by $l$.}
\end{fctproblem}

\begin{proposition}
\label{2nash-to-aclv}
{\sc 2-Nash} reduces in polynomial time to
{\sc Another Completely Labeled Vertex}.
\end{proposition}

Theorem \ref{unit-vector-dual-thm} gives the dual of Proposition
\ref{2nash-to-aclv}; since the construction of the polar polytope from
the original one is also polynomial, the reduction is to the problem
{\sc Another Completely Labeled Facet}.

\begin{fctproblem}
{Another Completely Labeled Facet}
{A simplicial $m$-dimensional polytope $Q$ with $m+n$ vertices;
a labeling $l_v:[m+n]\to [n]$;
a facet $F_0$ of $Q$, completely labeled by $l_v$.}
{A facet $F\neq F_0$ of $S$ completely labeled by $l_v$.}
\end{fctproblem}

\begin{proposition}
\label{2nash-to-aclf}
{\sc 2-Nash} reduces in polynomial time to
{\sc Another Completely Labeled Facet}.
\end{proposition}
