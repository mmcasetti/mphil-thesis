\section{Cyclic Polytopes and Gale Strings}

We now apply the results of the previous section to unit vector games
for which the best response polytope is the dual of a cyclic polytope.
These polytopes are characterized by their representation as
a combinatorial structure, called Gale strings.
We will first give the definition of cyclic polytope,
then of Gale string, then the theorem by Gale \cite{gale} that shows the
equivalence of the two representations.

The {\em moment curve} in dimension $d$ is defined as
\begin{equation}
\mu_d:\reals\longrightarrow\reals^d\qquad\qquad
\mu_d:t\longmapsto (t,t^2,\ldots,t^d)\T .
\end{equation}
The {\em cyclic polytope} $C_d(n)$ in dimension~$d$ with $n$
vertices, where $n>d$, is given as the convex hull of any
$n$ points on the moment curve, that is, by $n$ arbitrary reals
$t_1,\ldots,t_n$, where $ t_1<\cdots<t_n$, according to
\begin{equation}
\label{cycdef}
% affine independence is automatic
%C_d(n)=\conv\{ \mu_d(t_i)\ \mid\\text{ affinely independent } \}.
C_d(n)=\conv\{\,\mu_d(t_i)\,\mid\,1\le i\le n \}\,.
\end{equation}

\begin{example}
\label{cyc36-ex}
Figure \ref{cyc36-fig} shows the cyclic polytope in dimension 3 with 6 facets.

\begin{figure}[hbt]
\strut\hfill
\includegraphics[width=28ex]{chapter-2/fig-gale-def/cyc36.pdf}
\hfill\strut
\caption[The cyclic polytope $C_3(6)$]{The cyclic polytope $C_3(6)$.}
\label{cyc36-fig}
\end{figure}
\end{example}

Given $k\in\naturals$ and a set $S$, we can represent the function
$f:[k]\to S$ as the string $s=s(1)s(2)\cdots s(k)$; we have a {\em bitstring}
if $S=\{0,1\}$. A maximal substring of consecutive
1's in a bitstring is called a {\em run}.
A run is called even if its length is even, and odd if its
length is odd.
% Does not seem to be needed
% An {\em interior run} is bounded on both sides by 0's.
We will use the notation $\1^k$ for a run of
%\linebreak[5]
length~$k$ and $0^k$ for a string of 0's of
length $k$. A {\em Gale string of length $n$ and
dimension $d$}, where $n>d$, is a bitstring $s$ that satisfies the
following conditions:
\begin{enumerate}
\item exactly $d$ bits of $s$ are equal to $\1$;
\item ({\em Gale evenness condition})
$
\qquad 0\1^k0\text{ is a substring of }s\quad
{\Rightarrow}\quad
k\text{ is even.}
$
\end{enumerate}
Let $G(d,n)$ be the set of these Gale strings.

In general, the Gale evenness condition allows for Gale strings that start
or end with an odd-length run; if $d$ is even then $s$ can start with
an odd run if and only if it ends with an odd run.
When $d$ is even, we can therefore see the Gale strings in $G(d,n)$
as ``loops''
obtained by ``gluing together'' the endpoints of the strings; then
all runs on the loops are even.
Formally, we can see the bit positions in a Gale string $s\in G(d,n)$ with
$d$ even as equivalence classes
%\linebreak[4]
modulo~$n$.

\begin{example}\label{gs-example}
As an example for even $d$, we have
\begin{align*}
G(4,6) = \{ & \1\1\1\100, \1\1\100\1, \1\100\1\1, \100\1\1\1, 00\1\1\1\1, \\
            & 0\1\1\1\10, \1\10\1\10, \10\1\10\1, 0\1\10\1\1 \}
\end{align*}
The strings $\1\1\1\100$, $\1\1\100\1$, $\1\100\1\1$, $\100\1\1\1$,
$00\1\1\1\1$ and $0\1\1\1\10$ are
equivalent under a cyclic shift (if considering the strings as ``loops'', the
$\1$'s are all consecutive), as are the strings $\1\10\1\10$, $\10\1\10\1$
and $0\1\10\1\1$ (two runs of two $\1$'s separated by a single $0$).
As an example for odd $d$, we have
\[
G(3,5) = \{ \1\1\100, \10\1\10, \100\1\1, \1\100\1,
0\1\10\1, 00\1\1\1 \}\,.
\]
Notice that because $d$ is odd, a cyclic shift is
not allowed: $0\10\1\1$ is a shift of $\10\1\10$ but it is not a Gale
string.
\end{example}

The relation between cyclic polytopes and Gale strings was given by
% use the unbreakable space ~, not linebreaks! \linebreak[5]
Gale~\cite{gale}.

\begin{theorem}{\rm (Gale \cite{gale}.)}
\label{origgale-thm}
For any $d,n\in\naturals$, where $n>d$,
a set $F$ is a facet of $G_d(n)$ if and only if 
\begin{equation}
\label{facet-gs}
F = \conv\{ \mu(t_j)\,\mid\, s(j)=1 \quad\text{ for }s\in G(d,n) \}.
\end{equation}
\end{theorem}

\begin{proof}
First, a hyperplane in $\reals^d$ of the form
$\{x\in\reals^d\mid a\T x=a_0\}$ for some nonzero vector
$a=(a_1,\ldots,a_d)\T$ can contain at most $d$ points on the
moment curve, because otherwise the polynomial equation
with a polynomial of degree~$d$ given by
$-a_0+a_1t+a_2t^2+\cdots a_dt^d=0$ would have more than $d$
roots~$t$.
For the same reason, it is easy to see that any $d$ points
on the moment curve are affinely independent and define a
unique hyperplane through them, which the moment curve {\em
crosses} at these points (if it only touched the hyperplane,
then a slight perturbation of the hyperplane would contain
$d+1$ points on the moment curve).
%% Consider $C_d(n)$ as defined in
% \[
% C_d(n) = %
% \conv\{ \mu_d(t_j)\ \mid\ t_1 < \ldots < t_n \text{ affinely independent} \}.
% \]
Let $\overline{t_{1}} < \cdots < \overline{t_{d}}$ be a
choice of $d$ of the $t_j$'s in the definition
(\ref{cycdef}) of $C_d(n)$, and consider the hyperplane $H$
that contains the points $\mu_d(\overline{t_1})$, \ldots, $\mu_d(\overline{t_d})$.
The moment curve crosses $H$ at these points.
% sign does not make sense. The scalar product of \mu would
% have to be taken with respect to the normal vector of H.
That is, if $t,t'\notin\{ \overline{t_i} \}$ and
$t<\overline{t_i}<t'$ for exactly one $\overline{t_i}$ then
$\mu_d(t)$ and $\mu_d(t')$ are on opposite sides of~$H$.
A facet $F$ of the cyclic polytope $C_d(n)$ given by
$F=H\cap C_d(n)$ then corresponds to a choice of
$\overline{t_i}$'s such that for all the other $t_k$'s in
the definition of $C_d(n)$, that is, such that
$t_k\notin \{ \overline{t_i}\,\mid\,i\in [d] \}$,
the corresponding points $\mu_d(t_k)$ are on the {\em same} side of~$H$.
This can happen only if for every pair of these $t_k$'s the moment
curve has an even number of crossings of $H$ between them.
Hence, there is
an even number of $\overline{t_i}$'s between any two $t_k$'s.
Let $s$ be the bitstring in which the \1's correspond to the
$\overline{t_i}$'s and the 0's correspond to the other $t_k$'s.
Then the condition that the set $F$ in (\ref{facet-gs}) is a
facet is equivalent to the Gale evenness condition.
\end{proof}

Because the moment curve has at most $d$ points on any hyperplane,
each facet of $C_d(n)$ is a $d$-simplex, so $C_d(n)$ is simplicial
and the choice of the $t_j$'s in (\ref{cycdef}) is
irrelevant for the characterization of the facets of
$C_d(n)$ by Gale evenness strings.
However, note that the order of the points on the curve,
that is, of $t_1<\cdots<t_n$, defines the positions in the
bitstring.

\clearpage

\begin{example}
Consider the facet $F$ of the cyclic polytope $C_3(6)$ marked in blue
in Figure \ref{cyc36fac-fig}.
If we label the vertices on the moment curve as $t_i$, with $i\in [n]$, and
we set $s(i)=1$ if $t_i$ is a vertex of $F$ and $s(i)=0$
otherwise, we see that the corresponding Gale string $s\in G(3,6)$ is
$s=\100\1\10$.
Figure \ref{cyc36fac-hyper-fig} shows the intersection of the
moment curve and the hyperplane $H$ in $\reals^3$ defined by the
$t_i$ such that $s(i)=1$.

\begin{figure}[h]
\strut\hfill
\includegraphics[width=28ex]{chapter-2/fig-gale-def/cyc36fac.pdf}%
\hfill\strut
\caption[A facet of the cyclic polytope $C_3(6)$]{%
The facet of the cyclic polytope $C_3(6)$ corresponding to the Gale
string $s=\1 00 \1\1 0\in G(3,6)$.
}
\label{cyc36fac-fig}
\end{figure}
\begin{figure}[h]
\strut\hfill
\includegraphics[width=35ex]{chapter-2/fig-gale-def/100110-hyper.pdf}%
\hfill\strut
\caption[A facet of $C_3(6)$ as zeroes of the moment curve]{%
The facet of the cyclic polytope $C_3(6)$ given by the intersection
of the moment curve and the hyperplane $H$ (seen ``from the
side'') through the points
% not zero's of \mu!
$\mu_3(t_1),\mu_3(t_4),\mu_3(t_5)$ corresponds to the Gale string
$s=\1 00 \1\1 0\in G(3,6)$.
}
\label{cyc36fac-hyper-fig}
\end{figure}
\end{example}

\clearpage

\begin{example}
\label{c46-ex}
Figure \ref{c46-fig} shows the cyclic polytope $C_4(6)$, with the exterior
facet corresponding to the Gale string $s=\1\1\1\100$.
Figure \ref{cyc46fac-hyper-fig} shows the string $s=\1\1\1\100$ as
intersection of the moment curve and the hyperplane $H$.
\begin{figure}[h]
\strut\hfill
\includegraphics[width=45ex]{chapter-2/fig-gale-def/c46.pdf}%
\hfill\strut
\caption[The cyclic polytope $C_4(6)$]{%
The cyclic polytope $C_4(6)$. The thin lines represent the edges inside
the exterior facet, in bold lines; the latter corresponds to the Gale string
$s=\1\1\1\1 0\in G(4,6)$.
}
\label{c46-fig}
\end{figure}
\begin{figure}[hbt]
\strut\hfill
\includegraphics[width=40ex]{chapter-2/fig-gale-def/111100-hyper.pdf}%
\hfill\strut
\caption[A facet of $C_4(6)$ via the moment curve]{%
The facet of the cyclic polytope $C_4(6)$ given by the intersection of
the moment curve and the hyperplane $H$ through the points
$\mu_4(t_1),\mu_4(t_2),\mu_4(t_3),\mu_4(t_4)$,
corresponding to the Gale string $s=\1\1\1\1 00\in G(4,6)$.}
\label{cyc46fac-hyper-fig}
\end{figure}
\end{example}

\clearpage

\begin{example}
As a counterexample, consider Figure \ref{notfacet-even-hyper-fig}.
The points $t=t_4$ and $t'=t_6$ lie on the moment curve but
$\mu_4(t_4)$ and $\mu_4(t_6)$ are on opposite
sides of the hyperplane~$H$ defined by the other four points.
% corresponding bitstring is  $s=\1 0\1 0\1 0$
The corresponding bitstring is $s=\1\1\10\10$, which is
not a Gale string. The violation of the Gale evenness condition corresponds
to the change of side with respect to the hyperplane between the $t_j$'s.
% An analogous case is shown in Figure  \ref{notfacet-even-hyper-fig}; the

%% not a good figure because of the extra crossings of the
%% curve (only 3 allowed). Also not very different from the next figure hence
%% omitted.
% \begin{figure}[hp]
% \strut\hfill
% \includegraphics[width=45ex]{chapter-2/fig-gale-def/notfacet-odd-hyper.pdf}%
% \hfill\strut
% \caption[Not a facet of $C_3(6)$]{%
% There is a change of sign between two $t_j$'s. The bitstring $s=0\1\1 0\1 0$
% does not satisfy the Gale evenness condition, and the
% $\overline{t_i}$'s (for the \1\ bits) do not
% correspond to a facet of $C_3(6)$.
% }
% \label{notfacet-odd-hyper-fig}
% \end{figure}

\begin{figure}[hp]
\strut\hfill
\includegraphics[width=45ex]{chapter-2/fig-gale-def/notfacet-even-hyper.pdf}%
\hfill\strut
\caption[Not a facet of $C_4(6)$]{%
There is a change of side between two $\mu_4(t_j)$'s (for
the 0 bits). The bitstring $s=\1\1\1 0\1 0$
does not satisfy the Gale evenness condition, and the
set of $\mu_4(\overline{t_i})$'s (for the \1 bits)
does not define a facet of $C_4(6)$.
}
\label{notfacet-even-hyper-fig}
\end{figure}
\end{example}

\clearpage

We now apply Theorem \ref{origgale-thm} to the study of bimatrix games.
\linebreak[4]
Proposition \ref{nash-to-acl} states that 2-{\sc Nash} can be reduced to
{\sc Another Completely Labeled Facet}.
If the polytope $P^\Delta$ in Theorem \ref{unit-vector-dual-thm} is cyclic and we
define a labeling for Gale strings such that a completely labeled Gale
string corresponds to a completely labeled facet of~$P^\Delta$, then
we can study unit vector games and their dual cyclic best
response polytope as Gale strings.
We say that $s\in G(d,n)$ is a {\em completely labeled Gale string}
for some labeling function $l_s:[n]\to[d]$ if
$\{ l_s(i)\,\mid\,s(i)=\1\text{ for }i\in [n] \}=[d]$.
Since $s\in G(d,n)$ has exacty $d$ bits equal to \1, this means that for
each $j\in [d]$ there is exactly one
$i\in [n]$ such that $s(i)=\1$ and $l_s(i)=j$.
It may not always be possible to find a completely labeled Gale string.

\begin{example}
\label{no-clgs}
For $l_s = 121314$, there are no completely labeled Gale strings.
The labels $l_s(i)=2,3,4$ appear only once in $l_s$, so we must have
$s(2)=s(4)=s(6)=1$. We also must have $l_s(i)=1$ for exactly one $i=1,3,5$.
The candidate strings are then $s=\1\10\10\1$, $s'=0\1\1\10\1$,
$s''=0\10\1\1\1$, but none of these satisfies the Gale evenness condition.
\end{example}

A {\em Gale game} is a unit vector game $(U,B)$ where
% U is a matrix : not commas
$U=[e_{l(1)}\cdots e_{l(d)}]$ for some labeling $l:[n]\to [d]$ and
for which the dual of the best response polytope is a cyclic polytope
$P^\Delta=\conv\{ e_1,\ldots,e_d,c_1,\ldots,c_n \}$. Theorem
\ref{unit-vector-dual-thm} gives the labeling (\ref{vert-labeling-unitv})
for the $d+n$ vertices of $P^\Delta$ as
\[
\arraycolsep.2em
\begin{array}{rcll}
l_v(-e_i)&=&i\quad & \text{ for }i\in [d],\\
l_v(c_j)&=&l(j)\quad & \text{ for }j\in [n].
\end{array}
\]
We define the labeling $l_s:[d+n]\to [d]$ of $G(d,n)$ as
\begin{equation}
\label{gs-labeling-unitv}
\arraycolsep.2em
\begin{array}{rcll}
l_s(i)&=&i\quad & \text{ for }i\in [d],\\
l_s(d+j)&=&l(j)\quad & \text{ for }j\in [n].
\end{array}
\end{equation}
Then the Gale strings $s\in G(d,d+n)$ that are completely labeled by $l_s$
correspond exactly to facets of $P^\Delta$ that are completely labeled by $l_v$,
with the
\linebreak[5]
facet $F_0$ corresponding to the ``trivial'' completely labeled
string $\1^d 0^n$.

From this point forward, we will assume that $d$ is even;
we will also assume that the labeling $l_s:[d+n]\to [d]$ satisfies
$l_s(i)\neq l_s(i+1)$. This can be done without loss of generality,
given the following consideration.
Suppose that $l_s(i)=l_s(i + 1)$ for some index $i$, and let $s$ be a
completely labeled Gale string for $l_s$. Then only one of
$s(i)$ and $s(i+1)$ can be equal to \1 (it is possible that both
are equal to~0), so $s(i)s(i+1)$ will never be part of a run of even length that
``interferes'' with the Gale Evenness Condition. Therefore, we can
identify the indices $i$ and $i + 1$.

\begin{example}
\label{c46-123432-ex}
Given the string of labels $l_s=123432$, there are four associated completely
labeled Gale strings in $G(4,6)$:
$s_A=\1\1\1\100$, $s_B=\1\10\1\10$,
\linebreak[4]
$s_C=\100\1\1\1$ and
$s_D=\10\1\10\1$. These correspond to the completely labeled facets
for the labeling shown in Figure \ref{c46-123432-fig} on the left.
\begin{figure}[hbt]
\strut\hfill
\includegraphics[width=50ex]{chapter-2/fig-gale-def/123432-fac-name.pdf}%
\hfill
\small
\begin{tabular}{c | c @{ } c @{ } c @{ } c @{ } c @{ } c @{ } c }
facet & {\bf 1} & {\bf 2} & {\bf 3} & {\bf 4} & {\bf 3} & {\bf 2}\\
\hline
{\bf A} & \1 & \1 & \1 & \1 & 0 & 0 \\
{\bf B} & \1 & \1 & 0 & \1 & \1 & 0 \\
{\bf C} & \1 & 0 & 0 & \1 & \1 & \1 \\
{\bf D} & \1 & 0 & \1 & \1 & 0 & \1
\end{tabular}
\hfill\strut
\caption[A labeling of $C_4(6)$ and its completely labeled facets]{%
The cyclic polytope $C_4(6)$. The labeling of the vertices corresponds to
the labeling of $G(4,6)$ given by $l_s=123432$.
}
\label{c46-123432-fig}
\end{figure}
\end{example}

\clearpage

We can now define the problem \anothergale\ as in Table
\ref{another-gale}, where for brevity we write $n$ instead
of $d+n$.
% you have to alert the reader to changes in notation

\begin{problem}
{\anothergale}
{A labeling $l:[n]\to[d]$, where $d$ is even and $d<n$.
A Gale string $s\in G(d,n)$, completely labeled by $l$.}
{A Gale string $s'\in G(d,n)$, completely labeled by $l$,
such that $s' \neq s$.}
\label{another-gale}
\end{problem}

It takes polynomial time to translate
the facets of the cyclic polytope
\linebreak[6]
$C_d(d+n)$ % can be translated in polynomial time
into the corresponding Gale strings in $G(d,d+n)$, following
the proof of Theorem~\ref{origgale-thm}.
Furthermore, defining the labeling $l_s$ from the labeling $l_v$ also takes
polynomial time: for the labels $i\in [d]$ it is immediate, for the labels
$d+j$, where $j\in [n]$, we have to check the $d\times n$ matrix $U$
of the imitation game.
Therefore, by Proposition \ref{nash-to-acl}, we have a reduction from
{\sc Gale Nash} to \anothergale.

\begin{proposition}
\label{galenash-to-another-gale}
{\sc Gale Nash} is polynomial-time reducible to \anothergale.
\end{proposition}
