\section{Cyclic Polytopes and Gale Strings}

We now apply the results of the previous section to unit vector games
for which the best response polytope satisfies a further condition:
being the dual of a cyclic polytope. Cyclic polytopes can be represented
in a sintetic way using a combinatorial structure, the Gale strings.
We will first give the definition of cyclic polytope,
then of Gale string, then the theorem by Gale \cite{gale} about their
correspondence.

The {\em moment curve} in dimension $d$ is defined as
\begin{equation}
\mu_d:\reals\longrightarrow\reals^d\qquad\qquad
\mu_d:t\longmapsto (t,t^2,\ldots,t^d)\T .
\end{equation}
The {\em cyclic polytope} in dimension~$d$ with $n$ vertices, where $n>d$ is
\begin{equation}
C_d(n)=\conv\{ \mu_d(t_i)\ \mid\ t_1<\ldots<t_n\text{ affinely independent } \}.
\end{equation}

\begin{example}
\label{cyc36-ex}
The cyclic polytope in dimension 3 with 6 facets can be seen in figure
\ref{cyc36-fig}.

\begin{figure}[hbt]
\strut\hfill
\includegraphics[width=35ex]{chapter-2/fig-gale-def/cyc36.pdf}
\hfill\strut
\caption{The cyclic polytope $C_3(6)$
}
\label{cyc36-fig}
\end{figure}
\end{example}

Given $k\in\naturals$ and a set $S$, we can represent the function
$f:[k]\to S$ as the string $s=s(1)s(2)\cdots s(k)$; we have a {\em bitstring}
if $S=\{0,1\}$. A maximal substring of consecutive
1's in a bitstring is called a {\em run}; an {\em interior} run is
bounded on both sides by 0's. We will use the notation $\1^k$ for a run of
length $k$ (\1's will be in boldface for readability), and $0^k$ for a
string of 0's of length $k$ A {\em Gale string of length $n$ and
dimension $d$}, where $n>d$, is a bitstring $s\in G(d,n)$ satisfying the
following conditions:
\begin{enumerate}
\item exactly $d$ bits of $s$ are equal to $\1$;
\item ({\em Gale evenness condition})
\begin{equation}
0\1^k0\text{ is a substring of }s\quad
{\Longrightarrow}\quad
k\text{ is even.}
\end{equation}
\end{enumerate}

In general, the Gale evenness conditions allows for Gale strings that start
or end with an odd-length run; but if $d$ is even then $s$ can start with
an odd run if and only if it ends with an odd run.
We can then consider the Gale strings in $G(d,n)$ with even $d$ as ``loops''
obtained by ``glueing together'' the endpoints of the strings; then
all runs on the loops are even.
Formally, we can see the indices of a Gale string $s\in G(d,n)$ with
$d$ even as equivalence classes modulo $n$; this also makes the set of Gale
strings of even dimension invariant under a cyclic shift of the strings.

\begin{example}\label{gs-example}
As an example of even $d$, we have
\begin{align*}
G(4,6) = \{ & \1\1\1\100, \1\1\100\1, \1\100\1\1, \100\1\1\1, 00\1\1\1\1, \\
            & 0\1\1\1\10, \1\10\1\10, \10\1\10\1, 0\1\10\1\1 \}
\end{align*}
The strings $\1\1\1\100$, $\1\1\100\1$, $\1\100\1\1$, $\100\1\1\1$,
$00\1\1\1\1$ and $0\1\1\1\10$ are
equivalent under a cyclic shift (if considering the strings as ``loops'', the
$\1$'s are all consecutive), as are the strings $\1\10\1\10$, $\10\1\10\1$
and $0\1\10\1\1$ (if considering the strings as ``loops'', the even runs of
$\1$'s are two couples separated by a single $0$).

As an example of odd $d$, we have
\[
G(3,5) = \{ \1\1\100, \10\1\10, \100\1\1, \1\100\1, 0\1\10\1, 00\1\1\1 \}
\]
Note that $0\10\1\1$ is a cyclic shift of $\10\1\10$, but it is not a Gale
string.
\end{example}

The relation between cyclic polytopes and Gale strings has been given by
Gale \cite{gale}.

\begin{theorem}{\rm (Gale \cite{gale})}
\label{gale-thm}
For any positive integers $d,n$ with $n>d$
\begin{align}
\label{facet-gs}
    & F \text{ is a} \text{ facet of } C_d(n) & \nonumber \\
    & \Longleftrightarrow & \nonumber \\
    & F = \conv\{ \mu(t_j)\ \mid\ s(j)=1 \text{ for }s\in G(d,n) \}. &
\end{align}

\begin{proof}
We have that
$C_d(n) = %
\conv\{ \mu_d(t_j)\ \mid\ t_1 < \ldots < t_n \text{ affinely independent} \}$.
Let $\overline{t_1} < \ldots < \overline{t_d}$ be a choice of some of the
$t_j$'s in the definition of $C_d(n)$. Then the points
$\mu_d(\overline{t_i})$, where $i\in [d]$, define an hyperplane $H$ that
crosses the moment curve $\mu_d(t)$ at all and only the $\overline{t_i}$'s.
Given the definition of moment curve, the hyperplane $H$ is never
tangent to the moment curve, and every crossing gives a ``change of sign'';
that is, if there is one and only one $\overline{t_i}\in (t,t')$ then
$\mu_d(t)$ and $\mu_d(t')$ have opposite sign.
A facet $F$ of the cyclic polytope $C_d(n)$ then corresponds to a choice of
$\overline{t_i}$'s with $i\in [d]$, such that for all the $t_k$'s with
$k\in [n]$ and $t_k\notin \{ \overline{t_i}\ \mid\ i\in [d] \}$
all the $\mu_d(t_k)$'s have the same sign.
This can happen only if for every couple of $t_k$'s the moment
curve has an even number of changes of sign between them; therefore there is
an even number of $\overline{t_i}$'s between any two $t_k$'s.
Let $s$ be the bitstring in which the \1's correspond to the $t_i$'s and
and the 0's correspond to the other $t_k$'s; then the condition for being
a facet in (\ref{label-gs}) becomes the Gale evenness condition.
\end{proof}
\end{theorem}

Note also that the fact that the moment curve has exactly $d$ zeroes implies
that the each facet of $C_d(n)$ is a $d$-simplex, so $C_d(n)$ is simplicial
and the choice of the $t_j$'s in the proof is irrelevant.

\begin{example}
Consider the facet $F$ of the cyclic polytope $C_3(6)$ underlined in blue
in Figure \ref{cyc36fac-fig}.
\begin{figure}[hbt]
\strut\hfill
\includegraphics[width=35ex]{chapter-2/fig-gale-def/cyc36fac.pdf}%
\hfill\strut
\caption[A facet of the cyclic polytope $C_3(6)$]{%
A facet of the cyclic polytope $C_3(6)$.
}
\label{cyc36fac-fig}
\end{figure}
If we label the vertices on the moment curve as $i\in [n]$, and we set $i=1$
if $i$ is a vertex of $F$ and $i=0$ otherwise, we see that the corresponding
Gale string $s\in G(3,6)$ is $s=\100\1\10$.
On the hyperplane, the correspondence is as in figure \ref{cyc36fac-hyper-fig}.
\begin{figure}[hbt]
\strut\hfill
\includegraphics[width=45ex]{chapter-2/fig-gale-def/100110-hyper.pdf}%
\hfill\strut
\caption[A facet of $C_3(6)$ as hyperplane intersecting the moment curve]{%
The facet of the cyclic polytope $C_3(6)$ corresponding to the Gale string
$s=\1 00 \1\1 0\in G(3,6)$ as an hyperplane intersecting the moment curve.
}
\label{cyc36fac-hyper-fig}
\end{figure}
\end{example}

\begin{example}
\label{c46-ex}
Figure \ref{c46-fig} shows the cyclic polytope $C_4(6)$, with the exterior
facet corresponding to the Gale string $s=\1\1\1\100$.
\begin{figure}[hbt]
\strut\hfill
\includegraphics[width=60ex]{chapter-2/fig-gale-def/c46.pdf}%
\hfill\strut
\caption[The cyclic polytope $C_4(6)$]{%
The cyclic polytope $C_4(6)$. The thin lines represent the edges inside
the facet $1234$, drawn in bold lines.
}
\label{c46-fig}
\end{figure}
On the hyperplane, the string $s=\1\1\1\100$ can be seen as in figure
\ref{c46fac-hyper-fig}.
\begin{figure}[hbt]
\strut\hfill
\includegraphics[width=45ex]{chapter-2/fig-gale-def/111100-hyper.pdf}%
\hfill\strut
\caption[A facet of $C_4(6)$ as hyperplane intersecting the moment curve]{%
The facet of the cyclic polytope $C_4(6)$ corresponding to the Gale string
$s=\1\1\1\1 0\in G(4,6)$ as an hyperplane intersecting the moment curve.
}
\label{cyc46fac-hyper-fig}
\end{figure}
\end{example}

\begin{example}
As a counterexample, consider figure \ref{notfacet-even-hyper-fig}.
There are two points $t_j$ that lie on the moment curve but are on
two different sides of the hyperplane $H$, so they do not belong to the
same facet. The corresponding bitstring is $s=\1\1\10\10$, which is
not a Gale string; the violation of the Gale evenness condition correspond
to the change of sign between the $t_j$'s.
\begin{figure}[hbt]
\strut\hfill
\includegraphics[width=45ex]{chapter-2/fig-gale-def/notfacet-even-hyper.pdf}%
\hfill\strut
\caption[Not a facet of $C_4(6)$]{%
The point in pink and the point in green on the moment curve have
different sign, so they don't correspond to a facet; the
corresponding bitstring $s=\1\1\1 0\1 0$ does not satisfy the Gale evenness
condition.
}
\label{notfacet-even-hyper-fig}
\end{figure}
An analogous case is shown in Figure \ref{notfacet-odd-hyper-fig}; the
corresponding bitstring should be $s=\1 0\1 0\1 0$.
\begin{figure}[hbt]
\strut\hfill
\includegraphics[width=45ex]{chapter-2/fig-gale-def/notfacet-odd-hyper.pdf}%
\hfill\strut
\caption[Not a facet of $C_3(6)$]{%
The point in pink and the point in green on the moment curve have
different sign, so they don't correspond to a facet; the
corresponding bitstring $s=\1 0\1 0\1 0$ does not satisfy the Gale evenness
condition.
}
\label{notfacet-odd-hyper-fig}
\end{figure}
\end{example}



We now apply Theorem \ref{gale-thm} to the study of bimatrix games.
Proposition \ref{2nash-to-aclf} states that 2-{\sc Nash} can be reduced to
{\sc Another Completely Labeled Facet}.
If the polytope $Q$ in \ref{unit-vector-dual-thm} is cyclic and we
define a labeling for Gale strings such that a completely labeled Gale
string corresponds to a completely labeled facet of $Q$,
then we can study unit vector games and their dual cyclic best
response polytope as Gale strings.
We say that $s\in G(d,n)$ is a {\em completely labeled Gale string}
for some labeling function $l_s:[n]\to[d]$ if
$\{ l(i)\ \mid\ s(i)=\1\text{ for }i\in [n] \}=[d]$.
Since $s\in G(d,n)$ has exacty $d$ bits equal to \1, this means that for
each $j\in [d]$ there is exactly one
$i\in [n]$ such that $s(i)=\1$ and $l_s(i)=j$.
Note that it is not always possible to find a completely labeled Gale string.

\begin{example}
\label{no-clgs}
For $l = 121314$, there are no completely labeled Gale strings.
The labels $l(i)=2,3,4$ appear only once in $l$, as $l(2),l(4),l(6)$
respectively; therefore we must have $s(2)=s(4)=s(6)=1$. For every other
$i\in [n]$ we have $l(i)=1$, so we have $l(i)=1$ for exactly one $i=1,3,5$.
The candidate strings are then \1\10\10\1, 0\1\1\10\1, 0\10\1\1\1; but
none of these satisfies the Gale evenness condition.
\end{example}

A Gale game is a unit vector game $(U,B)$, where
$U=(e_{l(1),\ldots,e_{l(d)}})$ for some labeling $l:[n]\to [d]$,
for which the dual of the best response polytope is a cyclic polytope
$Q=\conv\{ e_1,\ldots,e_d,c_1,\ldots,c_n \}$. Theorem
\ref{unit-vector-dual-thm} gives a labeling $l_v$ of the $d+n$ vertices of
$Q$ in \ref{vert-labeling-unitv}
\begin{eqnarray*}
l_v(-e_i)=i\text{ for }i\in [m]; \\
l_v(c_j)=l(j)\text{ for }j\in [n].
\end{eqnarray*}
We define the labeling $l_s:[d+n]\to [d]$ of $G(d,n)$ as
\begin{eqnarray}\label{gs-labeling-unitv}
l_s(i)=i\text{ for }i\in [d]; \nonumber \\
l_s(d+j)=l(j)\text{ for }j\in [n].
\end{eqnarray}
Then the Gale strings $s\in G(d,d+n)$ that are completely labeled for $l_s$
correspond exactly to the completely labeled facets of $Q$, with the
facet $F_0$ corresponding to the ``trivial'' completely labeled
string $\1^d 0$.

From this point forward, we will assume that $d$ is even.
We will also assume that the labeling $l:[n]\to [d]$ is such
that $l(i)\neq l(i+1)$; this can be done without loss of generality,
given the following consideration.
Suppose that $l(i)=l(i + 1)$ for some index $i$, and let $s$ be a
completely labeled Gale string for $l$. Then only one of
$s(i)$ and $s(i+1)$ can be equal to \1 (note that it's possible that both
are 0s). So $s(i)s(i+1)$ will never be a run of even length that
interferes with the Gale Evenness Condition, so we can ``simplify''
by identifying the indices $i$ and $i + 1$.


\begin{example}
\label{c46-123432-ex}
Given the string of labels $l=123432$, there are four associated completely
labeled Gale strings in $G(4,6)$:
$s_A=\1\1\1\100$, $s_B=\1\10\1\10$, \mbox{$s_C=\100\1\1\1$} and
$s_D=\10\1\10\1$, as in Figure \ref{clgs-123432-fig}.
\begin{figure}[hbt]
\strut\hfill
\begin{tabular}{c | c @{ } c @{ } c @{ } c @{ } c @{ } c @{ } c }
{\small facet} & {\bf 1} & {\bf 2} & {\bf 3} & {\bf 4} & {\bf 3} & {\bf 2}\\
\hline
{\bf A} & \1 & \1 & \1 & \1 & 0 & 0 \\
{\bf B} & \1 & \1 & 0 & \1 & \1 & 0 \\
{\bf C} & \1 & 0 & 0 & \1 & \1 & \1 \\
{\bf D} & \1 & 0 & \1 & \1 & 0 & \1
\end{tabular}
\hfill\strut
\caption[Completely labeled Gale strings]{%
The completely labeled Gale strings for the labeling $l=123432$.
}
\label{clgs-123432-fig}
\end{figure}
On the cyclic polytope $C_4(6)$, these correspond to the facets as in figure
\ref{c46-123432-fig}.
\begin{figure}[hbt]
\strut\hfill
\includegraphics[width=60ex]{chapter-2/fig-gale-def/123432-fac-name.pdf}%
\hfill\strut
\caption[Completely labeled facets on a cyclic polytope]{%
The cyclic polytope $C_4(6)$ with vertices labeled following $l_s=123432$.
The completely labeled facets correspond to the
completely labeled Gale strings of Figure \ref{clgs-123432-fig}.
}
\label{c46-123432-fig}
\end{figure}
\end{example}

We can now define the problem \anothergale\ as follows:

\begin{problem}
{\anothergale}
{A labeling $l:[n]\to[d]$, where $d$ is even and $d<n$.
A Gale string $s\in G(d,n)$, completely labeled by $l$.}
{A Gale string $s'\in G(d,n)$, completely labeled by $l$,
such that $s' \neq s$.}
\end{problem}

It takes polynomial time to translate a cyclic polytope $C_d(n)$ into the
corresponding $G(d,n)$: the bitstrings of length $n$ with $d$ positive bits
are found in $O(n^2)$ time, and checking that a bitstring satisfies the
Gale evenness condition takes $O(d)$ time, so the operation takes
$O(n^3d)$ time.
Furthermore, building the labeling $l_s$ from the labeling $l_v$ also takes
polynomial time: for the labels $i\in [m]$ it is immediate, for the labels
$m+j$, where $j\in [n]$, we have to check the labeling $l:[n]\to [m]$,
that is, the $n\times m$ matrix $U$ in the imitation game.
Therefore, by Proposition \ref{2nash-to-aclf}, we have a reduction from
{\sc Gale Nash} to \anothergale.

\begin{proposition}
\label{galenash-to-another-gale}
{\sc Gale Nash} is polynomial-time reducible to \anothergale.
\end{proposition}
