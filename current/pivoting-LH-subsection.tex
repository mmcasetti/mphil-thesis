\documentclass[11pt, draft]{article}

%% TYPESETTING

% draft:
\linespread{1.3}
\usepackage{todonotes}

% final (?) TODO: check with LSE requirements
% page measurements
% \setlength{\hoffset}{0mm}
% \setlength{\oddsidemargin}{25mm}
% \setlength{\textwidth}{130mm}

%% PACKAGES

\usepackage{amssymb}
\usepackage{amsmath}
\usepackage{amsthm}

\usepackage[all]{xy}

\usepackage[ruled,vlined,linesnumbered]{algorithm2e}

%% THEOREMS and ENVIRONMENTS

% theorems

\newtheorem{theorem}{Theorem}
\newtheorem{property}{Property}[section]
\theoremstyle{definition}\newtheorem{definition}{Definition}
\theoremstyle{remark}\newtheorem{example}{Example}[section]

% environments

% computational problems
% TODO check p{textwidth} in final version

% decision problem

\newenvironment{decproblem}[3]
{
\vspace{2.5ex}
\noindent
\begin{tabular}{p{18mm} @{\textbf{:} } p{100mm}}
\hline
\multicolumn{2}{l}{\noindent {\sc #1}} \\
\hline
\textbf{input} & #2 \\
\textbf{question} & #3 \\
\hline
\end{tabular}
}
{
\vspace{1.5ex}
}

% function problem

\newenvironment{fctproblem}[3]
{
\vspace{2.5ex}
\noindent
\begin{tabular}{p{15mm} @{\textbf{:} } p{102mm}}
\hline
\multicolumn{2}{l}{\noindent {\sc #1}} \\
\hline
\textbf{input} & #2 \\
\textbf{output} & #3 \\
\hline
\end{tabular}
}
{
\vspace{1.5ex}
}

% use:
% \begin{problem}
% {name of problem}
% {input of problem}
% {output of problem}
% \end{problem}


%% SHORTCUTS

% definitions

\def\reals{{\mathbb R}}
\def\naturals{{\mathbb N}}
\def\conv{{\rm conv}}
\def\0{{\bf0}}
\def\1{{\bf1}}
\def\T{^{\top}}
\def\rone{{\1\T}}

% to choose the name of the problem - GALE or COMPLETELY LABELED GALE STRING

\def\gale{{\sc{Gale}}}
\def\anothergale{{\sc{Another Gale}}}


%%% END PREAMBLE

\begin{document}

\section{Introduction}

\section{Complexity, Games, Polytopes and Gale Strings}

\subsection{The Complexity Classes P and PPAD}

\subsection{Normal Form Games and Nash Equilibria}

\todo[inline]{file: background-subsection}

\subsection{Bimatrix Games and Best Response Polytopes}

\todo[inline]{file: polytopes-subsection}

\subsection{Cyclic Polytopes and Gale Strings}\label{gs-ssect}

\subsection{Labeling and the Problem \anothergale}

\todo[inline]{file: gale-def-subsection}

\section{Algorithmic and Complexity Results}
\label{main-thm-sect}

\subsection{Pivoting}

\subsection{The Lemke-Howson Algorithm and Parity}

\begin{definition}\label{almost-completely-labeled}
Let $l:[d]\to [n]$ be a labeling. An {\em almost completely labeled} Gale
string associated with $l$ is $s\in G(d,n)$ such that $|l(1(s))|=d-1$
\end{definition}

If $s$ is an almost completely labeled Gale string $s$ associated with the
labeling $l:[n]\to[d]$, then for each label $l(i)$ but one there is a bit
$s(i)=1$. Furthermore, there will be exactly one ``duplicate'' label $l(i)$
such that $s(i)=s(j)=1$ for exactly one $j\neq i$

Completely and almost completely labeled Gale strings are used to build the
{\em Lemke-Howson for Gale algorithm}. We first define its fundamental
subroutine, the {\em pivoting algorithm}.

\begin{definition}\label{pivoting}
We define as {\em pivoting} the operation defined in algorithm
\ref{pivoting-algorithm}, where we consider the Gale strings as ``loops'' by
identifying position $i$ with any position $i+kn$

\todo[inline]{use "modulo" notation}

\todo[inline]{for clgs, is dropped label necessarily double one?}

\begin{algorithm}\label{pivoting-algorithm}
\SetKwInOut{Input}{input}
\SetKwInOut{Output}{output}
\Input{A string of labels $l$ of length~$n$; a completely or almost
completely labeled Gale string for $l$; $i \in [n]$ such that $s(i)=1$}
\Output{A complete or almost complete Gale string for $l$.}
\BlankLine
set $s(i)=0$ \\
let $j$ be the length of the odd maximal run of $1$s created by this \\
\If{ the odd maximal run of $1$s is on the \textbf{right} of position $i$ }
{ set $s(i+j+1)=1$ }
\Else
{ set $s(i-j-1)=1$ }
\Return $s$
\caption{Pivoting on completely labeled Gale strings}
\end{algorithm}

The operation where we set $s(i)=0$ (line 1) is called {\em dropping
label $i$}.

\end{definition}

\begin{example}
Let $l = 123432$. The Lemke-Howson for Gale algorithm applied to the
completely labeled Gale string $s = 111100$ and the label
\underline{\underline{4}} in position 4 returns the almost completely labeled
Gale string $011110$.

\begin{center}
{\renewcommand{\tabcolsep}{2ex}
\begin{tabular}{|c|p{100mm}|}
\hline
\underline{123\underline{4}}32 &
drop the label \textbf{4} from $l$ \\
\textbf{111\underline{1}}00 &
taking the corresponding \textbf{1} in $s$ \\
\textbf{111}\underline{0}00 &
and setting it to \textbf{0} \\
\cline{2-2}
\textbf{111}00\textbf{1} &
set the other end of the odd run in $s$ to \textbf{1} \\
\cline{2-2}
\underline{1\underline{2}3}43\underline{\underline{2}} &
there is a \textbf{1} in $s$ in a position corresponding to the labels
\textbf{1}, \textbf{2} \underline{(twice)} and \textbf{3} in $l$;
there are only \textbf{0}'s in all the positions corresponding to the
label \textbf{4} in $l$ \\
\hline
\end{tabular}
}\\
\end{center}

\end{example}

\begin{example}
Let $l = 123424$. The Lemke-Howson for Gale algorithm applied to the almost
completely labeled Gale string $s = 110011$ and the duplicate label
\underline{\underline{2}} in position 2 returns the almost completely
labeled Gale string $100111$.

\begin{center}
{\renewcommand{\tabcolsep}{2ex}
\begin{tabular}{|c|p{100mm}|}
\hline
\underline{1\underline{2}}34\underline{24} &
drop the label \textbf{2} in position 2 from $l$ \\
\textbf{11}00\textbf{11} &
taking the corresponding \textbf{1} in $s$ \\
\textbf{1}\underline{0}00\textbf{11} &
and setting it to \textbf{0} \\
\cline{2-2}
\textbf{1}00\textbf{111} &
set the other end of the odd run (on the loop) in $s$ to \textbf{1} \\
\cline{2-2}
\underline{1}23\underline{\underline{4}2\underline{4}} &
there is a \textbf{1} in $s$ in a position corresponding to the labels
\textbf{1}, \textbf{2} and \textbf{4} \underline{(twice)} in $l$;
there are only \textbf{0}'s in all the positions corresponding to the
label \textbf{3} in $l$ \\
\hline
\end{tabular}
}\\
\end{center}

If we drop duplicate label \underline{\underline{2}} in position 5 instead,
the algorithm returns the completely labeled Gale string $111001$.

\begin{center}
{\renewcommand{\tabcolsep}{2ex}
\begin{tabular}{|c|p{100mm}|}
\hline
\underline{12}34\underline{\underline{2}4} &
drop the label \textbf{2} in position 5 from $l$ \\
\textbf{11}00\underline{\underline{1}1} &
taking the corresponding \textbf{1} in $s$ \\
\textbf{11}00\underline{0}\textbf{1} &
and setting it to \textbf{0} \\
\cline{2-2}
\textbf{111}00\textbf{1} &
set the other end of the odd run (on the loop) in $s$ to \textbf{1} \\
\cline{2-2}
\underline{123}42\underline{4} &
\underline{all the labels} in $l$ correspond to a \textbf{1} in $s$ \\
\hline
\end{tabular}
}\\
\end{center}

\end{example}

Note that, by the Gale evenness condition, we must have an the odd maximal
run of $1$'s either on the right or on the left of position $i$.

If the Gale string in the pivoting algorithm is completely labeled, we can
drop any label $l(i)$ such that $s(i) = 1$. We refer to these labels as {\em
free labels}.

Each pivoting can be seen as a step of a ``path'' through (almost) completely
labeled Gale strings. If the first and last step of the path are completely
labeled Gale strings, the path is described by the {\em Lemke-Howson for Gale
algorithm}.

\begin{definition}\cite{lhg-ref}

We define the {\em Lemke-Howson for Gale} algorithm as follows:

\begin{algorithm}\label{lhg-algorithm}
\SetKwInOut{Input}{input}
\SetKwInOut{Output}{output}
\Input{A $n$-string $l \in [n]$ where $d$ is even and $d< n$; a completely
labeled Gale string $s$ associated with $l$.}
\Output{A Gale string $s'\neq s$ associated with $l$.}
\BlankLine
set $s' = s$ \\
pivot any free label of s' \\
\While{ $s'$ is an almost completely labeled Gale string }
{pivot the duplicate label, not picked up by the previous pivot}
\Return $s'$
\caption{Lemke-Howson for Gale Algorithm}
\end{algorithm}

\end{definition}

\begin{example}\label{lhg-example}
Let's consider the label string $l = 123432$ and the associated completely
labeled Gale string $s = 111100$, as in example \ref{pivoting-example}.
The Lemke-Howson for Gale algorithm using $1$ as free label returns the
completely labeled Gale string $s' = 110110$.\\

\begin{center}
{\renewcommand{\tabcolsep}{2ex}
\begin{tabular}{|r|c|l|}
\hline
start &
\textbf{\underline{1}234}32 &
drop \textbf{1}  \\
\cline{1-1}
\cline{3-3}
 &
\textbf{\underline{1}111}00 &
pivot \\
 &
\underline{0}\textbf{111\underline{1}}0 &
 \\
\cline{1-1}
\cline{3-3}
pick \textbf{3}, duplicate &
1\textbf{2\underline{3}4\underline{3}}2 &
drop the other \textbf{3} \\
\cline{1-1}
\cline{3-3}
 &
0\textbf{1\underline{1}11}0 &
pivot \\
 &
\textbf{\underline{1}1}\underline{0}\textbf{11}0 &
 \\
\cline{1-1}
\cline{3-3}
pick \textbf{1}, starting label &
\textbf{\underline{1}2}3\textbf{43}2 &
end \\
\hline
\end{tabular}
}\\
\end{center}
\end{example}

Note how the last label that is picked up is the one dropped at the start,
that's been missing in all the intermediate step; otherwise, we would have
reached a completely labeled Gale string at an earlier iteration.

Using algorithm \ref{lhg-algorithm} we can show a fundamental property of
Gale strings.

\todo{edited from paper notes until here}

\begin{theorem}\label{even-number-gale}
For any labeling $l:[n]\to[d]$, where $d$ is even and $d<n$,
the number of completely labeled Gale strings associated with $l$ is even.
\end{theorem}

\begin{proof}
If there are no completely labeled Gale strings associated with $l$, the
theorem holds trivially.

Suppose now that there is at least one completely labeled Gale strings
associated with $l$.

First of all, note that a pivot is reversible. Suppose that we pivot on the
(almost) completely labeled Gale string $s$ by dropping the label $l(i)$ and
picking up the label $l(j)$. Then $s(j) = 0$ and it is adjacent to the opposite
side of the odd maximal run of $1$s starting at $i$ that was created by
dropping $l(i)$. Let $s'$ be the (almost) completely labeled Gale string
obtained from this pivot. Analogously, if we pivot on $s'$ by dropping
$l(j)$, we will have to pick up the label $l(i)$. The pivoting is therefore
reversible by simply dropping the label that was picked up.

As there are only a finite number of possible bitstrings for each label
string,  if cycling is not possible the algorithm must terminate by finding
another completely labeled Gale string in a finite number of steps.

\todo[inline]{edit rest of proof}

Cycling is not possible due to the following observations.
Suppose the algorithm returns to a bit assignment of $s$ other than the initial Gale string.
Then at this bit assignment of $s$, because each pivot is reversible, we would have to be able to pick up two labels.
This, however, is ruled out by the GEC as only one of the adjacent runs of the dropped label is odd.
Returning to initial position is only possible by reversing the initial pivot which is not allowed.
The only free choice we have is at the beginning of the algorithm where we drop one free label.
From then on the process of the algorithm is uniquely determined, thus terminating in a finite number of steps at another Gale string.

\end{proof}

Theorem \ref{even-number-gale} holds for odd $d$ as well.

\todo[inline]{expand - or cut?}

The reversibility of the pivoting steps leads to the following property

\begin{property}\label{lhg-labels-property}
Let $s\in G(d,n)$ be a completely labeled Gale string for a labeling $l$,
and let $s'$ be the completely labeled Gale string obtained by running the
Lemke-Howson algorithm on $s$ by dropping the label $l(i)$. Then running
the Lemke-Howson algorithm on $s'$ by dropping the label $l(i)$ returns $s$.

The converse does not hold: it's possible for two Gale strings $s$ and $s'$
to be the endpoints of the path given by the Lemke-Howson algorithm run by
dropping both the label $l(i)$ and $l(j)\neq l(i)$. This is trivially true
since it's possible that $|\{s\in G(d,n) completely labeled for l \}| < d$.
\end{property}

An interesting example is the following.

\begin{example}
For the labeling $l=12342314$, the completely labeled Gale strings in
$G(4,8)$ are 11110000, 00011110, 00001111, 11100001, 01100011, 10001101.
They are related as endpoints of the Lemke-Howson algorithm as shown in the
following graph:

\begin{displaymath}
\xymatrix
@M=5pt
{
11110000
\ar@{<->}[rr] |{1}
\ar@{<->}[drr] |<<<<<<<<{2,4}
\ar@{<->}[ddrr] |>>>>>>>>{3}
& & 00011110 \\
01100011
\ar@{<->}[urr] |>>>>>>{2}
\ar@{<->}[rr] |>>>>>>{1,3}
\ar@{<->}[drr] |{4}
& & 11100001 \\
00001111
\ar@{<->}[uurr] |<<<<<<{3,4}
\ar@{<->}[rr] |{1,2}
& & 10001101
}
\end{displaymath}

Note that the graph is bipartite: this holds in general, a property related
to further results that we will discuss in section \ref{further-section}.

Note also that it's not a complete bipartite graph: there isn't an edge
between 1110001 and 00001111.

To our knowledge, it is an open question if the graph has to be connected.
\todo[inline]{If this were the case, there would be Nash equilibria not reachable via LHG from artificial equilibrium}
\end{example}

Note that the parity result of theorem \ref{even-number-gale} is about
{\em completely labeled} Gale strings, not Gale strings $s\in G(d,n)$ in
general. For example, $|G(4,6)|=9$, as shown in \ref{gs-example}.

\todo[inline]{results on 11234 = 1234 and 12345236 = 123423 (can delete successive double - any length of run, can delete single occurrences if in odd run). The first one plays a part in main thms}

\todo[inline]{until here}

\subsection{The Complexity of \gale\ and \anothergale}
\label{main-result-subsection}

\todo[inline]{file: main-result-subsection}

\newpage

\begin{thebibliography}{00}

\frenchspacing\parskip-1ex
\small

\bibitem{main} M. M. Casetti, J. Merschen, B. von Stengel (2010).
Finding Gale Strings.
\emph{Electronic Notes in Discrete Mathematics}
\todo[inline]{issue, pp. n--m.}

\bibitem{cd} X. Chen, X. Deng (2006).
Settling the complexity of two-player Nash equilibrium.
\emph{Proc. 47th FOCS}, pp. 261--272.

\bibitem{dgp} C. Daskalakis, P. W. Goldberg, C. H. Papadimitriou (2006).
The complexity of computing a Nash equilibrium.
\emph{Proc. Ann. 38th STOC}, pp. 71--78.
\todo[inline]{change ref to econometrica(?)}

\bibitem{edm} J. Edmonds (1965).
Paths, trees, and flowers.
\emph{Canad. J. Math.} 17, pp. 449--467.

\bibitem{gale} D. Gale (1963),
Neighborly and cyclic polytopes.
\emph{Convexity, Proc. Symposia in Pure Math.}, Vol. 7, ed. V. Klee, American Math. Soc., Providence, Rhode Island, pp. 225--232.
\todo[inline]{check if right typography}

\bibitem{jm} J. Merschen (2012).
\todo[inline]{thesis}

\bibitem{lh} C. E. Lemke, J. T. Howson, Jr. (1964).
Equilibrium points of bimatrix games.
\emph{J.  Soc. Indust. Appl. Mathematics} 12, pp.  413--423.

\bibitem{ppad} C. H. Papadimitriou (1994).
On the complexity of the parity argument and other inefficient proofs of existence.
\emph{J. Comput. System Sci.} 48, pp. 498--532.

\bibitem{svs} R. Savani, B. von Stengel (2006).
Hard-to-solve bimatrix games.
\emph{Econometrica} 74, pp. 397--429.

\bibitem{vvs} L. V\'{egh}, B. von Stengel
\todo[inline]{ref}


\end{thebibliography}

\end{document}
