\documentclass[11pt, draft]{article}

%% TYPESETTING

% draft:
\linespread{1.3}
\usepackage{todonotes}

% final (?) TODO: check with LSE requirements
% page measurements
% \setlength{\hoffset}{0mm}
% \setlength{\oddsidemargin}{25mm}
% \setlength{\textwidth}{130mm}

%% PACKAGES

\usepackage{amssymb}
\usepackage{amsmath}
\usepackage{amsthm}

\usepackage[all]{xy}

\usepackage[ruled,vlined,linesnumbered]{algorithm2e}

%% THEOREMS and ENVIRONMENTS

% theorems

\newtheorem{theorem}{Theorem}
\newtheorem{proposition}{proposition}
\theoremstyle{remark}\newtheorem{example}{Example}[section]

% environments

% computational problems
% TODO check p{textwidth} in final version

% decision problem

\newenvironment{decproblem}[3]
{
\vspace{2.5ex}
\noindent
\begin{tabular}{p{18mm} @{\textbf{:} } p{100mm}}
\hline
\multicolumn{2}{l}{\noindent {\sc #1}} \\
\hline
\textbf{input} & #2 \\
\textbf{question} & #3 \\
\hline
\end{tabular}
}
{
\vspace{1.5ex}
}

% function problem

\newenvironment{fctproblem}[3]
{
\vspace{2.5ex}
\noindent
\begin{tabular}{p{15mm} @{\textbf{:} } p{102mm}}
\hline
\multicolumn{2}{l}{\noindent {\sc #1}} \\
\hline
\textbf{input} & #2 \\
\textbf{output} & #3 \\
\hline
\end{tabular}
}
{
\vspace{1.5ex}
}

% use:
% \begin{problem}
% {name of problem}
% {input of problem}
% {output of problem}
% \end{problem}


%% SHORTCUTS

% definitions

\def\reals{{\mathbb R}}
\def\naturals{{\mathbb N}}

% vectors and polytopes
\def\T{^{\top}}
\def\0{{\bf 0}}
\def\1{{\bf 1}}

\def\conv{{\rm conv}}

% to choose the name of the problem - GALE or COMPLETELY LABELED GALE STRING

\def\gale{{\sc{Gale}}}
\def\anothergale{{\sc{Another Gale}}}


%%% END PREAMBLE

\begin{document}

\section{Introduction}

\section{Complexity, Games, Polytopes and Gale Strings}

\subsection{Some Complexity Classes}

\subsection{Normal Form Games and Nash Equilibria}

\todo[inline]{file: background-subsection}

\subsection{Some Geometrical Notation}

\subsection{Bimatrix Games, Labels and Polytopes}

\todo[inline]{file: polytopes-subsection}

\begin{equation}\label{br-polytopes}
P = \{ x\in\reals^m | x\geq\0,\ B\T x\leq\1 \},\quad
Q = \{ y\in\reals^n | y\geq\0,\ A y\leq\1 \},\quad
\end{equation}

a point in $P$ has label $k$ iff
$x_k = 0$ for $k\in \{ 1,\ldots,m \}$ or $(B\T x)_{k - m} = 0$ for
$k\in \{ m+1,\ldots,m+n \}$;
analogously, a point in $Q$ has label $k$ if and only if either
$y_{k - m} = 0$ for $k\in \{ m+1,\ldots,m+n \}$ or $(A y)_{k}$ for
$k\in \{ m+1,\ldots,m+n \}$.


\begin{proposition}\label{symmetric-eq-thm}
Let $(A,B)$ be a bimatrix game and $(x,y)$ be one of its Nash equilibria.
Then $(z,z)$, where $z=(x,y)$, is a Nash equilibrium of the symmetric game
$(C,C\T)$, where

\[
C = \left(
    \begin{array}{cc}
    0 & A \\
    B\T & 0
    \end{array}
    \right).
\]
\end{proposition}

\begin{theorem}\cite{uvg}\label{unit-vector-thm}
Let $l:[n]\to [m]$, and let $(U,B)$ be the unit vector game where
$U=(e_{l(1)}\ \cdots\ e_{l(n)})$. Consider the polytopes $P^l$ and $Q^l$
where

\begin{equation}\label{p-l-unitv}
P^l = \{ x\in\reals^m | x\geq\0,\ B\T x\leq\1 \}
\end{equation}

\begin{equation}
Q^l = \{ y\in\reals^n | y\geq\0,\
\sum_{\substack{j\in N_i \\ i\in [m]}} y_j\leq 1 \}
\end{equation}

where $N_i = \{ j\in [n] | l(j)=i \}$ for $i\in [m]$.

Give a labeling $l_f$ of the facets of $P^l$ according to the
inequality defining it, as follows:

\begin{eqnarray}\label{facet-labeling-unitv}
x_i\geq 0\text{ has label }i\text{ for }i\in [m] \\
B\T x)_j \leq 1\text{ has label }l(j)\text{ for }j\in [n]
\end{eqnarray}

Then $x\in P^l$ is a completely labeled point of $P^l\setminus\{\0\}$
if and only if there is some $y\in Q^l$ such that, after scaling,
the pair $(x,y)$ is a Nash equilibrium of $(U,B)$

\end{theorem}

\begin{theorem}\cite{balthasar}\label{unit-vector-dual-thm}
Let $Q$ be a labeled $m$-dimensional simplicial polytope with \0 in
its interior, with vertices $e_1,\ldots,e_m,c_1,\ldots,c_n$, so that
$F_0 = \conv\{ e_i\ |\ i\in [m] \}$ is a facet of $Q$.

Let $l:[n]\to [m]$, and let $(U,B)$ be the unit vector game with
$U=(e_{l(1)}\ \cdots\ e_{l(n)})$ and $B = (b_1\ \cdots\ b_n)$,
where $b_j = \frac{c_j}{1 + \1\T c_j}$ for $j\in [n]$.

Label the vertices of $Q$ as follows:
\begin{eqnarray}\label{vert-labeling-unitv}
l_v(-e_i)=i\text{ for }i\in [m] \\
l_v(c_j)=l(j)\text{ for }j\in [n]
\end{eqnarray}

Then a facet $F\neq F_0$ of $Q$ with normal vector $v$ is completely
labeled if and only if $(x,y)$ is a Nash equilibrium of $(U,B)$, where
$x = \frac{v + \1}{\1\T (v + \1)}$,
and $x_i = 0$ if and only if $−e_i\in F$ for $i\in [m]$.
Any $j$ so that $c_j$ is a vertex of $F$ represents a pure best reply to $x$;
the mixed strategy $y$ is the uniform distribution on the set of the pure best
replies to $x$.
\end{theorem}

\subsection{Cyclic Polytopes and Gale Strings}\label{gs-ssect}

\subsection{Labeling and the Problem \anothergale}

\todo[inline]{file: gale-def-subsection}

\begin{eqnarray}\label{gs-labeling-unitv}
l_s(i)=i\text{ for }i\in [d] \\
l_s(d+j)=l(j)\text{ for }j\in [n].
\end{eqnarray}

\newpage

\section{Algorithmic and Complexity Results}
\label{main-thm-sect}

\subsection{Lemke Paths and the Lemke-Howson for Gale Algorithm}

\todo[inline]{
We have NEs $\Leftrightarrow$ completely labeled things (facets, vertices, GS)
We give now different versions of fundam algorithm to deal with labeling
looking for compl.label. - in particular in these cases

first in version on simple polytopes with labeled facets,
(name: Lemke-Howson;
Lemke-Howson 1964, Shapley 1974 beautiful exposition)

then dual case with labeled vertices (name???? exchange? cite from???)

then in special case Gale strings (name: Lemke-Howson for Gale,
cite from???).

Mention general version - or leave it further results? (maybe better)
(name: exchange algorithm, Edmonds - Sanit\`{a}). (This case: index more
problematic - see )
}


Consider a labeling $l:[n]\to [m]$, for a set $X$ with $|X|=n$.
Then $x=(x_1,\ldots,x_m)\in X^m$ is {\em almost completely labeled} if
$|\{ j\in [n]\ |\ x_i=j\text{ for some }i\in [m]  \}|
= [m]\setminus \{ k \}$ for exactly one $k\in [m]$.
This mean that all labels appear once in $x$,
except for the {\em missing label} $k$,
and a {\em duplicate label} $\bar{k}\in [m]$ that appears twice.

\todo[inline]{now we see in poly with labeled facets, cl vertices}

Let $P$ be a simple polytope in dimension $m$ with $n$ facets.
We define the operation of {\em pivoting on vertices} as moving from a
vertex $x$ of $P$ to another vertex $y$ such that there is an edge
between $x$ and $y$. Note that, since $P$ is simple, there are exactly
$m$ possible choices for $y$.

Now let $l_f:[n]\to [m]$ be a labeling of the facets of $P$ such that there
is at least one completely labeled vertex $x_0$ of $P$.
Note that if we pivot from vertex $v$ we ``leave behind''
a facet $F$, that has label $k$; we call this {\em dropping label $k$}.
We will then reach a vertex $w$ that shares with $v$ all facets except
$F$ (that contains $v$ but not $w$) and another facet $G$
(that contains $w$ but not $v$) that has label $j$;
we will call this {\em picking up label $j$}.
We give the {\em Lemke-Howson algorithm} \todo{reference!}
as in \ref{lh-alg}.

\begin{algorithm}\label{lh-alg}
\SetKwInOut{Input}{input}
\SetKwInOut{Output}{output}
\Input{
A simple $m$-polytope $P$ with $n$ facets.
A labeling $l_f:[n]\to [m]$ of the facets of $P$.
A vertex $x_0$ of $P$, completely labeled for $l$.
}
\Output{
A completely labeled vertex $y\neq x_0$ of $P$.
}
\BlankLine
choose a label $k\in [n]$ \\
pivot from $x_0$ to $y$ dropping label $k$ \\
\While{ $y$ is not completely labeled }
{
pivot from $y$ to $y'$ dropping duplicate label $j$, moving away from $x_0$ \\
rename $x_0 = y$, $y = y'$
}
\Return $y$
\caption{Lemke-Howson algorithm}
\end{algorithm}

The steps of the Lemke-Howson algorithm result in a {\em Lemke path}
that connects two completely labeled vertices through
{\em $k$-almost complementary} vertices and edges, that is,
almost completely labeled vertices and edges where the only missing label
is $k$.
It remains to show that $y\neq x_0$. This comes from the fact that the Lemke
paths are {\em simple paths}, that is, there are no ``loops'' where a vertex
is visited more than once.
This is not possible because at each vertex there are
only two edges corresponding to the missing label $k$, since $P$ is not
degenerate; one is the edge that is traversed to get to the vertex, one is
the one that is traversed to leave it in the next step.
This proves the following.

\begin{theorem}\label{lh-works-thm}
The Lemke-Howson algorithm \ref{lh-alg} returns a solution to the problem
{\sc Another Completely Labeled Vertex}.
\end{theorem}

In the context of finding the Nash equilibrium of a bimatrix game $(A,B)$,
there are two equivalent implementations of the Lemke-Howson algorithm.

We can consider the game $C$ as in proposition \ref{symmetric-eq-thm},
and the associated polytope
$S = \{ z\in\reals^{m+n}\ |\ z\geq\0,\ Cz\leq\1 \}$,
labeling the $2(m+n)$ inequalities defining the facets of of $S$
as $1,\ldots,m+n,1,\ldots,m+n$.
Then applying the Lemke-Howson algorithm starting from
vertex $\0$ returns a Nash equilibrium $(z,z)$ of $C$ and a corresponding
$(x,y)=z$ a Nash equilibrium of $(A,B)$.

We can also follow the ``traditional'' version of the Lemke-Howson algorithm;
a very clear exposition of this can be found in Shapley \cite{shapley}.
Let $P$ and $Q$ be the best response polytopes of $(A,B)$ as in
\ref{br-polytopes}. We then move alternately on $P$ and $Q$, starting from
the couple of vertices $(\0,\0)$.
Since we move in $\reals^m$ and $\reals^n$ instead of $\reals^{m+n}$,
this version is more practical to visualize,
as shown in the following example.

\begin{example}

\todo[inline]{ex Savani - von Stengel, pag. 11; fig 8 are Schegel diagrams of
BR polytopes.}

\end{example}

Theorem \ref{lh-works-thm} has a straightforward dual version.
Let $Q$ be a simplicial polytope in dimension $m$ with $n$ vertices.
We {\em pivot on facets} by moving from facet $F$ to facet $G$
that shares an edge with $F$. Since $P$ is simplicial,
there are exactly $m$ possible choices for $G$.
Let $l_v:[n]\to [m]$ be a labeling of the vertices of $P$ such that there
is at least one completely labeled facet $F_0$.
We {\em drop label $k$} and {\em pick up label $j$} when pivoting
from a facet $F$ to a facet $G$ that shares with $F$ all vertices
except a vertex $v$ with label $k$ that belongs to $F$ but not $G$,
and another vertex $w$ with label $j$ that belongs to $G$ but not $F$.
The Lemke-Howson algorithm then becomes the theorem in \ref{lh-dual-alg}.
\todo{reference!}

\begin{algorithm}\label{lh-dual-alg}
\SetKwInOut{Input}{input}
\SetKwInOut{Output}{output}
\Input{
A simplicial $m$-polytope $Q$ with $n$ vertices.
A labeling $l_v:[n]\to [m]$ of the vertices of $P$.
A vertex $F_0$ of $Q$, completely labeled for $l$.
}
\Output{
A completely labeled facet $F\neq F_0$ of $Q$.
}
\BlankLine
choose a label $k\in [n]$ \\
pivot from $F_0$ to $F$ dropping label $k$ \\
\While{ $F$ is not completely labeled }
{
pivot from $F$ to $F'$ dropping duplicate label $j$, moving away from $F_0$ \\
rename $F_0 = F$, $F = F'$
}
\Return $F$
\caption{Lemke-Howson algorithm on facets}
\end{algorithm}

Considering the dual of Lemke paths on (almost) completely labeled facets,
we get the dual result to theorem \ref{lh-works-thm}.

\begin{theorem}\label{lh-dual-works-thm}
The algorithm \ref{lh-dual-alg} returns a solution to the problem
{\sc Another Completely Labeled Facet}.
\end{theorem}

To find a Nash equilibrium of a unit vector game $(U,B)$, where
$U=(e_{l(1)}\cdots e_{l(n)})$ for a labeling $l:[n]\to [m]$,
we can apply theorem \ref{unit-vector-thm} and algorithm \ref{lh-alg},
or we can apply the dual theorem \ref{unit-vector-dual-thm} and
algorithm \ref{lh-dual-alg}.
The first case relies on the polytope
$P^l= \{ x\in\reals^m\ |\ x\geq\0,\ B\T x\leq\1 \}$ defined
in \ref{p-l-unitv}; theorem \ref{unit-vector-thm} shows that
$P^l$ encodes all the Nash equilibria of $(U,B)$ as completely labeled
vertices, with an ``artificial'' equilibrium corresponding to the
vertex $\0$.
The second case relies on the polytope $Q$, defined
as the convex hull of vertices $-e_i$ for $i\in [m]$ and
$c_j=\frac{b_j}{1 - \1\T b_j}$ for $j\in [n]$; analogously, theorem
\ref{unit-vector-dual-thm} shows that $Q$ encodes all the Nash equilibria
of $(U,B)$ as completely labeled facets, with the ``artificial''
equilibrium corresponding to the facet $F_0=\conv(-e_1,\ldots,-e_m)$.

On the other hand, we can consider $(U,B)$ as any bimatrix game, and
apply algorithm \ref{lh-alg} to the product of the best response
polytopes $P$ and $Q$.
The projection of a Lemke path for a missing label $i\in [m]$
on $P\times Q$ to $P$ defines a Lemke path in $P^l$. \todo{how?}
However, $P\times Q$ has $m+n$ labels, therefore there could be
Lemke paths for a missing label $m+j$ with $j\in [n]$
on $P\times Q$ that get lost in the projection on $P^l$.
The following theorem, proved by Savani and von Stengel in \cite{uvg},
shows that there is no loss of generality in studying Lemke paths on
$P^l$; an analogous result holds for the dual case.

\begin{theorem}\label{p-l-general}
Let $(U,B)$ be a unit vector game, with
$U=(e_{l(1)}\cdots e_{l(n)})$ for a labeling $l:[n]\to [m]$;
let $P = \{ x\in\reals^m | x\geq\0,\ B\T x\leq\1 \}$ and
$Q = \{ y\in\reals^n | y\geq\0,\ A y\leq\1 \}$, as in \ref{br-polytopes};
and let $P^l= \{ x\in\reals^m\ |\ x\geq\0,\ B\T x\leq\1 \}$
as in \ref{p-l-unitv}. Then the Lemke path on $P\times Q$ for the missing
label $k$ projects to a path on $P$ that is the Lemke path on $P^l$
for missing label $k$ if $k\in [m]$,
and for missing label $l(j)$ if $k=m+j$ with $j\in [n]$.
\end{theorem}

As before, we now consider the case of unit vector games
where the simplicial polytope $Q$ is cyclic; that is,
the case that we can study from the point of view of Gale strings.




\todo[inline]{
sp case cycl poly leads to gs

thanks to dual of theorem SvS-15 \ref{p-l-general}, when doing labeling
as in \ref{gs-labeling-unitv}
we can take the str of labels $l(n+j)\cdots l(n+m)$
instead of $l(1)\cdots l(n+m)$,
that is, we could cut the ``artificial'' first labels $12...n$.



After all, in main we're studying ANOTHER GALE
in general, not nec starting from $12...n$; and we're interested in finding
{\em one} eq that's not the one we started from
(and is at other end of LPath, since index and so on),
{\em not all equilibria};
but the eq we started from is not nec the artificial one - actually,
if we go with this we can take any NE to start looking for another,
and we're sure to find a ``non-artificial'' one.
Note: if we were looking for all NE, LH doesn't work anyway - see ex by
Wilson in Shapley, where ``disconnected'' paths between equilibria.

----

complexity considerations, PPAD

Needed for why result interesting;
it can be done at the end, with Morris paths
}


\subsection{The Complexity of \gale\ and \anothergale}
\label{main-result-subsection}

\todo[inline]{file: main-result-subsection}


\begin{thebibliography}{00}

\frenchspacing\parskip0.3ex
\small

\bibitem{balthasar} A. V. Balthasar (2009).
``Geometry and equilibria in bimatrix games.''
PhD Thesis, London School of Economics and Political Science.

\bibitem{brightwell} G. R. Brightwell, P. Winkler (2004).
``Note on Counting Eulerian Circuits.''
CDAM Research Report LSE-CDAM-2004-12.

\bibitem{msc-diss} M. M. Casetti (2008).
``PPAD Completeness of Equilibrium Computation.''
MSc Thesis, London School of Economics and Political Science.

\bibitem{main} M. M. Casetti, J. Merschen, B. von Stengel (2010).
``Finding Gale Strings.''
\emph{Electronic Notes in Discrete Mathematics} 36, pp. 1065--1082.

\bibitem{cd} X. Chen, X. Deng (2006).
``Settling the Complexity of 2-Player Nash Equilibrium.''
\emph{Proc. 47th Annual IEEE Symposium on Foundations of
Computer Science (FOCS)}, pp. 261--272.

\bibitem{dgp} C. Daskalakis, P. W. Goldberg, C. H. Papadimitriou (2009).
``The Complexity of Computing a Nash Equilibrium.''
\emph{SIAM Journal on Computing} 39, pp. 195--259.

\bibitem{edm} J. Edmonds (1965).
``Paths, Trees, and Flowers.''
\emph{Canad. J. Math.} 17, pp. 449--467.

\bibitem{edm-oiks} J. Edmonds (2009).
``Euler complexes.''
In: \emph{Research Trends in Combinatorial Optimization}, eds. W.
Cook, L. Lovasz, and J. Vygen, Springer, Berlin, pp. 65--68.

\bibitem{edm-sanita} J. Edmonds, L. Sanit\`a (2010).
``On finding another room-partitioning of the vertices.''
\emph{Electronic Notes in Discrete Mathematics} 36, pp. 1257--1264.

\bibitem{gale} D. Gale (1963).
``Neighborly and Cyclic Polytopes.''
In: \emph{Convexity, Proc. Symposia in Pure Math.}, Vol. 7,
ed. V. Klee, American Math. Soc., Providence, Rhode Island, pp. 225--232.

\bibitem{gale-kuhn-tucker} D. Gale, H. W. Kuhn, A. W. Tucker (1950).
``On Symmetric Games.''
In: \emph{Contributions to the Theory of Games}~I, eds. H. W. Kuhn and A. W.
Tucker, \emph{Annals of Mathematics Studies} 24, Princeton University Press,
Princeton, pp. 81--87.

\bibitem{gilboa-zemel} I. Gilboa, E. Zemel (1989).
``Nash and correlated equilibria: some complexity considerations.''
\emph{Games and Economic Behavior} 1, pp. 80--93.

\bibitem{wicdiv} K. Gillen, J. McKelvie, M. Wilson (2015).
``Fear and Loathing in Eternity.''
{\em The Wicked + The Divine}, issue 9, ed. Image Comics.

\bibitem{gps} P. W. Goldberg, C. H. Papadimitriou, R. Savani (2011).
``The Complexity of the Homotopy Method, Equilibrium Selection, and
Lemke-Howson solutions.''
\emph{Proc. 52nd Annual IEEE Symposium on Foundations of Computer Science (FOCS)}, pp. 67--76.

\bibitem{lh} C. E. Lemke, J. T. Howson, Jr. (1964).
``Equilibrium Points of Bimatrix Games.''
\emph{J.  Soc. Indust. Appl. Mathematics} 12, pp.  413--423.

\bibitem{mclennan-tourky} A. McLennan, R. Tourky (2010).
``Imitation Games and Computation.''
\emph{Games and Economic Behavior} 70, pp. 4--11.

\bibitem{megiddo-papad} N. Megiddo, C. H. Papadimitriou (1991).
``On Total Functions, Existence Theorems and Computational Complexity.''
\emph{Theoretical Computer Science} 81, pp. 317--324.

\bibitem{jm} J. Merschen (2012).
``Nash Equilibria, Gale Strings, and Perfect Matchings.''
PhD Thesis, London School of Economics and Political Science.

\bibitem{morris} W. D. Morris Jr. (1994).
``Lemke Paths on Simple Polytopes.''
\emph{Math. Oper. Res.} 19, pp. 780--789.

\bibitem{nash} J. F. Nash (1951).
``Noncooperative games.''
\emph{Annals of Mathematics}, 54, pp. 289--295.

\bibitem{gth} M. J. Osborne, A. Rubinstein (1994).
{\em A Course in Game Theory.}
The MIT Press, Cambridge, Massachusetts.

\bibitem{papad-cc} C. H. Papadimitriou (1994).
{\em Computational Complexity.}
Addison-Wesley, Reading, MA.

\bibitem{ppad} C. H. Papadimitriou (1994).
``On the Complexity of the Parity Argument and Other Inefficient Proofs of
Existence.''
\emph{J. Comput. System Sci.} 48, pp. 498--532.

\bibitem{svs} R. Savani, B. von Stengel (2006).
``Hard-to-solve Bimatrix Games.''
\emph{Econometrica} 74, pp. 397--429.

\bibitem{uvg} R. Savani, B. von Stengel (2015).
``Unit Vector Games.''
arXiv:1501.02243v1 [cs.GT]

\bibitem{shapley} L. S. Shapley (1974).
``A Note on the Lemke-Howson Algorithm.''
\emph{Mathematical Programming Study 1: Pivoting and Extensions}, pp. 175--189

\bibitem{vvs} L. A. V\'{e}gh, B. von Stengel (2015),
``Oriented Euler Complexes and Signed Perfect Matchings.''
\emph{Mathematical Programming Series B} 150, pp. 153--178.

\bibitem{vn28} J. von Neumann (1928).
``Zur Theorie der Gesellschaftspiele.''
\emph{Mathematische Annalen} 100, pp. 295--320.

\bibitem{vs-agt} B. von Stengel (2007).
``Equilibrium computation for two-player games in strategic and extensive
form.'' Chapter 3, ``Algorithmic Game Theory,''
eds. N. Nisan, T. Roughgarden, E. Tardos, V. Vazirani.
Cambridge Univ. Press, Cambridge, pp. 53--78.

\bibitem{vs-noclf} B. von Stengel (2012).
``Completely Labeled Facet is NP-Complete.''
Manuscript, 6 pp.

\bibitem{ziegler} G. M. Ziegler (1995).
{\em Lectures on Polytopes.}
Springer, New York.

\end{thebibliography}


\end{document}
