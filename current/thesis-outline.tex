\documentclass[preprint,12pt]{article}

\newtheorem{theorem}{Theorem}[section]
\newtheorem{lemma}[theorem]{Lemma}
\newtheorem{proposition}[theorem]{Proposition}
\newtheorem{corollary}[theorem]{Corollary}

\newenvironment{proof}[1][Proof]
{\begin{trivlist}
\item[\hskip \labelsep {\bfseries #1}]}
{\end{trivlist}}

\newenvironment{definition}[1][Definition]
{\begin{trivlist}
\item[\hskip \labelsep {\bfseries #1}]}
{\end{trivlist}}

\newenvironment{example}[1][Example]
{\begin{trivlist}
\item[\hskip \labelsep {\bfseries #1}]}
{\end{trivlist}}

\newenvironment{remark}[1][Remark]
{\begin{trivlist}
\item[\hskip \labelsep {\bfseries #1}]}
{\end{trivlist}}



\usepackage{amssymb}
\usepackage[ruled,vlined]{algorithm2e}
\usepackage{amsmath}
\usepackage{amsfonts}
\usepackage{mathptmx}
\usepackage[latin1]{inputenc}
\usepackage{graphicx}
\usepackage[all]{xy}

\include{pstricks}
\include{cases}


\begin{document}

\title{The Gale String Problem for Equilibrium Computation in Games}

\begin{abstract}

At most 300 words!

\end{abstract}

\section*{Introduction}

What, why and how

\section{Bimatrix games}

\subsection{Basic definitions}

NE, BR poly

basic definitions: see section 2 in ge; improved version of thm 2.1 in ge is found in Prop 1 in VvS (general, not just cyclic)

\subsection{The Lemke-Howson algorithm}

LH

\subsection{Gale strings}

The background of the main result

SvS (include in subsection above)

\subsection{Pivoting and the class PPAD}

pivoting in general as on of the reasons to introduce PPA(D). just give the def of directed, the idea of pivoting + sign will be discussed in further results (it's the open problem following from our result solved in JM/VvS, not our main result)

oiks (so you can mention that EulG - as the ones used for MAIN are oik)

\section{The complexity of \textsc{Completely labeled Gale string} and \textsc{Another completely labeled Gale string}}

MAIN!
reduction to Perfect matching; both problems are in P

[why not call them GALE and ANOTHER GALE?]

\section{Further results}

The framework provided by our result led to further questions, related to the issue of the \textit{sign} of an index.

orientation, sign... (what follows, in JM, VvS)
Open problems

\end{document}
