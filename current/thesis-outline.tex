\documentclass[preprint,12pt]{article}

\newtheorem{theorem}{Theorem}[section]
\newtheorem{lemma}[theorem]{Lemma}
\newtheorem{proposition}[theorem]{Proposition}
\newtheorem{corollary}[theorem]{Corollary}

\newenvironment{proof}[1][Proof]
{\begin{trivlist}
\item[\hskip \labelsep {\bfseries #1}]}
{\end{trivlist}}

\newenvironment{definition}[1][Definition]
{\begin{trivlist}
\item[\hskip \labelsep {\bfseries #1}]}
{\end{trivlist}}

\newenvironment{example}[1][Example]
{\begin{trivlist}
\item[\hskip \labelsep {\bfseries #1}]}
{\end{trivlist}}

\newenvironment{remark}[1][Remark]
{\begin{trivlist}
\item[\hskip \labelsep {\bfseries #1}]}
{\end{trivlist}}



\usepackage{amssymb}
\usepackage[ruled,vlined]{algorithm2e}
\usepackage{amsmath}
\usepackage{amsfonts}
\usepackage{mathptmx}
\usepackage[latin1]{inputenc}
\usepackage{graphicx}
\usepackage[all]{xy}

\include{pstricks}
\include{cases}


\begin{document}

\title{The Gale String Problem for Equilibrium Computation in Games}

\begin{abstract}

(See other file)

\end{abstract}


\section{Introduction}

What, why and overview of tools used


\section{Basic definitions}

\subsection{Bimatrix games}

Nash Equilibria, Best response polytopes

see section 2 in ENDM article; improved version of thm 2.1 can be found in Prop 1 in VvS (general, not just cyclic polytopes)

\subsection{The Lemke-Howson algorithm}

The Lemke-Howson algorithm

\subsection{Gale strings}

Definition, Lemke-Howson for Gale (SvS)

\subsection{Pivoting and the class PPAD}

touch on pivoting as one of the reasons to introduce PPA(D). *just give the def of directed*, the idea of pivoting + sign will be discussed in "further results" section. The focus is "why the main result is relevant"

mention oiks, so you can later mention that EulG - as the ones used for MAIN are oik. Again: not too much.


\section{The complexity of \textsc{Completely labeled Gale string} and \textsc{Another completely labeled Gale string}}

Note: why not call them GALE and ANOTHER GALE? It would make it more readable.

**Main result!** - the reduction to Perfect matching; both GALE and ANOTHER GALE are in P, we're happy.


\section{Further results}

The framework provided by our result led to further questions, related to the issue of the *sign* of an index - and so on (Merschen, VvS)

Open problems (?)

\end{document}
