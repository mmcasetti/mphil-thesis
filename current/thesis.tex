\documentclass[preprint,12pt]{article}

\newtheorem{theorem}{Theorem}[section]
\newtheorem{lemma}[theorem]{Lemma}
\newtheorem{proposition}[theorem]{Proposition}
\newtheorem{corollary}[theorem]{Corollary}

\newenvironment{proof}[1][Proof]
{\begin{trivlist}
\item[\hskip \labelsep {\bfseries #1}]}
{\end{trivlist}}

\newenvironment{definition}[1][Definition]
{\begin{trivlist}
\item[\hskip \labelsep {\bfseries #1}]}
{\end{trivlist}}

\newenvironment{example}[1][Example]
{\begin{trivlist}
\item[\hskip \labelsep {\bfseries #1}]}
{\end{trivlist}}

\newenvironment{remark}[1][Remark]
{\begin{trivlist}
\item[\hskip \labelsep {\bfseries #1}]}
{\end{trivlist}}

\usepackage{amssymb}
\usepackage[ruled,vlined]{algorithm2e}
\usepackage{amsmath}
\usepackage{amsfonts}
\usepackage{mathptmx}
\usepackage[latin1]{inputenc}
\usepackage{graphicx}
\usepackage[all]{xy}

\include{pstricks}
\include{cases}

\begin{document}

\title{The Gale String Problem for Equilibrium Computation in Games}

\begin{abstract}

% New version from 10/12

This thesis presents a report on original research, published as conjoint work with Merschen and von Stengel in ENDM (2010). Our result shows a polynomial time algorithm to find a Nash equilibrium for a particular class of games, which was previously used by Savani and von Stengel (2006) as an example of exponential time for the classical Lemke-Howson algorithm for bimatrix games (1964).

It was conjectured that solving these games via the Lemke-Howson algorithm was complete in the class $\mathbf{PPAD}$ (Proof by Parity Argument, Directed version). A major motivation for the definition of this class by Papadimitriou (1994) was, in turn, to capture the pivoting technique of many results related to the Nash equilibrium, including the Lemke-Howson algorithm. A $\mathbf{PPAD}$-completness proof of the games we consider would have provided a traceable proof of the Daskalakis, Goldberg and Papaditrimiou (2005) and Chen and Deng (2009) results about the $\mathbf{PPAD}$-completeness of every normal form game. Our result of polynomial-time solvability, on the other hand, indicates the existence of a special class of games, unless $\mathbf{PPAD} = \mathbf{P}$.

Our proof exploits two results. The first one is the representation of the Nash equilibria of these games as a string of labels and an associated string of $0$s and $1$s satisfying some conditions, called {\em Gale conditions}, as seen in Savani and von Stengel (2006). The second one is the polynomial-time solvability of the problem of finding a perfect matching in a graph, solved by Edmonds (1965).

Further results by Merschen (2012) and V\'{e}gh and von Stengel (2014) solved the open problem of the {\em sign} of the equilibrium found in polynomial time.

\end{abstract}


\section{Introduction}

What, why and overview of tools used


\section{Definitions}

\subsection{Gale strings}
\label{galestrings}

Definition

\subsection{Bimatrix games}
\label{games}

% Nash Equilibria, Best response polytopes

% see section 2 in ENDM article; improved version of thm 2.1 can be found in Prop 1 in VvS (general, not just cyclic polytopes)

[basic on games, NE]

Gale strings can be used to study the problem of finding the Nash equilibrium in a special class of games. These games are characterised by having a \textit{cyclic polytope} as their best response polytope.

A ($d$-dimensional) {\em simplicial polytope} $P$ is the convex hull of a set of at least $d+1$ points $v$ in $\reals^d$ in general position, that is, no $d+1$ of them are on a common hyperplane.

If a point $v$ cannot be omitted from these points without changing $P$ then $v$ is called a {\em vertex} of $P$. A {\em facet} of $P$ is the convex hull $\conv\,F$ of a set $F$ of $d$ vertices of $P$ that lie on a hyperplane $\{ x\in \reals^d\mid a\T x=a_0\}$ so that $a\T u<a_0$ for all other vertices $u$ of $P$; the vector $a$ (unique up to a scalar multiple) is called the {\em normal vector} of the facet. We often identify the facet with its set of vertices~$F$.

A {\em cyclic polytope\/} $P$ in dimension~$d$ with $n$
vertices is the convex hull of $n$ points $\mu(t_j)$ on the
{\em moment curve\/} $\mu\colon t\mapsto
(t,t^2,\ldots,t^d)^\top$ for $j\in [n]$.
Suppose that $t_1<t_2<\cdots< t_n$.
Then the facets of $P$ are encoded by $G(d,n)$, that is,
\[
F \hbox{ is a facet of }P
\quad
\Longleftrightarrow
\quad
F = \conv\{\mu(t_i)\mid i\in 1(s)\}
\hbox { for some }s\in G(d,n),
\]
as shown by Gale \cite{gale}.
% Essentially, this holds because any set $S\subset [n]$
% the moment curve defines a unique hyperplane which is crossed
% (and not just touched) by the moment curve; if the bitstring
% $s$ that encodes $F$ as $1(s)$ has a substring $01^k0$
For this cyclic polytope $P$, a labeling $l:[n]\to[d]$ can
be understood as a label $l(j)$ for each vertex $\mu(t_j)$
for $j\in [n]$.
A completely labeled Gale string $s$ therefore represents a
facet $F$ of $P$ that is completely labeled.

The following theorem, due to Balthasar and von Stengel
\cite{B09,BvS10}, establishes a connection between general
labeled polytopes and equilibria of certain $d\times n$
bimatrix games $(U,B)$.

\begin{theorem}\label{t-unitv}
Consider a labeled  $d$-dimensional simplicial polytope $Q$ with $\0$ in
its interior, with vertices $-e_1,\ldots,-e_d,c_1,\ldots,c_n$,
% We need to assume that the c_i are pairwise distinct, otherwise
% a vertex can have several labels.
% We need to assume that c_i vertex of Q for the following reason:
% the definition of completely labeled facet is difficult if
% c_i is in a facet but not vertex of the facet.
so that % $F_0$ in {\rm(\ref{F0})}
$F_0=\conv\{-e_1,\ldots,-e_d\}$
is a facet of $Q$.
Let $-e_i$ have label $i$ for $i\in[d]$, and let $c_j$
have label $l(j)\in[d]$ for $j\in[n]$.
Let $(U,B)$ be the $d\times n$ bimatrix game with
$U=[e_{l(1)}\cdots\,e_{l(n)}]$ and $B=[b_1\,\cdots\,b_n]$,
where $b_j=c_j/(1+\rone c_j)$ for $j\in[n]$.
Then the completely labeled facets $F$ of $Q$, with the
exception of~$F_0$, are in one-to-one correspondence to the
Nash equilibria $(x,y)$ of the game $(U,B)$ as follows:
if $v$ is the normal vector of $F$, then
$x=(v+\1)/\rone (v+\1)$,
and $x_i=0$ if and only if $-e_i\in F$ for $i\in[d]$;
any other label~$j$ of $F$, so that $c_j$ is a
vertex of~$F$, represents a pure best reply to~$x$.
The mixed strategy $y$ is the uniform distribution on
the set of pure best replies to~$x$.
\end{theorem}

In the preceding theorem, any simplicial polytope can take
the role of $Q$ as long as it has one completely labeled
facet~$F_0$.
Then an affine transformation, which does not change the
incidences of the facets of $Q$, can be used to map $F_0$ to
the negative unit vectors $-e_1,\ldots,-e_d$ as described,
with $Q$ if necessary expanded in the direction $\1$ so that
$\0$ is in its interior.

A $d\times n$ bimatrix game $(U,B)$ is a {\em unit vector game}
if all columns of $U$ are unit vectors.
For such a game $B$ with $B=[b_1\cdots b_n]$, the columns
$b_j$ for $j\in[n]$ can be obtained from $c_j$ as in
Theorem~\ref{t-unitv} if $b_j>\0$ and $\1\T b_j<1$.
This is always possible via a positive-affine transformation
of the payoffs in~$B$, which does not change the game.
The unit vectors $e_{l(j)}$ that constitute the columns of
$U$ define the labels of the vertices $c_j$.
The corresponding polytope with these vertices is simplicial
if the game $(U,B)$ is nondegenerate \cite{vS02}, which here
means that no mixed strategy $x$ of the row player has more
than $|\{i\in[d]\mid x_i>0\}|$ pure best replies.
Any game can be made nondegenerate by a suitable
``lexicographic'' perturbation of $B$, which can be
implemented symbolically.

Unit vector games encode arbitrary bimatrix games:
An $m\times n$ bimatrix game $(A,B)$ with (w.l.o.g.{})
positive payoff matrices $A,B$ can be symmetrized so
that its Nash equilibria are in one-to-correspondence to the
symmetric equilibria of the $(m+n)\times(m+n)$
symmetric game $(C\T,C)$ where
\[
\label{symmetrize}
C=\biggl(\begin{matrix} 0 & B\cr A^\top & 0\cr\end{matrix}\biggr).
\]
In turn, as shown by McLennan and Tourky \cite{mt},
the symmetric equilibria $(x,x)$ of any symmetric game
$(C\T,C)$ are in one-to-one correspondence to the Nash
equilibria $(x,y)$ of the ``imitation game'' $(I,C)$ where
$I$ is the identity matrix; the mixed strategy $y$ of the
second player is simply the uniform distribution on the
set $\{i\mid x_i>0\}$.
Clearly, $I$ is a matrix of unit vectors, so $(I,C)$ is a
special unit vector game.

Special games are obtained by using cyclic polytopes in
Theorem~\ref{t-unitv}, suitably affinely transformed with
a completely labeled facet $F_0$.
When $Q$ is a cyclic polytope in dimension $d$ with $d+n$
vertices, then the string of labels $l(1)\cdots l(n)$ in
Theorem~\ref{t-unitv} defines a labeling $l':[d+n]\to [d]$
where $l'(i)=i$ for $i\in [d]$ and
$l'(d+j)=l(j)$ for $j\in [n]$.
In other words, the string of labels $l(1)\cdots l(n)$ is
just prefixed with the string $1\,2\cdots d$ to give $l'$.
Then $l'$ has a trivial completely labeled Gale string
$1^d0^n$ which defines the facet $F_0$.
Then the problem \textsc{Another completely labeled Gale
String} defines exactly the problem of finding a Nash
equilibrium of the unit vector game $(I,B)$.
Note again that $B$ is here not a general matrix (which would
define a general game) but obtained from the last $n$ of
$d+n$ vertices of a cyclic polytope in dimension~$d$.

\subsection{The Lemke-Howson algorithm}

The Lemke-Howson algorithm

for Gale

\subsection{Pivoting and the class PPAD}

touch on pivoting as one of the reasons to introduce PPA(D). *just give the def of directed*, the idea of pivoting + sign will be discussed in "further results" section. The focus is "why the main result is relevant"

mention oiks, so you can later mention that EulG - as the ones used for MAIN are oik. Again: not too much.


\section{The complexity of \textsc{Completely labeled Gale string} and \textsc{Another completely labeled Gale string}}

Note: why not call them GALE and ANOTHER GALE? It would make it more readable.

**Main result!** - the reduction to Perfect matching; both GALE and ANOTHER GALE are in P, we're happy.


\section{Further results}

The framework provided by our result led to further questions, related to the issue of the *sign* of an index - and so on (Merschen, VvS)

Open problems (?)


\section*{Appendix: Notation}

For a matrix $A$ we denote its transpose with $A\T$. We treat vectors $u,v$ in $\reals^d$ as column vectors, so $u\T v$ is their scalar product. By $\0$ we denote a vector of all $0$'s, of suitable dimension, by $\1$ a vector of all $1$'s. A unit vector, which has a 1 in its $i$th component and 0 otherwise, is denoted by $e_i$. Inequalities like $u\ge\0$ hold for all components. For a set of points $S$ we denote its convex hull by $\conv\,S$.



\begin{thebibliography}{00}

% As in ENDM, so far

\frenchspacing\parskip-1ex
\small

\bibitem{B09}
A. V. Balthasar (2009),
Geometry and Equilibria in Bimatrix Games.
PhD Thesis, London School of Economics.

\bibitem{BvS10}
A. Balthasar, B. von Stengel (2010),
Index and uniqueness of symmetric equilibria.
In preparation.

\bibitem{cd} X. Chen, X. Deng (2006).
Settling the complexity of two-player Nash equilibrium.
\emph{Proc. 47th FOCS}, pp. 261--272.

\bibitem{dgp} C. Daskalakis, P. W. Goldberg, C. H. Papadimitriou (2006).
The complexity of computing a Nash equilibrium.
\emph{Proc. Ann. 38th STOC}, pp. 71--78.

% \bibitem{dgp2} C. Daskalakis, P. W. Goldberg, C. H.  Papadimitriou (2009).
% The complexity of computing a Nash equilibrium.
% Commun.\ ACM 52, pp. 89--97.

\bibitem{edm} J. Edmonds (1965).
Paths, trees, and flowers. \emph{Canad. J. Math.} 17, pp. 449--467.

\bibitem{je07}
J. Edmonds (2007). Euler complexes. Manuscript,  5 pp.

\bibitem{gale} D. Gale (1963),
Neighborly and cyclic polytopes.
In: Convexity, Proc. Symposia in Pure Math., Vol. 7, ed. V. Klee, American Math. Soc., Providence, Rhode Island, pp. 225--232.

% \bibitem{gz} I. Gilboa and E. Zemel. Nash and correlated equilibria: Some complexity considerations. Games and Economic Behavior, 1989.

% \bibitem{k} L. G. Khachiyan (1979). A polynomial algorithm in linear programming. \emph{Soviet Mathematics Doklady}, 20(1), pp. 191--194.

\bibitem{lg}
C. E. Lemke, S. J. Grotzinger (1976).
On generalizing Shapley's index theory to labelled
pseudomanifolds.
Math. Programming 10, 245--262.

\bibitem{lh} C. E. Lemke, J. T. Howson, Jr. (1964).
Equilibrium points of bimatrix games.
\emph{J.  Soc. Indust. Appl. Mathematics} 12, pp.  413--423.

\bibitem{mt}
A. McLennan, R. Tourky (2009).
Simple complexity from imitation games.
\emph{Games and Economic Behavior},
in press, doi:10.1016/j.geb.2009.10.003

\bibitem{morris}
W. D. Morris Jr. (1994).
Lemke paths on simple polytopes.
\emph{Math. Oper. Res.} 19, pp. 780--789.

% \bibitem{nash} J. Nash (1951). Noncooperative games.
% \emph{Ann. Math.} 54, pp. 289--295.

\bibitem{ppad} C. H. Papadimitriou (1994).
On the complexity of the parity argument and other inefficient proofs of existence.
\emph{J. Comput. System Sci.} 48, pp. 498--532.

\bibitem{svs} R. Savani, B. von Stengel (2006).
Hard-to-solve bimatrix games.
\emph{Econometrica} 74, pp. 397--429.

% \bibitem{shapley} L. S. Shapley (1974), A Note on the Lemke-Howson Algorithm. Mathematical Programming Study, 1, pp. 175--189.

\bibitem{todd} M. J. Todd (1976), Orientation in
complementary pivot algorithms.
\emph{Math. Oper. Res.} 1, pp. 54--66.

\bibitem{vS02}
B. von~Stengel (2002).
Computing equilibria for two-person games.
In: \emph{Handbook of Game Theory, Vol.~3,}
eds. R. J. Aumann and S. Hart, North-Holland, Amsterdam,
pp. 1723--1759.


% \bibitem{val} L. G. Valiant (1979), The complexity of computing the permanent. Theoretical Computer Science 8, pp. 89--201.

% \bibitem{vN}  J. von Neumann (1928). Zur Theorie der Gesellschaftspiele, \emph{Mathematische Annalen}, 100, pp. 295--320.


\end{thebibliography}

\end{document}
