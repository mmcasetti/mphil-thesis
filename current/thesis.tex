\documentclass[11pt, draft]{article}

%% TYPESETTING

% draft:
\linespread{1.3}
\usepackage{todonotes}

% final (?) TODO: check with LSE requirements
% page measurements
% \setlength{\hoffset}{0mm}
% \setlength{\oddsidemargin}{25mm}
% \setlength{\textwidth}{130mm}

%% PACKAGES

\usepackage{amssymb}
\usepackage{amsmath}
\usepackage{amsthm}

\usepackage[all]{xy}

\usepackage[ruled,vlined,linesnumbered]{algorithm2e}

%% THEOREMS and ENVIRONMENTS

% theorems

\newtheorem{theorem}{Theorem}
\newtheorem{property}{Property}[section]
\theoremstyle{definition}\newtheorem{definition}{Definition}
\theoremstyle{remark}\newtheorem{example}{Example}[section]

% environments

% computational problem
% TODO check p{textwidth} in final version

\newenvironment{problem}[3]
{
\noindent
\begin{tabular}{p{13mm} @{\textbf{:} } p{105mm}}
\hline
\multicolumn{2}{l}{\noindent \textsc{#1}} \\
\hline
\textbf{input} & #2 \\
\textbf{output} & #3 \\
\hline
\end{tabular}
}
{\\}

% use:
% \begin{problem}
% {name of problem}
% {input of problem}
% {output of problem}
% \end{problem}


%% SHORTCUTS

% definitions

\def\reals{{\mathbb R}}
\def\naturals{{\mathbb N}}
\def\conv{{\rm conv}}
\def\0{{\bf0}}
\def\1{{\bf1}}
\def\T{^{\top}}
\def\rone{{\1\T}}

% to choose the name of the problem - GALE or COMPLETELY LABELED GALE STRING

\def\gale{{\sc{Gale}}}
\def\anothergale{{\sc{Another Gale}}}


%%% END PREAMBLE

\begin{document}

\title{Complexity of the Gale String Problem for Equilibrium
Computation in Games}

\begin{abstract}
This thesis presents a report on original research, published as joint
work with Merschen and von Stengel in {\em Electronic Notes in Discrete}
\mbox{{\em Mathematics} \cite{main}.} Our result shows a polynomial time
algorithm to solve two problems related to labeled Gale strings, a
combinatorial structure consisting a string of labels and a bitstring
satisfying certain conditions introduced by Gale \mbox{in \cite{gale}.}

Gale strings can be used in the representation of a particular class
of games that Savani and von \mbox{Stengel \cite{svs}} used as an example
of exponential running time for the classical \mbox{Lemke-Howson}
algorithm to find a Nash equilibrium of a bimatrix \mbox{game \cite{lh}.}
It was conjectured that solving these games via the Lemke-Howson algorithm
was complete in the \mbox{class PPAD} (Proof by Parity Argument, Directed
version). A major motivation for the definition of this class by
\mbox{Papadimitriou \cite{ppad}} was, in turn, to capture the pivoting
technique of many results related to the Nash equilibrium, including
the Lemke-Howson algorithm.

Our result, on the contrary, sets apart this class of games as a case
for which there is a polynomial-time algorithm to find a Nash equilibrium.
Since Daskalakis, Goldberg and \mbox{Papaditrimiou \cite{dgp}} and
Chen and \mbox{Deng \cite{cd}} proved the \mbox{PPAD-completeness} of
finding a Nash equilibrium in general normal-form games, we have a special
class of games, unless \mbox{PPAD = P}.

Our proof exploits two results. The first one is the representation
of the Nash equilibria of these games as Gale strings, as seen in Savani
and von \mbox{Stengel \cite{svs}.} The second one is the polynomial-time
solvability of the problem of finding a perfect matching in a graph, proven
by \mbox{Edmonds \cite{edm}.}

Further results by \mbox{Merschen \cite{jm}} and V\'{e}gh and von
\mbox{Stengel \cite{vvs}} will be mentioned.

An appendix relates an amendment to the proof of the
\linebreak[5]
\mbox{PPAD-completeness} result by Daskalakis, Goldberg and
\mbox{Papaditrimiou \cite{dgp}.}
\end{abstract}

\newpage

\section{Introduction}

\section{Games, Polytopes and Gale Strings}

\subsection{Games and Nash Equilibria}

\subsection{Bimatrix Games and Best Response Polytopes}

\todo[inline]{possibly one subsection only}

\subsection{Cyclic Polytopes and Gale Strings}

A special case of games is obtained by taking a particular case of best
response polytope in theorem \ref{t-unitv}.

\begin{definition}\label{cyclic-polytope}
A {\em cyclic polytope} $P$ in dimension~$d$ with $n$ vertices is the
convex hull of distinct points $\mu(t_j)$, where $j\in [n]$ and $\mu$ is the
{\em moment curve}
\[
\mu\colon t\mapsto(t,t^2,\ldots,t^d)^\top
\]
\end{definition}

Cyclic polytopes can be represented through a combinatorial structure, the
{\em Gale strings}. This makes their study particularly interesting, and, as
we will see, it can be used to obtain very elegant proofs.

\begin{definition}\label{bitstring-def}
For any integer $k$ and any set $S$, we can represent the function
$f_s:[k]\to S$ as the string $s = s(1)s(2)\cdots s(k)$. If $S=\{0,1\}$ we
denote
\begin{align*}
\mathnormal{1}(s) & = s^{-1}(1) \\
     & = \{j\in [k]\mid s(j)=1\}
\end{align*}

The indicator function of $\mathnormal{1}(s)$ will then correspond to a
{\em bitstring} $s$, a sequence of $0$'s and $1$'s.

A maximal substring of consecutive $1$'s in a bitstring is called a
{\em run}.
\end{definition}

\begin{example}
Let $k = 6$, and let $f_s(j)=0$ if $j$ is even and $f_s(j)=1$ if $j$ is odd.
Then $s = 101010$ and $1(s) = {1,3,5}$.
\end{example}

We can now give the definition of {\em Gale string}.

\begin{definition}\label{gs-def}
We denote as $G(d,n)$ the set of all bitstrings $s$ of length $n$ such that

\begin{enumerate}
\item exactly $d$ bits in $s$ are $1$ and
\item $s$ fulfills the {\em Gale evenness condition}:
\[
01^k0\hbox{ is a substring of }s\quad{\Rightarrow}\quad k\hbox{ is even.}
\]
\end{enumerate}

An element of of $G(d,n)$ is called a {\em Gale string of dimension~$d$ and
length~$n$}.
\end{definition}

Definition \ref{gs-def} characterises Gale strings as bitstrings of length~$n$ with exactly~$d$ elements equal to $1$, such that {\em interior}
runs (that is, runs bounded on both sides by~$0$s) must be of even length.
Note that this condition allows Gale strings to start or end with an
odd-length run.

This leads to an important consequence when~$d$ is even.

\begin{property}\label{even-d-gs}
Let $d$ be even, and let $s$ in $G(d,n)$. Then if $s$ starts with an odd run
it will also end with an odd run, and if $s$ starts with an even run it will
end with an even run.

\todo[inline]{use "modulo" - even more than "cyclic shift"}
That is, the set of Gale strings of even dimension is therefore invariant
under a cyclic shift of the strings.
\end{property}

We can then consider the Gale strings in $G(d,n)$ with even $d$ as a ``loop''
obtained by ``glueing together'' the extremes of the string to form an even
run.

\begin{example}\label{gs-example}
We consider $G(4,6)$. We have
\begin{align*}
G(4,6) = \{ & 111100, \\
        & 111001, \\
        & 110011, \\
        & 100111, \\
        & 001111, \\
        & 011110, \\
        & 110110, \\
        & 101101, \\
        & 011011\}
\end{align*}

\todo[inline]{change mention of "cyclic shift" if not used before}
The strings $111100$, $111001$, $110011$, $100111$, $001111$ and $011110$ are
equivalent under a cyclic shift, as are the strings $110110$, $101101$ and
$011011$.
\end{example}

\todo[inline]{here to end subsect: polytopes}

The relation between cyclic polytopes and Gale strings is given by the
following theorem by Gale \cite{gale-cyclicpoly}.

\begin{theorem}[\cite{gale-cyclicpoly}]\label{cp-gs-gale}
For any positive integer $n$, assume that $t_1 < t_2 < \cdots < t_n$ and let
P be the cyclic polytope obtained by taking $t_j$, where $j \in [n]$, in
definition \ref{cyclic-polytope}.

Then the facets of $P$ are encoded by $G(d,n)$; that is, $F$ is a facet of
$P$ if and only if
\[
F = \conv\{\mu(t_i)\mid i\in 1(s)\} \qquad \hbox{ for some }s\in G(d,n)
\]
\end{theorem}

\todo[inline]{sketch of pf if not too long and it uses relevant techniques}

\todo[inline]{graphics of cyclic polytope - cfr vS articles and talks}

From this point forward, we will assume that $d$ is even.

\todo[inline]{give something to generalise to odd case}

\subsection{Labeling and the Problem \anothergale}

\begin{definition}
Given a set $G$ of bitstrings of length $n$ and a parameter $d$, a
{\em labeling} is a function $l:[n]\to[d]$. A string $s$ in $G(d,n)$ is
{\em completely labeled} if $l(\mathnormal{1}(s))=[d]$. Any $l(i)\in [d]$
is called a {\em label}
\end{definition}

If $s \in G(d,n)$ is completely labeled for the labeling $l:[n]\to[d]$, then
for each label $l(i)$ there is a bit $s(i)=1$. We therefore have exactly
$d$ positions $i$ for which $s(i)=1$; hence, $|l(\mathnormal{1}(s))|=d$.

\begin{example}
Given the string of labels $l=123432$, there are four associated completely
labeled Gale strings: $111100$, $110110$, $100111$ and $101101$.\\

\begin{center}
{\renewcommand{\tabcolsep}{2ex}
\begin{tabular}{|c|c|c|c|}
\hline
\textbf{1234}32 &
\textbf{12}3\textbf{43}2 &
\textbf{1}23\textbf{432} &
\textbf{1}2\textbf{34}3\textbf{2} \\
\textbf{1111}00 &
\textbf{11}0\textbf{11}0 &
\textbf{1}00\textbf{111} &
\textbf{1}0\textbf{11}0\textbf{1} \\
\hline
\end{tabular}
}\\
\end{center}

\end{example}

Sometimes there aren't any completely labeled Gale strings that are
associated with a given labeling.

\begin{example}
For $l = 121314$, there are no completely labeled Gale strings.
\end{example}

\todo[inline]{here to end subsect: polytopes}
\todo[inline]{graphics of labeled cyclic polytope}

% Essentially, this holds because any set $S\subset [n]$
% the moment curve defines a unique hyperplane which is crossed
% (and not just touched) by the moment curve; if the bitstring
% $s$ that encodes $F$ as $1(s)$ has a substring $01^k0$
For this cyclic polytope $P$, a labeling $l:[n]\to[d]$ can
be understood as a label $l(j)$ for each vertex $\mu(t_j)$
for $j\in [n]$.
A completely labeled Gale string $s$ therefore represents a
facet $F$ of $P$ that is completely labeled.

Special games are obtained by using cyclic polytopes in
Theorem~\ref{t-unitv}, suitably affinely transformed with
a completely labeled facet $F_0$.
When $Q$ is a cyclic polytope in dimension $d$ with $d+n$
vertices, then the string of labels $l(1)\cdots l(n)$ in
Theorem~\ref{t-unitv} defines a labeling $l':[d+n]\to [d]$
where $l'(i)=i$ for $i\in [d]$ and
$l'(d+j)=l(j)$ for $j\in [n]$.
In other words, the string of labels $l(1)\cdots l(n)$ is
just prefixed with the string $1\,2\cdots d$ to give $l'$.
Then $l'$ has a trivial completely labeled Gale string
$1^d0^n$ which defines the facet $F_0$.
Then the problem \anothergale\ defines exactly the problem of finding a Nash
equilibrium of the unit vector game $(I,B)$.
Note again that $B$ is here not a general matrix (which would
define a general game) but obtained from the last $n$ of
$d+n$ vertices of a cyclic polytope in dimension~$d$.\\

\begin{problem}
{\anothergale}
{A labeling $l:[n]\to[d]$, where $d$ is even and $d<n$, and an associated
completely labeled Gale string $s$ in $G(d,n)$.}
{A completely labeled Gale string $s'$ in $G(d,n)$ associated with $l$, such
that $s' \neq s$.}
\end{problem}

\newpage

\section{Algorithmic and Complexity Results}
\label{main-thm-sect}

\subsection{Pivoting}

\subsection{The Lemke-Howson Algorithm and Parity}

\begin{definition}\label{almost-completely-labeled}
Let $l:[d]\to [n]$ be a labeling. An {\em almost completely labeled} Gale
string associated with $l$ is $s\in G(d,n)$ such that $|l(1(s))|=d-1$
\end{definition}

If $s$ is an almost completely labeled Gale string $s$ associated with the
labeling $l:[n]\to[d]$, then for each label $l(i)$ but one there is a bit
$s(i)=1$. Furthermore, there will be exactly one ``duplicate'' label $l(i)$
such that $s(i)=s(j)=1$ for exactly one $j\neq i$

Completely and almost completely labeled Gale strings are used to build the
{\em Lemke-Howson for Gale algorithm}. We first define its fundamental
subroutine, the {\em pivoting algorithm}.

\begin{definition}\label{pivoting}
We define as {\em pivoting} the operation defined in algorithm
\ref{pivoting-algorithm}, where we consider the Gale strings as ``loops'' by
identifying position $i$ with any position $i+kn$

\todo[inline]{use "modulo" notation}

\todo[inline]{for clgs, is dropped label necessarily double one?}

\begin{algorithm}\label{pivoting-algorithm}
\SetKwInOut{Input}{input}
\SetKwInOut{Output}{output}
\Input{A string of labels $l$ of length~$n$; a completely or almost
completely labeled Gale string for $l$; $i \in [n]$ such that $s(i)=1$}
\Output{A complete or almost complete Gale string for $l$.}
\BlankLine
set $s(i)=0$ \\
let $j$ be the length of the odd maximal run of $1$s created by this \\
\If{ the odd maximal run of $1$s is on the \textbf{right} of position $i$ }
{ set $s(i+j+1)=1$ }
\Else
{ set $s(i-j-1)=1$ }
\Return $s$
\caption{Pivoting on completely labeled Gale strings}
\end{algorithm}

The operation where we set $s(i)=0$ (line 1) is called {\em dropping
label $i$}.

\end{definition}

\begin{example}
Let $l = 123432$. The Lemke-Howson for Gale algorithm applied to the
completely labeled Gale string $s = 111100$ and the label
\underline{\underline{4}} in position 4 returns the almost completely labeled
Gale string $011110$.

\begin{center}
{\renewcommand{\tabcolsep}{2ex}
\begin{tabular}{|c|p{100mm}|}
\hline
\underline{123\underline{4}}32 &
drop the label \textbf{4} from $l$ \\
\textbf{111\underline{1}}00 &
taking the corresponding \textbf{1} in $s$ \\
\textbf{111}\underline{0}00 &
and setting it to \textbf{0} \\
\cline{2-2}
\textbf{111}00\textbf{1} &
set the other end of the odd run in $s$ to \textbf{1} \\
\cline{2-2}
\underline{1\underline{2}3}43\underline{\underline{2}} &
there is a \textbf{1} in $s$ in a position corresponding to the labels
\textbf{1}, \textbf{2} \underline{(twice)} and \textbf{3} in $l$;
there are only \textbf{0}'s in all the positions corresponding to the
label \textbf{4} in $l$ \\
\hline
\end{tabular}
}\\
\end{center}

\end{example}

\begin{example}
Let $l = 123424$. The Lemke-Howson for Gale algorithm applied to the almost
completely labeled Gale string $s = 110011$ and the duplicate label
\underline{\underline{2}} in position 2 returns the almost completely
labeled Gale string $100111$.

\begin{center}
{\renewcommand{\tabcolsep}{2ex}
\begin{tabular}{|c|p{100mm}|}
\hline
\underline{1\underline{2}}34\underline{24} &
drop the label \textbf{2} in position 2 from $l$ \\
\textbf{11}00\textbf{11} &
taking the corresponding \textbf{1} in $s$ \\
\textbf{1}\underline{0}00\textbf{11} &
and setting it to \textbf{0} \\
\cline{2-2}
\textbf{1}00\textbf{111} &
set the other end of the odd run (on the loop) in $s$ to \textbf{1} \\
\cline{2-2}
\underline{1}23\underline{\underline{4}2\underline{4}} &
there is a \textbf{1} in $s$ in a position corresponding to the labels
\textbf{1}, \textbf{2} and \textbf{4} \underline{(twice)} in $l$;
there are only \textbf{0}'s in all the positions corresponding to the
label \textbf{3} in $l$ \\
\hline
\end{tabular}
}\\
\end{center}

If we drop duplicate label \underline{\underline{2}} in position 5 instead,
the algorithm returns the completely labeled Gale string $111001$.

\begin{center}
{\renewcommand{\tabcolsep}{2ex}
\begin{tabular}{|c|p{100mm}|}
\hline
\underline{12}34\underline{\underline{2}4} &
drop the label \textbf{2} in position 5 from $l$ \\
\textbf{11}00\underline{\underline{1}1} &
taking the corresponding \textbf{1} in $s$ \\
\textbf{11}00\underline{0}\textbf{1} &
and setting it to \textbf{0} \\
\cline{2-2}
\textbf{111}00\textbf{1} &
set the other end of the odd run (on the loop) in $s$ to \textbf{1} \\
\cline{2-2}
\underline{123}42\underline{4} &
\underline{all the labels} in $l$ correspond to a \textbf{1} in $s$ \\
\hline
\end{tabular}
}\\
\end{center}

\end{example}

Note that, by the Gale evenness condition, we must have an the odd maximal
run of $1$'s either on the right or on the left of position $i$.

If the Gale string in the pivoting algorithm is completely labeled, we can
drop any label $l(i)$ such that $s(i) = 1$. We refer to these labels as {\em
free labels}.

Each pivoting can be seen as a step of a ``path'' through (almost) completely
labeled Gale strings. If the first and last step of the path are completely
labeled Gale strings, the path is described by the {\em Lemke-Howson for Gale
algorithm}.

\begin{definition}\cite{lhg-ref}

We define the {\em Lemke-Howson for Gale} algorithm as follows:

\begin{algorithm}\label{lhg-algorithm}
\SetKwInOut{Input}{input}
\SetKwInOut{Output}{output}
\Input{A $n$-string $l \in [n]$ where $d$ is even and $d< n$; a completely
labeled Gale string $s$ associated with $l$.}
\Output{A Gale string $s'\neq s$ associated with $l$.}
\BlankLine
set $s' = s$ \\
pivot any free label of s' \\
\While{ $s'$ is an almost completely labeled Gale string }
{pivot the duplicate label, not picked up by the previous pivot}
\Return $s'$
\caption{Lemke-Howson for Gale Algorithm}
\end{algorithm}

\end{definition}

\begin{example}\label{lhg-example}
Let's consider the label string $l = 123432$ and the associated completely
labeled Gale string $s = 111100$, as in example \ref{pivoting-example}.
The Lemke-Howson for Gale algorithm using $1$ as free label returns the
completely labeled Gale string $s' = 110110$.\\

\begin{center}
{\renewcommand{\tabcolsep}{2ex}
\begin{tabular}{|r|c|l|}
\hline
start &
\textbf{\underline{1}234}32 &
drop \textbf{1}  \\
\cline{1-1}
\cline{3-3}
 &
\textbf{\underline{1}111}00 &
pivot \\
 &
\underline{0}\textbf{111\underline{1}}0 &
 \\
\cline{1-1}
\cline{3-3}
pick \textbf{3}, duplicate &
1\textbf{2\underline{3}4\underline{3}}2 &
drop the other \textbf{3} \\
\cline{1-1}
\cline{3-3}
 &
0\textbf{1\underline{1}11}0 &
pivot \\
 &
\textbf{\underline{1}1}\underline{0}\textbf{11}0 &
 \\
\cline{1-1}
\cline{3-3}
pick \textbf{1}, starting label &
\textbf{\underline{1}2}3\textbf{43}2 &
end \\
\hline
\end{tabular}
}\\
\end{center}
\end{example}

Note how the last label that is picked up is the one dropped at the start,
that's been missing in all the intermediate step; otherwise, we would have
reached a completely labeled Gale string at an earlier iteration.

Using algorithm \ref{lhg-algorithm} we can show a fundamental property of
Gale strings.

\todo{edited from paper notes until here}

\begin{theorem}\label{even-number-gale}
For any labeling $l:[n]\to[d]$, where $d$ is even and $d<n$,
the number of completely labeled Gale strings associated with $l$ is even.
\end{theorem}

\begin{proof}
If there are no completely labeled Gale strings associated with $l$, the
theorem holds trivially.

Suppose now that there is at least one completely labeled Gale strings
associated with $l$.

First of all, note that a pivot is reversible. Suppose that we pivot on the
(almost) completely labeled Gale string $s$ by dropping the label $l(i)$ and
picking up the label $l(j)$. Then $s(j) = 0$ and it is adjacent to the opposite
side of the odd maximal run of $1$s starting at $i$ that was created by
dropping $l(i)$. Let $s'$ be the (almost) completely labeled Gale string
obtained from this pivot. Analogously, if we pivot on $s'$ by dropping
$l(j)$, we will have to pick up the label $l(i)$. The pivoting is therefore
reversible by simply dropping the label that was picked up.

As there are only a finite number of possible bitstrings for each label
string,  if cycling is not possible the algorithm must terminate by finding
another completely labeled Gale string in a finite number of steps.

\todo[inline]{edit rest of proof}

Cycling is not possible due to the following observations.
Suppose the algorithm returns to a bit assignment of $s$ other than the initial Gale string.
Then at this bit assignment of $s$, because each pivot is reversible, we would have to be able to pick up two labels.
This, however, is ruled out by the GEC as only one of the adjacent runs of the dropped label is odd.
Returning to initial position is only possible by reversing the initial pivot which is not allowed.
The only free choice we have is at the beginning of the algorithm where we drop one free label.
From then on the process of the algorithm is uniquely determined, thus terminating in a finite number of steps at another Gale string.

\end{proof}

Theorem \ref{even-number-gale} holds for odd $d$ as well.

\todo[inline]{expand - or cut?}

The reversibility of the pivoting steps leads to the following property

\begin{property}\label{lhg-labels-property}
Let $s\in G(d,n)$ be a completely labeled Gale string for a labeling $l$,
and let $s'$ be the completely labeled Gale string obtained by running the
Lemke-Howson algorithm on $s$ by dropping the label $l(i)$. Then running
the Lemke-Howson algorithm on $s'$ by dropping the label $l(i)$ returns $s$.

The converse does not hold: it's possible for two Gale strings $s$ and $s'$
to be the endpoints of the path given by the Lemke-Howson algorithm run by
dropping both the label $l(i)$ and $l(j)\neq l(i)$. This is trivially true
since it's possible that $|\{s\in G(d,n) completely labeled for l \}| < d$.
\end{property}

An interesting example is the following.

\begin{example}
For the labeling $l=12342314$, the completely labeled Gale strings in
$G(4,8)$ are 11110000, 00011110, 00001111, 11100001, 01100011, 10001101.
They are related as endpoints of the Lemke-Howson algorithm as shown in the
following graph:

\begin{displaymath}
\xymatrix
@M=5pt
{
11110000
\ar@{<->}[rr] |{1}
\ar@{<->}[drr] |<<<<<<<<{2,4}
\ar@{<->}[ddrr] |>>>>>>>>{3}
& & 00011110 \\
01100011
\ar@{<->}[urr] |>>>>>>{2}
\ar@{<->}[rr] |>>>>>>{1,3}
\ar@{<->}[drr] |{4}
& & 11100001 \\
00001111
\ar@{<->}[uurr] |<<<<<<{3,4}
\ar@{<->}[rr] |{1,2}
& & 10001101
}
\end{displaymath}

Note that the graph is bipartite: this holds in general, a property related
to further results that we will discuss in section \ref{further-section}.

Note also that it's not a complete bipartite graph: there isn't an edge
between 1110001 and 00001111.

To our knowledge, it is an open question if the graph has to be connected.
\todo[inline]{If this were the case, there would be Nash equilibria not reachable via LHG from artificial equilibrium}
\end{example}

Note that the parity result of theorem \ref{even-number-gale} is about
{\em completely labeled} Gale strings, not Gale strings $s\in G(d,n)$ in
general. For example, $|G(4,6)|=9$, as shown in \ref{gs-example}.

\todo[inline]{results on 11234 = 1234 and 12345236 = 123423 (can delete successive double - any length of run, can delete single occurrences if in odd run). The first one plays a part in main thms}

\newpage

\subsection{The Complexity of \gale\ and \anothergale}
\label{main-thm-subsection}

We will now give our main result: \anothergale\ can be solved in polynomial
time. Therefore, it takes polynomial time to find a Nash Equilibrium of a
bimatrix game for which the best response polytope is cyclic.

Our proof will be based on a simple graph construction, and it will exploit
the following result by Edmonds \ref{edm-flowers} on the problem
\textsc{Perfect Matching}, defined as follows.\\

\begin{problem}
{\textsc{Perfect Matching}}
{A graph $G = (V,E)$.}
{Whether there is a set $M\subseteq E$ of pairwise non-adjacent edges so that
every vertex $v \in V$ is incident to exactly one edge in~$M$.}
\end{problem}

\begin{theorem}[\ref{edm-flowers}]\label{pm-poly}
The problem \textsc{Perfect Matching} is solvable in polynomial time.
\end{theorem}

We begin by considering the accessory problem \gale, and proving that it is
solvable in polynomial time.\\

\begin{problem}
{\gale}
{A labeling $l:[n]\to[d]$, where $d$ is even and $d<n$.}
{Whether there is a completely labeled Gale string~$s$ in~$G(d,n)$
associated with $l$.}
\end{problem}

\begin{theorem}\label{gale-p-thm}
The problem \gale\ is solvable in polynomial time.
\end{theorem}
\begin{proof}

\todo[inline]{proof from old draft: edit!}

We give a rather simple reduction to \textsc{Perfect Matching}.
Given the labeling $l:[n]\to[d]$, construct the
(multi-)graph $G$ with vertex set $V=[d]$ and up to $n$
(possibly parallel) edges with endpoints $l(i),l(i+1)$ for
$i\in [n]$ whenever these endpoints are distinct (so $G$ has
no loops); here we let $n+1=1$ (``modulo~$n$'') so that
$n,n+1$ is to be understood as $n,1$.
Then a completely labeled Gale string $s$ in $G(d,n)$ splits
into a number of runs which are uniquely split into $d/2$
pairs $i,i+1$ so that the labels $l(i)$ and $l(i+1)$ are
distinct, and all labels $1,\ldots,n$ occur among them.
So this defines a perfect matching for~$G$.

Conversely, a perfect matching $M$ of $G$ defines a Gale
string $s$ where $s(i)=s({i+1})=1$ if the edge that joins
$l(i)$ and $l(i+1)$ is in $M$ and $s(i)=0$ otherwise, so $s$
is completely labeled.
This shows how \textsc{Completely labeled Gale string}
reduces to \textsc{Perfect Matching}.
Finding a perfect matching, or deciding that $G$ has none,
can be done in polynomial time~\cite{edm-flowers}.
\end{proof}

Two examples of the construction used in theorem \ref{gale-p-thm} follows.
In the first one, there is a perfect matching and a corresponding Gale
string.

\begin{example}
Let $l=12343122$ be a string of labels. Then we have the edges $e_i$ as
follows:

\[
\underbrace{
1\overbrace{\phantom{\scriptscriptstyle 1}}^{e_1}
2\overbrace{\phantom{\scriptscriptstyle 1}}^{e_2}
3\overbrace{\phantom{\scriptscriptstyle 1}}^{e_3}
4\overbrace{\phantom{\scriptscriptstyle 1}}^{e_4}
3\overbrace{\phantom{\scriptscriptstyle 1}}^{e_5}
1\overbrace{\phantom{\scriptscriptstyle 1}}^{e_6}
2\overbrace{\phantom{\scriptscriptstyle 1}}^{\scriptscriptstyle cycle}
2
}_{e_7}
\]

So the graph $G$ will be:

\begin{displaymath}
\xymatrix
@M=5pt
{
1
\ar@/^1pc/@{-}[rr] |{e_1}
\ar@{-}[rr] |{e_6}
\ar@/_1pc/@{-}[rr] |{e_7}
\ar@{-}[ddrr] |{e_5}
& & 2
\ar@{-}[dd] |{e_2}
\\
\\
3
\ar@{-}[rr] |{e_3}
\ar@/_1pc/@{-}[rr] |{e_4}
& & 4
}
\end{displaymath}

A possible completely labeled Gale string for $l$ is $11011000$, which
corresponds to $e_1, e_4$ in $G$:

\[
\underbrace{
\mathbf{1}\overbrace{\phantom{\scriptscriptstyle 1}}^{e_1}
\mathbf{1}\overbrace{\phantom{\scriptscriptstyle 1}}
0\overbrace{\phantom{\scriptscriptstyle 1}}
\mathbf{1}\overbrace{\phantom{\scriptscriptstyle 1}}^{e_4}
\mathbf{1}\overbrace{\phantom{\scriptscriptstyle 1}}
0\overbrace{\phantom{\scriptscriptstyle 1}}
0\overbrace{\phantom{\scriptscriptstyle 1}}
0
}
\]


\end{example}

An example where there isn't a perfect matching, and therefore there isn't
any possible Gale string for the labeling is the following.

\begin{example}
Let us consider the labeling $l=121314$. The associated graph $G$ will be

\begin{displaymath}
\xymatrix
@M=5pt
{
1
\ar@/^1pc/@{-}[rr] |{e_1}
\ar@{-}[rr] |{e_2}
\ar@/^1pc/@{-}[ddrr] |{e_3}
\ar@{-}[ddrr] |{e_4}
\ar@{-}[dd] |{e_5}
\ar@/_1pc/@{-}[dd] |{e_6}
& & 2
\\
\\
3
& & 4
}
\end{displaymath}

It's trivial to see that it's not possible to find a perfect matching for
$G$.
\end{example}

We finally extend the proof of theorem \ref{gale-p-thm} to show that
\anothergale\ is polynomial-time solvable.

\begin{theorem}\label{gale-p-thm}
The problem \anothergale\ is solvable in polynomial time.
\end{theorem}
\begin{proof}

\todo[inline]{proof from old draft: to edit!}

The reduction for \anothergale\
is an extension of this.
Consider the given completely labeled Gale string $s$ and
the matching $M$ for it.
If $G$ has multiple edges between two nodes and one of them
is in $M$, simply replace that edge by a parallel edge to
obtain another completely labeled Gale string~$s'$.
Hence, we can assume that $M$ has no edges that have a
parallel edge.
Another completely labeled Gale string $s'$ exists by
Theorem~\ref{t-even}.
The corresponding matching $M'$ does not use at least one
edge in $M$.
Hence, at least one of the $d/2$ graphs $G$ which have one
of the edges of $M$ removed has a perfect matching $M'$,
which is a perfect matching of $G$, and which defines
a completely labeled Gale string $s'$ different from~$s$.
The search for $M'$ takes again polynomial time.
\end{proof}

\todo[inline]{examples for second PM (one w/ double edges, one without)}

\newpage

\section{Further results}\label{further-section}

\section*{Appendix A: A result about PPAD completeness of \textsc{Nash}}

\section*{Appendix B: Notation}

For a matrix $A$ we denote its transpose with $A^T$.

Vectors $u,v$ in $\reals^d$ as column vectors

$u^T v$ is their scalar product.

$\0$ we denote a vector of all $0$'s, of suitable dimension, by $\1$ a vector of all $1$'s.

A unit vector, which has a 1 in its $i$th component and 0 otherwise, is denoted by $e_i$.

Inequalities like $u\ge\0$ hold for all components.

For a set of points $S$ we denote its convex hull by $\conv\,S$.

For $n \in \naturals$ we denote $[n] = {1,2,\ldots,n}$

\newpage

\begin{thebibliography}{00}

\frenchspacing\parskip-1ex
\small

\bibitem{main} M. M. Casetti, J. Merschen, B. von Stengel (2010).
Finding Gale Strings.
\emph{Electronic Notes in Discrete Mathematics}
\todo[inline]{issue, pp. n--m.}

\bibitem{cd} X. Chen, X. Deng (2006).
Settling the complexity of two-player Nash equilibrium.
\emph{Proc. 47th FOCS}, pp. 261--272.

\bibitem{dgp} C. Daskalakis, P. W. Goldberg, C. H. Papadimitriou (2006).
The complexity of computing a Nash equilibrium.
\emph{Proc. Ann. 38th STOC}, pp. 71--78.
\todo[inline]{change ref to econometrica(?)}

\bibitem{edm} J. Edmonds (1965).
Paths, trees, and flowers.
\emph{Canad. J. Math.} 17, pp. 449--467.

\bibitem{gale} D. Gale (1963),
Neighborly and cyclic polytopes.
\emph{Convexity, Proc. Symposia in Pure Math.}, Vol. 7, ed. V. Klee, American Math. Soc., Providence, Rhode Island, pp. 225--232.
\todo[inline]{check if right typography}

\bibitem{jm} J. Merschen (2012).
\todo[inline]{thesis}

\bibitem{lh} C. E. Lemke, J. T. Howson, Jr. (1964).
Equilibrium points of bimatrix games.
\emph{J.  Soc. Indust. Appl. Mathematics} 12, pp.  413--423.

\bibitem{ppad} C. H. Papadimitriou (1994).
On the complexity of the parity argument and other inefficient proofs of existence.
\emph{J. Comput. System Sci.} 48, pp. 498--532.

\bibitem{svs} R. Savani, B. von Stengel (2006).
Hard-to-solve bimatrix games.
\emph{Econometrica} 74, pp. 397--429.

\bibitem{vvs} L. V\'{egh}, B. von Stengel
\todo[inline]{ref}


\end{thebibliography}

\end{document}
