\documentclass[preprint,12pt]{article}
\linespread{1.2}

\usepackage{amssymb}
\usepackage[ruled,vlined]{algorithm2e}
\usepackage{amsmath}
\usepackage{amsfonts}
\usepackage{mathptmx}
\usepackage[latin1]{inputenc}
\usepackage{graphicx}
\usepackage[all]{xy}
\usepackage{verbatim}


\newtheorem{theorem}{Theorem}[section]
\newtheorem{lemma}[theorem]{Lemma}
\newtheorem{proposition}[theorem]{Proposition}
\newtheorem{corollary}[theorem]{Corollary}

\newenvironment{proof}[1][Proof]
{\begin{trivlist}
\item[\hskip \labelsep {\bfseries #1}]}
{\end{trivlist}}

\newenvironment{definition}[1][Definition]
{\begin{trivlist}
\item[\hskip \labelsep {\bfseries #1}]}
{\end{trivlist}}

\newenvironment{example}[1][Example]
{\begin{trivlist}
\item[\hskip \labelsep {\bfseries #1}]}
{\end{trivlist}}

\newenvironment{remark}[1][Remark]
{\begin{trivlist}
\item[\hskip \labelsep {\bfseries #1}]}
{\end{trivlist}}


\newdimen\einr\einr2em
\def\abs#1{\par\hangafter=1\hangindent=\einr
  \noindent\hbox to\einr{\ignorespaces#1\hfill}\ignorespaces}


\def\reals{{\mathbb R}}
\def\naturals{{\mathbb N}}
\def\conv{{\rm conv}}
\def\0{{\bf0}}
\def\1{{\bf1}}
\def\T{^{\top}}
\def\rone{{\1\T}}
\def\reals{{\mathbb R}}

\def\gale{\textsc{Completely Labeled Gale String}}
\def\anothergale{\textsc{Another Completely Labeled Gale String}}

\include{pstricks}
\include{cases}



\def\proof{\noindent{\em Proof.\enspace}}
\def\proofof#1{\noindent{\em Proof of #1.\enspace}}
\def\endproof{\hfill\strut\nobreak\hfill\tombstone\par\medbreak}
\def\tombstone{\hbox{\lower.4pt\vbox{\hrule\hbox{\vrule
  \kern7.6pt\vrule height7.6pt}\hrule}\kern.5pt}}


\begin{document}

\title{The Gale String Problem for Equilibrium Computation in Games}

\begin{abstract}

This thesis presents a report on original research,
published as joint work with Merschen and von Stengel in
{\em Electronic Notes in Discrete Mathematics} (2010). Our result
shows a polynomial time algorithm to find a Nash equilibrium
for a particular class of games, which was previously used
by Savani and von Stengel (2006) as an example of exponential
time for the classical Lemke-Howson algorithm for bimatrix
games (1964).

It was conjectured that solving these games via the
Lemke-Howson algorithm was complete in the class
$\mathrm{PPAD}$ (Proof by Parity Argument, Directed
version). A major motivation for the definition of this
class by Papadimitriou (1994) was, in turn, to capture the
pivoting technique of many results related to the Nash
equilibrium, including the Lemke-Howson algorithm. A
$\mathrm{PPAD}$-completeness proof of the games we consider
would have provided a traceable proof of the Daskalakis,
Goldberg and Papaditrimiou (2005) and Chen and Deng (2009)
results about the $\mathrm{PPAD}$-completeness of every
normal form game. Our result of polynomial-time solvability,
on the other hand, indicates the existence of a special
class of games, unless $\mathrm{PPAD} = \mathrm{P}$.

Our proof exploits two results. The first one is the
representation of the Nash equilibria of these games as a
string of labels and an associated string of $0$s and $1$s
satisfying some conditions, called \textit{Gale conditions},
as seen in Savani and von Stengel (2006). The second one is
the polynomial-time solvability of the problem of finding a
perfect matching in a graph, solved by Edmonds (1965).

Further results by Merschen (2012) and V\'{e}gh and von
Stengel (2014) solved the open problem of the {\em sign} of
the equilibrium found in polynomial time.

\end{abstract}


\section{Introduction}

% What, why


\section{Bimatrix Games and Polytopes}
\label{gamespoly}

\tiny

A ($d$-dimensional) {\em simplicial polytope} $P$ is the convex hull of a set of at least $d+1$ points $v$ in $\reals^d$ in general position, that is, no $d+1$ of them are on a common hyperplane.

If a point $v$ cannot be omitted from these points without changing $P$ then $v$ is called a {\em vertex} of $P$. A {\em facet} of $P$ is the convex hull $\conv\,F$ of a set $F$ of $d$ vertices of $P$ that lie on a hyperplane $\{ x\in \reals^d\mid a^T x=a_0\}$ so that $a^T u<a_0$ for all other vertices $u$ of $P$; the vector $a$ (unique up to a scalar multiple) is called the {\em normal vector} of the facet. We often identify the facet with its set of vertices~$F$.


The following theorem, due to Balthasar and von Stengel
\cite{B09,BvS10}, establishes a connection between general
labeled polytopes and equilibria of certain $d\times n$
bimatrix games $(U,B)$.

THIS FROM VvS

\begin{theorem}\label{t-unitv}
Consider a labeled  $d$-dimensional simplicial polytope $Q$ with $\0$ in
its interior, with vertices $-e_1,\ldots,-e_d,c_1,\ldots,c_n$,
% We need to assume that the c_i are pairwise distinct, otherwise
% a vertex can have several labels.
% We need to assume that c_i vertex of Q for the following reason:
% the definition of completely labeled facet is difficult if
% c_i is in a facet but not vertex of the facet.
so that % $F_0$ in {\rm(\ref{F0})}
$F_0=\conv\{-e_1,\ldots,-e_d\}$
is a facet of $Q$.
Let $-e_i$ have label $i$ for $i\in[d]$, and let $c_j$
have label $l(j)\in[d]$ for $j\in[n]$.
Let $(U,B)$ be the $d\times n$ bimatrix game with
$U=[e_{l(1)}\cdots\,e_{l(n)}]$ and $B=[b_1\,\cdots\,b_n]$,
where $b_j=c_j/(1+\rone c_j)$ for $j\in[n]$.
Then the completely labeled facets $F$ of $Q$, with the
exception of~$F_0$, are in one-to-one correspondence to the
Nash equilibria $(x,y)$ of the game $(U,B)$ as follows:
if $v$ is the normal vector of $F$, then
$x=(v+\1)/\rone (v+\1)$,
and $x_i=0$ if and only if $-e_i\in F$ for $i\in[d]$;
any other label~$j$ of $F$, so that $c_j$ is a
vertex of~$F$, represents a pure best reply to~$x$.
The mixed strategy $y$ is the uniform distribution on
the set of pure best replies to~$x$.
\end{theorem}

In the preceding theorem, any simplicial polytope can take
the role of $Q$ as long as it has one completely labeled
facet~$F_0$.
Then an affine transformation, which does not change the
incidences of the facets of $Q$, can be used to map $F_0$ to
the negative unit vectors $-e_1,\ldots,-e_d$ as described,
with $Q$ if necessary expanded in the direction $\1$ so that
$\0$ is in its interior.

A $d\times n$ bimatrix game $(U,B)$ is a {\em unit vector game}
if all columns of $U$ are unit vectors.
For such a game $B$ with $B=[b_1\cdots b_n]$, the columns
$b_j$ for $j\in[n]$ can be obtained from $c_j$ as in
Theorem~\ref{t-unitv} if $b_j>\0$ and $\1^T b_j<1$.
This is always possible via a positive-affine transformation
of the payoffs in~$B$, which does not change the game.
The unit vectors $e_{l(j)}$ that constitute the columns of
$U$ define the labels of the vertices $c_j$.
The corresponding polytope with these vertices is simplicial
if the game $(U,B)$ is nondegenerate \cite{vS02}, which here
means that no mixed strategy $x$ of the row player has more
than $|\{i\in[d]\mid x_i>0\}|$ pure best replies.
Any game can be made nondegenerate by a suitable
``lexicographic'' perturbation of $B$, which can be
implemented symbolically.

Unit vector games encode arbitrary bimatrix games:
An $m\times n$ bimatrix game $(A,B)$ with (w.l.o.g.{})
positive payoff matrices $A,B$ can be symmetrized so
that its Nash equilibria are in one-to-correspondence to the
symmetric equilibria of the $(m+n)\times(m+n)$
symmetric game $(C^T,C)$ where
\[
\label{symmetrize}
C=\biggl(\begin{matrix} 0 & B\cr A^\top & 0\cr\end{matrix}\biggr).
\]
In turn, as shown by McLennan and Tourky \cite{mt},
the symmetric equilibria $(x,x)$ of any symmetric game
$(C^T,C)$ are in one-to-one correspondence to the Nash
equilibria $(x,y)$ of the ``imitation game'' $(I,C)$ where
$I$ is the identity matrix; the mixed strategy $y$ of the
second player is simply the uniform distribution on the
set $\{i\mid x_i>0\}$.
Clearly, $I$ is a matrix of unit vectors, so $(I,C)$ is a
special unit vector game.


\section{Cyclic Polytopes and Gale Strings}

\normalsize

\subsection{Gale Strings}

% PBL: we use G(d,n) for cyclic poly and Gale string (OK, the same, but)
% also: s as function and as string (idem)

Let $[k]=\{1,\ldots,k\}$ for any positive integer $k$.
For any set $S$, we can represent the function $s:[k]\to S$
as the string $s = s(1)s(2)\cdots s(k)$. Let $A \subset [k]$, and
$S=\{0,1\}$. We will then have a correspondence between the function
$s:A\to \{0,1\}$ and a {\em bitstring} $s$, that in turn corresponds
to the indicator function of the set

\begin{align*}
1(s) & = s^{-1}(1) \\
     & = \{j\in [k]\mid s(j)=1\}
\end{align*}

For a bitstring $s$, a maximal substring of $s$ of consecutive $1$'s is
called a {\em run}.

\begin{definition}\label{gale-string}
$G(d,n)$ is the set of all bitstrings $s$ of length $n$
such that
\begin{enumerate}
\item exactly $d$ bits in $s$ are $1$ and
\item $s$ fulfills the {\em Gale evenness condition}:
\[
01^k0\hbox{ is a substring of }s\quad{\Rightarrow}\quad k\hbox{ is even.}
\]
\end{enumerate}
An element of of $G(d,n)$ is called a {\em Gale string of dimension~$d$ and length~$n$}.
\end{definition}

Definition \ref{gale-string} characterises Gale strings as bitstrings of length~$n$ with exactly~$d$ elements equal to $1$, such that interior runs
(that is, runs bounded on both sides by~$0$s) must be of even length. Note
that this condition allows Gale strings to start or end with an odd-length
run. If~$d$ is even, then any $s$ in $G(d,n)$ that starts with an odd run
also ends with an odd run; we can then consider the Gale string as a ``loop''
by ``glueing together'' the extremes of the string to form an even run. The
set of Gale strings of even dimension is therefore invariant under a cyclic
shift of the strings.

As an example, we can consider $G(4,6)$. We have
\[
G(4,6) = \{111100, 111001, 110011, 100111, 001111, 110110, 101101, 011011\}
\]
The strings $111100$, $111001$, $110011$, $100111$ and $001111$ are
equivalent under a cyclic shift, as are the strings $110110$, $101101$ and
$011011$.

From this point forward, we will assume that $d$ is even.

%% give something to generalise to odd case
%% even case used to define orientation

\tiny

\subsection{Cyclic Polytopes and Labeling}

A {\em cyclic polytope\/} $P$ in dimension~$d$ with $n$
vertices is the convex hull of $n$ points $\mu(t_j)$ on the
{\em moment curve\/}

\[
\mu\colon t\mapsto(t,t^2,\ldots,t^d)^\top \hbox{ for } j\in [n]
\]

Suppose that $t_1<t_2<\cdots< t_n$. Then the facets of $P$ are encoded
by $G(d,n)$, that is $F$ is a facet of $P$ if and only if
\[
F = \conv\{\mu(t_i)\mid i\in 1(s)\} \hbox { for some }s\in G(d,n)
\]
as shown by Gale \cite{gale}.

% NOW! labels, connection to polytopes used for specific games

% def: "completely labeled GS *associated with* labeling" (it makes writing  style more pleasant)

% def of ALMOST COMPLETELY


Given a set $G$ of bit strings of length $n$ and a parameter
$d$, a {\em labeling\/} is a function $l:[n]\to[d]$.
Given a labeling, a string $s$ in $G$ is called
{\em completely labeled\/} if $l(1(s))=[d]$, that is, if
every label in $[d]$ appears as $l(i)$ for at least one bit
$s(i)$ so that $s(i)=1$.


Clearly, if $s$ is completely labeled, then $s$ has
at least $d$ bits that are~$1$, and
if exactly $d$ bits in $s$ are~$1$, then every label
in $[d]$ occurs exactly once.


% Essentially, this holds because any set $S\subset [n]$
% the moment curve defines a unique hyperplane which is crossed
% (and not just touched) by the moment curve; if the bitstring
% $s$ that encodes $F$ as $1(s)$ has a substring $01^k0$
For this cyclic polytope $P$, a labeling $l:[n]\to[d]$ can
be understood as a label $l(j)$ for each vertex $\mu(t_j)$
for $j\in [n]$.
A completely labeled Gale string $s$ therefore represents a
facet $F$ of $P$ that is completely labeled.

% from here...

For example, given the string of labels $l=123432$, we see that there are
four associated completely labeled Gale strings: $111100$, $110110$, $100111$
and $101101$.

{\renewcommand{\tabcolsep}{2ex}
\begin{tabular}{|c|c|c|c|}
\hline
\textbf{\underline{1234}}32 &
\textbf{\underline{12}}3\textbf{\underline{43}}2 &
\textbf{\underline{1}}23\textbf{\underline{432}} &
\textbf{\underline{1}}2\textbf{\underline{34}}3\textbf{\underline{2}} \\
\hline
111100 &
110110 &
100111 &
101101 \\
\hline
\end{tabular}
}

For $121314$, there are no completely labeled Gale strings.

% until here: checked

Special games are obtained by using cyclic polytopes in
Theorem~\ref{t-unitv}, suitably affinely transformed with
a completely labeled facet $F_0$.
When $Q$ is a cyclic polytope in dimension $d$ with $d+n$
vertices, then the string of labels $l(1)\cdots l(n)$ in
Theorem~\ref{t-unitv} defines a labeling $l':[d+n]\to [d]$
where $l'(i)=i$ for $i\in [d]$ and
$l'(d+j)=l(j)$ for $j\in [n]$.
In other words, the string of labels $l(1)\cdots l(n)$ is
just prefixed with the string $1\,2\cdots d$ to give $l'$.
Then $l'$ has a trivial completely labeled Gale string
$1^d0^n$ which defines the facet $F_0$.
Then the problem \textsc{Another completely labeled Gale
String} defines exactly the problem of finding a Nash
equilibrium of the unit vector game $(I,B)$.
Note again that $B$ is here not a general matrix (which would
define a general game) but obtained from the last $n$ of
$d+n$ vertices of a cyclic polytope in dimension~$d$.


\noindent
\textsc{Another completely labeled Gale string}
{\parskip0pt
\abs{\textbf{Input}:}
% Positive integers $n,d$, where $d$ is even and $d<n$, and
% a string of {\em labels} $l:[n]\to[d]$.
A labeling $l:[n]\to[d]$, where $d$ is even and $d<n$,
and a completely labeled Gale string $s$ in $G(d,n)$.
\abs{\textbf{Output}:}
A completely labeled Gale string $s'$ in
$G(d,n)$ where $s'\ne s$.
\par}

\section{Main Complexity Results}
\label{main-thm-sect}
% \section{The Complexity of \anothergale}
% \section{Complexity Results}

\subsection{Pivoting, Parity and the Lemke-Howson Algorithm}\label{lhg-sect}

\begin{definition}\label{almost-completely-labeled}
(\emph{almost labeled condition}) the number of 1's in $s$ is equal to $d$ and $\left\{l(i)|s(i) = 1\right\} \subset \left\{1,\ldots,d\right\}$ with $|\left\{l(i)|s(i) = 1\right\}|=d-1$.
\end{definition}

\normalsize

\begin{definition}\label{pivoting}

We define as {\em pivoting} the operation defined in algorithm
\ref{pivoting-algorithm}, where we consider the Gale strings as ``loops'' by
identifying position $i$ with any position $i+kn$

\begin{algorithm}\label{pivoting-algorithm}
\SetKwInOut{Input}{input}
\SetKwInOut{Output}{output}
\Input{A string of labels $l$ of length~$n$; a Gale string $s$ that describes
a a complete or almost complete labeling of~$l$; $i \in [n]$ such that
$s(i)=1$}
\Output{A different Gale string $s'$ that describes a complete or almost
complete labeling of~$l$.}
\BlankLine
set $s' = s$ \\
set $s'(i)=0$ (we call this {\em dropping the label} $l(i)$) \\
let $j$ be the length of the odd maximal run of $1$s created by this \\
\If{ the odd maximal run of $1$s is on the \textbf{right} of position $i$ }
{ set $s'(i+j+1)=1$ }
\Else
{ set $s'(i-j-1)=1$ }
\Return $s'$
\caption{Pivoting}
\end{algorithm}
\end{definition}

Note that, by the Gale evenness condition, we must have an the odd maximal
run of $1$s either on the right or on the left of position $i$.

If the Gale string in the pivoting algorithm is completely labeled, we can
drop any label $l(i)$ such that $s(i) = 1$. We refer to these labels as {\em
free labels}.

Each pivoting can be seen as a step of a ``path'' between (almost) completely
labeled Gale strings. If the first and last step of the path are completely
labeled Gale strings, the path is described by algorithm \ref{lhg-algorithm},
the {\em Lemke-Howson algorithm}:

\begin{algorithm}\label{lhg-algorithm}
\SetKwInOut{Input}{input}
\SetKwInOut{Output}{output}
\Input{A $n$-string $l \in [n]$ where $d$ is even and $d< n$; a completely labeled Gale string $s$ associated with $l$.}
\Output{A Gale string $s'\neq s$ associated with $l$.}
\BlankLine
set $s' = s$ \\
pivot any free label of s' \\
\While{ $s'$ is an almost completely labeled Gale string }
{pivot the duplicate label, not picked up by the previous pivot}
\Return $s'$
\caption{Lemke-Howson for Gale Algorithm}
\end{algorithm}

\begin{example}
\end{example}

Using algorithm \ref{lhg-algorithm} we can show a fundamental property of
Gale strings.

\begin{theorem}\label{even-number-gale}
For any labeling $l:[n]\to[d]$, where $d$ is even and $d<n$,
the number of completely labeled Gale strings associated with $l$ is even.
\end{theorem}

\begin{proof}
If there are no completely labeled Gale strings associated with $l$, the
theorem holds trivially.

Suppose now that there is at least one completely labeled Gale strings
associated with $l$.

First of all, note that a pivot is reversible. Suppose that we pivot on the
(almost) completely labeled Gale string $s$ by dropping the label $l(i)$ and
picking up the label $l(j)$. Then $s(j) = 0$ and it is adjacent to the opposite
side of the odd maximal run of $1$s starting at $i$ that was created by
dropping $l(i)$. Let $s'$ be the (almost) completely labeled Gale string
obtained from this pivot. Analogously, if we pivot on $s'$ by dropping
$l(j)$, we will have to pick up the label $l(i)$. The pivoting is therefore
reversible by simply dropping the label that was picked up.

% zig-zag image!!!

As there are only a finite number of possible bitstrings for each label string and if cycling is not possible the algorithm must terminate by finding another Gale string in a finite number of steps.
Cycling is not possible due to the following observations.
Suppose the algorithm returns to a bit assignment of $s$ other than the initial Gale string.
Then at this bit assignment of $s$, because each pivot is reversible, we would have to be able to pick up two labels.
This, however, is ruled out by the GEC as only one of the adjacent runs of the dropped label is odd.
Returning to initial position is only possible by reversing the initial pivot which is not allowed.
The only free choice we have is at the beginning of the algorithm where we drop one free label.
From then on the process of the algorithm is uniquely determined, thus terminating in a finite number of steps at another Gale string. As this initial choice is somewhat arbitrary the resulting Gale strings and NE for different first steps can vary.

\end{proof}

Theorem \ref{even-number-gale} holds for odd $d$ as well.

% expand - or cut?

% graphs of LHG paths - if not in proof too?
% graphs of endpoints of LHG - that will be used in Further Results for mentioning index? (and anyway it's something)

\subsection{The Complexity of \gale and \anothergale}\label{main-thm-subsect}

We consider the following decision problem.

\einr 5em

\noindent
\gale
{\parskip0pt
\abs{\textbf{Input}:}
A labeling $l:[n]\to[d]$, where $d$ is even and $d<n$.
\abs{\textbf{Question}:}
Is there a completely labeled Gale string~$s$ in~$G(d,n)$ associated with
$l$?
\par}


Theorem~\ref{even-number-gale} implies that if there is one completely
labeled Gale string, there is also a second one.
The following function problem asks to compute a completely
labeled Gale string if one such string is already given.


We now show that both \gale and \anothergale
can be solved in polynomial time.
The proof uses a reduction to the following problem,
which was first shown to be solvable in polynomial time by
Edmonds~\cite{edm}.

\noindent \textsc{Perfect Matching}
{\parskip0pt
\abs{\textbf{Input}:} Graph $G = (V,E)$.
\abs{\textbf{Question}:}
Is there a set $M\subseteq E$ of pairwise non-adjacent edges
so that every vertex $v \in V$ is incident to exactly one
edge in~$M$?
\par}

\begin{theorem}
\label{t-main}
The problems
\textsc{Completely labeled Gale string} and
\textsc{Almost completely labeled Gale string}
can be solved in polynomial time.
\end{theorem}

\proof
We give a rather simple reduction to \textsc{Perfect Matching}.
Given the labeling $l:[n]\to[d]$, construct the
(multi-)graph $G$ with vertex set $V=[d]$ and up to $n$
(possibly parallel) edges with endpoints $l(i),l(i+1)$ for
$i\in [n]$ whenever these endpoints are distinct (so $G$ has
no loops); here we let $n+1=1$ (``modulo~$n$'') so that
$n,n+1$ is to be understood as $n,1$.
Then a completely labeled Gale string $s$ in $G(d,n)$ splits
into a number of runs which are uniquely split into $d/2$
pairs $i,i+1$ so that the labels $l(i)$ and $l(i+1)$ are
distinct, and all labels $1,\ldots,n$ occur among them.
So this defines a perfect matching for~$G$.

Conversely, a perfect matching $M$ of $G$ defines a Gale
string $s$ where $s(i)=s({i+1})=1$ if the edge that joins
$l(i)$ and $l(i+1)$ is in $M$ and $s(i)=0$ otherwise, so $s$
is completely labeled.
This shows how \textsc{Completely labeled Gale string}
reduces to \textsc{Perfect Matching}.
Finding a perfect matching, or deciding that $G$ has none,
can be done in polynomial time~\cite{edm}.

The reduction for \textsc{Another completely labeled Gale string}
is an extension of this.
Consider the given completely labeled Gale string $s$ and
the matching $M$ for it.
If $G$ has multiple edges between two nodes and one of them
is in $M$, simply replace that edge by a parallel edge to
obtain another completely labeled Gale string~$s'$.
Hence, we can assume that $M$ has no edges that have a
parallel edge.
Another completely labeled Gale string $s'$ exists by
Theorem~\ref{t-even}.
The corresponding matching $M'$ does not use at least one
edge in $M$.
Hence, at least one of the $d/2$ graphs $G$ which have one
of the edges of $M$ removed has a perfect matching $M'$,
which is a perfect matching of $G$, and which defines
a completely labeled Gale string $s'$ different from~$s$.
The search for $M'$ takes again polynomial time.
\endproof

% Theorem~\ref{t-main} has a simple proof which raises some
% interesting observations on its own.
The significance of Theorem~\ref{t-main} is to be understood
in the context of equilibrium computation for games, which
we discuss next.
The remainder of this paper contains only known results.

%% In addition, the multigraph $G$ in the proof of
%% Theorem~\ref{t-main} is of interest of its own:
%% It is a Euler graph (each node is the endpoint of an even
%% number of edges), which is also known as a 1-oik (see
%% Edmonds \cite{je07}), and its perfect matchings are known as
%% room partitions.
%% The set of such room partitions is even, shown by the
%% ``exchange algorithm'', also due to Edmonds \cite{je07}.
%% This provides an alternative proof of Theorem~\ref{t-even}.
%% However, the exchange algorithm for perfect matchings of Euler
%% graphs is different (and apparently faster) than the
%% Lemke--Howson algorithm for labeled Gale strings, which
%% can be exponential (a result due to Morris \cite{morris},
%% see below).


\section{Further results}

% The framework provided by our result led to further questions, related to the issue of the *sign* of an index - and so on (Merschen, VvS)

% Open problems (?)


\section*{Appendix A: Notation}

For a matrix $A$ we denote its transpose with $A^T$. We treat vectors $u,v$ in $\reals^d$ as column vectors, so $u^T v$ is their scalar product. By $\0$ we denote a vector of all $0$'s, of suitable dimension, by $\1$ a vector of all $1$'s. A unit vector, which has a 1 in its $i$th component and 0 otherwise, is denoted by $e_i$. Inequalities like $u\ge\0$ hold for all components. For a set of points $S$ we denote its convex hull by $\conv\,S$.

For $n \in \naturals$ we denote $[n] = {1,2,\ldots,n}$

\section*{Appendix B: A result about PPAD completeness of \textsc{Nash}}

% the result of MSc dissertation

\begin{thebibliography}{00}

\frenchspacing\parskip-1ex
\small

\end{thebibliography}

\end{document}
