\documentclass[11pt, draft]{article}

%% TYPESETTING

% draft:
\linespread{1.3}
\usepackage{todonotes}

% final (?) TODO: check with LSE requirements
% page measurements
% \setlength{\hoffset}{0mm}
% \setlength{\oddsidemargin}{25mm}
% \setlength{\textwidth}{130mm}

%% PACKAGES

\usepackage{amssymb}
\usepackage{amsmath}
\usepackage{amsthm}

\usepackage[all]{xy}

\usepackage[ruled,vlined,linesnumbered]{algorithm2e}

%% THEOREMS and ENVIRONMENTS

% theorems

\newtheorem{theorem}{Theorem}
\newtheorem{property}{Property}[section]
\theoremstyle{remark}\newtheorem{example}{Example}[section]

% environments

% computational problems
% TODO check p{textwidth} in final version

% decision problem

\newenvironment{decproblem}[3]
{
\vspace{2.5ex}
\noindent
\begin{tabular}{p{18mm} @{\textbf{:} } p{100mm}}
\hline
\multicolumn{2}{l}{\noindent {\sc #1}} \\
\hline
\textbf{input} & #2 \\
\textbf{question} & #3 \\
\hline
\end{tabular}
}
{
\vspace{1.5ex}
}

% function problem

\newenvironment{fctproblem}[3]
{
\vspace{2.5ex}
\noindent
\begin{tabular}{p{15mm} @{\textbf{:} } p{102mm}}
\hline
\multicolumn{2}{l}{\noindent {\sc #1}} \\
\hline
\textbf{input} & #2 \\
\textbf{output} & #3 \\
\hline
\end{tabular}
}
{
\vspace{1.5ex}
}

% use:
% \begin{problem}
% {name of problem}
% {input of problem}
% {output of problem}
% \end{problem}


%% SHORTCUTS

% definitions

\def\reals{{\mathbb R}}
\def\naturals{{\mathbb N}}

% vectors and polytopes
\def\T{^{\top}}
\def\0{{\bf 0}}
\def\1{{\bf 1}}

\def\conv{{\rm conv}}


% to choose the name of the problem - GALE or COMPLETELY LABELED GALE STRING

\def\gale{{\sc{Gale}}}
\def\anothergale{{\sc{Another Gale}}}


%%% END PREAMBLE

\begin{document}

\title{Complexity of the Gale String Problem for Equilibrium
Computation in Games}

\begin{abstract} % 298 words!
This thesis presents a report on original research, published as joint
work with Merschen and von Stengel in {\em Electronic Notes in Discrete}
\mbox{{\em Mathematics} \cite{main}.} Our result shows a polynomial time
algorithm to solve two problems related to labeled Gale strings, a
combinatorial structure consisting a string of labels and a bitstring
satisfying certain conditions introduced by Gale \mbox{in \cite{gale}.}

Gale strings can be used in the representation of a particular class
of games that Savani and von \mbox{Stengel \cite{svs}} used as an example
of
\linebreak[5]
exponential running time for the classical \mbox{Lemke-Howson}
algorithm to find a Nash equilibrium of a bimatrix \mbox{game \cite{lh}.}
It was
\linebreak[5]
conjectured that solving these games via the Lemke-Howson
\linebreak[5]
algorithm
was complete in the \mbox{class PPAD} (Proof by Parity
\linebreak[5]
Argument, Directed version). A major motivation for the definition of
this class by \mbox{Papadimitriou \cite{ppad}} was, in turn, to capture
the pivoting technique of many results related to the Nash equilibrium,
including the Lemke-Howson algorithm.

Our result, on the contrary, sets apart this class of games as a case
for which there is a polynomial-time algorithm to find a Nash equilibrium.
Since Daskalakis, Goldberg and \mbox{Papaditrimiou \cite{dgp}} and
Chen and \mbox{Deng \cite{cd}} proved the \mbox{PPAD-completeness} of
finding a Nash equilibrium in general normal-form games, we have a special
class of games, unless \mbox{PPAD = P}.

Our proof exploits two results. The first one is the representation
of the Nash equilibria of these games as Gale strings, as seen in Savani
and von \mbox{Stengel \cite{svs}.} The second one is the polynomial-time
solvability of the problem of finding a perfect matching in a graph, proven
by \mbox{Edmonds \cite{edm}.}

Further results by \mbox{Merschen \cite{jm}} and V\'{e}gh and von
\mbox{Stengel \cite{vvs}} will be mentioned.

An appendix relates an amendment to the proof of the
\linebreak[5]
\mbox{PPAD-completeness} result by Daskalakis, Goldberg and
\linebreak[5]
\mbox{Papaditrimiou \cite{dgp}.}
\end{abstract}

\newpage

\section{Introduction}

\section{Complexity, Games, Labels, Polytopes, and Gale Strings}


\subsection{Some Complexity Classes}

\todo[inline]{references - cite Papadimitriou (book) for general;
Papadimitriou 1994 for PPAD}

A {\em computational problem} is given by the combination of an {\em input}
and a related {\em output}. A specific input gives an {\em instance} of the
problem.

Computational problems can be classified according to the form of their
output. A {\em function problem} $P$ returns for an intance $x$ an output
$y$ that satisfies a given binary relation $R(x,y)$. In the case of a
{\em decision problems}, $y$ is either ``YES'' or ``NO''. The
{\em complement} of a decision problem $P$ is the problem $\bar{P}$
that returns ``NO'' for each instance of $P$ that returns ``YES'', and
vice versa.

{\em Search problems} are function problems that return either an output
$y$ such that $R(x,y)$, or ``NO'', if it's not
possible to find any such $y$. If $y$ is guaranteed to exist, the problem
is called a {\em total function problem}. {\em Counting problems}
return the {\em number} of $y$'s that satisfy $R(x,y)$; given a problem
$R$ we denote the associated counting problem $\# R$.

An example of decision problem is: ``(input) given a graph, (question)
is it possible to find an Euler tour of the graph?'' Its complement
is ``(input) given a graph, (question) is it possible that there isn't
any Euler of the graph?'' A search problem is: ``(input) given a graph,
(output) return one Euler tour of the graph, or ``NO'' if no such tour
exists.'' A total function problem is: ``(input) given an Euler graph,
(output) return one of its Euler tours.'' Finally, a counting problem
is ``(input) given a graph, (output) return the number of its Euler tours.''

Computational problems are also classified according to their
{\em computational complexity}, given by the {\em reducibility} from each
other.

\todo[inline]{Turing machines: here - not that in the following deterministic
TM}

Let $P_1$ be a computational problem. For an instance $x$ of $P_1$, let
$|x|$ be the the number of bits needed to encode $x$. $P_1$
{\em reduces to the problem $P_2$ in polynomial time}, denoted
$P_1\leq_P P_2$, if there exists a {\em polynomial-time reduction}, that is,
a function $f: \{0,1\}^\ast \to \{0,1\}^\ast$ and a Turing machine
$\mathcal{M}$ such that for all $x\in\{0,1\}^\ast$

\begin{enumerate}
\item $x\in P_1\quad\iff\quad f(x)\in P_2$;
\item $\mathcal{M}$ computes $f(x)$;
\item $\mathcal{M}$ stops after $p(|x|)$ steps, where $p$ is a polynomial.
\end{enumerate}

For any class $\mathrm{C}$ of decision problems, the class of all complements
of the problems in $C$ is the {\em complement class} $\mathrm{co-C}$.
A problem $P$ is {\em hard} for a class $\mathrm{C}$ if for every problem
$P_C$ in $\mathrm{C}$ there is a polynomial-time reduction to $P$; that is,
if $P$ is hard to solve at least as every problem in $\mathrm{C}$. A
$\mathrm{C}-hard$ problem in $\mathrm{C}$ is {\em complete} for
$\mathrm{C}$.

The complexity class $\mathrm{\mathbf{P}}$ contains all the
{\em polynomially decidable problems}, that is, all problems $P$ such that
there exists a Turing machine $\mathcal{M}$ that outputs either ``YES'' or
``NO''  for all inputs $x\in\{0,1\}^\ast$ of $P$ after $p(|x|)$ steps,
where $p$ is a polynomial. Intuitively, a decision problem is in
$\mathrm{\mathbf{P}}$ if the answer to its question can be found in a
number of steps that is polynomial in the input of the problem.

A problem $P$ belongs to the class $\mathrm{\mathbf{NP}}$,
{\em non-deterministic polynomial-time problems}, if there exists a
Turing machine $\mathcal{M}$ and polynomials $p_1,p_2$ such that

\begin{enumerate}
\item for all $x\in P$ there exists a {\em certificate} $y\in \{0,1\}^\ast$
which satisfies $|y|\leq p_1(|x|)$;
\item $\mathcal{M}$ accepts the combined input $xy$, stopping after at most
$p_2(|x| + |y|)$ steps;
\item for all $x\notin P$ there does not exist $y\in \{0,1\}^\ast$ such
that $\mathcal{M}$ accepts the combined input $xy$.
\end{enumerate}

This means that a decision problem is in $\mathrm{\mathbf{NP}}$ if it takes
polynomial time to verify whether the ``certificate solution'' $y$ is,
indeed, a correct answer to the question posed by the problem.
The class $\mathrm{\mathbf{\# P}}$ is the class of all problems that output
the number of possible certificates for a problem in $\mathrm{\mathbf{NP}}$.

\todo[inline]{check formal def of \# P (?)}

In \cite{megiddo-papad}, Megiddo and Papadimitriou introduce the classes
$\mathrm{\mathbf{FNP}}$, {\em function non-deterministic polynomial}, and
$\mathrm{\mathbf{TFNP}}$, {\em total function non-deterministic polynomial}.
The former is defined as the class of binary relations $R(x,y)$ such that
there is a polynomial-time algorithm that decides whether $R(x,y)$ holds
for given $x,y$ satisfying $|y|\leq p(|x|)$, where $p$ is a polynomial. The
latter is the class of all such problems for which $y$ is guaranteed to
exist. Intuitively, $\mathrm{\mathbf{FNP}}$ and $\mathrm{\mathbf{TFNP}}$ are
similar to $\mathrm{\mathbf{NP}}$, but they allow for problems of
(respectively) function and total function form.

In \cite{megiddo-papad}, Megiddo and Papadimitriou also prove that, unless
$\mathrm{\mathbf{NP}}=\mathrm{\mathbf{co-NP}}$, it's impossible to find a
$\mathrm{\mathbf{TFNP}}$-complete problem.
To circumvent this limitation of $\mathrm{\mathbf{TFNP}}$, Papadimitriou
(\cite{ppad}) focused on the problems for which the existence of a solution
is proved by a ``parity argument'', introducing the classes
$\mathrm{\mathbf{PPA}}$ ({\em Proof by Parity Argument}) and
$\mathrm{\mathbf{PPAD}}$ ({\em Proof by Parity Argument, Directed version}).

\todo[inline]{definition of PPA(D): one in Papadimitriou 1994 and one in DGP
the second w END OF THE LINE, use that one.}

\todo[inline]{as an example, BROUWER, SPERNER (look at Papadimitriou 1994)}

\subsection{Normal Form Games and Nash Equilibria}

A {\em finite normal-form game} $\Gamma=(P,S,u)$,
where $S=\times_{p\in P} S_p$ and $u=\times_{p\in P} u^p$, and both
$P$ and $S$ are finite,
is a model of a strategic interaction.
Each {\em player} $p\in P$ chooses a probability distribution
$x^p=(x_1,\ldots,x_{|S_p|})$ over a set of {\em strategies} $s^p\in S_p$.
Since $x^p$ is a probability distribution, we have ${x_s}^p\geq 0$ and
$\sum_{s}{x_s}^p$ for every $p\in P$ and $s\in S_p$.
If ${x_s}^p = 0$ for every strategy $s^p$ but $\bar{s^p}$, the strategy
$x^p$ is the {\em pure strategy} $\bar{s^p}$; otherwise, $x^p$ is a
{\em mixed strategy}.
The {\em strategy profile} $(x^1,\ldots,x^{|P|})$ influences the
{\em payoff} $u^p: S\to\reals$ of each player $p$. For each player
$p$ we denote the set of strategy profiles of all his
opponents as $S_{-p}=\times_{q\neq p}{x_s}^q$, and for $s\in S_{-p}$ we
denote

\todo[inline]{(what exactly is this?!) as $x_s=\prod_{q\neq p} {x_{s_q}}^q$}

A {\em Nash equilibrium} of a game is a strategy profile in which each
player cannot improve his expected payoff by unilaterally changing his
strategy. Formally, a Nash equilibrium is a strategy profile $x$ such that
for every $p\in P$ and every $i,j\in S_p$
\[
\sum_{s\in S_{-p}} u^p(j,s) x_s > \sum_{s\in S_{-p}} u^p(k,s) x_s
\Rightarrow
{x_k}^p = 0
\]

The existence of a Nash equilibrium is guaranteed by the following theorem
by Nash (\cite{nash}).

\begin{theorem}\label{nash-thm}(Nash, \cite{nash})
Every finite game in normal form has a Nash equilibrium.
\end{theorem}

\todo[inline]{sketch of pf, brouwer}

\todo[inline]{as example, (generalised) matching pennies, used in DGP, incl
appendix here}

\todo[inline]{n-NASH in TFNP and typical problem pushing the def of PPAD
(\cite{ppad}); in fact, PPAD-complete (\cite{dgp} for $n\geq 3$, \cite{cd}
for $n=2$).}



\subsection{Some Geometrical Notation}

\todo[inline]{so results in labels section - after this - don't get lost in
boredom, and each section in background is about 150-200 lines (see: getting
lost).

Maybe turn this into an appendix? It would make sense if something more about
proof of Nash}

We denote the transpose of a matrix $A$ as $A\T$.
We consider vectors $u,v\in\reals^d$ as column vectors, so $u\T v$ is
their scalar product. A vector in $\reals^d$ for which all components
are $0$'s will be denoted as $\0$; similarly, a vectors for which all
components are $1$'s will be denoted as $\1$.
The {\em unit vector} $e_i$ is the vector that has $i$-th component
\smash{${e_i}_i = 1$} and \smash{${e_i}_j=0$} for all other components.
When writing an inequality of the form $u\geq v$ (and analogous), we mean
that it holds for every component; that is, $u_i\geq v_i$ for all
$i\in [d]$.

An {\em affine combination} of points in an Euclidean space $z_1,\ldots,z_n$
is
\[
\sum_{i=1}^n \lambda_i z_i \quad \text{where }\lambda_i\in\reals
\text{ such that }\sum_{i=1}^n \lambda_i = 1
\]

The points $z_1,\ldots,z_n$ are {\em affinely independent} if none of them
is an affine combination of the others.

A {\em convex combination} of points $z_1,\ldots,z_n$ is an affine
combination where $\lambda_i\geq 0$ for all $i\in [n]$.
Note that such $\lambda_i$'s can be seen as a probability distribution over
the $z_i$'s.

\todo[inline]{def simplex: here?}

A set of point $Z$ is {\em convex} if it is closed under forming convex
combinations, that is, if $\bar{z}=\sum_{i=1}^n \lambda_i z_i$,
where $z_i\in Z$, $\lambda_i\geq 0$ and $\sum_{i=1}^n \lambda_i = 1$,
then $\bar{z}\in Z$. A convex set has {\em dimension} $d$ if it has exactly
$d + 1$ affinely independent points.


\todo[inline]{def simplex: here?}


\todo[inline]{convex hull (needed for def cyclic poly);

pow hyperplanes;

polyhedron, polytopes}



\todo[inline]{from here: notes - copy-paste}

A ($d$-dimensional) {\em simplicial polytope} $P$ is the convex hull of a set
of at least $d+1$ points $v$ in $\reals^d$ in general position, that is, no
$d+1$ of them are on a common hyperplane.

If a point $v$ cannot be omitted from these points without changing $P$ then
$v$ is called a {\em vertex} of $P$. A {\em facet} of $P$ is the convex hull
$\conv\,F$ of a set $F$ of $d$ vertices of $P$ that lie on a hyperplane
$\{ x\in \reals^d\mid a^T x=a_0\}$ so that $a^T u<a_0$ for all other vertices
$u$ of $P$; the vector $a$ (unique up to a scalar multiple) is called the
{\em normal vector} of the facet. We often identify the facet with its set of
vertices~$F$.

\newpage

\subsection{Bimatrix Games, Labels and Polytopes}

In the rest of this thesis we will focus on two-player normal-form games.
For sake of readability, we will use feminine pronouns when referring to
player 1 and masculine pronouns when referring to player 2.

Two-player normal-form games are
also called {\em bimatrix games}, since they can be characterized by the
$m \times n$ payoff matrices $A$ and $B$, where $a_{ij}$ and $b_{ij}$ are
the payoffs of player 1 and 2 when she plays her $i$th pure strategy
and he plays his $j$th pure strategy.
We will assume that $(A,B)$ are non-negative, and that $A$ and $B\T$ have
no zero column. This can be easily obtained without loss of generality via
an affine transformation that will not affect the equilibria of the game.

The Nash equilibria of bimatrix games can be analysed from a combinatorial
point of view using {\em labels}. This method is due to Shapley
\cite{shapley}, in a study building on ideas introduced in a
paper by Lemke and Howson \cite{lh}.


Let $(A,B)$ be bimatrix game. The mixed-strategy simplices of player 1 and 2
are, respectively

\begin{equation}
X = \{ x\in\reals^m | x\geq\0,\ \1\T x = 1 \},\quad
Y = \{ y\in\reals^n | y\geq\0,\ \1\T y = 1 \}
\end{equation}

A {\em labeling} of the game is then given as follows:

\begin{enumerate}
\item the $m$ pure strategies of player 1 are identified by $1,\ldots,m$;
\item the $n$ pure strategies of player 2 are identified by $m+1,\ldots,m+n$;
\item each mixed strategy $x\in X$ of player 1 has
    \begin{itemize}
    \item label $i$ for each $i\in [m]$ such that $x_i = 0$, that is if in
    $x$ player 1 does not play her $i$th pure strategy;
    \item label $m + j$ for each $j\in [n]$ such that the $j$th pure strategy
    of player 2 is a best response to $x$;
    \end{itemize}
\item each mixed strategy $y\in Y$ of player 2 has
    \begin{itemize}
    \item label $m + j$ for each $j\in [n]$ such that $y_j = 0$, that is if in
    $y$ player 2 does not play his $j$th pure strategy;
    \item label $i$ for each $i\in [m]$ such that the $i$th pure strategy
    of player 1 is a best response to $y$;
    \end{itemize}
\end{enumerate}

A strategy profile $(x,y)\in X\times Y$ is {\em completely labeled} if every
label $1,\ldots,m+n$ is a label of either $x$ or $y$. We have the following
theorem (Theorem 1 in \cite{shapley}):

\begin{theorem}\label{comp-label-bimatrix-thm}
Let $(x,y)\in X\times Y$; then $(x,y)$ is a Nash equilibrium of the bimatrix
game $(A,B)$ if and only if $(x,y)$ is completely labeled.

\begin{proof}
The mixed strategy $x\in X$ has label $m + j$ for some $j\in [n]$ if and
only if the $j$th pure strategy of player 2 is a best response to $x$; this,
in turn, is a necessary and sufficient condition for player 2 to play his
$j$th strategy at an equilibrium against $x$. Therefore, at an equilibrium
$(x,y)$ all labels $m + 1,\ldots,m + n$ will appear either as labels of
$x$ or of $y$. The analogous holds for the labels $i\in [n]$.
\end{proof}
\end{theorem}

An useful geometrical representation of labels can be given on the mixed
strategies simplices $X$ and $Y$. The outside of each simplex is
labeled according to the player's own pure strategies that are {\em not}
played; so, for instance, the outside of $X$ will have labels
$1,\ldots,n$. The interior of each simplex is subdivided in closed polyhedral
sets, called {\em best-response regions}. These are labeled according to
the other player's pure strategy that is a best response in that set;
so, for instance, the inside of $X$ will have labels $m + 1,\ldots,m + n$.

We give an example of this construction.

\begin{example}

\todo[inline]{page 3--4 of Savani, von Stengel, Unit Vector Games.

With graphics.}

\end{example}

We will now give a description of labeling on polytopes equivalent to
the construction based on best-response regions.

We begin by noticing that the best-response regions can be obtained as
projections on $X$ and $Y$ of the {\em best-response facets} of
the polyhedra

\begin{equation}\label{br-polyhedron}
\bar{P} = \{ (x,v)\in X\times\reals | B\T x\leq\1 v \},\quad
\bar{Q} = \{ (y,u)\in Y\times\reals | A y\leq\1 u \}.
\end{equation}

These facets in $\bar{P}$ are defined as the points $(x,v)\in X\times\reals$
such that $(B\T x)_j = v$. These points represent the strategies $x\in X$ of
player 1 that give exactly payoff $v$ to player 2 when he plays strategy $j$.
The projection of the facet defined by $(B\T x)_j = v$ to $X$ will have
label $j$. Analogously, in $\bar{Q}$, the facets are the points
$(y,u)\in Y\times\reals$ such that $a_i y = u$, and their projection to $Y$
will be the best-response region with label $i$.

\begin{example}

\todo[inline]{cont of ex above, page 4--5, image on page 5 left}

\end{example}

Given the assumptions on non-negativity of $A$ and $B\T$, we can give a
change coordinates to $x_i / v$ and $y_j / u$ and replace $\bar{P}$ and
$\bar{Q}$ with the {\em best-response polytopes}

\begin{equation}\label{br-polytopes}
P = \{ x\in\reals^m | x\geq\0,\ B\T x\leq\1 \},\quad
Q = \{ y\in\reals^n | y\geq\0,\ A y\leq\1 \},\quad
\end{equation}

Each one of these polytope is defined by half spaces corresponding to
either the player's own strategy that is not being played or the other
player's best response; each one of the facets of the polytope is labeled
by the strategy corresponding to the relative half-space.

This means that a point in $P$ has label $k$ if and only if either
$x_k = 0$ for $k\in \{ 1,\ldots,m \}$ or $(B\T x)_{k - m} = 0$ for
$k\in \{ m+1,\ldots,m+n \}$;
analogously, a point in $Q$ has label $k$ if and only if either
$y_{k - m} = 0$ for $k\in \{ m+1,\ldots,m+n \}$ or $(A y)_{k}$ for
$k\in \{ m+1,\ldots,m+n \}$. A point $(x,y)\in P\times Q$ is
{\em completely labeled} if every $k\in [m + n]$ is a label of $x$ or $y$.
Note that the point $(\0,\0)$ is completely labeled. Rescaling back to
$\bar{P}$ and $\bar{Q}$, all the non-zero completely labeled points give
exactly all the equilibria of $(A,B)$. In this construction, we will
call $(\0,\0)$ {\em artificial equilibrium}.

\begin{example}

\todo[inline]{ex in Savani, von Stengel, image on page 5 right}

\end{example}

A characterization of the completely labeled pairs in $P\times Q$ can be
given as follows.

\begin{proposition}\label{compl-orth-cond}
The pair $(x,y)\in P\times Q$ is completely labeled if and only if one of
the following condition holds:
\begin{itemize}
\item {\em (Complementarity condition)}

\begin{equation}
x_i = 0\text{ or }(Ay)_i = 1\text{ for all }i\in [m],\quad
y_j = 0\text{ or }(B\T x)_j\text{ for all }j\in [n]
\end{equation}

\item {\em (Orthogonality condition)}

\begin{equation}
x\T (\1 - Ay) = 0,\quad
y\T (\1 - B\T x) = 0
\end{equation}
\end{itemize}
\end{proposition}

Proposition \ref{compl-orth-cond} can be used to prove a useful property:
{\em symmetric games}, that is, games that have payoff matrix of the form
$(C,C\T)$ for some matrix $C$, can be used to study generic bimatrix games
without loss of generality. The result is due to Gale, Kuhn and Tucker
\cite{gale-kuhn-tucker} for zero-sum games; its extension to non-zero-sum
games is a folklore result.

\begin{proposition}
Let $(A,B)$ be a bimatrix game and $(x,y)$ be one of its Nash equilibria.
Then $(z,z)$, where $z=(x,y)$, is a Nash equilibrium of the symmetric game
$(C,C\T)$, where

\[
C = \left(
    \begin{array}{cc}
    0 & A \\
    B\T & 0
    \end{array}
    \right).
\]
\end{proposition}

The converse has been proved by McLennan and Tourky \cite{mclennan-tourky} in
their study of {\em imitiation games}, that is, bimatrix games of the form
$(I,B)$.

\begin{proposition}\label{imitation-thm}
The pair $(x,x)$ is a symmetric Nash equilibrium of the symmetric bimatrix
game $(C,C\T)$ if and only if there is some $y$ such that $(x,y)$ is a
Nash equilibrium of the imitation game $(I,C\T)$.
\end{proposition}

\begin{example}
Consider the symmetric game $(C,C\T)$, where $C\T = B$ in the previous
examples.

\todo[inline]{ex Savani, von Stengel, pg 8}

\end{example}

Balthasar \cite{balthasar} and V\'{e}gh and von Stengel \cite{vvs} extended
proposition \ref{imitation-thm} to {\em unit vector games}, that is, games
of the form $(U,B)$, where the columns of the matrix $U$ are unit vectors.
The form of the theorem that we give here follows the version in Savani
and von Stengel \cite{svs-15}, dual to the one in Balthasar \cite{balthasar}.

\begin{theorem}\label{unit-vector-thm}
Let $l:[n]\to [m]$, and let $(U,B)$ be the unit vector game where
$U=(e_{l(1)}\ \cdots\ e_{l(n)})$. Consider the polytopes $P^l$ and $Q^l$
where

\begin{equation}
P^l = \{ x\in\reals^m | x\geq\0,\ B\T x\leq\1 \}
\end{equation}

\begin{equation}
Q^l = \{ y\in\reals^n | y\geq\0,\
\sum_{\substack{j\in N_i \\ i\in [m]}} y_j\leq 1 \}
\end{equation}

where $N_i = \{ j\in [n] | l(j)=i \}$ for $i\in [m]$.

Label every facet of $P^l$ according to the inequality defining it,
as follows:

\begin{itemize}
\item $x_i\geq 0$ has label $i$, for $i\in [m]$
\item $(B\T x)_j \leq 1$ has label $l(j)$, for $j\in [n]$
\end{itemize}

Then $x\in P^l$ is a completely labeled point of $P^l\setminus\{\0\}$
if and only if there is some $y\in Q^l$ such that, after scaling,
the pair $(x,y)$ is a Nash equilibrium of $(U,B)$

\begin{proof}
Let $P,Q$ be the polytopes associated to the game $(U,B)$ as before.

Let $(x,y)\in P\times Q\setminus\{ \0,\0 \}$ be a Nash equilibrium of
$(U,B)$, therefore completely labeled in $[m + n]$.
Then, if $x_i=0$, then $x$ has label $i\in m$.
If $x_i > 0$ instead, then $y$ has label $i$, therefore $(Uy)_i = 1$,
therefore for some $j\in [n]$ we have $y_j > 0$ and $U_j = e_i$, so $l(j)=i$.
Since $y_j > 0$ and $(x,y)$ is completely labeled, $x\in P$ has label $m+j$,
that is, $(B\T x)_j = 1$, therefore $x\in P^l$ has label $l(j) = i$.
Hence, $x$ is a completely labeled point of $P^l$.

Conversely, let $x\in P^l\setminus \{ \0 \}$ be completely labeled.
If $x_i > 0$, then there is $j\in [m]$ such that $(B\T x) = j$ and
$l(j) = i$, that is, $j\in N_i$. For all $i$ such that $x_i >0 $,
define $y$ as follows: $y_h = 0$ for all $h\in N_i\setminus \{ j \}$,
$y_j = 1$. Then $(x,y)\in P\times Q$ is completely labeled.
\end{proof}
\end{theorem}


\todo[inline]{
nondegeneracy; made nondegenerate by ``lexicographic'' perurbation
(what does it mean?);

ex pg 9; odd no eq, mention homotopy method (find ref)
(tie with Nash, again?)
}


\subsection{Cyclic Polytopes and Gale Strings}\label{gs-ssect}

A special case of games is obtained by taking a particular case of best
response polytope in theorem \ref{t-unitv}.

A {\em cyclic polytope} $P$ in dimension~$d$ with $n$ vertices is the
convex hull of distinct points $\mu(t_j)$, where $j\in [n]$ and $\mu$ is the
{\em moment curve}
\[
\mu\colon t\mapsto(t,t^2,\ldots,t^d)^\top
\]

Restricting the study of best response polytopes to the case of cyclic
polytopes gives an interesting case, since cyclic polytopes can be
represented as a combinatorial structure, called {\em Gale strings}.
These are a case of {\em bitstrings}, that is a string of $0$'s and $1$'s.

Formally: given an integer $k$ and a set $S$, we can represent the function
$f_s:[k]\to S$ as the string $s = s(1)s(2)\cdots s(k)$. In the case where
$S=\{0,1\}$ we call $s$ a bitstring.

A maximal substring of consecutive $1$'s in a bitstring is called a {\em run}.

We denote with $G(d,n)$ the set of all {\em Gale strings of length $n$ and
dimension $d$}, defined as the set of all bitstrings $s$ of length $n$
such that  {\em Gale string} is a

\begin{enumerate}
\item exactly $d$ bits in $s$ are $1$ and
\item $s$ fulfills the {\em Gale evenness condition}:
\[
01^k0\hbox{ is a substring of }s\quad{\Rightarrow}\quad k\hbox{ is even.}
\]
\end{enumerate}

The Gale evenness condition characterises Gale strings in $G(d,n)$ as
the bitstrings of length~$n$ with exactly~$d$ elements equal to $1$,
such that {\em interior} runs (that is, runs bounded on both sides by~$0$s)
must be of even length. In general, this condition allows Gale strings to
start or end with an odd-length run. When $d$ is even, on the other hand,
$s$ starts with an odd run if and only if it ends with an odd run.
We can then consider the Gale strings in $G(d,n)$ with even $d$ as a ``loop''
obtained by ``glueing together'' the extremes of the string to form an even
run; more formally, we can see the indices of the string as equivalence
classes modulo $n$, so that we identify $s(i+n)=s(i)$. This also implies
that the set of Gale strings of even dimension is therefore invariant
under a cyclic shift of the strings.

\begin{example}\label{gs-example}
We consider $G(4,6)$. We have
\begin{align*}
G(4,6) = \{ & 111100, \\
        & 111001, \\
        & 110011, \\
        & 100111, \\
        & 001111, \\
        & 011110, \\
        & 110110, \\
        & 101101, \\
        & 011011\}
\end{align*}

The strings $111100$, $111001$, $110011$, $100111$, $001111$ and $011110$ are
equivalent under a cyclic shift (if considering the strings as loops, the
$1$'s are all consecutive), as are the strings $110110$, $101101$ and
$011011$ (if considering the strings as loops, the even runs of $1$'s are
two couples separated by a single $0$).
\end{example}

\todo[inline]{here to end subsect: polytopes - edit all anyway}

The relation between cyclic polytopes and Gale strings is given by the
following theorem by Gale \cite{gale-cyclicpoly}.

\begin{theorem}[\cite{gale-cyclicpoly}]\label{cp-gs-gale}
For any positive integer $n$, assume that $t_1 < t_2 < \cdots < t_n$ and let
P be the cyclic polytope obtained by taking $t_j$, where $j \in [n]$, in
definition \ref{cyclic-polytope}.

Then the facets of $P$ are encoded by $G(d,n)$; that is, $F$ is a facet of
$P$ if and only if
\[
F = \conv\{\mu(t_i)\mid i\in 1(s)\} \qquad \hbox{ for some }s\in G(d,n)
\]
\end{theorem}

\todo[inline]{sketch of pf if not too long and it uses relevant techniques}

\todo[inline]{graphics of cyclic polytope - parallel to gale string}

From this point forward, we will assume that $d$ is even.

\todo[inline]{give something to generalise to odd case}

\subsection{Labeling and the Problem \anothergale}

Given a set $G$ of bitstrings of length $n$ and a parameter $d$, a
{\em labeling} is a function $l:[n]\to[d]$. A string $s$ in $G(d,n)$ is
{\em completely labeled} if $l(\mathnormal{1}(s))=[d]$. Any $l(i)\in [d]$
is called a {\em label}

If $s \in G(d,n)$ is completely labeled for the labeling $l:[n]\to[d]$, then
for each label $l(i)$ there is a bit $s(i)=1$. We therefore have exactly
$d$ positions $i$ for which $s(i)=1$; hence, $|l(\mathnormal{1}(s))|=d$.

\begin{example}
Given the string of labels $l=123432$, there are four associated completely
labeled Gale strings: $111100$, $110110$, $100111$ and $101101$.\\

\begin{center}
{\renewcommand{\tabcolsep}{2ex}
\begin{tabular}{|c|c|c|c|}
\hline
\textbf{1234}32 &
\textbf{12}3\textbf{43}2 &
\textbf{1}23\textbf{432} &
\textbf{1}2\textbf{34}3\textbf{2} \\
\textbf{1111}00 &
\textbf{11}0\textbf{11}0 &
\textbf{1}00\textbf{111} &
\textbf{1}0\textbf{11}0\textbf{1} \\
\hline
\end{tabular}
}\\
\end{center}

\end{example}

Sometimes there aren't any completely labeled Gale strings that are
associated with a given labeling.

\begin{example}
For $l = 121314$, there are no completely labeled Gale strings.
\end{example}

\todo[inline]{here to end subsect: polytopes}
\todo[inline]{graphics of labeled cyclic polytope}

% Essentially, this holds because any set $S\subset [n]$
% the moment curve defines a unique hyperplane which is crossed
% (and not just touched) by the moment curve; if the bitstring
% $s$ that encodes $F$ as $1(s)$ has a substring $01^k0$
For this cyclic polytope $P$, a labeling $l:[n]\to[d]$ can
be understood as a label $l(j)$ for each vertex $\mu(t_j)$
for $j\in [n]$.
A completely labeled Gale string $s$ therefore represents a
facet $F$ of $P$ that is completely labeled.

Special games are obtained by using cyclic polytopes in
Theorem~\ref{t-unitv}, suitably affinely transformed with
a completely labeled facet $F_0$.
When $Q$ is a cyclic polytope in dimension $d$ with $d+n$
vertices, then the string of labels $l(1)\cdots l(n)$ in
Theorem~\ref{t-unitv} defines a labeling $l':[d+n]\to [d]$
where $l'(i)=i$ for $i\in [d]$ and
$l'(d+j)=l(j)$ for $j\in [n]$.
In other words, the string of labels $l(1)\cdots l(n)$ is
just prefixed with the string $1\,2\cdots d$ to give $l'$.
Then $l'$ has a trivial completely labeled Gale string
$1^d0^n$ which defines the facet $F_0$.
Then the problem \anothergale\ defines exactly the problem of finding a Nash
equilibrium of the unit vector game $(I,B)$.
Note again that $B$ is here not a general matrix (which would
define a general game) but obtained from the last $n$ of
$d+n$ vertices of a cyclic polytope in dimension~$d$.

\begin{fctproblem}
{\anothergale}
{A labeling $l:[n]\to[d]$, where $d$ is even and $d<n$, and an associated
completely labeled Gale string $s$ in $G(d,n)$.}
{A completely labeled Gale string $s'$ in $G(d,n)$ associated with $l$, such
that $s' \neq s$.}
\end{fctproblem}


\newpage

\section{Algorithmic and Complexity Results}
\label{main-thm-sect}

\subsection{Pivoting}

\subsection{The Lemke-Howson Algorithm and Parity}

\todo[inline]{file: pivoting-LH-subsection}

\section{The Complexity of \gale\ and \anothergale}

We will now give our main result: \anothergale\ can be solved in polynomial
time; therefore, it takes polynomial time to find a Nash Equilibrium of a
bimatrix game with dual cyclic best response polytope. Our proof will
rely on the construction of a graph and, if possible, a perfect matching
for it.
A {\em perfect matching} of a multigraph $G=(V,E)$ is a set $M\subseteq E$
of pairwise non-adjacent edges so that every vertex $v \in V$ is incident
to exactly one edge in~$M$. A theorem by Edmonds (\cite{edm}) gives
the complexity of the associated problem {\sc Perfect Matching}.

\begin{fctproblem}
{Perfect Matching}
{A multigraph $G = (V,E)$.}
{A perfect matching for $G$, or {\sc No} if there is no possible perfect
matching for $G$.}
\end{fctproblem}

\begin{theorem}{\rm (Edmonds \cite{edm})}\label{pm-thm}
The problem {\sc Perfect Matching} can be solved in polynomial time.
\end{theorem}

To prove our main result on \anothergale, we will first focus on the
accessory problem \gale, and we will use theorem \ref{pm-thm} to prove
that it is solvable in polynomial time. We will consider every Gale string
as a ``loop.''

\begin{fctproblem}
{\gale}
{A labeling $l:[n]\to[d]$, where $d$ is even and $d<n$.}
{A Gale string $s\in G(d,n)$ that is completely labeled by $l$}
\end{fctproblem}

\begin{theorem}\label{gale-thm}
The problem \gale\ is solvable in polynomial time.

\begin{proof}
We give a reduction of \gale\ to {\sc Perfect Matching}.

Consider the multigraph $G=(V,E)$ with $V=[d]$, so that the vertices of $G$
correspond to the labels $l(i)\in [d]$, and
$E=\{(l(i),l(i+1))\text{ for }i\in[n] \}$, so that there is an edge
between two vertices if and only if the corresponding labels are
next to each other at some index $i$.
Let $s\in G(d,n)$ be a completely labeled Gale string. By Gale evenness
condition, every run of $s$ corresponds uniquely to $d/2$ pairs of indices
$(i,i+1)$ with $s(i)=s(i+1)=1$, and since $s$ is completely labeled, all
labels $l(i)\in [d]$ occur at exactly one of these indices. Then the edges
$(l(i),l(i+1))$ form a perfect matching of $G$.

Conversely, let $l:[n]\to [d]$ be a labeling, and let $M$ be a perfect
matching for $G$. Consider a bitstring $s$ with $s(i)=s(i+1)$ for every
$(l(i),l(i+1))\in M$ and $s(i)=0$ otherwise.
Since $M$ is a matching, all the $(l(i),l(i+1))\in M$ are
disjoint, so, considering $s$ as a ``loop,'' every run of $s$ is of even
length, thus satisfying the Gale evenness condition. Since $M$ is perfect,
every vertex $v\in [d]$ is the endpoint of an edge $(l(i),l(i+1))$, so $s$
has exactly $d$ bits equal to $\1$, so it is completely labeled.

We have therefore reduced the problem \gale to {\sc Perfect Matching}, that
by theorem \ref{gale-thm} can be solved in polynomial time.
\end{proof}
\end{theorem}

We give two examples of the construction used in theorem \ref{gale-thm}.

\begin{example}\label{gs-pm-ex}
Consider the labeling $l:[6]\to [4]$ with $l=123423$. To find a
Gale string $s\in G(4,6)$ that is completely labeled by $l$ we must look for
a perfect matching $M$ of $G=([4],\{ e_i=(l(i),l(i+1))\ |\ i\in [6] \})$,
as seen in figure \ref{perfect-matching}.

\begin{figure}[hbt]
\strut\hfill
\includegraphics[width=40ex]{chapter-2/fig/perfect-matching.pdf}%
\hfill\strut
\caption[The graph from a labeling]{%
The graph $G$ associated to the labeling $l=123423$.
}
\label{perfect-matching}
\end{figure}

The matching $M$, in turn, will give the completely labeled Gale string $s$
as $s(i)=s(i+1)=1$ for $e_i\in M$, $s(j)=0$ otherwise.

For instance, if we take the perfect matching $M=\{ e_1,e_3 \}$, we have
the string $s=111100$. If we take the perfect matching $M'=\{ e_4,e_6 \}$
instead, we have the string $s'=1000111$.
\end{example}

A perfect matching for a graph, and therefore a Gale string for a labeling,
is not always possible, as shown in the next example.

\begin{example}
Consider the labeling $l=121314$. The graph $G$ is shown in figure
\ref{no-matching}

\begin{figure}[hbt]
\strut\hfill
\includegraphics[width=27ex]{chapter-2/fig/no-matching.pdf}%
\hfill\strut
\caption[A graph without a perfect matching]{%
The graph for the labeling $l=121314$
}
\label{no-matching}
\end{figure}

Since there aren't any disjoint edges, it's not possible to find a perfect
matching for $G$. We have already seen in example \ref{no-clgs} that
there isn't any possible completely labeled Gale string for $l=121314$.
\end{example}

We finally extend the proof of theorem \ref{gale-thm} to \anothergale.

\begin{theorem}\label{anothergale-thm}
The problem \anothergale\ is solvable in polynomial time.

\begin{proof}
Let $G=(V,E)$ be the graph corresponding to the labeling $l:[n]\to [d]$ as
in the proof of theorem \ref{gale-thm} and let $M$ be the perfect matching
of $G$ corresponding to the completely labeled Gale string $s\in G(d,n)$.

If there are two edges $e,e'\in E$ such that $e\in M$, both $e$ and $e'$
have endpoints $l(i),l(i+1)$, but $e\neq e'$ (recall that $G$ can be a
multigraph), the matching $M'=(M\setminus \{ e \})\cup\{ e' \}$
is perfect.
The corresponding completely labeled Gale string
$s'\in G(d,n)$ satisfies $s'\neq s$, since in $s$ the \1's corresponding
to the labels $l(i),l(i+1)$ are in the positions given by the edge
$e$, while in $s'$ they are in the positions given by $e'\neq e$.
It takes time $d/2$ to check all edges of $M$, the time required
is still polynomial.

We now assume that all the edges in every perfect matching $M$ for $G$
don't have a parallel edge.
Since by theorem \ref{lhg-works-thm} there is an even number of
completely labeled Gale strings, the existence of $s$ guarantees the
existence of another completely labeled Gale string $s'\neq s$ and the
corresponding perfect matching $M'\neq M$.
Since $M'\neq M$, there is at least one edge $e'\in M$ such that
$e'\notin M'$. Consider the $d/2$ graphs $G_i=(V,E_i)$, where
$E_i=E\setminus\{ e_i \}$ for $e_i\in M$. Since $V(G)=V(G')$ and
$E(G)\subset E(G')$, every perfect matching for one of these $G_i$
is a perfect matching for $G$ as well.
With a brute force approach, we look for a perfect matching in each $G_i$;
this will be $M'$. Since there are $i\in [d/2]$, the time to find it
will be still polynomial.
\end{proof}
\end{theorem}

We give two examples of the construction of theorem \ref{anothergale-thm}.

\begin{example}
The labeling $l=1234324$ gives the graph $G$ in figure
\ref{matching-2-edges}. Suppose that Edmonds' algorithm returns the matching
$M=\{ e_1,e_3 \}$, associated to the completely labeled Gale string
$s=1111000$. The edge $e_3$ has a parallel edge, $e_4$; we immediately
have a second perfect matching in $M'=\{ e_1,e_4 \}$, associated to the
Gale string $s'=1101100$.

\begin{figure}[hbt]
\strut\hfill
\includegraphics[width=40ex]{chapter-2/fig/matching-2-edges.pdf}%
\hfill\strut
\caption[A matching with a parallel edge]{%
The graph for the labeling $l=1234324$.
}
\label{matching-2-edges}
\end{figure}
\end{example}

A case in which every edge in every perfect matching does not have a
parallel one is the one given in example \ref{gs-pm-ex}.

\begin{example}
Consider the labeling $l=123423$; the associated graph is shown on
the left in figure \ref{intermediate-matching}. There are only two possible
matchings, and neither has a parallel edge.
Note that $G$ is a multigraph: we look for parallel edges in the matching,
not in all the edges of the graph.
Suppose that Edmonds' algorithm returns the perfect matching
$M=\{ e_1,e_3 \}$; we can then delete the edge $e_1$ to obtain the graph
$G_1$, seen on figure \ref{intermediate-matching} right. The graph $G_1$
has a perfect matching in $M'=\{ e_4,e_6 \}$, that is also a perfect
matching of $G$, associated to the string $s'=100111$.

\begin{figure}[hbt]
\strut\hfill
\includegraphics[width=35ex]{chapter-2/fig/perfect-matching.pdf}%
\hfill
\hfill
\includegraphics[width=35ex]{chapter-2/fig/intermediate-matching.pdf}%
\hfill\strut
\caption[A matching without parallel edges]{%
Left: the graph $G=(V,E)$ for the labeling $123423$. \\
Right: the graph $G_1=(V,E\setminus\{ e_1 \})$.
}
\label{intermediate-matching}
\end{figure}
\end{example}


\newpage

\section{Further results}\label{further-section}

\newpage

\section*{A result about the PPAD completeness of \textsc{Nash}}

\section*{Notation}

\newpage


\begin{thebibliography}{00}

\frenchspacing\parskip0.3ex
\small

\bibitem{balthasar} A. V. Balthasar (2009).
``Geometry and equilibria in bimatrix games.''
PhD Thesis, London School of Economics and Political Science.

\bibitem{brightwell} G. R. Brightwell, P. Winkler (2004).
``Note on Counting Eulerian Circuits.''
CDAM Research Report LSE-CDAM-2004-12.

\bibitem{msc-diss} M. M. Casetti (2008).
``PPAD Completeness of Equilibrium Computation.''
MSc Thesis, London School of Economics and Political Science.

\bibitem{main} M. M. Casetti, J. Merschen, B. von Stengel (2010).
``Finding Gale Strings.''
\emph{Electronic Notes in Discrete Mathematics} 36, pp. 1065--1082.

\bibitem{cd} X. Chen, X. Deng (2006).
``Settling the Complexity of 2-Player Nash Equilibrium.''
\emph{Proc. 47th Annual IEEE Symposium on Foundations of
Computer Science (FOCS)}, pp. 261--272.

\bibitem{dgp} C. Daskalakis, P. W. Goldberg, C. H. Papadimitriou (2009).
``The Complexity of Computing a Nash Equilibrium.''
\emph{SIAM Journal on Computing} 39, pp. 195--259.

\bibitem{edm} J. Edmonds (1965).
``Paths, Trees, and Flowers.''
\emph{Canad. J. Math.} 17, pp. 449--467.

\bibitem{edm-oiks} J. Edmonds (2009).
``Euler complexes.''
In: \emph{Research Trends in Combinatorial Optimization}, eds. W.
Cook, L. Lovasz, and J. Vygen, Springer, Berlin, pp. 65--68.

\bibitem{edm-sanita} J. Edmonds, L. Sanit\`a (2010).
``On finding another room-partitioning of the vertices.''
\emph{Electronic Notes in Discrete Mathematics} 36, pp. 1257--1264.

\bibitem{gale} D. Gale (1963).
``Neighborly and Cyclic Polytopes.''
In: \emph{Convexity, Proc. Symposia in Pure Math.}, Vol. 7,
ed. V. Klee, American Math. Soc., Providence, Rhode Island, pp. 225--232.

\bibitem{gale-kuhn-tucker} D. Gale, H. W. Kuhn, A. W. Tucker (1950).
``On Symmetric Games.''
In: \emph{Contributions to the Theory of Games}~I, eds. H. W. Kuhn and A. W.
Tucker, \emph{Annals of Mathematics Studies} 24, Princeton University Press,
Princeton, pp. 81--87.

\bibitem{gilboa-zemel} I. Gilboa, E. Zemel (1989).
``Nash and correlated equilibria: some complexity considerations.''
\emph{Games and Economic Behavior} 1, pp. 80--93.

\bibitem{wicdiv} K. Gillen, J. McKelvie, M. Wilson (2015).
``Fear and Loathing in Eternity.''
{\em The Wicked + The Divine}, issue 9, ed. Image Comics.

\bibitem{gps} P. W. Goldberg, C. H. Papadimitriou, R. Savani (2011).
``The Complexity of the Homotopy Method, Equilibrium Selection, and
Lemke-Howson solutions.''
\emph{Proc. 52nd Annual IEEE Symposium on Foundations of Computer Science (FOCS)}, pp. 67--76.

\bibitem{lh} C. E. Lemke, J. T. Howson, Jr. (1964).
``Equilibrium Points of Bimatrix Games.''
\emph{J.  Soc. Indust. Appl. Mathematics} 12, pp.  413--423.

\bibitem{mclennan-tourky} A. McLennan, R. Tourky (2010).
``Imitation Games and Computation.''
\emph{Games and Economic Behavior} 70, pp. 4--11.

\bibitem{megiddo-papad} N. Megiddo, C. H. Papadimitriou (1991).
``On Total Functions, Existence Theorems and Computational Complexity.''
\emph{Theoretical Computer Science} 81, pp. 317--324.

\bibitem{jm} J. Merschen (2012).
``Nash Equilibria, Gale Strings, and Perfect Matchings.''
PhD Thesis, London School of Economics and Political Science.

\bibitem{morris} W. D. Morris Jr. (1994).
``Lemke Paths on Simple Polytopes.''
\emph{Math. Oper. Res.} 19, pp. 780--789.

\bibitem{nash} J. F. Nash (1951).
``Noncooperative games.''
\emph{Annals of Mathematics}, 54, pp. 289--295.

\bibitem{gth} M. J. Osborne, A. Rubinstein (1994).
{\em A Course in Game Theory.}
The MIT Press, Cambridge, Massachusetts.

\bibitem{papad-cc} C. H. Papadimitriou (1994).
{\em Computational Complexity.}
Addison-Wesley, Reading, MA.

\bibitem{ppad} C. H. Papadimitriou (1994).
``On the Complexity of the Parity Argument and Other Inefficient Proofs of
Existence.''
\emph{J. Comput. System Sci.} 48, pp. 498--532.

\bibitem{svs} R. Savani, B. von Stengel (2006).
``Hard-to-solve Bimatrix Games.''
\emph{Econometrica} 74, pp. 397--429.

\bibitem{uvg} R. Savani, B. von Stengel (2015).
``Unit Vector Games.''
arXiv:1501.02243v1 [cs.GT]

\bibitem{shapley} L. S. Shapley (1974).
``A Note on the Lemke-Howson Algorithm.''
\emph{Mathematical Programming Study 1: Pivoting and Extensions}, pp. 175--189

\bibitem{vvs} L. A. V\'{e}gh, B. von Stengel (2015),
``Oriented Euler Complexes and Signed Perfect Matchings.''
\emph{Mathematical Programming Series B} 150, pp. 153--178.

\bibitem{vn28} J. von Neumann (1928).
``Zur Theorie der Gesellschaftspiele.''
\emph{Mathematische Annalen} 100, pp. 295--320.

\bibitem{vs-agt} B. von Stengel (2007).
``Equilibrium computation for two-player games in strategic and extensive
form.'' Chapter 3, ``Algorithmic Game Theory,''
eds. N. Nisan, T. Roughgarden, E. Tardos, V. Vazirani.
Cambridge Univ. Press, Cambridge, pp. 53--78.

\bibitem{vs-noclf} B. von Stengel (2012).
``Completely Labeled Facet is NP-Complete.''
Manuscript, 6 pp.

\bibitem{ziegler} G. M. Ziegler (1995).
{\em Lectures on Polytopes.}
Springer, New York.

\end{thebibliography}


\end{document}
