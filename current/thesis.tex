\documentclass[11pt, a4paper]{report}

\usepackage{todonotes}

\author{Marta Maria Casetti}

\title{Complexity of the Gale String Problem for Equilibrium
Computation in Games}

%% TYPESETTING

\linespread{1.5}
\setlength{\textwidth}{13cm}
\setlength{\oddsidemargin}{2.5cm}
\usepackage[T1]{fontenc}

%% PACKAGES

%% maths symbols and fonts

\usepackage{amssymb}
\usepackage{amsmath}
\usepackage{amsthm}
\usepackage[ruled,vlined,linesnumbered]{algorithm2e}

%% images (TODO: remove xy)

\usepackage[all]{xy}

\usepackage{xcolor,graphicx}
\usepackage[font=small,format=hang,justification=justified,labelfont=bf,labelsep=quad]{caption}


%% hyperref

\usepackage[pdftex,bookmarks,colorlinks]{hyperref}
\hypersetup{citecolor=red,linkcolor=blue}

% When printing
% \hypersetup{colorlinks=false}

%% THEOREMS and ENVIRONMENTS

%% theorems

\newtheorem{theorem}{Theorem}
\newtheorem{proposition}{Proposition}[chapter]
\theoremstyle{remark}\newtheorem{example}{Example}[chapter]

%% environments

% decision problem

\newenvironment{decproblem}[3]
{
\vspace{2.5ex}
\noindent
\begin{tabular}{p{17mm} @{\textbf{:} } p{106mm}}
\hline
\multicolumn{2}{l}{\noindent {\sc #1}} \\
\hline
\textbf{input} & #2 \\
\textbf{question} & #3 \\
\hline
\end{tabular}
}
{
\vspace{1.5ex}
}

% function problem

\newenvironment{fctproblem}[3]
{
\vspace{2.5ex}
\noindent
\begin{tabular}{p{13mm} @{\textbf{:} } p{110mm}}
\hline
\multicolumn{2}{l}{\noindent {\sc #1}} \\
\hline
\textbf{input} & #2 \\
\textbf{output} & #3 \\
\hline
\end{tabular}
}
{
\vspace{1.5ex}
}

% use:
% \begin{problem}
% {name of problem}
% {input of problem}
% {output of problem}
% \end{problem}


% Macro created by Bernhard von Stengel
% MACROS FOR CREATING bimatrix games
% registers allocated once only
\newcount\rows
\newcount\cols
\newcount\rowcoord
\newcount\colcoord
\newcount\m
\newcount\n
% the crucial variable-length-parameter macro \dosth
\def\dosth#1{\ifx###1##\else\dofirst#1\anytoken\fi}
\def\doagain#1\anytoken{\dosth{#1}}
% example of \dofirst
% \def\dofirst#1{{$\langle#1\rangle$}\doagain}
% example of \dosth
% \dosth{1234x{x^3}y}
\def\payoffpairs#1#2#3{\m=#1\multiply\m by 4 \advance\m by -1 \n=1
  \def\dofirst##1{\put(\n,-\m){\makebox(0,0){\strut##1}}\advance\n by 4 \doagain}%
  \dosth{#2\strut}%
  \m=#1\multiply\m by 4 \advance\m by -3 \n=3 \dosth{#3\strut}}
\def\singlepayoffs#1#2{\m=#1\multiply\m by 4 \advance\m by -2 \n=2
  \def\dofirst##1{\put(\n,-\m){\makebox(0,0){\strut##1}}\advance\n by 4 \doagain}%
  {\large\dosth{#2\strut}}}
% the bimatrix game command
\newcommand{\bimatrixgame}[8]{%
\setlength{\unitlength}{#1}%
\rows=#2
\cols=#3
\rowcoord=\rows
\colcoord=\cols
\multiply\rowcoord by 4
\multiply\colcoord by 4
\m=\rowcoord
\n=\colcoord
\advance\m by 2 % 2 units left of payoff table
\advance\n by 2 % 2 units above payoff table
\begin{picture}(\n,\m)(-2,-\rowcoord)
\m=\rows
\n=\cols
\advance\m by 1
\advance\n by 1
\thinlines
\multiput(0,0)(0,-4){\m}{\line(1,0){\colcoord}}
\multiput(0,0)(4,0){\n}{\line(0,-1){\rowcoord}}
\put(0,0){\line(-1,1){2}}
\put(-1.5,0.5){\makebox(0,0)[r]{#4}}  % name player I
\put(-.7,1.7){\makebox(0,0)[l]{#5}}   % name player II
%row annotations - even with long strategy names, stick out to the left
\n=2
\def\dofirst##1{\put(-0.8,-\n){\makebox(0,0)[r]{\strut##1}}\advance\n by 4
   \doagain}
\dosth{#6\strut}
%column annotations
\n=2
\def\dofirst##1{\put(\n,1.0){\makebox(0,0){\strut##1}}\advance\n by 4
   \doagain}
\dosth{#7\strut}#8%
\end{picture}}
% example usage:
% \def\mm#1{\makebox(0,0){\strut#1}}%
%
% \bimatrixgame{4mm}{3}{4}{I}{II}{TMB}{lcr{\it out}}
% {
% \payoffpairs{1}{00{$a^2$}0}{1{\fbox 3}{\fbox 3}2}
% \payoffpairs{2}{0000}{1111}
% \singlepayoffs{3}{5555}
% % \multiput(0,-0.13)(.16,-.16){75}{\tiny.}
% \put(10,-2){\mm{*}}
% }


%% SHORTCUTS

%% numerical sets
\def\reals{{\mathbb R}}
\def\naturals{{\mathbb N}}

%% vectors and polytopes
\def\T{^{\top}}
\def\0{{\bf 0}}
\def\1{{\bf 1}}
\def\conv{{\rm conv}}

%% lemke-howson for gale
\def\tdot{$\cdot$}
\def\u1{${\bf \overline{1}}$}
\def\d1{${\bf \underline{1}}$}

%% useful
\def\gale{{\sc{Gale}}}
\def\anothergale{{\sc{Another Gale}}}

%%% END PREAMBLE

\begin{document}

\begin{center}
\hspace{15cm}

{\Large\bf Complexity of the Gale String Problem for Equilibrium Computation
in Games}

\vspace*{4cm}

{\large Marta Maria Casetti}

\vspace*{7cm}

{\small Thesis submitted to the Department of Mathematics\\
London School of Economics and Political Science\\
for the degree of Master of Philosophy}

\vspace*{3.5cm}

{\small London, April 2015}

\end{center}

\newpage

\section*{Declaration}

I certify that chapter \ref{main-chapter} of this thesis I have presented
for examination for the MPhil degree of the London School of Economics and
Political Science is based on joint work with Julian Merschen and Bernhard
von Stengel, published in \cite{main}.

The copyright of this thesis rests with the author. Quotation from it is
permitted, provided that full acknowledgement is made. This thesis may not
be reproduced without my prior written consent.

I warrant that this authorisation does not, to the best of my belief,
infringe the rights of any third party.


\newpage

\section*{Abstract}

This thesis presents a report on original research, published as joint
work with Merschen and von Stengel in {\em Electronic Notes in Discrete
Mathematics}~\cite{main}. Our result shows a polynomial time
algorithm to solve two problems related to labeled Gale strings, a
combinatorial structure consisting a string of labels and a bitstring
satisfying certain conditions introduced by Gale \mbox{in \cite{gale}.}

Gale strings can be used in the representation of a particular class
of games that Savani and von Stengel~\cite{svs} used as an example
of exponential running time for the classical \mbox{Lemke-Howson}
algorithm to find a Nash equilibrium of a bimatrix game~\cite{lh}.
It was conjectured that solving these games via the Lemke-Howson
algorithm was complete in the class {\bf PPAD} (Proof by Parity
Argument, Directed version). A major motivation for the definition of
this class by Papadimitriou~\cite{ppad} was, in turn, to capture
the pivoting technique of many results related to the Nash equilibrium,
including the Lemke-Howson algorithm.

Our result, on the contrary, sets apart this class of games as a case
for which there is a polynomial-time algorithm to find a Nash equilibrium.
Since Daskalakis, Goldberg and Papaditrimiou~\cite{dgp} and
Chen and Deng~\cite{cd} proved the \mbox{{\bf PPAD}-completeness} of
finding a Nash equilibrium in general normal-form games, we have a special
class of games, unless \mbox{{\bf PPAD} = {\bf P}}.

Our proof exploits two results. The first one is the representation
of the Nash equilibria of these games as Gale strings, as seen in Savani
and von Stengel~\cite{svs}. The second one is the polynomial-time
solvability of the problem of finding a perfect matching in a graph, proven
by Edmonds~\cite{edm}.

Merschen~\cite{jm} and V\'{e}gh and von Stengel~\cite{vvs} expanded our
technique to prove further interesting results.


\tableofcontents

\listoffigures

\chapter*{Introduction}


The topic of this thesis is a problem in the field of
{\em algorithmic game theory}, that is, the study of game-theoretic problems
from the point of view of computer science. In particular, we focus on the
computational complexity of a particular class of games. These

General refs for comp compl \cite{papad-cc}

General refs for geometry \cite{ziegler}


\chapter{Polytopes, Games, Complexity, Labels and Gale Strings}


\subsection{Some Geometrical Notation}

\todo[inline]{so results in labels section - after this - don't get lost in
boredom, and each section in background is about 150-200 lines (see: getting
lost).

Maybe turn this into an appendix? It would make sense if something more about
proof of Nash}

We denote the transpose of a matrix $A$ as $A\T$.
We consider vectors $u,v\in\reals^d$ as column vectors, so $u\T v$ is
their scalar product. A vector in $\reals^d$ for which all components
are $0$'s will be denoted as $\0$; similarly, a vectors for which all
components are $1$'s will be denoted as $\1$.
The {\em unit vector} $e_i$ is the vector that has $i$-th component
\smash{${e_i}_i = 1$} and \smash{${e_i}_j=0$} for all other components.
When writing an inequality of the form $u\geq v$ (and analogous), we mean
that it holds for every component; that is, $u_i\geq v_i$ for all
$i\in [d]$.

An {\em affine combination} of points in an Euclidean space $z_1,\ldots,z_n$
is
\[
\sum_{i=1}^n \lambda_i z_i \quad \text{where }\lambda_i\in\reals
\text{ such that }\sum_{i=1}^n \lambda_i = 1
\]

The points $z_1,\ldots,z_n$ are {\em affinely independent} if none of them
is an affine combination of the others.

A {\em convex combination} of points $z_1,\ldots,z_n$ is an affine
combination where $\lambda_i\geq 0$ for all $i\in [n]$.
Note that such $\lambda_i$'s can be seen as a probability distribution over
the $z_i$'s.

\todo[inline]{def simplex: here?}

A set of point $Z$ is {\em convex} if it is closed under forming convex
combinations, that is, if $\bar{z}=\sum_{i=1}^n \lambda_i z_i$,
where $z_i\in Z$, $\lambda_i\geq 0$ and $\sum_{i=1}^n \lambda_i = 1$,
then $\bar{z}\in Z$. A convex set has {\em dimension} $d$ if it has exactly
$d + 1$ affinely independent points.


\todo[inline]{def simplex: here?}


\todo[inline]{convex hull (needed for def cyclic poly);

pow hyperplanes;

polyhedron, polytopes}



\todo[inline]{from here: notes - copy-paste}

A ($d$-dimensional) {\em simplicial polytope} $P$ is the convex hull of a set
of at least $d+1$ points $v$ in $\reals^d$ in general position, that is, no
$d+1$ of them are on a common hyperplane.

If a point $v$ cannot be omitted from these points without changing $P$ then
$v$ is called a {\em vertex} of $P$. A {\em facet} of $P$ is the convex hull
$\conv\,F$ of a set $F$ of $d$ vertices of $P$ that lie on a hyperplane
$\{ x\in \reals^d\mid a^T x=a_0\}$ so that $a^T u<a_0$ for all other vertices
$u$ of $P$; the vector $a$ (unique up to a scalar multiple) is called the
{\em normal vector} of the facet. We often identify the facet with its set of
vertices~$F$.

\newpage

\subsection{Bimatrix Games, Labels and Polytopes}

In the rest of this thesis we will focus on two-player normal-form games.
For sake of readability, we will use feminine pronouns when referring to
player 1 and masculine pronouns when referring to player 2.

Two-player normal-form games are
also called {\em bimatrix games}, since they can be characterized by the
$m \times n$ payoff matrices $A$ and $B$, where $a_{ij}$ and $b_{ij}$ are
the payoffs of player 1 and 2 when she plays her $i$th pure strategy
and he plays his $j$th pure strategy.
We will assume that $(A,B)$ are non-negative, and that $A$ and $B\T$ have
no zero column. This can be easily obtained without loss of generality via
an affine transformation that will not affect the equilibria of the game.

The Nash equilibria of bimatrix games can be analysed from a combinatorial
point of view using {\em labels}. This method is due to Shapley
\cite{shapley}, in a study building on ideas introduced in a
paper by Lemke and Howson \cite{lh}.


Let $(A,B)$ be bimatrix game. The mixed-strategy simplices of player 1 and 2
are, respectively

\begin{equation}
X = \{ x\in\reals^m | x\geq\0,\ \1\T x = 1 \},\quad
Y = \{ y\in\reals^n | y\geq\0,\ \1\T y = 1 \}
\end{equation}

A {\em labeling} of the game is then given as follows:

\begin{enumerate}
\item the $m$ pure strategies of player 1 are identified by $1,\ldots,m$;
\item the $n$ pure strategies of player 2 are identified by $m+1,\ldots,m+n$;
\item each mixed strategy $x\in X$ of player 1 has
    \begin{itemize}
    \item label $i$ for each $i\in [m]$ such that $x_i = 0$, that is if in
    $x$ player 1 does not play her $i$th pure strategy;
    \item label $m + j$ for each $j\in [n]$ such that the $j$th pure strategy
    of player 2 is a best response to $x$;
    \end{itemize}
\item each mixed strategy $y\in Y$ of player 2 has
    \begin{itemize}
    \item label $m + j$ for each $j\in [n]$ such that $y_j = 0$, that is if in
    $y$ player 2 does not play his $j$th pure strategy;
    \item label $i$ for each $i\in [m]$ such that the $i$th pure strategy
    of player 1 is a best response to $y$;
    \end{itemize}
\end{enumerate}

A strategy profile $(x,y)\in X\times Y$ is {\em completely labeled} if every
label $1,\ldots,m+n$ is a label of either $x$ or $y$. We have the following
theorem (Theorem 1 in \cite{shapley}):

\begin{theorem}\label{comp-label-bimatrix-thm}
Let $(x,y)\in X\times Y$; then $(x,y)$ is a Nash equilibrium of the bimatrix
game $(A,B)$ if and only if $(x,y)$ is completely labeled.

\begin{proof}
The mixed strategy $x\in X$ has label $m + j$ for some $j\in [n]$ if and
only if the $j$th pure strategy of player 2 is a best response to $x$; this,
in turn, is a necessary and sufficient condition for player 2 to play his
$j$th strategy at an equilibrium against $x$. Therefore, at an equilibrium
$(x,y)$ all labels $m + 1,\ldots,m + n$ will appear either as labels of
$x$ or of $y$. The analogous holds for the labels $i\in [n]$.
\end{proof}
\end{theorem}

An useful geometrical representation of labels can be given on the mixed
strategies simplices $X$ and $Y$. The outside of each simplex is
labeled according to the player's own pure strategies that are {\em not}
played; so, for instance, the outside of $X$ will have labels
$1,\ldots,n$. The interior of each simplex is subdivided in closed polyhedral
sets, called {\em best-response regions}. These are labeled according to
the other player's pure strategy that is a best response in that set;
so, for instance, the inside of $X$ will have labels $m + 1,\ldots,m + n$.

We give an example of this construction.

\begin{example}

\todo[inline]{page 3--4 of Savani, von Stengel, Unit Vector Games.

With graphics.}

\end{example}

We will now give a description of labeling on polytopes equivalent to
the construction based on best-response regions.

We begin by noticing that the best-response regions can be obtained as
projections on $X$ and $Y$ of the {\em best-response facets} of
the polyhedra

\begin{equation}\label{br-polyhedron}
\bar{P} = \{ (x,v)\in X\times\reals | B\T x\leq\1 v \},\quad
\bar{Q} = \{ (y,u)\in Y\times\reals | A y\leq\1 u \}.
\end{equation}

These facets in $\bar{P}$ are defined as the points $(x,v)\in X\times\reals$
such that $(B\T x)_j = v$. These points represent the strategies $x\in X$ of
player 1 that give exactly payoff $v$ to player 2 when he plays strategy $j$.
The projection of the facet defined by $(B\T x)_j = v$ to $X$ will have
label $j$. Analogously, in $\bar{Q}$, the facets are the points
$(y,u)\in Y\times\reals$ such that $a_i y = u$, and their projection to $Y$
will be the best-response region with label $i$.

\begin{example}

\todo[inline]{cont of ex above, page 4--5, image on page 5 left}

\end{example}

Given the assumptions on non-negativity of $A$ and $B\T$, we can give a
change coordinates to $x_i / v$ and $y_j / u$ and replace $\bar{P}$ and
$\bar{Q}$ with the {\em best-response polytopes}

\begin{equation}\label{br-polytopes}
P = \{ x\in\reals^m | x\geq\0,\ B\T x\leq\1 \},\quad
Q = \{ y\in\reals^n | y\geq\0,\ A y\leq\1 \},\quad
\end{equation}

Each one of these polytope is defined by half spaces corresponding to
either the player's own strategy that is not being played or the other
player's best response; each one of the facets of the polytope is labeled
by the strategy corresponding to the relative half-space.

This means that a point in $P$ has label $k$ if and only if either
$x_k = 0$ for $k\in \{ 1,\ldots,m \}$ or $(B\T x)_{k - m} = 0$ for
$k\in \{ m+1,\ldots,m+n \}$;
analogously, a point in $Q$ has label $k$ if and only if either
$y_{k - m} = 0$ for $k\in \{ m+1,\ldots,m+n \}$ or $(A y)_{k}$ for
$k\in \{ m+1,\ldots,m+n \}$. A point $(x,y)\in P\times Q$ is
{\em completely labeled} if every $k\in [m + n]$ is a label of $x$ or $y$.
Note that the point $(\0,\0)$ is completely labeled. Rescaling back to
$\bar{P}$ and $\bar{Q}$, all the non-zero completely labeled points give
exactly all the equilibria of $(A,B)$. In this construction, we will
call $(\0,\0)$ {\em artificial equilibrium}.

\begin{example}

\todo[inline]{ex in Savani, von Stengel, image on page 5 right}

\end{example}

A characterization of the completely labeled pairs in $P\times Q$ can be
given as follows.

\begin{proposition}\label{compl-orth-cond}
The pair $(x,y)\in P\times Q$ is completely labeled if and only if one of
the following condition holds:
\begin{itemize}
\item {\em (Complementarity condition)}

\begin{equation}
x_i = 0\text{ or }(Ay)_i = 1\text{ for all }i\in [m],\quad
y_j = 0\text{ or }(B\T x)_j\text{ for all }j\in [n]
\end{equation}

\item {\em (Orthogonality condition)}

\begin{equation}
x\T (\1 - Ay) = 0,\quad
y\T (\1 - B\T x) = 0
\end{equation}
\end{itemize}
\end{proposition}

Proposition \ref{compl-orth-cond} can be used to prove a useful property:
{\em symmetric games}, that is, games that have payoff matrix of the form
$(C,C\T)$ for some matrix $C$, can be used to study generic bimatrix games
without loss of generality. The result is due to Gale, Kuhn and Tucker
\cite{gale-kuhn-tucker} for zero-sum games; its extension to non-zero-sum
games is a folklore result.

\begin{proposition}
Let $(A,B)$ be a bimatrix game and $(x,y)$ be one of its Nash equilibria.
Then $(z,z)$, where $z=(x,y)$, is a Nash equilibrium of the symmetric game
$(C,C\T)$, where

\[
C = \left(
    \begin{array}{cc}
    0 & A \\
    B\T & 0
    \end{array}
    \right).
\]
\end{proposition}

The converse has been proved by McLennan and Tourky \cite{mclennan-tourky} in
their study of {\em imitiation games}, that is, bimatrix games of the form
$(I,B)$.

\begin{proposition}\label{imitation-thm}
The pair $(x,x)$ is a symmetric Nash equilibrium of the symmetric bimatrix
game $(C,C\T)$ if and only if there is some $y$ such that $(x,y)$ is a
Nash equilibrium of the imitation game $(I,C\T)$.
\end{proposition}

\begin{example}
Consider the symmetric game $(C,C\T)$, where $C\T = B$ in the previous
examples.

\todo[inline]{ex Savani, von Stengel, pg 8}

\end{example}

Balthasar \cite{balthasar} and V\'{e}gh and von Stengel \cite{vvs} extended
proposition \ref{imitation-thm} to {\em unit vector games}, that is, games
of the form $(U,B)$, where the columns of the matrix $U$ are unit vectors.
The form of the theorem that we give here follows the version in Savani
and von Stengel \cite{svs-15}, dual to the one in Balthasar \cite{balthasar}.

\begin{theorem}\label{unit-vector-thm}
Let $l:[n]\to [m]$, and let $(U,B)$ be the unit vector game where
$U=(e_{l(1)}\ \cdots\ e_{l(n)})$. Consider the polytopes $P^l$ and $Q^l$
where

\begin{equation}
P^l = \{ x\in\reals^m | x\geq\0,\ B\T x\leq\1 \}
\end{equation}

\begin{equation}
Q^l = \{ y\in\reals^n | y\geq\0,\
\sum_{\substack{j\in N_i \\ i\in [m]}} y_j\leq 1 \}
\end{equation}

where $N_i = \{ j\in [n] | l(j)=i \}$ for $i\in [m]$.

Label every facet of $P^l$ according to the inequality defining it,
as follows:

\begin{itemize}
\item $x_i\geq 0$ has label $i$, for $i\in [m]$
\item $(B\T x)_j \leq 1$ has label $l(j)$, for $j\in [n]$
\end{itemize}

Then $x\in P^l$ is a completely labeled point of $P^l\setminus\{\0\}$
if and only if there is some $y\in Q^l$ such that, after scaling,
the pair $(x,y)$ is a Nash equilibrium of $(U,B)$

\begin{proof}
Let $P,Q$ be the polytopes associated to the game $(U,B)$ as before.

Let $(x,y)\in P\times Q\setminus\{ \0,\0 \}$ be a Nash equilibrium of
$(U,B)$, therefore completely labeled in $[m + n]$.
Then, if $x_i=0$, then $x$ has label $i\in m$.
If $x_i > 0$ instead, then $y$ has label $i$, therefore $(Uy)_i = 1$,
therefore for some $j\in [n]$ we have $y_j > 0$ and $U_j = e_i$, so $l(j)=i$.
Since $y_j > 0$ and $(x,y)$ is completely labeled, $x\in P$ has label $m+j$,
that is, $(B\T x)_j = 1$, therefore $x\in P^l$ has label $l(j) = i$.
Hence, $x$ is a completely labeled point of $P^l$.

Conversely, let $x\in P^l\setminus \{ \0 \}$ be completely labeled.
If $x_i > 0$, then there is $j\in [m]$ such that $(B\T x) = j$ and
$l(j) = i$, that is, $j\in N_i$. For all $i$ such that $x_i >0 $,
define $y$ as follows: $y_h = 0$ for all $h\in N_i\setminus \{ j \}$,
$y_j = 1$. Then $(x,y)\in P\times Q$ is completely labeled.
\end{proof}
\end{theorem}


\todo[inline]{
nondegeneracy; made nondegenerate by ``lexicographic'' perurbation
(what does it mean?);

ex pg 9; odd no eq, mention homotopy method (find ref)
(tie with Nash, again?)
}


\section{Normal Form Games and Nash Equilibria}

We now give the game-theoretic background that will be used in this thesis.
A {\em game}, as first defined by von Neumann in \cite{vnm28}, is a model
of strategic interaction. A {\em finite normal form game} is
$\Gamma=(P,S,u)$ with $S=\times_{p\in P} S_p$ and $u=\times_{p\in P}u^p$
where both the set of {\em players} $P$ and the set of
{\em pure strategies} $S_p$ for every player $p\in P$, and therefore the set
of {\em pure strategy profiles} $S$, are finite. We will use the
notation $S_{-p}=\times{q\neq p}S_p$.
The purpose of each player $p\in P$ is to maximize her {\em payoff function}
$u^p:S\to\reals$. In the following pages, by ``game'' we will always mean
``finite normal form game.'' If there are only two players, we will refer to
player 1 using feminine pronouns and to player 2 using masculine ones.
Such games are called {\em bimatrix games} since they can be characterized by
the $m \times n$ payoff matrices $A$ and $B$, where $a_{ij}$ and $b_{ij}$ are
the payoffs of player 1 and 2 when the former plays her $i$th pure strategy
and the latter plays his $j$th pure strategy. A bimatrix game is
{\em zero-sum} if $B=-A$, and {\em symmetric} if  $B=A\T$.

A {\em mixed strategy} of player $p$ is a probability
distribution on $S_P$; it can be described as a point
$x=(x^p_1,\ldots,x^p_{|S_p|})$ on the
$(|S_p|-1)$-dimensional {\em mixed strategy simplex} $\Delta_p$;
the {\em mixed strategy profiles} will be the simplicial polytope
$\Delta=\times_{p\in P}\Delta_p$.
We extend the payoff functions as $u^p:\Delta\to\reals$ by linearity.

A {\em Nash equilibrium} of a game is a strategy profile in which each
player cannot improve his expected payoff by unilaterally changing her
strategy. Such a strategy is called {\em best response strategy} (or simply
{\em best response}); a strategy that is not a best response is called
{\em dominated}.
Formally: for $s\in S_{-p}$ let $x_s=\prod_{q\neq p} x^q_{s_q}$;
then a Nash equilibrium is a strategy profile
$x$ such that for every $p\in P$ and every $\sigma,\tau\in S_p$
\begin{equation}
\sum_{s\in S_{-p}} u^p(\sigma,s) x_s > \sum_{s\in S_{-p}} u^p(\tau,s) x_s
\quad
\Rightarrow
\quad
x_\tau^p = 0
\end{equation}

The existence of a Nash equilibrium is guaranteed by the fundamental theorem
by Nash (\cite{nash}).

\begin{theorem}\label{nash-thm}
Every finite game in normal form has a Nash equilibrium.
\end{theorem}

We give a classic example of game: matching pennies.

\begin{example}
In the $n\times n$ bimatrix game {\em generalized matching pennies}, both
players have payoff zero unless they pay the same strategy. In this case,
player 1 (the {\em pursuer}) has payoff $1$ and player 2 (the \emph{evader}),
has payoff $-1$. If $n=2$ the game (called simply {\em matching pennies}) is

\begin{center}
\def\mm#1{\makebox(0,0){\strut#1}}
\bimatrixgame{3mm}{2}{2}{{{\scriptsize pursuer}}}{{{\scriptsize evader}}}
{{{\scriptsize head}}{{\scriptsize tail}}}
{{\scriptsize head}{\scriptsize tail}}
{
\payoffpairs{1}{{1}0}{{-1}0}
\payoffpairs{2}{0{1}}{0{-1}}
}
\end{center}

At the unique equilibrium of generalised matching pennies, each player
follows the uniform distribution over their strategies.
\end{example}


Consider the problem $n$-{\sc Nash}, as follows.

\begin{fctproblem}
{$n$-Nash}
{A $n$-player game $\Gamma$.}
{A Nash equilibrium of $\Gamma$.}
\end{fctproblem}

By theorem \ref{nash-thm}, {\sc $n$-Nash} is a total function
problem; Megiddo and Papadimitriou (\cite{megiddo-papad}) proved that it is
in {\bf TFNP}. Daskalakis, Goldberg and Papadimitriou \cite{dgp} and Chen
and Deng \cite{cd} have proven its {\bf PPAD}-completeness, the former for
$n\geq 3$ and the latter for $n\geq 2$.

\begin{theorem}\label{nash-ppad-complete}
For $n\geq 2$, the problem {\sc $n$-Nash} is {\bf PPAD}-complete.
\end{theorem}


\section{Some Complexity Classes}

A {\em Turing machine} $\mathcal{M}$ is a representation of a
{\em program} that takes an {\em input}, runs a {\em program}
manipulating the input, and either does not come to a
{\em halting state} or it returns an {\em output}; the latter can
be {\sc Yes} (in which case the Turing machine {\em accepts} the
input), {\sc No} (the Turing machine {\em rejects} the input), or a
string $\mathcal{M}(x)$. Formally, $\mathcal{M}=(K,\Sigma,\delta,s)$.
The finite set $K$ is the set of {\em states}; $\Sigma$ is a finite set of
{\em symbols} (the {\em alphabet} of $\mathcal{M}$) such that
$\Sigma\cap K=\varnothing$, and $\Sigma$ always contains the symbols
$\sqcup$ ({\em blank}) and $\rhd$ ({\em first symbol}); $\delta$ is the
{\em transition function}
\[
\delta:K\times \Sigma\longrightarrow (K\cup\{ h, {\sc Yes}, {\sc No} \})%
\times \Sigma \times \{ \leftarrow,\rightarrow,- \}
\]
where $h$ is the {\em halting state}, {\sc Yes} is the {\em accepting state},
{\em No} is the {\em rejecting state}, and
$\{ \leftarrow,\rightarrow,- \}\nsubseteq(K\cup\Sigma)$ correspond to
the {\em cursor directions} ``left,'' ``right'' and ``stay.''

A {\em language} is a set of strings of symbols
$L\subset(\Sigma\setminus\{ \sqcup \})^\ast$; a Turing machine $\mathcal{M}$
{\em decides} $L$ if for every $x\in (\Sigma\setminus\{ \sqcup \})^\ast$
$\mathcal{M}(x)={\sc Yes}$ if $x\in L$ and
$\mathcal{M}(x)={\sc No}$ if $x\notin L$.
We say that $\mathcal{M}$
{\em accepts} $L$ if for every $x\in (\Sigma\setminus\{ \sqcup \})^\ast$
$\mathcal{M}(x)={\sc Yes}$ if $x\in L$ and
$\mathcal{M}(x)$ does not halt if $x\notin L$.
Given a function $f:(\Sigma\setminus\{ \sqcup \})^\ast\to\Sigma^\ast$,
we say that $\mathcal{M}$ {\em computes} $f$ if for every
$x\in (\Sigma\setminus\{ \sqcup \})^\ast$ $\mathcal{M}(x)=f(x)$.

Given a problem, a specific input of a problem is called an {\em instance}.
The output of a {\em decision problem} $P$ is either {\sc Yes} or {\sc No}.
Its {\em complement} is the problem $\bar{P}$ that outputs ``{\sc No}''
for each instance of $P$ that outputs ``{\sc Yes}'', and vice versa.
A {\em function problem} outputs a more generic $y$ that satisfies
a binary relation $R(x,y)$, where $x$ is the instance of the problem.
{\em Search problems} are function problems that return either $y$ such
that $R(x,y)$, or ``{\sc No}'' if it's not possible to find any such $y$.
If $y$ is guaranteed to exist, the problem is called a
{\em total function problem}.

An example of decision problem is: ``(input) given a graph, (question)
is it possible to find an Euler tour of the graph?'' Its complement
is ``(input) given a graph, (question) is it possible that there isn't
any Euler of the graph?'' A search problem is: ``(input) given a graph,
(output) return one Euler tour of the graph, or ``NO'' if no such tour
exists.'' A total function problem is: ``(input) given an Euler graph,
(output) return one of its Euler tours.''

Let $P_1$ be a problem and let $x$ be an instance of $P_1$ that is encoded
in $|x|$ bits. $P_1$ {\em reduces to the problem $P_2$ in polynomial time},
denoted $P_1\leq_P P_2$, if there exists a {\em polynomial-time reduction},
that is, a function $f: \{0,1\}^\ast \to \{0,1\}^\ast$ and a Turing machine
$\mathcal{M}$ such that for all $x\in\{0,1\}^\ast$
\begin{enumerate}
\item $x\in P_1\quad\iff\quad f(x)\in P_2$;
\item $\mathcal{M}$ computes $f(x)$;
\item $\mathcal{M}$ stops after $p(|x|)$ steps, where $p$ is a polynomial.
\end{enumerate}

Intuitively, if $P_1$ is polynomial-time reducible to $P_2$, it takes
polynomial time to ``translate'' $P_1$ to $P_2$, and then to
``translate back'' a solution of $P_2$ as a solution of $P_1$.
This is particularly useful if $P_2$ is ``difficult to solve''; then
the problem $P_1$ is at least as ``difficult.''

A {\em class} is a set of languages.
For any class $\mathrm{C}$ of decision problems, the class of all complements
of the problems in $\mathrm{C}$ is the {\em complement class}
$\mathrm{co-C}$. A problem $P$ is {\em hard} for a class $\mathrm{C}$
if for every problem $P_{\mathrm{C}}$ in $\mathrm{C}$ there is a
polynomial-time reduction to $P$; that is,
if $P$ is hard to solve at least as every problem in $\mathrm{C}$. A
$\mathrm{C}-hard$ problem in $\mathrm{C}$ is {\em complete} for
$\mathrm{C}$.

The complexity class $\mathrm{\mathbf{P}}$ contains all the
{\em polynomially decidable problems}; that is, all problems $P$ such that
there exists a Turing machine $\mathcal{M}$ that outputs either {\sc Yes} or
{\sc No}  for all inputs $x\in\{0,1\}^\ast$ of $P$ after $p(|x|)$ steps,
where $p$ is a polynomial. Intuitively, a decision problem is in
$\mathrm{\mathbf{P}}$ if the answer to its question can be found in a
number of steps that is polynomial in the input of the problem.
A problem $P$ belongs to the class $\mathrm{\mathbf{NP}}$,
{\em non-deterministic polynomial-time problems}, if there exists a
Turing machine $\mathcal{M}$ and polynomials $p_1,p_2$ such that
\begin{enumerate}
\item for all $x\in P$ there exists a {\em certificate} $y\in \{0,1\}^\ast$
which satisfies $|y|\leq p_1(|x|)$;
\item $\mathcal{M}$ accepts the combined input $xy$, stopping after at most
$p_2(|x| + |y|)$ steps;
\item for all $x\notin P$ there does not exist $y\in \{0,1\}^\ast$ such
that $\mathcal{M}$ accepts the combined input $xy$.
\end{enumerate}
Intuitively, a decision problem is in $\mathrm{\mathbf{NP}}$ if it takes
polynomial time to verify whether the ``certificate solution'' $y$ is,
indeed, a correct answer to the question posed by the problem.
The class $\mathrm{\mathbf{\# P}}$ is the class of all problems that output
the number of possible certificates for a problem in $\mathrm{\mathbf{NP}}$.

In \cite{megiddo-papad}, Megiddo and Papadimitriou introduced the classes
$\mathrm{\mathbf{FNP}}$, {\em function non-deterministic polynomial}, and
$\mathrm{\mathbf{TFNP}}$, {\em total function non-deterministic polynomial}.
The former is defined as the class of binary relations $R(x,y)$ such that
there is a polynomial-time algorithm that decides $R(x,y)$
for $x,y$ such that $|y|\leq p(|x|)$, where $p$ is a polynomial. The
latter is the class of all such problems for which $y$ is guaranteed to
exist. Intuitively, $\mathrm{\mathbf{FNP}}$ and $\mathrm{\mathbf{TFNP}}$ are
similar to $\mathrm{\mathbf{NP}}$, but they allow for problems of
(respectively) function and total function form.
Also in \cite{megiddo-papad}, Megiddo and Papadimitriou proved that, unless
$\mathrm{\mathbf{NP}}=\mathrm{\mathbf{co-NP}}$, $\mathrm{\mathbf{TFNP}}$ is a
{\em semantic} class, that is, a class without complete problems.
To circumvent this limitation of $\mathrm{\mathbf{TFNP}}$, Papadimitriou
\cite{ppad} focused on the problems for which the existence of a solution
is proved by a ``parity argument'', introducing the classes
$\mathrm{\mathbf{PPA}}$ ({\em Proof by Parity Argument}) and
$\mathrm{\mathbf{PPAD}}$ ({\em Proof by Parity Argument, Directed version}).

Intuitively, {\bf PPA} is the class of problems for which the existence of
a solution can be proved using the argument ``in any undirected graph with
one odd-degree node there must be another odd-degree node.'' It is
interesting to note that no {\bf PPA} problems have been proven
{\bf PPA}-complete.
The existence of the problems in {\bf PPAD}, on the other hand, can be
proven using the argument ``in any directed graph with one unbalanced node
(that is, with outdegree different from its indegree) there must be another
unbalanced node.'' The latter can simplified without loss of generality or
computational power to the case of indegree and outdegree at most one,
that is, ``in any directed graph in which all vertices have indegree and
outdegree at most one, if there is a {\em source} (a node with indegree
zero), then there must be a {\em sink} (a node with outdegree zero).''
Formally, we can define {\bf PPAD} as the class of problems reducible to
the problem {\sc End Of The Line}.

\begin{fctproblem}
{End Of The Line}
{Two circuits $S$ and $P$ with $n$ input bits and $n$ output bits such that
$P(0^n)=0^n\neq S(0^n)$.}
{An input $x\in \{ 0,1 \}^n$ such that $P(S(x)\neq x)$ or
$S(P(x))\neq x\neq 0^n$}
\end{fctproblem}

This is the definition given in Daskalakis, Goldberg and Papadimitriou
\cite{dgp}; Papadimitriou \cite{ppad} defines the class in terms of Turing
machines.
A circuit is formally defined as a directed acyclic graph with $n$
vertices with indegree $0$ called {\em input nodes}, $m$ vertices with
outdegree $0$ called {\em output nodes}, and {\em internal nodes} with
indegree 1 or 2; when each input node receives an input in $\{ 0,1 \}$,
the internal nodes with indegree $2$ compute the Boolean functions $and$ or
$or$, the internal nodes with indegree $1$ compute the Boolean function
$not$ and each output node returns a value in $\{ 0,1 \}$ accordingly.
The problems in {\bf PPAD} can be seen as a circuit $S$ (``successor''),
and a circuit $P$ (``predecessor'') that are used to build a directed
graph with an edge $(x,y)$ if and only if $S(x)=y$ and $P(y)=x$, with a
{\em standard source} $0^n$; the output is either a sink or a
non-standard source. Problems in {\bf PPA} can be defined analogously,
such that the resulting graph is not directed and instead of sources and
sinks there are generic endpoints.
We have that
${\bf{\rm PPAD}}\subset{\bf{\rm PPA}}$, and there are {\bf PPA} problems
that are not {\bf PPAD}, as we will see in section \ref{lh-section}.
Figure \ref{ppad-graph} presents an example of graph of a {\bf PPAD}
problem; a graph for a {\bf PPA} problem is analogous, but undirected.

\begin{figure}[hbt]
\strut\hfill
\includegraphics[width=70ex]{chapter-1/fig/PPAD.pdf}%
\hfill\strut
\caption[A PPAD problem]{%
In green, the circuit $S$; in red, the circuit $P$. The standard source
is the black node; the green nodes are the other sources; the red nodes are
the sinks. The graph can include paths, cycles and isolated points.
}
\label{ppad-graph}
\end{figure}

By theorem \ref{nash-thm}, the problem $n$-{\sc Nash}, defined as follows,
is a total function problem.

\begin{fctproblem}
{$n$-Nash}
{A $n$-player game.}
{A Nash equilibrium of the game.}
\end{fctproblem}

Megiddo and Papadimitriou (\cite{megiddo-papad}) proved that it is
in {\bf TFNP}. Daskalakis, Goldberg and Papadimitriou \cite{dgp} and Chen
and Deng \cite{cd} have proven its {\bf PPAD}-completeness, the former for
$n\geq 3$ and the latter for $n\geq 2$.
A small amendment of the proof in \cite{dgp} can be found in Casetti
\cite{msc-diss}.

\begin{theorem}{\rm (Daskalakis, Goldberg and Papadimitriou \cite{dgp};
Chen and Deng \cite{cd})}\label{nash-ppad-complete}
For $n\geq 2$, the problem {\sc $n$-Nash} is {\bf PPAD}-complete.
\end{theorem}

We will see more problems in {\bf PPA} and {\bf PPAD} in chapter
\ref{main-chapter}; in fact, our main result can be seen as a
negative result on the {\bf PPAD} complexity of a case of $2$-Nash.


\section{Bimatrix Games and Labels}

In the rest of this thesis we will focus on two-player normal-form games.
For sake of readability, we will use feminine pronouns when referring to
player 1 and masculine pronouns when referring to player 2.

Two-player normal-form games are
also called {\em bimatrix games}, since they can be characterized by the
$m \times n$ payoff matrices $A$ and $B$, where $a_{ij}$ and $b_{ij}$ are
the payoffs of player 1 and 2 when she plays her $i$th pure strategy
and he plays his $j$th pure strategy.
We will assume that $(A,B)$ are non-negative, and that $A$ and $B\T$ have
no zero column. This can be easily obtained without loss of generality via
an affine transformation that will not affect the equilibria of the game.

The Nash equilibria of bimatrix games can be analysed from a combinatorial
point of view using {\em labels}. This method is due to Shapley
\cite{shapley}, in a study building on ideas introduced in a
paper by Lemke and Howson \cite{lh}.


Let $(A,B)$ be bimatrix game. The mixed-strategy simplices of player 1 and 2
are, respectively

\begin{equation}
X = \{ x\in\reals^m | x\geq\0,\ \1\T x = 1 \},\quad
Y = \{ y\in\reals^n | y\geq\0,\ \1\T y = 1 \}
\end{equation}

A {\em labeling} of the game is then given as follows:

\begin{enumerate}
\item the $m$ pure strategies of player 1 are identified by $1,\ldots,m$;
\item the $n$ pure strategies of player 2 are identified by $m+1,\ldots,m+n$;
\item each mixed strategy $x\in X$ of player 1 has
    \begin{itemize}
    \item label $i$ for each $i\in [m]$ such that $x_i = 0$, that is if in
    $x$ player 1 does not play her $i$th pure strategy;
    \item label $m + j$ for each $j\in [n]$ such that the $j$th pure strategy
    of player 2 is a best response to $x$;
    \end{itemize}
\item each mixed strategy $y\in Y$ of player 2 has
    \begin{itemize}
    \item label $m + j$ for each $j\in [n]$ such that $y_j = 0$, that is if in
    $y$ player 2 does not play his $j$th pure strategy;
    \item label $i$ for each $i\in [m]$ such that the $i$th pure strategy
    of player 1 is a best response to $y$;
    \end{itemize}
\end{enumerate}

A strategy profile $(x,y)\in X\times Y$ is {\em completely labeled} if every
label $1,\ldots,m+n$ is a label of either $x$ or $y$. We have the following
theorem (Theorem 1 in \cite{shapley}):

\begin{theorem}\label{comp-label-bimatrix-thm}
Let $(x,y)\in X\times Y$; then $(x,y)$ is a Nash equilibrium of the bimatrix
game $(A,B)$ if and only if $(x,y)$ is completely labeled.

\begin{proof}
The mixed strategy $x\in X$ has label $m + j$ for some $j\in [n]$ if and
only if the $j$th pure strategy of player 2 is a best response to $x$; this,
in turn, is a necessary and sufficient condition for player 2 to play his
$j$th strategy at an equilibrium against $x$. Therefore, at an equilibrium
$(x,y)$ all labels $m + 1,\ldots,m + n$ will appear either as labels of
$x$ or of $y$. The analogous holds for the labels $i\in [n]$.
\end{proof}
\end{theorem}

An useful geometrical representation of labels can be given on the mixed
strategies simplices $X$ and $Y$. The outside of each simplex is
labeled according to the player's own pure strategies that are {\em not}
played; so, for instance, the outside of $X$ will have labels
$1,\ldots,n$. The interior of each simplex is subdivided in closed polyhedral
sets, called {\em best-response regions}. These are labeled according to
the other player's pure strategy that is a best response in that set;
so, for instance, the inside of $X$ will have labels $m + 1,\ldots,m + n$.

We give an example of this construction.

\begin{example}

\todo[inline]{page 3--4 of Savani, von Stengel, Unit Vector Games.

With graphics.}

\end{example}

We will now give a description of labeling on polytopes equivalent to
the construction based on best-response regions.

We begin by noticing that the best-response regions can be obtained as
projections on $X$ and $Y$ of the {\em best-response facets} of
the polyhedra

\begin{equation}\label{br-polyhedron}
\bar{P} = \{ (x,v)\in X\times\reals | B\T x\leq\1 v \},\quad
\bar{Q} = \{ (y,u)\in Y\times\reals | A y\leq\1 u \}.
\end{equation}

These facets in $\bar{P}$ are defined as the points $(x,v)\in X\times\reals$
such that $(B\T x)_j = v$. These points represent the strategies $x\in X$ of
player 1 that give exactly payoff $v$ to player 2 when he plays strategy $j$.
The projection of the facet defined by $(B\T x)_j = v$ to $X$ will have
label $j$. Analogously, in $\bar{Q}$, the facets are the points
$(y,u)\in Y\times\reals$ such that $a_i y = u$, and their projection to $Y$
will be the best-response region with label $i$.

\begin{example}

\todo[inline]{cont of ex above, page 4--5, image on page 5 left}

\end{example}

Given the assumptions on non-negativity of $A$ and $B\T$, we can give a
change coordinates to $x_i / v$ and $y_j / u$ and replace $\bar{P}$ and
$\bar{Q}$ with the {\em best-response polytopes}

\begin{equation}\label{br-polytopes}
P = \{ x\in\reals^m | x\geq\0,\ B\T x\leq\1 \},\quad
Q = \{ y\in\reals^n | y\geq\0,\ A y\leq\1 \},\quad
\end{equation}

Each one of these polytope is defined by half spaces corresponding to
either the player's own strategy that is not being played or the other
player's best response; each one of the facets of the polytope is labeled
by the strategy corresponding to the relative half-space.

This means that a point in $P$ has label $k$ if and only if either
$x_k = 0$ for $k\in \{ 1,\ldots,m \}$ or $(B\T x)_{k - m} = 0$ for
$k\in \{ m+1,\ldots,m+n \}$;
analogously, a point in $Q$ has label $k$ if and only if either
$y_{k - m} = 0$ for $k\in \{ m+1,\ldots,m+n \}$ or $(A y)_{k}$ for
$k\in \{ m+1,\ldots,m+n \}$. A point $(x,y)\in P\times Q$ is
{\em completely labeled} if every $k\in [m + n]$ is a label of $x$ or $y$.
Note that the point $(\0,\0)$ is completely labeled. Rescaling back to
$\bar{P}$ and $\bar{Q}$, all the non-zero completely labeled points give
exactly all the equilibria of $(A,B)$. In this construction, we will
call $(\0,\0)$ {\em artificial equilibrium}.

\begin{example}

\todo[inline]{ex in Savani, von Stengel, image on page 5 right}

\end{example}

A characterization of the completely labeled pairs in $P\times Q$ can be
given as follows.

\begin{proposition}\label{compl-orth-cond}
The pair $(x,y)\in P\times Q$ is completely labeled if and only if one of
the following condition holds:
\begin{itemize}
\item {\em (Complementarity condition)}

\begin{equation}
x_i = 0\text{ or }(Ay)_i = 1\text{ for all }i\in [m],\quad
y_j = 0\text{ or }(B\T x)_j\text{ for all }j\in [n]
\end{equation}

\item {\em (Orthogonality condition)}

\begin{equation}
x\T (\1 - Ay) = 0,\quad
y\T (\1 - B\T x) = 0
\end{equation}
\end{itemize}
\end{proposition}

Proposition \ref{compl-orth-cond} can be used to prove a useful property:
{\em symmetric games}, that is, games that have payoff matrix of the form
$(C,C\T)$ for some matrix $C$, can be used to study generic bimatrix games
without loss of generality. The result is due to Gale, Kuhn and Tucker
\cite{gale-kuhn-tucker} for zero-sum games; its extension to non-zero-sum
games is a folklore result.

\begin{proposition}\label{symmetric-eq-thm}
Let $(A,B)$ be a bimatrix game and $(x,y)$ be one of its Nash equilibria.
Then $(z,z)$, where $z=(x,y)$, is a Nash equilibrium of the symmetric game
$(C,C\T)$, where

\[
C = \left(
    \begin{array}{cc}
    0 & A \\
    B\T & 0
    \end{array}
    \right).
\]
\end{proposition}

The converse has been proved by McLennan and Tourky \cite{mclennan-tourky} in
their study of {\em imitiation games}, that is, bimatrix games of the form
$(I,B)$.

\begin{proposition}\label{imitation-thm}
The pair $(x,x)$ is a symmetric Nash equilibrium of the symmetric bimatrix
game $(C,C\T)$ if and only if there is some $y$ such that $(x,y)$ is a
Nash equilibrium of the imitation game $(I,C\T)$.
\end{proposition}

\begin{example}
Consider the symmetric game $(C,C\T)$, where $C\T = B$ in the previous
examples.

\todo[inline]{ex Savani, von Stengel, pg 8}

\end{example}

Savani and von Stengel \cite{uvg} extended the study of imitation games to
{\em unit vector games} $(U,B)$, where the columns of the matrix $U$ are
unit vectors. For these games, the use of labeling in polytopes to
characterize Nash equilibria of the game is given by the following
theorem, first proved in dual form by Balthasar \cite{balthasar}.

\begin{theorem}\cite{uvg}\label{unit-vector-thm}
Let $l:[n]\to [m]$, and let $(U,B)$ be the unit vector game where
$U=(e_{l(1)}\ \cdots\ e_{l(n)})$. Consider the polytopes $P^l$ and $Q^l$
where

\begin{equation}\label{p-l-unitv}
P^l = \{ x\in\reals^m | x\geq\0,\ B\T x\leq\1 \}
\end{equation}

\begin{equation}
Q^l = \{ y\in\reals^n | y\geq\0,\
\sum_{\substack{j\in N_i \\ i\in [m]}} y_j\leq 1 \}
\end{equation}

where $N_i = \{ j\in [n] | l(j)=i \}$ for $i\in [m]$.

Give a labeling $l_f$ of the facets of $P^l$ according to the
inequality defining it, as follows:

\begin{eqnarray}\label{facet-labeling-unitv}
x_i\geq 0\text{ has label }i\text{ for }i\in [m] \\
(B\T x)_j \leq 1\text{ has label }l(j)\text{ for }j\in [n]
\end{eqnarray}

Then $x\in P^l$ is a completely labeled point of $P^l\setminus\{\0\}$
if and only if there is some $y\in Q^l$ such that, after scaling,
the pair $(x,y)$ is a Nash equilibrium of $(U,B)$

\begin{proof}
Let $P,Q$ be the polytopes associated to the game $(U,B)$ as before.

Let $(x,y)\in P\times Q\setminus\{ \0,\0 \}$ be a Nash equilibrium of
$(U,B)$, therefore completely labeled in $[m + n]$.
Then, if $x_i=0$, then $x$ has label $i\in m$.
If $x_i > 0$ instead, then $y$ has label $i$, therefore $(Uy)_i = 1$,
therefore for some $j\in [n]$ we have $y_j > 0$ and $U_j = e_i$, so $l(j)=i$.
Since $y_j > 0$ and $(x,y)$ is completely labeled, $x\in P$ has label $m+j$,
that is, $(B\T x)_j = 1$, therefore $x\in P^l$ has label $l(j) = i$.
Hence, $x$ is a completely labeled point of $P^l$.

Conversely, let $x\in P^l\setminus \{ \0 \}$ be completely labeled.
If $x_i > 0$, then there is $j\in [m]$ such that $(B\T x) = j$ and
$l(j) = i$, that is, $j\in N_i$. For all $i$ such that $x_i >0 $,
define $y$ as follows: $y_h = 0$ for all $h\in N_i\setminus \{ j \}$,
$y_j = 1$. Then $(x,y)\in P\times Q$ is completely labeled.
\end{proof}
\end{theorem}

Theorem \ref{unit-vector-thm} gives a correspondence between the
completely labeled vertices of the polytope $P^l$ and the equilibria
of the unit vector game $(U,B)$, with an ``artificial''
equilibrium corresponding to the vertex $\0$.

The dual version of theorem \ref{unit-vector-thm}, given in
\cite{balthasar}, is constructed as follows.
We translate the polytope $P^l$ in theorem \ref{unit-vector-thm} to
$P = \{ x - \1\ |\ x\in P^l \}$. Multiplying all payoffs in $B$ by a
constant if necessary (an operation that does not change the game),
we can have \1 is in the interior of $P^l$ and \0 in the interior of $P$.
We have $x\in P$ if and only if $x + \1\geq\0$ and
$B\T(x + \1)=(x + \1)\T B\leq\1$;
that is, if and only if $-x_i\leq 1$ for $i\in [m]$ and
$x\T \frac{b_j}{1 - \1\T b_j} \leq 1$ for $j\in [n]$.
The polar of $P$ is then
\begin{equation}
P^\Delta = \conv(\{−e_i\ |\ i\in [m] \} \cup \{ \frac{b_j}{1 - \1\T b_j} \})
\end{equation}

\todo[inline]{$P$ is simple (why? because game nondegenerate!
show it before), so $P^\Delta$ is simplicial}

Since \0 is in the interior of $P$, we have that $P^{\Delta\Delta}=P$,
and the facets of $P^\Delta$ correspond to
the vertices of $P$ and vice versa. We can then label the vertices
of $P\Delta$ with the labels of the corresponding facets in $P^l$, so
the completely labeled facets of $P\Delta$ will correspond to the
completely labeled vertices of $P^l$.
In particular, the facet corresponding to \0 is
\begin{equation}
F_0 = \{ x\in P^\Delta\ |\ -\1\T x = 1 \} = \conv\{ e_i\ |\ i\in [m] \}.
\end{equation}
Theorem \ref{unit-vector-thm} then translates as follows.

\begin{theorem}\cite{balthasar}\label{unit-vector-dual-thm}
Let $Q$ be a labeled $m$-dimensional simplicial polytope with \0 in
its interior, with vertices $e_1,\ldots,e_m,c_1,\ldots,c_n$, so that
$F_0 = \conv\{ e_i\ |\ i\in [m] \}$ is a facet of $Q$.

Let $l:[n]\to [m]$, and let $(U,B)$ be the unit vector game with
$U=(e_{l(1)}\ \cdots\ e_{l(n)})$ and $B = (b_1\ \cdots\ b_n)$,
where $b_j = \frac{c_j}{1 + \1\T c_j}$ for $j\in [n]$.

Label the vertices of $Q$ as follows:
\begin{eqnarray}\label{vert-labeling-unitv}
l_v(-e_i)=i\text{ for }i\in [m] \\
l_v(c_j)=l(j)\text{ for }j\in [n]
\end{eqnarray}

Then a facet $F\neq F_0$ of $Q$ with normal vector $v$ is completely
labeled if and only if $(x,y)$ is a Nash equilibrium of $(U,B)$, where
$x = \frac{v + \1}{\1\T (v + \1)}$,
and $x_i = 0$ if and only if $−e_i\in F$ for $i\in [m]$.
Any $j$ so that $c_j$ is a vertex of $F$ represents a pure best reply to $x$;
the mixed strategy $y$ is the uniform distribution on the set of the pure best
replies to $x$.
\end{theorem}

In theorem \ref{unit-vector-dual-thm} we have a correspondence between
the completely labeled facets of the polytope $Q$
(the completely labeled vertices of $P^l$ in theorem \ref{unit-vector-thm})
and the equilibria of the unit vector game $(U,B)$,
with the ``artificial'' equilibrium corresponding to the
facet $F_0$ (the vertex $\0$ in theorem \ref{unit-vector-thm}).
This proves the following proposition.

\begin{proposition}\label{ne-cl-pbl}
The problem 2-{\sc Nash} for unit vector games is
polynomial-time reducible
\todo{do we have to prove ``polynomial''?}
to the problems {\sc Another Completely Labeled Vertex} and its dual
{\sc Another Completely Labeled Facet}, where

\begin{fctproblem}
{\sc Another Completely Labeled Vertex}
{A simple $m$-dimensional polytope $S$ with $m+n$ facets;
a labeling $l:[m+n]\to [n]$;
a facet $F_0$ of $S$, completely labeled by $l$.}
{A facet $F\neq F_0$ of $S$, completely labeled by $l$.}
\end{fctproblem}

\begin{fctproblem}
{\sc Another Completely Labeled Vertex}
{A simplicial $m$-dimensional polytope $S$ with $m+n$ vertices;
a labeling $l:[m+n]\to [n]$;
a vertex $v_0$ of $S$, completely labeled by $l$.}
{A vertex $v\neq v_0$ of $S$, completely labeled by $l$.}
\end{fctproblem}

\end{proposition}

\todo[inline]{
nondegeneracy; made nondegenerate by ``lexicographic'' perurbation
(what does it mean?);

nondegeneracy -> br polytope $P$ is simple -> $P^\Delta$ is simplicial

so: this goes before!

ex pg 9; odd no eq, mention homotopy method (find ref)
(tie with Nash, again?)
}


\subsection{Cyclic Polytopes and Gale Strings}\label{gs-ssect}

A special case of games is obtained by taking a particular case of best
response polytope in theorem \ref{t-unitv}.

A {\em cyclic polytope} $P$ in dimension~$d$ with $n$ vertices is the
convex hull of distinct points $\mu(t_j)$, where $j\in [n]$ and $\mu$ is the
{\em moment curve}
\[
\mu\colon t\mapsto(t,t^2,\ldots,t^d)^\top
\]

Restricting the study of best response polytopes to the case of cyclic
polytopes gives an interesting case, since cyclic polytopes can be
represented as a combinatorial structure, called {\em Gale strings}.
These are a case of {\em bitstrings}, that is a string of $0$'s and $1$'s.

Formally: given an integer $k$ and a set $S$, we can represent the function
$f_s:[k]\to S$ as the string $s = s(1)s(2)\cdots s(k)$. In the case where
$S=\{0,1\}$ we call $s$ a bitstring.

A maximal substring of consecutive $1$'s in a bitstring is called a {\em run}.

We denote with $G(d,n)$ the set of all {\em Gale strings of length $n$ and
dimension $d$}, defined as the set of all bitstrings $s$ of length $n$
such that  {\em Gale string} is a

\begin{enumerate}
\item exactly $d$ bits in $s$ are $1$ and
\item $s$ fulfills the {\em Gale evenness condition}:
\[
01^k0\hbox{ is a substring of }s\quad{\Rightarrow}\quad k\hbox{ is even.}
\]
\end{enumerate}

The Gale evenness condition characterises Gale strings in $G(d,n)$ as
the bitstrings of length~$n$ with exactly~$d$ elements equal to $1$,
such that {\em interior} runs (that is, runs bounded on both sides by~$0$s)
must be of even length. In general, this condition allows Gale strings to
start or end with an odd-length run. When $d$ is even, on the other hand,
$s$ starts with an odd run if and only if it ends with an odd run.
We can then consider the Gale strings in $G(d,n)$ with even $d$ as a ``loop''
obtained by ``glueing together'' the extremes of the string to form an even
run; more formally, we can see the indices of the string as equivalence
classes modulo $n$, so that we identify $s(i+n)=s(i)$. This also implies
that the set of Gale strings of even dimension is therefore invariant
under a cyclic shift of the strings.

\begin{example}\label{gs-example}
We consider $G(4,6)$. We have
\begin{align*}
G(4,6) = \{ & 111100, \\
        & 111001, \\
        & 110011, \\
        & 100111, \\
        & 001111, \\
        & 011110, \\
        & 110110, \\
        & 101101, \\
        & 011011\}
\end{align*}

The strings $111100$, $111001$, $110011$, $100111$, $001111$ and $011110$ are
equivalent under a cyclic shift (if considering the strings as loops, the
$1$'s are all consecutive), as are the strings $110110$, $101101$ and
$011011$ (if considering the strings as loops, the even runs of $1$'s are
two couples separated by a single $0$).
\end{example}

\todo[inline]{here to end subsect: polytopes - edit all anyway}

The relation between cyclic polytopes and Gale strings is given by the
following theorem by Gale \cite{gale-cyclicpoly}.

\begin{theorem}[\cite{gale-cyclicpoly}]\label{cp-gs-gale}
For any positive integer $n$, assume that $t_1 < t_2 < \cdots < t_n$ and let
P be the cyclic polytope obtained by taking $t_j$, where $j \in [n]$, in
definition \ref{cyclic-polytope}.

Then the facets of $P$ are encoded by $G(d,n)$; that is, $F$ is a facet of
$P$ if and only if
\[
F = \conv\{\mu(t_i)\mid i\in 1(s)\} \qquad \hbox{ for some }s\in G(d,n)
\]
\end{theorem}

\todo[inline]{sketch of pf if not too long and it uses relevant techniques}

\todo[inline]{graphics of cyclic polytope - parallel to gale string}

From this point forward, we will assume that $d$ is even.

\todo[inline]{give something to generalise to odd case}

\subsection{Labeling and the Problem \anothergale}

Given a set $G$ of bitstrings of length $n$ and a parameter $d$, a
{\em labeling} is a function $l:[n]\to[d]$. A string $s$ in $G(d,n)$ is
{\em completely labeled} if $l(\mathnormal{1}(s))=[d]$. Any $l(i)\in [d]$
is called a {\em label}

If $s \in G(d,n)$ is completely labeled for the labeling $l:[n]\to[d]$, then
for each label $l(i)$ there is a bit $s(i)=1$. We therefore have exactly
$d$ positions $i$ for which $s(i)=1$; hence, $|l(\mathnormal{1}(s))|=d$.

\begin{example}
Given the string of labels $l=123432$, there are four associated completely
labeled Gale strings: $111100$, $110110$, $100111$ and $101101$.\\

\begin{center}
{\renewcommand{\tabcolsep}{2ex}
\begin{tabular}{|c|c|c|c|}
\hline
\textbf{1234}32 &
\textbf{12}3\textbf{43}2 &
\textbf{1}23\textbf{432} &
\textbf{1}2\textbf{34}3\textbf{2} \\
\textbf{1111}00 &
\textbf{11}0\textbf{11}0 &
\textbf{1}00\textbf{111} &
\textbf{1}0\textbf{11}0\textbf{1} \\
\hline
\end{tabular}
}\\
\end{center}

\end{example}

Sometimes there aren't any completely labeled Gale strings that are
associated with a given labeling.

\begin{example}
For $l = 121314$, there are no completely labeled Gale strings.
\end{example}

\todo[inline]{here to end subsect: polytopes}
\todo[inline]{graphics of labeled cyclic polytope}

% Essentially, this holds because any set $S\subset [n]$
% the moment curve defines a unique hyperplane which is crossed
% (and not just touched) by the moment curve; if the bitstring
% $s$ that encodes $F$ as $1(s)$ has a substring $01^k0$
For this cyclic polytope $P$, a labeling $l:[n]\to[d]$ can
be understood as a label $l(j)$ for each vertex $\mu(t_j)$
for $j\in [n]$.
A completely labeled Gale string $s$ therefore represents a
facet $F$ of $P$ that is completely labeled.

Special games are obtained by using cyclic polytopes in
Theorem~\ref{t-unitv}, suitably affinely transformed with
a completely labeled facet $F_0$.
When $Q$ is a cyclic polytope in dimension $d$ with $d+n$
vertices, then the string of labels $l(1)\cdots l(n)$ in
Theorem~\ref{t-unitv} defines a labeling $l':[d+n]\to [d]$
where $l'(i)=i$ for $i\in [d]$ and
$l'(d+j)=l(j)$ for $j\in [n]$.
In other words, the string of labels $l(1)\cdots l(n)$ is
just prefixed with the string $1\,2\cdots d$ to give $l'$.
Then $l'$ has a trivial completely labeled Gale string
$1^d0^n$ which defines the facet $F_0$.
Then the problem \anothergale\ defines exactly the problem of finding a Nash
equilibrium of the unit vector game $(I,B)$.
Note again that $B$ is here not a general matrix (which would
define a general game) but obtained from the last $n$ of
$d+n$ vertices of a cyclic polytope in dimension~$d$.

\begin{fctproblem}
{\anothergale}
{A labeling $l:[n]\to[d]$, where $d$ is even and $d<n$, and an associated
completely labeled Gale string $s$ in $G(d,n)$.}
{A completely labeled Gale string $s'$ in $G(d,n)$ associated with $l$, such
that $s' \neq s$.}
\end{fctproblem}


\chapter{Algorithmic and Complexity Results}
\label{main-chapter}

\section{The Lemke-Howson Algorithm}
\label{lh-sect}

Theorem \ref{nash-ppad-complete-thm} makes the study of solutions of
$n$-Nash as endpoints of paths particularly interesting. In this
section we will see an algorithm that describes exactly this idea.

Let $P$ be a simple $d$-polytope with $n$ facets.
We {\em pivot on the vertices} of $P$ by moving from a
vertex $x$ to another vertex $y$ connected to $x$ by an edge.
Note that, since $P$ is simple, there are exactly
$d$ possible choices for $y$.
Analogously, we {\em pivot on the facets} of a simplicial
polytope $P^\Delta$ in
dimension $d$ by moving from a facet $F$ to a facet $G$ that
shares all vertices but one with~$F$. Since $P^\Delta$ is
simplicial, there are $d$ possible choices for $G$.

Suppose now that there is a labeling $l_f:[n]\to [d]$ of the facets of the
simple polytope $P$.
If we pivot from vertex $x$ to vertex $x'$ we ``leave behind'' a facet $F$
to which $x$, but not $x'$, belong; let $F$ have label $k$. At the same
time, we ``reach'' a facet $F'$, to which $x$ does not belong, but $x'$
does; let $F'$ have label $h$.
Therefore, if $x$ has labels $(l_1,\ldots,k,\ldots,l_d)$, then
$x'$ has labels $(l_1,\ldots,h,\ldots,l_d)$. We call this
{\em dropping label $k$ and picking up label $h$}, or
{\em pivoting on label $k$}; see Figure
\ref{pivot-vertex-fig} for illustration.
Analogously, if there is a labeling $l_v:[n]\to [d]$ of the vertices
of the simplicial poytope $P^\Delta$
and we pivot from a facet $F$ with labels $(l_1,\ldots,k,\ldots,l_d)$
to a facet $F'$ with labels $(l_1,\ldots,h,\ldots,l_d)$, we say that we
{\em drop label $k$ and pick up label $h$}, or that we
{\em pivot on label $k$}; see Figure \ref{pivot-facet-fig}.

\begin{figure}[hp]
\strut\hfill
\includegraphics[width=45ex]{chapter-3/fig-lh/cube-pivot.pdf}%
\hfill\strut
\caption[A pivot on the vertices of the cube]{%
A pivot on label $k$, dropping vertex $x$ with labels $(l_1,l_2,k)$ and
picking up vertex $x'$ with labels $(l_1,l_2,h)$.
}
\label{pivot-vertex-fig}
\end{figure}

\begin{figure}[hp]
\strut\hfill
\includegraphics[width=55ex]{chapter-3/fig-lh/octahedron-pivot.pdf}%
\hfill\strut
\caption[A pivot on the facets of the octahedron]{%
A pivot on label $k$, dropping facet $F$ with labels $(l_1,l_2,k)$ and
picking up facet $F'$ with labels $(l_1,l_2,h)$.
}
\label{pivot-facet-fig}
\end{figure}

\clearpage

% again, I don't think what you do here makes sense
%Let $m,n\in \naturals$ with $m\leq n$; consider a set $X$
%and a labeling $l:X\to [m]$. The $n$-uple $x=(x_1,\ldots,x_n)\in X^n$
Consider a labeling function $l:[n]\to[d]$, and a subset $S$
of $[n]$ with $|S|=d$.
Then $S$ is called {\em almost completely labeled} if
\begin{equation}
\label{acl-equation}
l(S)~=~\{\,l(s)\mid s\in S\}~=~[d]\setminus\{k\}\,,
\end{equation}
that is, all labels appear once in $S$ except for one {\em
missing label} $k\in [d]$.
Because $|S|=d$, in that case there is one
{\em duplicate label} $h\in [d]$ that appears twice in~$S$.

The set $S$ will be the set of labels of a vertex in a
simple polytope (as the set of labels of the facets containing
that vertex), or the set of labels of a facet in a
simplicial polytope (as the set of labels of the vertices on
that facet).
Correspondingly, we call such a vertex (or facet) ``almost
completely labeled''.
It is easy to see that if we pivot from an almost completely
labeled vertex (or facet) on the duplicate label, or from a
completely labeled vertex (or facet) by on any label (which
will become the missing label~$k$), then we reach either an
almost completely labeled or a completely labeled vertex
(or facet).

The algorithm by Lemke and Howson \cite{lh} finds one Nash
equilibrium of a bimatrix game.
In a modern description (e.g., Savani and von Stengel \cite{svs}),
it employs pivoting on the vertices of a simple polytope,
moving through a succession of almost completely labeled
vertices with missing label~$k$, where this polytope is
the product $P\times Q$ of the best-response polytopes.
This can be abstracted slightly further by considering only
a single polytope $P$ in dimension $d$ with facets labels
from $[d]$.
The resulting algorithm is also known as computing a ``Lemke
path'', in the terminology used by Morris~\cite{morris}, and
shown in Algorithm~\ref{lh-alg}.
We will use Lemke paths to prove some fundamental properties
of both the Lemke-Howson Algorithm and the problem {\sc
Another Completely Labeled} {\sc Vertex}.

\begin{algorithm}[htb]
\SetKwInOut{Input}{input}
\SetKwInOut{Output}{output}
\Input{%
A simple $d$-polytope $P$ with $n$ facets and a
labeling $l_f:[n]\to [d]$ of the facets of $P$.
A vertex $x_0$ of $P$ that is completely labeled by $l_f$.
}
\Output{%
A vertex $x\neq x_0$ of $P$ that is completely labeled by $l_f$.
}
\BlankLine
choose any label $k\in [d]$ as missing label \\
pivot on label $k$ from $x_0$ to $x$ reaching a new facet
with label $h$\\
\While{ $h\ne k$, so $x$ is not completely labeled ~}
{
% you don't use x_0 in the while loop
pivot away from the other facet with label $h$ from $x$ to $x'$  \\
let $h$ be the label of the new facet of $x'$ and
set $x = x'$
}
\Return $x$
\caption{Lemke Path}
\label{lh-alg}
\end{algorithm}

\begin{proposition}\label{lh-works-ppa-thm}
The Lemke Path Algorithm \ref{lh-alg} returns a solution to
the {\bf PPA} problem {\sc Another Completely Labeled Vertex}.
Furthermore, the number of completely labeled vertices in a simple
polytope with labeled facets is even.
\end{proposition}

\begin{proof}
We first show that the Lemke Path Algorithm works.
From the completely labeled vertex $x_0$, there is a unique
edge that leaves the facet with label~$k$ which leads to a
new vertex $x$ as in step~2 of the algorithm.
If $x$ is completely labeled, then the algorithm terminates
with output $x$ where clearly $x\ne x_0$.
Otherwise, $x$ is an almost completely labeled vertex with
duplicate label~$h$, where one of the facets that contain $x$
and have label~$h$ is a ``new'' facet that did not contain
the preceding vertex on the Lemke path.
Because $S$ is simple, $x$ is always on exactly $d$ facets
and the duplicate label is unique.
Hence no vertex (including $x_0$) can ever be re-visited
on the path because it would otherwise offer an alternative
way to proceed when the vertex was encountered for the first
time.

The parity is proven by the following argument: each Lemke path is
uniquely determined by its missing label and its starting point, so the
Lemke path from the endpoint with the same missing label will lead back
to the starting point. Since the endpoint and the starting point are
different, the Lemke paths must connect an even number of points.

Finally, for each label $k\in [d]$ chosen in line 1 of
Algorithm~\ref{lh-alg}, the Lemke paths are disjoint paths connecting
all the completely labeled vertices of $P$, with a standard
starting point $x_0$.
The problem {\sc Another Completely Labeled Vertex} correspond to finding
a non-standard endpoints of this graph, which is a {\bf PPA} problem.
\end{proof}

Applying the parity result of Proposition \ref{lh-works-ppa-thm} to the
case of a bimatrix game (not necessarily a unit vector game), and
remembering that the point $(\0,\0)$ corresponds to the
``artificial'' equilibrium, we have the following result, due to Lemke and
Howson \cite{lh}.

\begin{theorem}{\rm (Lemke-Howson \cite{lh}.)}
Every non-degenerate bimatrix game has an odd number of Nash equilibria.
\end{theorem}

There are two ways of using the Lemke-Howson Algorithm to find a Nash
equilibrium of a bimatrix game $(A,B)$.
The first one is to ``symmetrize'' the game as in Proposition
\ref{symmetrize-c}. Let
$R = \{ z\in\reals^{m+n}\ |\ z\geq\0,\ Cz\leq\1 \}$ be the
polytope associated to the game $(C,C\T)$, where
\[
C=\binom{~0~~~ A\,}{B\T~ 0\,}.
\]
The facets of $C$ correspond to $2(m+n)$ inequalities. We label both
the $i$-th and the $(m+n+i)$-th inequality as $i\in [m+n]$ and we apply the
Lemke Path algorithm starting from the vertex $\0$. This returns a Nash
equilibrium $(z,z)$ of $C$, which corresponds to a Nash equilibrium
$(x,y)=z$ of $(A,B)$.

We can also follow the ``traditional'' exposition of the Lemke-Howson
Algorithm by alternating a move on the best response polytopes $P$ and a move
on the best response polytope $Q$ of (\ref{br-polytopes}).
Since the polytopes $P$ and $Q$ are in $\reals^m$ and $\reals^n$, whereas $R$
is a polytope in $\reals^{m+n}$, this second version is much easier to
visualize.

\begin{example}({\rm Savani and von Stengel \cite{uvg}}.)
Consider the $3\times 3$ game $(A,B)$ of Example \ref{br-game-ex}.
\begin{equation*}
A = \left(\begin{matrix}1&0&0\\ 0&1&0\\
0&0&1\end{matrix}\right),
\qquad
B = \left(\begin{matrix}0&2&4\\ 3&2&0\\
0&2&0\end{matrix}\right).
\end{equation*}
The best response polytopes can be represented as the best response regions
of Figure \ref{br-regions-fig} extended to the origin $\0$,
as in Figure~\ref{lh-path-fig}, which is in fact the
original exposition by Shapley~\cite{shapley}.
The path starts from $(\0,\0)$.
We choose the missing label~2 and move in the polytope $P$.
Then label~6 is duplicate; so we drop it and we make the next move on the
polytope $Q$, and so on until we reach the point $x$ in $P$
and $y$ in~$Q$, which gives here the only Nash equilibrium $(x,y)$ of $(A,B)$.
\begin{figure}[hbt]
\strut\hfill
\includegraphics[width=75ex]{chapter-3/fig-lh/lemke-path.pdf}%
\hfill\strut
\caption[A Lemke path for a bimatrix game]{%
Lemke path for missing label 2 on the best response polytopes $P$ (left)
and $Q$ (right) of game (\ref{AB}).
}
\label{lh-path-fig}
\end{figure}
\end{example}

%\newpage

It is possible to have an equilibrium that cannot be reached applying the
Lemke-Howson Algorithm from the artificial equilibrium, or even from
the endpoint of a Lemke path from the artificial equilibrium. This is
can be seen in the next example. Notice that, by the parity result in
Proposition \ref{lh-works-ppa-thm}, there must be at least another equilibrium
that is ``disconnected'' from the artificial one in this way.

\begin{example}(Shapley \cite{shapley}.)
Consider the symmetric game $(C,C\T)$ with
\begin{equation}
\label{disj-lp-game}
C = \left(
    \begin{matrix}
        0&3&0 \\
        2&2&0 \\
        3&0&1
    \end{matrix}
    \right).
\end{equation}
There are three equilibria of $(C,C\T)$, all of them symmetric,
at $(x_i,x_i)$ with
$x_1=(0,0,1)$,
$x_2=(1/6,1/3,1/2)$ and
$x_3=(1/3,2/3,0)$. All Lemke paths from the
artificial equilibrium $(0,0)$ end at $(x_1,x_1)$, and consequently all
other Lemke paths connect $(x_2,x_2)$ and $(x_3,x_3)$; see
Figure \ref{disj-lp-br-poly}.

\begin{figure}[hbt]
\strut\hfill
\includegraphics[width=80ex]{chapter-3/fig-lh/shapley-game.pdf}%
\hfill\strut
\caption[A game with disjoint Lemke paths]{%
Lemke paths for missing label 1 (blue), 2 (green) and 3 (pink) on the best
response polytopes of game (\ref{disj-lp-game}). The paths for missing label
4, 5 and 6 on the best response polytope of player 1 and 2 are the same
as the paths of 1, 2 and 3 on the best response polytope of player 2 and~1,
after the corresponding change of labels.
}
\label{disj-lp-br-poly}
\end{figure}
\end{example}

\clearpage

The dual version of the Lemke Path Algorithm \ref{lh-alg} and of
Proposition \ref{lh-works-ppa-thm} is straightforward.

\begin{algorithm}
\SetKwInOut{Input}{input}
\SetKwInOut{Output}{output}
\Input{%
A simplicial $m$-polytope $P^\Delta$ with $n$ vertices and a
labeling $l_v:[n]\to [d]$ of the vertices of $P^\Delta$.
A facet $F_0$ of $P^\Delta$ that is completely labeled by $l_v$.
}
\Output{%
A facet $F\neq F_0$ of $P^\Delta$ that is completely labeled by $l_v$.
}
\BlankLine
choose any label $k\in [d]$ as missing label \\
pivot on label $k$ from $F_0$ to $F$ which has a new verte
with label $h$ \\
\While{ $h\ne k$, so $F$ is not completely labeled ~}
{
pivot away from the other vertex with label $h$ from $F$ to $F'$  \\
let $h$ be the label of the new vertex of $F'$ and
set $F = F'$
}
\Return $F$
\caption{Dual Lemke Path}
\label{lh-dual-alg}
\end{algorithm}

\begin{proposition}\label{lh-dual-works-ppa-thm}
The Dual Lemke Path Algorithm \ref{lh-dual-alg} returns
a solution to the {\bf PPA} problem {\sc Another Completely Labeled Facet}.
Furthermore, the number of completely labeled facets
in a simplicial polytope with labeled vertices is even.
\end{proposition}

By Theorem \ref{unit-vector-thm} and Theorem \ref{unit-vector-dual-thm},
in the case of unit vector games it is enough to apply the
Lemke Path Algorithm~\ref{lh-alg} to the polytope $P^l$ in
(\ref{p-l}), or the Dual Lemke Path Algorithm
(\ref{lh-dual-alg}) to the polytope $P^\Delta$ in
(\ref{p-l-dual}).
The following theorem by Savani and von Stengel \cite{uvg} guarantees that
not only does this yield a Nash equilibrium, but no potential solutions
are ``lost'' considering the polytope $P^l$ with $m$ labels
instead of the product of polytopes $P\times Q$ with $m + n$ labels;
an analogous result holds for the dual case.

\begin{theorem}\label{unit-paths}
Let $(U,B)$ be a unit vector game, with
$U=[e_{l(1)}\cdots e_{l(n)}]$ for a labeling $l:[n]\to [m]$.
Let
\[
\arraycolsep.3em
\begin{array}{rcll}
P&=&\{ x\in\reals^m&\mid\ x\geq\0,\ B\T x\leq\1 \}, \\
Q&=&\{ y\in\reals^n&\mid\ A y\leq\1,~y\geq\0 \}, \\
\end{array}
\]
as in $(\ref{br-polytopes})$, and let
\[
P^l=\{ x\in\reals^m \mid\ x\geq\0,\ B\T x\leq\1 \}
\quad \hbox{with labels in $[m]$ as in $(\ref{facet-labeling-unitv})$}
\]
as in $(\ref{p-l})$.
Then for the missing label $k\in [m]$
the Lemke path on $P\times Q$ projects to a path on $P$ that corresponds
to the Lemke path on $P^l$ for the missing label~$k$. For the missing label
$k=m+j$, where $j\in [n]$, the Lemke path on $P\times Q$ projects to a
path on $Q$ that corresponds to the Lemke path on $P^l$ for the
missing label~$l(j)$.
\end{theorem}

We finally focus on the case of Gale games.
We will consider the strings $s\in G(d,n)$, with $d$ even,
as ``wrapped-around strings''.
Let $s(i)=1$ for an index $i\in [n]$. Then, by the Gale evenness condition,
there is an odd run of \1's either on the left or on the right
of position $i$ in $s$. Let $j$ be the first index after this run.
A {\em pivot from $s$ to $s'$} is given by setting $s'(i)=0$
and $s'(j)=1$ to yield the new string~$s'$ which otherwise
agrees with~$s$.
If there is a labeling $l_s:[n]\to [d]$, we say that we
{\em pivot on label $l_s(i)$} (where we have to specify $i$
%% something you missed in your Algorithms
when the label $l_s(i)$ does not uniquely identify~$i$),
{\em dropping label $l_s(i)$} and
{\em picking up label $l_s(j)$}.
The {\em Lemke Path for Gale Algorithm} is given
in Algorithm~\ref{lhg-alg}.

\clearpage

\begin{algorithm}%\label{lhg-alg}
\SetKwInOut{Input}{input}
\SetKwInOut{Output}{output}
\Input{%
A labeling $l_s:[n]\to [d]$, where $d$ is even, such that there is a
completely labeled Gale string $s_0 \in G(d,n)$.
}
\Output{%
A Gale string $s\in G(d,n)$ that $s$ is completely labeled by $l_s$,
such that $s\neq s_0$.
}
\BlankLine
choose a missing label $k\in [d]$ \\
pivot on label $k$ from $s_0$ to $s$ reaching a new \1 bit
with label $h$\\
\While{ $h\ne k$, so $s$ is not completely labeled ~}
{
pivot away from the other \1 bit in $s$ with label $h$ from $s$ to $s'$  \\
let $h$ be the label of the new \1 bit in $s'$ and
set $s = s'$
}
\Return $s$
\caption{Lemke Path for Gale}
\label{lhg-alg}
\end{algorithm}

The next example illustrates the correspondence between the
Lemke Path Algorithm and the Lemke Path for Gale Algorithm.

\begin{example}
Figure \ref{lhg-123432-fig} shows the cyclic polytope $C_4(6)$
with the labeling
\[
\begin{array}{rll}
l_v(i)= & i\quad & \text{ for }i\in [4], \\
l_v(5)= & 3\,,  \\
l_v(6)= & 2\,.
\end{array}
\]
This corresponds to the labeling $l_s$ for $G(4,6)$ given in
Example \ref{c46-123432-ex}, which gives the four completely labeled Gale
strings
$s_A=\1\1\1\100$, $s_B=\1\10\1\10$, $s_C=\100\1\1\1$ and
$s_D=\10\1\10\1$. These correspond to the facets $A$, $B$, $C$ and
$D$ of $C_4(6)$.
Pivoting from $s_A=\1\1\1\100$ with missing label 3 returns $s_B=\1\10\1\10$.
Analogously, pivoting from facet $A$ with missing label 3 returns facet $B$.

\clearpage

\begin{figure}[ht]
\strut\hfill
\includegraphics[width=40ex]{chapter-3/fig-lh/123432-lh-pivot.pdf}%
\hfill
\small
\begin{tabular}{c | c @{ } c @{ } c @{ } c @{ } c @{ } c @{ } c }
facet & {\bf 1} & {\bf 2} & {\bf 3} & {\bf 4} & {\bf 3} & {\bf 2}\\
\hline
{\bf A} & \1 & \1 & $\underline{\1}$ & \1 & 0               & 0 \\
{\bf B} & \1 & \1 & 0                & \1 & $\overline{\1}$ & 0 \\
\hline
facet & {\bf 1} & {\bf 2} & {\bf 3} & {\bf 4} & {\bf 3} & {\bf 2}
\end{tabular}
\hfill\strut
\caption[A pivot on $G(4,6)$ and on $C_4(6)$]{%
The pivoting to $s_A=\1\1\1\1 00$ to $s_B=\1\10\1\10$ in the
Lemke Path
for Gale Algorithm corresponds to the pivoting from facet $A$ to facet $B$
in the Dual Lemke Path Algorithm.%
}
\label{lhg-123432-fig}
\end{figure}
\end{example}

The membership of \anothergale\ in the complexity class
{\bf PPA} follows from an argument similar to Proposition
\ref{lh-works-ppa-thm}.
We extend the result to prove its membership in {\bf PPAD}.

A {\em permutation} of elements of an ordered set $S$ is a sequence
without repetition. This gives a rearrangement of the elements of $S$.
A {\em transposition} is a permutation of exactly two elements.
The {\em sign of a permutation} is
$\sign(\sigma)=(-1)^m$, where $m$ is the number of transpositions needed
to get the {\em natural order} $\sigma_0=1\ldots n$ from $\sigma$.
It is immediate to see that any two permutations that differ in only
one transposition have opposite sign.
We define the {\em sign of a completely labeled Gale string} $s\in G(d,n)$
%% what is "relative labeling"?
% as follows: let $l:[n]\to [d]$ be the relative labeling of $G(d,n)$, and
as follows: let $l:[n]\to [d]$ be the labeling of $G(d,n)$, and
let $l_0$ be the string of labels $l(i)$ such that $s(i)=\1$ and that
two labels corresponding to a run in $l$ are adjacent in $l_0$. Then we
define $\sign(s)=\sign(l_0)$.
Notice that if $l(i)=i$ for $i\in [d]$ then the sign of the completely
labeled Gale string $\1^d 0^{(n-d)}$ is always positive.
The {\em sign of an almost completely labeled Gale string} $s\in G(d,n)$ with
missing label~$k$ and duplicate label $h$ is defined on two different strings.
Let $i_1$ be the index of $h$ reached by the last pivot
(the ``new'' position of the \1) and let $i_2$ be the index of $h$ such that
$s(i_2)=\1$ before the last pivot (the ``old'' position of the \1).
Let $l_1$ be the string obtained as $l_0$ substituting $k$ to
$h$ at index $i_1$, and let $l_2$ be the string obtained as $l_0$
substituting $k$ to $h$ at index $i_2$. Notice that $\sign(l_1)=-\sign(l_2)$,
since they can be obtained from each other applying the transposition $(kh)$.

Consider now the steps of the Lemke paths in the Lemke Path for Gale
Algorithm in the case where $\sign(s_0)=+1$.
The negative case is analogous, with opposite signs.
If the first pivot returns another completely labeled Gale
string $s$, this must have negative sign because it has been obtained
``jumping'' over an odd number of \1's.
For the same reason, if the pivoting returns an almost completely labeled
Gale string, we have that $\sign(l_1)=-1$, which implies $\sign(l_2)=+1$.
The next pivoting step drops the label $h$ from index $i_2$, so again we
change sign. This shows that
the Lemke Path for Gale Algorithm results in the sign
of the completely and almost completely labeled Gale strings
``swinging'' as in Table \ref{lhg-sign-figure}.
Notice that all this construction can be done in polynomial time.
Orienting all Lemke paths from positive to negative reduces the problem
\anothergale\ to {\sc End Of The Line}.
\begin{table}[hbt]
$\xymatrix@C=30mm@R-=5mm@M=3mm{
\text{\footnotesize completely labeled string} \ar[r]^{pivot}
    & \text{\footnotesize missing label in new position}  \ar[dl]|{=}  \\
\text{\footnotesize missing label in old position} \ar[r]^{pivot}
    & \qquad\qquad\cdots\qquad\qquad \ar[dl]|{=} \\
\qquad\qquad\cdots\qquad\qquad \ar[r]^{pivot}
    & \text{\footnotesize completely labeled string}
}$
\caption[Sign switching on the Lemke Path for Gale Algorithm]
{Sign switching on the Lemke Path for Gale Algorithm.}
\label{lhg-sign-figure}
\end{table}

\begin{proposition}\label{lhg-works-ppad-thm}
The Lemke Path for Gale Algorithm~\ref{lhg-alg} returns a
solution to the {\bf PPAD} problem \anothergale.
% d even assumed in Algo
Furthermore, the number of completely labeled Gale strings
$s\in G(d,n)$ is even, and the two completely labeled Gale
strings at opposite ends of any Lemke path have opposite
sign.
\end{proposition}

\begin{example}
Let $l_s=123432$. Consider the Lemke path from the completely labeled
Gale string $s=\1\1\1\100$ with missing label $1$.
Figure \ref{lhg-sign-ex-figure} shows the graph of Table
\ref{lhg-sign-figure}. Notice that
$\sign(\10\1\10\1)=\sign(l(6)l(1)l(3)l(4))=\sign((2134))$,
since $s(6)=s(1)=1$ and therefore the indices $6$ and $1$ are consecutive
in the same run.
\begin{figure}[h]
\strut\hfill
\includegraphics[width=60ex]{chapter-3/fig-lh/lhg-sign.pdf}%
\hfill\strut
\caption[Pivoting with sign]
{Pivoting with sign on $123432$.}
\label{lhg-sign-ex-figure}
\end{figure}
\end{example}

A generalization of Proposition \ref{lhg-works-ppad-thm}
is given in Shapley \cite{shapley}: two Nash equilibria at the ends of a
Lemke path have opposite {\em index}, a concept analogous to the sign but
defined using determinants on the payoff matrices for the equilibrium
support.
The index of a Nash equilibrium is usually normalized (by
multiplication with $-1$ of all signs if necessary) so that
the artificial equilibrium has index $-1$.
Then a nondegenerate game with $n$ Nash equilibria
with index $+1$ has $n-1$ Nash equilibria with index $-1$.

\clearpage

Morris \cite{morris} gave an example of a labeling
$l:[2d]\to [d]$ where the length of the Lemke paths on the
cyclic polytope $C_d(2d)$ for the Lemke Path
Algorithm~\ref{lh-alg}
grows exponentially in $d$ for every missing label.
These paths are therefore also exponential on $G(d,2d)$ for
the Lemke Path for Gale Algorithm~\ref{lhg-alg}.
The labeling by Morris is given for $d$ both
even and odd as follows:
\[
\arraycolsep.2em
\begin{array}{rcll}
l(k)&=&k          & \quad\quad\text{for }k\in [d],    \\
l(d+1)&=&d,       & \\
l(d+k)&=&d-k      & \quad\quad\text{for }2\leq k< d,\ k\text{ even },   \\
l(d+k)&=&d-k+2    & \quad\quad\text{for }2\leq k\leq d,\ k\text{ odd }, \\
l(2d)&=&1         & \quad\quad\text{for }k\text{ even}.
\end{array}
\]
Without repetitions of labels in consecutive positions,
and restricted to the case of even~$d$,
Morris's labeling is simplified to $l:[2d-2]\to [d]$ as follows:
\[
\arraycolsep.2em
\begin{array}{rcll}
l(k)&=&k          & \quad\quad\text{for }k\in [d],    \\
l(d+k)&=&d-k+1    & \quad\quad\text{for }k\in [d],\ k\text{ even},   \\
l(d+k)&=&d-k-1    & \quad\quad\text{for }k\in [d],\ k\text{ odd}.
\end{array}
\]
Savani and von Stengel \cite{svs} \cite{uvg} extend Morris's
example in order to construct different ``hard to solve'' games.
These results give a strong motivation to study the
complexity of \anothergale.
Our main result, in the next section, will give a
{\bf FP} algorithm that circumvents the problem of any exponential running
time of the Lemke Path for Gale Algorithm by using a
different algorithm.

\clearpage

\begin{example}
\label{morris-ex}
Consider the labeling $l=1234564523$ for $G(6,10)$. The only two completely
labeled Gale string  are $s=\1\1\1\1\1\100$ and $s'=\1 00000\1\1\1\1\1$.
Table \ref{morris-6-fig} shows the Lemke path for missing label~1.
\begin{table}[hbt]
\begin{center}
\large
\begin{tabular}{c @{ } c @{ } c @{ } c @{ } c @{ } c @{ } c @{ } c @{ } c @{ } c @{ } c }
{\bf 1} & {\bf 2} & {\bf 3} & {\bf 4} & {\bf 5} & {\bf 6} & {\bf 4} & {\bf 5} & {\bf 2} & {\bf 3} \\
\hline
\d1 & \1 & \1 & \1 & \1 & \1 & 0 & 0 & 0 & 0 \\
0 & \1 & \1 & \d1 & \1 & \1 & \u1 & 0 & 0 & 0 \\
0 & \1 & \1 & 0 & \d1 & \1 & \1 & \u1 & 0 & 0 \\
0 & \d1 & \1 & 0 & 0 & \1 & \1 & \1 & \u1 & 0 \\
0 & 0 & \1 & \u1 & 0 & \1 & \d1 & \1 & \1 & 0 \\
0 & 0 & \1 & \1 & \u1 & \1 & 0 & \d1 & \1 & 0 \\
0 & 0 & \d1 & \1 & \1 & \1 & 0 & 0 & \1 & \u1 \\
0 & 0 & 0 & \d1 & \1 & \1 & \u1 & 0 & \1 & \1 \\
0 & 0 & 0 & 0 & \d1 & \1 & \1 & \u1 & \1 & \1 \\
\u1 & 0 & 0 & 0 & 0 & \1 & \1 & \1 & \1 & \1 \\
\hline
{\bf 1} & {\bf 2} & {\bf 3} & {\bf 4} & {\bf 5} & {\bf 6} & {\bf 4} & {\bf 5} & {\bf 2} & {\bf 3} \\
\end{tabular}
\normalsize
\end{center}
\caption[A Lemke path for the Morris labeling]
{The Lemke path for the Morris labeling on $G(4,6)$ with missing label 1.
The bit position \1 that is dropped in the next pivoting step
is underlined, the bit \1 that has just been picked up is
overlined.}
\label{morris-6-fig}
\end{table}
\end{example}


\section{The Complexity of the Lemke-Howson Algorithms}


\todo[inline]{complexity of LH algorithms

pbl index (that gives PPA vs PPAD):

endpoints of LHG have opposite sign (Z graph)
so the graph of "what eq can you reach from eq" is bipartite (not nec connected,
ex in shapley 74)

but - ex octahedron - every rp reachable from each other rp
}


\section{The Complexity of \gale\ and \anothergale}

We will now give our main result: \anothergale\ can be solved in polynomial
time; therefore, it takes polynomial time to find a Nash Equilibrium of a
bimatrix game with dual cyclic best response polytope. Our proof will
rely on the construction of a graph and, if possible, a perfect matching
for it.
A {\em perfect matching} of a multigraph $G=(V,E)$ is a set $M\subseteq E$
of pairwise non-adjacent edges so that every vertex $v \in V$ is incident
to exactly one edge in~$M$. A theorem by Edmonds (\cite{edm}) gives
the complexity of the associated problem {\sc Perfect Matching}.

\begin{fctproblem}
{Perfect Matching}
{A multigraph $G = (V,E)$.}
{A perfect matching for $G$, or {\sc No} if there is no possible perfect
matching for $G$.}
\end{fctproblem}

\begin{theorem}{\rm (Edmonds \cite{edm})}\label{pm-thm}
The problem {\sc Perfect Matching} can be solved in polynomial time.
\end{theorem}

To prove our main result on \anothergale, we will first focus on the
accessory problem \gale, and we will use theorem \ref{pm-thm} to prove
that it is solvable in polynomial time. We will consider every Gale string
as a ``loop.''

\begin{fctproblem}
{\gale}
{A labeling $l:[n]\to[d]$, where $d$ is even and $d<n$.}
{A Gale string $s\in G(d,n)$ that is completely labeled by $l$}
\end{fctproblem}

\begin{theorem}\label{gale-thm}
The problem \gale\ is solvable in polynomial time.

\begin{proof}
We give a reduction of \gale\ to {\sc Perfect Matching}.

Consider the multigraph $G=(V,E)$ with $V=[d]$, so that the vertices of $G$
correspond to the labels $l(i)\in [d]$, and
$E=\{(l(i),l(i+1))\text{ for }i\in[n] \}$, so that there is an edge
between two vertices if and only if the corresponding labels are
next to each other at some index $i$.
Let $s\in G(d,n)$ be a completely labeled Gale string. By Gale evenness
condition, every run of $s$ corresponds uniquely to $d/2$ pairs of indices
$(i,i+1)$ with $s(i)=s(i+1)=1$, and since $s$ is completely labeled, all
labels $l(i)\in [d]$ occur at exactly one of these indices. Then the edges
$(l(i),l(i+1))$ form a perfect matching of $G$.

Conversely, let $l:[n]\to [d]$ be a labeling, and let $M$ be a perfect
matching for $G$. Consider a bitstring $s$ with $s(i)=s(i+1)$ for every
$(l(i),l(i+1))\in M$ and $s(i)=0$ otherwise.
Since $M$ is a matching, all the $(l(i),l(i+1))\in M$ are
disjoint, so, considering $s$ as a ``loop,'' every run of $s$ is of even
length, thus satisfying the Gale evenness condition. Since $M$ is perfect,
every vertex $v\in [d]$ is the endpoint of an edge $(l(i),l(i+1))$, so $s$
has exactly $d$ bits equal to $\1$, so it is completely labeled.

We have therefore reduced the problem \gale to {\sc Perfect Matching}, that
by theorem \ref{gale-thm} can be solved in polynomial time.
\end{proof}
\end{theorem}

We give two examples of the construction used in theorem \ref{gale-thm}.

\begin{example}\label{gs-pm-ex}
Consider the labeling $l:[6]\to [4]$ with $l=123423$. To find a
Gale string $s\in G(4,6)$ that is completely labeled by $l$ we must look for
a perfect matching $M$ of $G=([4],\{ e_i=(l(i),l(i+1))\ |\ i\in [6] \})$,
as seen in figure \ref{perfect-matching}.

\begin{figure}[hbt]
\strut\hfill
\includegraphics[width=40ex]{chapter-2/fig/perfect-matching.pdf}%
\hfill\strut
\caption[The graph from a labeling]{%
The graph $G$ associated to the labeling $l=123423$.
}
\label{perfect-matching}
\end{figure}

The matching $M$, in turn, will give the completely labeled Gale string $s$
as $s(i)=s(i+1)=1$ for $e_i\in M$, $s(j)=0$ otherwise.

For instance, if we take the perfect matching $M=\{ e_1,e_3 \}$, we have
the string $s=111100$. If we take the perfect matching $M'=\{ e_4,e_6 \}$
instead, we have the string $s'=1000111$.
\end{example}

A perfect matching for a graph, and therefore a Gale string for a labeling,
is not always possible, as shown in the next example.

\begin{example}
Consider the labeling $l=121314$. The graph $G$ is shown in figure
\ref{no-matching}

\begin{figure}[hbt]
\strut\hfill
\includegraphics[width=27ex]{chapter-2/fig/no-matching.pdf}%
\hfill\strut
\caption[A graph without a perfect matching]{%
The graph for the labeling $l=121314$
}
\label{no-matching}
\end{figure}

Since there aren't any disjoint edges, it's not possible to find a perfect
matching for $G$. We have already seen in example \ref{no-clgs} that
there isn't any possible completely labeled Gale string for $l=121314$.
\end{example}

We finally extend the proof of theorem \ref{gale-thm} to \anothergale.

\begin{theorem}\label{anothergale-thm}
The problem \anothergale\ is solvable in polynomial time.

\begin{proof}
Let $G=(V,E)$ be the graph corresponding to the labeling $l:[n]\to [d]$ as
in the proof of theorem \ref{gale-thm} and let $M$ be the perfect matching
of $G$ corresponding to the completely labeled Gale string $s\in G(d,n)$.

If there are two edges $e,e'\in E$ such that $e\in M$, both $e$ and $e'$
have endpoints $l(i),l(i+1)$, but $e\neq e'$ (recall that $G$ can be a
multigraph), the matching $M'=(M\setminus \{ e \})\cup\{ e' \}$
is perfect.
The corresponding completely labeled Gale string
$s'\in G(d,n)$ satisfies $s'\neq s$, since in $s$ the \1's corresponding
to the labels $l(i),l(i+1)$ are in the positions given by the edge
$e$, while in $s'$ they are in the positions given by $e'\neq e$.
It takes time $d/2$ to check all edges of $M$, the time required
is still polynomial.

We now assume that all the edges in every perfect matching $M$ for $G$
don't have a parallel edge.
Since by theorem \ref{lhg-works-thm} there is an even number of
completely labeled Gale strings, the existence of $s$ guarantees the
existence of another completely labeled Gale string $s'\neq s$ and the
corresponding perfect matching $M'\neq M$.
Since $M'\neq M$, there is at least one edge $e'\in M$ such that
$e'\notin M'$. Consider the $d/2$ graphs $G_i=(V,E_i)$, where
$E_i=E\setminus\{ e_i \}$ for $e_i\in M$. Since $V(G)=V(G')$ and
$E(G)\subset E(G')$, every perfect matching for one of these $G_i$
is a perfect matching for $G$ as well.
With a brute force approach, we look for a perfect matching in each $G_i$;
this will be $M'$. Since there are $i\in [d/2]$, the time to find it
will be still polynomial.
\end{proof}
\end{theorem}

We give two examples of the construction of theorem \ref{anothergale-thm}.

\begin{example}
The labeling $l=1234324$ gives the graph $G$ in figure
\ref{matching-2-edges}. Suppose that Edmonds' algorithm returns the matching
$M=\{ e_1,e_3 \}$, associated to the completely labeled Gale string
$s=1111000$. The edge $e_3$ has a parallel edge, $e_4$; we immediately
have a second perfect matching in $M'=\{ e_1,e_4 \}$, associated to the
Gale string $s'=1101100$.

\begin{figure}[hbt]
\strut\hfill
\includegraphics[width=40ex]{chapter-2/fig/matching-2-edges.pdf}%
\hfill\strut
\caption[A matching with a parallel edge]{%
The graph for the labeling $l=1234324$.
}
\label{matching-2-edges}
\end{figure}
\end{example}

A case in which every edge in every perfect matching does not have a
parallel one is the one given in example \ref{gs-pm-ex}.

\begin{example}
Consider the labeling $l=123423$; the associated graph is shown on
the left in figure \ref{intermediate-matching}. There are only two possible
matchings, and neither has a parallel edge.
Note that $G$ is a multigraph: we look for parallel edges in the matching,
not in all the edges of the graph.
Suppose that Edmonds' algorithm returns the perfect matching
$M=\{ e_1,e_3 \}$; we can then delete the edge $e_1$ to obtain the graph
$G_1$, seen on figure \ref{intermediate-matching} right. The graph $G_1$
has a perfect matching in $M'=\{ e_4,e_6 \}$, that is also a perfect
matching of $G$, associated to the string $s'=100111$.

\begin{figure}[hbt]
\strut\hfill
\includegraphics[width=35ex]{chapter-2/fig/perfect-matching.pdf}%
\hfill
\hfill
\includegraphics[width=35ex]{chapter-2/fig/intermediate-matching.pdf}%
\hfill\strut
\caption[A matching without parallel edges]{%
Left: the graph $G=(V,E)$ for the labeling $123423$. \\
Right: the graph $G_1=(V,E\setminus\{ e_1 \})$.
}
\label{intermediate-matching}
\end{figure}
\end{example}


\chapter{Further results}

\chapter*{Acknowledgements}
\addcontentsline{toc}{chapter}{Acknowledgements}

\todo[inline]{appendix/acknowledgments}


\begin{thebibliography}{00}

\frenchspacing\parskip0.3ex
\small

\bibitem{balthasar} A. V. Balthasar (2009).
``Geometry and equilibria in bimatrix games.''
PhD Thesis, London School of Economics and Political Science.

\bibitem{brightwell} G. R. Brightwell, P. Winkler (2004).
``Note on Counting Eulerian Circuits.''
CDAM Research Report LSE-CDAM-2004-12.

\bibitem{msc-diss} M. M. Casetti (2008).
``PPAD Completeness of Equilibrium Computation.''
MSc Thesis, London School of Economics and Political Science.

\bibitem{main} M. M. Casetti, J. Merschen, B. von Stengel (2010).
``Finding Gale Strings.''
\emph{Electronic Notes in Discrete Mathematics} 36, pp. 1065--1082.

\bibitem{cd} X. Chen, X. Deng (2006).
``Settling the Complexity of 2-Player Nash Equilibrium.''
\emph{Proc. 47th Annual IEEE Symposium on Foundations of
Computer Science (FOCS)}, pp. 261--272.

\bibitem{dgp} C. Daskalakis, P. W. Goldberg, C. H. Papadimitriou (2009).
``The Complexity of Computing a Nash Equilibrium.''
\emph{SIAM Journal on Computing} 39, pp. 195--259.

\bibitem{edm} J. Edmonds (1965).
``Paths, Trees, and Flowers.''
\emph{Canad. J. Math.} 17, pp. 449--467.

\bibitem{edm-oiks} J. Edmonds (2009).
``Euler complexes.''
In: \emph{Research Trends in Combinatorial Optimization}, eds. W.
Cook, L. Lovasz, and J. Vygen, Springer, Berlin, pp. 65--68.

\bibitem{edm-sanita} J. Edmonds, L. Sanit\`a (2010).
``On finding another room-partitioning of the vertices.''
\emph{Electronic Notes in Discrete Mathematics} 36, pp. 1257--1264.

\bibitem{gale} D. Gale (1963).
``Neighborly and Cyclic Polytopes.''
In: \emph{Convexity, Proc. Symposia in Pure Math.}, Vol. 7,
ed. V. Klee, American Math. Soc., Providence, Rhode Island, pp. 225--232.

\bibitem{gale-kuhn-tucker} D. Gale, H. W. Kuhn, A. W. Tucker (1950).
``On Symmetric Games.''
In: \emph{Contributions to the Theory of Games}~I, eds. H. W. Kuhn and A. W.
Tucker, \emph{Annals of Mathematics Studies} 24, Princeton University Press,
Princeton, pp. 81--87.

\bibitem{gilboa-zemel} I. Gilboa, E. Zemel (1989).
``Nash and correlated equilibria: some complexity considerations.''
\emph{Games and Economic Behavior} 1, pp. 80--93.

\bibitem{wicdiv} K. Gillen, J. McKelvie, M. Wilson (2015).
``Fear and Loathing in Eternity.''
{\em The Wicked + The Divine}, issue 9, ed. Image Comics.

\bibitem{gps} P. W. Goldberg, C. H. Papadimitriou, R. Savani (2011).
``The Complexity of the Homotopy Method, Equilibrium Selection, and
Lemke-Howson solutions.''
\emph{Proc. 52nd Annual IEEE Symposium on Foundations of Computer Science (FOCS)}, pp. 67--76.

\bibitem{lh} C. E. Lemke, J. T. Howson, Jr. (1964).
``Equilibrium Points of Bimatrix Games.''
\emph{J.  Soc. Indust. Appl. Mathematics} 12, pp.  413--423.

\bibitem{mclennan-tourky} A. McLennan, R. Tourky (2010).
``Imitation Games and Computation.''
\emph{Games and Economic Behavior} 70, pp. 4--11.

\bibitem{megiddo-papad} N. Megiddo, C. H. Papadimitriou (1991).
``On Total Functions, Existence Theorems and Computational Complexity.''
\emph{Theoretical Computer Science} 81, pp. 317--324.

\bibitem{jm} J. Merschen (2012).
``Nash Equilibria, Gale Strings, and Perfect Matchings.''
PhD Thesis, London School of Economics and Political Science.

\bibitem{morris} W. D. Morris Jr. (1994).
``Lemke Paths on Simple Polytopes.''
\emph{Math. Oper. Res.} 19, pp. 780--789.

\bibitem{nash} J. F. Nash (1951).
``Noncooperative games.''
\emph{Annals of Mathematics}, 54, pp. 289--295.

\bibitem{gth} M. J. Osborne, A. Rubinstein (1994).
{\em A Course in Game Theory.}
The MIT Press, Cambridge, Massachusetts.

\bibitem{papad-cc} C. H. Papadimitriou (1994).
{\em Computational Complexity.}
Addison-Wesley, Reading, MA.

\bibitem{ppad} C. H. Papadimitriou (1994).
``On the Complexity of the Parity Argument and Other Inefficient Proofs of
Existence.''
\emph{J. Comput. System Sci.} 48, pp. 498--532.

\bibitem{svs} R. Savani, B. von Stengel (2006).
``Hard-to-solve Bimatrix Games.''
\emph{Econometrica} 74, pp. 397--429.

\bibitem{uvg} R. Savani, B. von Stengel (2015).
``Unit Vector Games.''
arXiv:1501.02243v1 [cs.GT]

\bibitem{shapley} L. S. Shapley (1974).
``A Note on the Lemke-Howson Algorithm.''
\emph{Mathematical Programming Study 1: Pivoting and Extensions}, pp. 175--189

\bibitem{vvs} L. A. V\'{e}gh, B. von Stengel (2015),
``Oriented Euler Complexes and Signed Perfect Matchings.''
\emph{Mathematical Programming Series B} 150, pp. 153--178.

\bibitem{vn28} J. von Neumann (1928).
``Zur Theorie der Gesellschaftspiele.''
\emph{Mathematische Annalen} 100, pp. 295--320.

\bibitem{vs-agt} B. von Stengel (2007).
``Equilibrium computation for two-player games in strategic and extensive
form.'' Chapter 3, ``Algorithmic Game Theory,''
eds. N. Nisan, T. Roughgarden, E. Tardos, V. Vazirani.
Cambridge Univ. Press, Cambridge, pp. 53--78.

\bibitem{vs-noclf} B. von Stengel (2012).
``Completely Labeled Facet is NP-Complete.''
Manuscript, 6 pp.

\bibitem{ziegler} G. M. Ziegler (1995).
{\em Lectures on Polytopes.}
Springer, New York.

\end{thebibliography}

\addcontentsline{toc}{chapter}{Bibliography}

\end{document}
