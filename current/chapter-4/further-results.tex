Algorithm \ref{anothergale-alg} allows us to find a Nash equilibrium of a
Gale game in polynomial time starting from another equilibrium (usually the
artificial one), but it does not give any information on the relationship
of these equilibria as endpoints of the Lemke-Howson algorithm. In
particular, the issue of the sign of the equilibrium is left open.
Following the construction in Proposition \ref{lhg-works-ppad-thm},
we can reduce
the problem ``given a Nash equilibrium of a Gale game, find another one
of opposite sign'' to the {\bf PPADS} problem {\sc Opposite Sign Gale} of
\linebreak[4]
Table \ref{opposite-gale}.

\begin{problem}
{Opposite Sign Gale}
{A labeling $l:[n]\to [d]$ and a completely labeled Gale string
\linebreak[4]
$s_0\in G(d,n)$.}
{A completely labeled Gale string $s\in G(d,n)$ such that
\linebreak[4]
$\sign(s)=-\sign(s_0)$.}
\label{opposite-gale}
\end{problem}

A polynomial-time algorithm for {\sc Opposite Sign Gale} where the
labeling for which the Gale graph $G=(V,E)$ in (\ref{edges-graph})
is planar has been given by Merschen \cite{jm}. The result for any labeling
has been given by V\'{e}gh and von Stengel \cite{vvs}.
The latter uses the definition of a general framework to deal with
pivoting algorithms, called
{\em Complementary Pivoting with Direction Algorithm}. This is defined not
only for perfect matchings of a Gale graph, but for the wider class of
room partitions of Euler complexes, first introduced by
Edmonds \cite{edm-oiks} and Edmonds and Sanit\`{a} \cite{edm-sanita}.

A {\em $d$-dimensional Euler complex} (in the following: {\em $d$-oik})
%% do not use M which was a matching before
%% this definition of M is not clear
is $\mathcal{C}=(V,R)$, where $V$ is a finite set of {\em nodes} and
%$M=\{ \{ R_i \} \ \mid\ R_i\subseteq V,\ |R_i|=d \}$ is a family of
$R$ is a family of of subsets $R_i$ of $V$, called {\em rooms}, such that
$|R_i|=d$ for all $i$ and any set of $d-1$ nodes (called a
{\em wall}) is contained in an {\em even} number of rooms.
%% Now you can use M because it generalizes matchings
A {\em room partitioning} $M$ of $(V,R)$ is a subset $M$ of
$R$ such that each node $v\in V$ is in exactly one room $R_i\in M$.
Edmonds and Sanit\`{a} used an ``Exchange Algorithm'' and a parity argument
analogous to Proposition \ref{lh-works-ppa-thm}
to show that there is an even number of room partitions of an oik.
In the case where the rooms in the oik are defined as the
sets of facets incident to the vertices of the best response polytopes of
a bimatrix game and
the room partitions correspond to completely labeled vertex pairs,
the Lemke-Howson Algorithm is a special case of the Exchange Algorithm.
An simpler example of an oik is an Euler graph, with
vertices as nodes of the graph, rooms as the edges of the
Euler graph, and the perfect matchings as room partitions.

The Gale graph is an Euler graph, since it has been defined through an
Euler tour of the vertices: this points towards a further connection
between oiks and the topic of our study.
This connection is less trivial than it appears, as it can be seen as
soon as we look into the issue of giving a sign to room partitions and, by
extension, to Nash equilibria.

\clearpage

\begin{example}
Consider the octahedron with vertices labeled as in
Figure \ref{octa-poly-fig} (left). The endpoints of the Lemke paths
in the Dual Lemke Path
%\linebreak[5]
Algorithm~\ref{lh-dual-alg} when dropping different labels are shown in
Figure \ref{octa-poly-fig} (right). Notice that this graph is bipartite:
this corresponds to a parity argument with sign similar to Proposition
\ref{lhg-works-ppad-thm}.

\begin{figure}[hp]
\strut\hfill
\includegraphics[height=45ex]{chapter-4/fig/octa-poly-facets-v2.pdf}%
\hfill
\includegraphics[width=25ex]{chapter-4/fig/octa-poly-paths-v2-vert.pdf}%
\hfill\strut
\caption[The endpoints of the Dual Lemke Path   Algorithm]
{Left: A labeled octahedron and its completely labeled facets.\\
Right: The endpoints of the Lemke paths on the octahedron.}
\label{octa-poly-fig}
\end{figure}

On the other hand, consider the same octahedron from the point of view of
room partitions. Figure \ref{exchange-rp-fig} shows the
Exchange Algorithm of Edmonds and Sanit\`{a} \cite{edm-sanita}
from room partition $B$ dropping vertex $v=1$.

\begin{figure}[h]
\strut\hfill
\begin{center}
\includegraphics[height=45ex]{chapter-4/fig/octa-exchange-1.pdf}%
\vspace{10ex}
\includegraphics[height=45ex]{chapter-4/fig/octa-exchange-2.pdf}%
\end{center}
\hfill\strut
\caption[The Exchange Algorithm]
{Top: Pivot on vertex $1$ from room $B_1$ to room $A_1$.
The new vertex $4$ is duplicate in room $A_1$ and room $B_2$.
Bottom: Pivot on vertex $4$ from the old room $B_2$ to
room $A_2$, picking up the missing vertex.}
\label{exchange-rp-fig}
\end{figure}

\clearpage

The graph describing the endpoints of the Exchange Algorithm,
analogous to Figure \ref{octa-poly-fig}, is shown in
Figure \ref{octa-rp-fig} and it is not bipartite.
Giving a sign to the room partitions cannot be done simply through
Edmonds's Exchange Algorithm.

\begin{figure}[hp]
\strut\hfill
\includegraphics[height=45ex]{chapter-4/fig/octa-rp.pdf}%
\hfill
\includegraphics[width=25ex]{chapter-4/fig/octa-rp-paths-vert.pdf}%
\hfill\strut
\caption[The endpoints of the Exchange Algorithm]
{Left: The room partitions of an octahedron.\\
Right: The endpoints of Exchange Algorithm on the octahedron.}
\label{octa-rp-fig}
\end{figure}
\end{example}

V\'{e}gh and von Stengel \cite{vvs} proved that the endpoints of the
Complementary Pivoting with Direction Algorithm have opposite orientation
as long as the room partitions are defined on an oriented
oik, where the order of the rooms in the partition matters
if the dimension $d$ of the oik matters if $d$ is odd.
If $d$ is even, as in the case of perfect matchings of an
Euler graph, the order of the rooms does not matter.
The parity result follows similar to the previous cases. Unfortunately, as
for the Lemke-Howson Algorithm, the
Complementary Pivoting with Direction Algorithm may take exponential time
in the general case.
Despite this, V\'{e}gh and von Stengel \cite{vvs} give a near-linear-time
algorithm that, given a perfect matching of an Euler graph, finds a perfect
matching of opposite sign. This allows to find a solution to
{\sc Opposite Sign Gale} in polynomial time.

The wider issue of the complexity of the Lemke-Howson Algorithm has been
solved by Goldberg, Papadimitriou and Savani \cite{gps} as
{\bf PSPACE}-com\-plete. This invites, once more, a further investigation of
exchange-like algorithms to construct ``hard to solve'' games.
Since restricting to games that can be reduced to oiks of dimension $2$
seems to give games that are, after all, easily solvable, a possible
direction for research could be the study of games based on oiks of
dimension $3$ or higher. Studying products of these could also be
interesting, although the use of products of polytopes to make the
solvability via enumeration support harder in Savani and
von Stengel \cite{svs} was circumvented together with the simpler
case of Gale games by our result.

Another result worth mentioning is found in Merschen \cite{jm}: finding the
number of equilibria of a Gale game is {\bf \#P}-complete.
Gilboa and Zemel \cite{gilboa-zemel} have proven that deciding the
uniqueness of Nash equilibria is a
\linebreak[5]
{\bf co-NP}-complete problem. A proof from the point of view of completely
\linebreak[5]
labeled facets of a
polytope was given by von Stengel \cite{vs-noclf} with the result of
{\bf NP}-completeness of the problem of finding a completely labeled facet
in a generic labeled polytope. Given these results, it could be interesting
to find a polynomial-time algorithm to decide the uniqueness of Nash
equilibria in Gale games, or to find yet another open
borderline case for the complexity of games in a more general framework.
