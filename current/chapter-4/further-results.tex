Algorithm \ref{anothergale-alg} allows us to find a Nash equilibrium of a
Gale game in polynomial time starting from another equilibrium (usually the
artificial one), but it ignores the relationship
of these equilibria as endpoints of the Lemke-Howson algorithm. In
particular, it does not give any information about the index of the
equilibrium (we mentioned the concept in Section \ref{lh-sect}, for a
clear introduction see Shapley \cite{shapley} and V\'{e}gh and
von Stengel \cite{vvs}).
Following the construction in Proposition \ref{lhg-works-ppad-thm},
we can reduce
the problem ``given a Nash equilibrium of a Gale game, find another one
of opposite index'' to the {\bf PPADS} problem {\sc Opposite Sign Gale} of
Table \ref{opposite-gale}.

\begin{problem}
{Opposite Sign Gale}
{A labeling $l:[n]\to [d]$ and a completely labeled Gale string
\linebreak[4]
$s_0\in G(d,n)$.}
{A completely labeled Gale string $s\in G(d,n)$ such that
\linebreak[4]
$\sign(s)=-\sign(s_0)$.}
\label{opposite-gale}
\end{problem}

A polynomial-time algorithm for {\sc Opposite Sign Gale} where the
labeling for which the Gale graph $G=(V,E)$ in (\ref{edges-graph})
is planar has been given by Merschen \cite{jm}; the result for any labeling
has been given by V\'{e}gh and von Stengel \cite{vvs}.
The latter uses the definition of a general framework to deal with
pivoting algorithms, called
{\em Complementary Pivoting with Direction Algorithm}. This is defined not
only for perfect matchings of a Gale graph, but for the wider class of
room partitions of Euler complexes, first introduced by
Edmonds \cite{edm-oiks} and Edmonds and Sanit\`{a} \cite{edm-sanita}.

A {\em $d$-dimensional Euler complex} (in the following: {\em $d$-oik})
is $\mathcal{C}=(V,R)$, where $V$ is a finite set of {\em nodes} and
$R$ is a family of of subsets $R_i$ of $V$, called {\em rooms}, such that
$|R_i|=d$ for all $i$ and any set of $d-1$ nodes (called a
{\em wall}) is contained in an {\em even} number of rooms.
A {\em room partitioning} $M$ of $(V,R)$ is a subset $M$ of
$R$ such that each node $v\in V$ is in exactly one room $R_i\in M$.
An example is given in Figure \ref{oik-ex-fig}: an octahedron where the nodes
are the vertices and the rooms are the facets.
Edmonds and Sanit\`{a} used an {\em Exchange Algorithm} and a parity argument
analogous to Proposition \ref{lh-works-ppa-thm}
to show that there is an even number of room partitions of an oik.
Figure \ref{exchange-rp-fig} shows an example of the
Exchange Algorithm on the octahedron of Figure \ref{oik-ex-fig}.

\begin{figure}[h]
\strut\hfill
\begin{center}
\includegraphics[width=50ex]{chapter-4/fig/octa-oik.pdf}%
\end{center}
\hfill\strut
\caption[An Euler complex]
{An octahedron as an Euler complex and its room partitions, considering
the vertices as nodes and the facets as rooms.}
\label{oik-ex-fig}
\end{figure}

\clearpage

\begin{figure}[ph]
\strut\hfill
\begin{center}
\includegraphics[height=43ex]{chapter-4/fig/octa-exalg-1.pdf}%
\vspace{10ex}
\includegraphics[height=43ex]{chapter-4/fig/octa-exalg-2.pdf}%
\end{center}
\hfill\strut
\caption[The Exchange Algorithm]
{Top: Pivot on vertex $1$ from room $B_1$ to room $A_2$.
The new vertex $4$ is duplicate, as it appears in room $A_2$ and room $B_2$.

Bottom: Pivot on the duplicate vertex $4$ from the old room $B_2$ to
room $A_1$. This picks up up the missing vertex, and concludes the algorithm.}
\label{exchange-rp-fig}
\end{figure}

\clearpage

In the case where the rooms in the oik are defined as the
sets of facets incident to the vertices of the best response polytopes of
a bimatrix game and
the room partitions correspond to completely labeled vertex pairs,
the Lemke-Howson Algorithm is a special case of the Exchange Algorithm.
A simpler example of oik is an Euler graph, considering its vertices as
nodes and its edges as rooms; then the walls are its vertices, and
the room partitions are given by its perfect matchings.
As mentioned before, the Gale graph used in the proof of
Theorem \ref{gale-thm} and Theorem \ref{anothergale-thm} is an Euler graph.
This points towards a further connection
between oiks and the topic of our study.
Unfortunately, this connection is less trivial than it appears at first,
as it can be seen by looking once more into the issue of sign.

\begin{example}
Consider a octahedron with a labeling as in
Figure \ref{octa-poly-fig} (left). The endpoints of the Lemke paths
in the Dual Lemke-Howson Algorithm~\ref{lh-dual-alg} are shown in
Figure \ref{octa-poly-fig} (right). Notice that this graph is bipartite:
this corresponds to the division between facets with positive and negative
sign. This, in turn, can be proven by a parity argument with sign
used in the proof that {\sc Another Completely Labeled Facet} is {\bf PPAD},
similar to the proof of Proposition \ref{lhg-works-ppad-thm}
for {\bf Another Gale}.

\begin{figure}[hp]
\strut\hfill
\includegraphics[height=45ex]{chapter-4/fig/octa-lh-facets.pdf}%
\hfill
\includegraphics[height=35ex]{chapter-4/fig/octa-lh-paths.pdf}%
\hfill\strut
\caption[The endpoints of the Dual Lemke-Howson Algorithm on an octahedron]
{Left: The completely labeled facets of a labeled octahedron.

Right: The endpoints of the Lemke paths.}
\label{octa-poly-fig}
\end{figure}

On the other hand, consider an octahedron from the point of view of
room partitions, as given in Figure \ref{oik-ex-fig} (left). Notice that only
the rooms matter, not the labels assigned to the vertices.
The graph describing the endpoints of the Exchange Algorithm,
analogous to Figure \ref{octa-poly-fig} (right), is shown in
Figure \ref{octa-rp-fig} (right) and it is not bipartite.
Giving a sign to the room partitions cannot be done simply through
Edmonds's Exchange Algorithm.

\begin{figure}[hp]
\strut\hfill
\includegraphics[height=45ex]{chapter-4/fig/octa-rp.pdf}%
\hfill
\includegraphics[height=35ex]{chapter-4/fig/octa-rp-paths.pdf}%
\hfill\strut
\caption[The endpoints of the Exchange Algorithm]
{Left: The room partitions of an octahedron.

Right: The endpoints of the Exchange Algorithm.}
\label{octa-rp-fig}
\end{figure}
\end{example}

\clearpage

V\'{e}gh and von Stengel \cite{vvs} proved that the endpoints of the
Complementary Pivoting with Direction Algorithm have opposite orientation
as long as the room partitions are defined on an oriented
oik, where the order of the rooms in the partition matters
if the dimension $d$ of the oik matters if $d$ is odd.
If $d$ is even, as in the case of perfect matchings of an
Euler graph, the order of the rooms does not matter.
The parity result follows similar to the previous cases. Unfortunately, as
for the Lemke-Howson Algorithm, the
Complementary Pivoting with Direction Algorithm may take exponential
\linebreak[5]
time in the general case.
Despite this, V\'{e}gh and von Stengel \cite{vvs} give a near-linear-time
algorithm that, given a perfect matching of an Euler graph, finds a perfect
matching of opposite sign. This allows to find a solution to
{\sc Opposite Sign Gale} in polynomial time.

The wider issue of the complexity of the Lemke-Howson Algorithm has been
solved by Goldberg, Papadimitriou and Savani \cite{gps} as
{\bf PSPACE}-com\-plete. From a more general point of view,
Fearnley and Savani \cite{fs-simplex} have proven that it is
{\bf PSPACE}-complete to find a solution that is computed by a pivoting
method, such as the simplex algorithm.
This invites, once more, a further investigation of
exchange-like algorithms to construct ``hard to solve'' games.
Since restricting to games that can be reduced to oiks of dimension $2$
seems to give games that are, after all, easily solvable, a possible
direction for research could be the study of games based on oiks of
dimension $3$ or higher. Studying products of these could also be
interesting, although the use of products of polytopes to make the
solvability via enumeration support harder in Savani and
von Stengel \cite{svs} was circumvented together with the simpler
case of Gale games by our result.

A further complexity-related result in the case of Gale games is given in
Merschen \cite{jm}: finding the number of equilibria of a Gale game
is {\bf \#P}-complete.
Gilboa and Zemel \cite{gilboa-zemel} have proven that deciding the
uniqueness of a Nash equilibrium is a
{\bf co-NP}-complete problem. A proof from the point of view of completely
labeled facets of a
polytope was given by von Stengel \cite{vs-noclf}. Also in \cite{vs-noclf},
von Stengel proved that the finding a completely labeled facet
in a generic labeled polytope is {\bf NP}-complete.
Given these results, it could be interesting
to find a polynomial-time algorithm to decide the uniqueness of Nash
equilibria in Gale games, or to find yet another open
borderline case for the complexity of normal-form games in a more general
framework.
