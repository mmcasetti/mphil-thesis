\subsection{Cyclic Polytopes and Gale Strings}\label{gs-ssect}

In the last section we have built a correspondence between labeling of
best response polytopes and bimatrix games.
We now focus on a special case of games, where the polytope in
theorem \ref{unit-vector-dual-thm} is a {\em cyclic polytope}, a
particular kind of polytope that can be represented as a
combinatorial structure called {\em Gale string}.
We give first the definition of cyclic polytope, then the definition
of Gale string, and we show their relation as proven by Gale in
\cite{gale}.

A {\em cyclic polytope} $P$ in dimension~$d$ with $n$ vertices is the
convex hull of $n$ distinct points $\mu_d (t_j)$ on the {\em moment curve}
$\mu_d :\ t\mapsto (t,t^2,\ldots,t^d)\T$ for $j\in [n]$ such that
$t_1 < t_2 < \cdots < t_n$.

\todo[inline]{example(s) with graphics! There's one in Ziegler}

A {\em bitstring} is a string of labels that are either $0$s or $1$s.
Formally: given an integer $k$ and a set $S$, we can represent the function
$f_s:[k]\to S$ as the string $s = s(1)s(2)\cdots s(k)$; we have a bitstring
in the case where $S=\{0,1\}$. A maximal substring of consecutive
$1$'s in a bitstring is called a {\em run}.

A {\em Gale string of length $n$ and dimension $d$} is a bitstring of length
$n$, denoted as $s\in G(d,n)$, such that

\begin{enumerate}
\item exactly $d$ bits in $s$ are $1$ and
\item ({\em Gale evenness condition})
\begin{equation}
01^k0\text{ is a substring of }s\quad
{\Longrightarrow}\quad
k\text{ is even.}
\end{equation}
\end{enumerate}

The Gale evenness condition characterises Gale strings in $G(d,n)$ as
the bitstrings of length~$n$ with exactly~$d$ elements equal to $1$,
such that {\em interior} runs (that is, runs bounded on both sides by~$0$s)
must be of even length. In general, this condition allows Gale strings to
start or end with an odd-length run; but when $d$ is even this means that
$s$ starts with an odd run if and only if it ends with an odd run.
We can then consider the Gale strings in $G(d,n)$ with even $d$ as a ``loop''
obtained by ``glueing together'' the extremes of the string to form an even
run. Formally, we can see the indices of a Gale string $s\in G(d,n)$ with
$d$ even as equivalence classes modulo $n$, identifying $s(i+n)=s(i)$.
This also shows that the set of Gale strings of even dimension is invariant
under a cyclic shift of the strings.

\begin{example}\label{gs-example}
As an example of $d$ even, we have
\begin{align*}
G(4,6) = \{ & \1\1\1\100, \1\1\100\1, \1\100\1\1, \100\1\1\1, 00\1\1\1\1, \\
            & 0\1\1\1\10, \1\10\1\10, \10\1\10\1, 0\1\10\1\1 \}
\end{align*}

The strings $\1\1\1\100$, $\1\1\100\1$, $\1\100\1\1$, $\100\1\1\1$,
$00\1\1\1\1$ and $0\1\1\1\10$ are
equivalent under a cyclic shift (if considering the strings as ``loops'', the
$\1$'s are all consecutive), as are the strings $\1\10\1\10$, $\10\1\10\1$
and $0\1\10\1\1$ (if considering the strings as ``loops'', the even runs of
$\1$'s are two couples separated by a single $0$).

As an example for $d$ odd, we have
\[
G(3,5) = \{ \1\1\100, \10\1\10, \100\1\1, \1\100\1, 0\1\10\1, 00\1\1\1 \}
\]

Note how $0\10\1\1$ is a cyclic shift of $\10\1\10$, but it is not a Gale
string.
\end{example}

The relation between cyclic polytopes and Gale strings is given by the
following theorem by Gale \cite{gale}.

\begin{theorem}[\cite{gale}]\label{cp-gs-gale-thm}
For any positive integers $d,n$ let $P$ be the cyclic polytope in dimension
$d$ with $n$ vertices. Then the facets of $P$ are encoded by $G(d,n)$;
that is,
\begin{align}
F \text{ is a} & \text{ facet of } P \nonumber \\
& \Longleftrightarrow \nonumber \\
F = \conv\{ \mu(t_j)\ |\ s(j)=1 & \text{ for some }j\in[n]
\text{ and }s\in G(d,n) \} \nonumber
\end{align}

\end{theorem}

\todo[inline]{pf - see Ziegler

\small
Essentially, this holds because any set $S\subset [n]$
the moment curve defines a unique hyperplane which is crossed
(and not just touched) by the moment curve; if the bitstring
$s$ that encodes $F$ as $1(s)$ has a substring $01^k0$
\normalsize

example of cyclic polytope + equivalent gale string - pg 35 JM has nice one}

We now give define a {\em labeling} for Gale strings, corresponding to
the labeling of best-response polytopes.

A string $s$ is
{\em completely labeled} for some labeling function $l:[n]\to[d]$ if
$\{ \bar{l}\in [d] | s(i)=1\text{ and }l(i)=\bar{l}
\text{ for some }i\in [n] \}=[d]$.
If $s\in G(d,n)$, this implies that for every $\bar{l}\in [d]$ there is
exactly one $s(i)$ such that $s(i) = 1$ and $l(i) = \bar{l}$,
since there are exactly $d$ positions such that $s(i) = 1$.

\begin{example}
Given the string of labels $l=123432$, there are four associated completely
labeled Gale strings: $\1\1\1\100$, $\1\10\1\10$, $\100\1\1\1$ and
$\10\1\10\1$.

\begin{tabular}{c @{ } c @{ } c @{ } c @{ } c @{ } c @{ } c}
{\bf 1} & {\bf 2} & {\bf 3} & {\bf 4} & {\bf 3} & {\bf 2} \\
\hline
\1 & \1 & \1 & \1 & \tdot & \tdot \\
\1 & \1 & \tdot & \1 & \1 & \tdot \\
\1 & \tdot & \tdot & \1 & \1 & \1 \\
\1 & \tdot & \1 & \1 & \tdot & \1
\end{tabular}
\end{example}

Sometimes it is not possible to find a completely labeled Gale string.

\begin{example}
For $l = 121314$, there are no completely labeled Gale strings.

The labels $l(i)=2,3,4$ appear only once in $l$, as $l(2),l(4),l(6)$
respectively; therefore we must have $s(2)=s(4)=s(6)=1$. For every other
$i\in [n]$ we have $l(i)=1$, so we have $l(i)=1$ for exactly one $i=1,3,5$.
The candidate strings are then \1\10\10\1, 0\1\1\10\1, 0\10\1\1\1; but
none of these satisfies the Gale evenness condition.
\end{example}

From this point forward, we will assume that $d$ is even.
We will also assume that the labeling $l:[n]\to [d]$ is such
that $l(i)\neq l(i+1)$; this can be done without loss of generality,
given the following consideration.
Suppose that $l(i)=l(i + 1)$ for some index $i$, and let $s$ be a
completely labeled Gale string for $l$. Then only one of
$s(i)$ and $s(i+1)$ can be equal to \1 (note that it's possible that both
are 0s). So $s(i)s(i+1)$ will never be a run of even length that
``interferes'' with the Gale Evenness Condition, so we can ``simplify''
by identifying the indices $i$ and $i + 1$.

We will now focus on the problem of finding the Nash equilibria of
a unit vector game $(U,B)$ where $U=(e_{l(1)}\cdots e_{l(d)})$ for
some labeling $l:[n]\to [d]$ and the best-response polytope $Q$ is
cyclic, exploiting theorems \ref{unit-vector-dual-thm} and
\ref{cp-gs-gale-thm}.
To do so, we must find a labeling $l$ such that the Gale strings
$s\in G(d,d+n)$ encoding the completely labeled facets of the
corresponding cyclic $d$-polytope $Q$ with $d+n$ vertices are
exactly the Gale strings of dimension $d$ and length $d+n$ that are
completely labeled for $l$.

Theorem \ref{unit-vector-dual-thm} relies on a labeling of the
vertices $l_v:[d+n]\to [d]$ defined in \ref{vert-labeling-unitv}
such that $l_v(-e_i)=i$ for $i\in [d]$,
and $l_v(c_j)=l(j)$ for the vertices $c_j\neq -e_i$, where $j\in [n]$.
We define the labeling $l_s:[d+n]\to [d]$ as follows:
\begin{eqnarray}\label{gs-labeling-unitv}
l_s(i)=i\text{ for }i\in [d] \\
l_s(d+j)=l(j)\text{ for }j\in [n].
\end{eqnarray}

The Gale strings $s\in G(d,d+n)$ that are completely labeled for $l_s$
correspond exactly to the completely labeled facets of $Q$, with the
facet $F_0$ corresponding to the ``trivial'' completely labeled
string $\1^d 0$.

Then, by proposition \ref{ne-cl-pbl}, we have the following theorem.

\begin{theorem}\label{ne-another-gale-pbl}
The problem of finding a Nash equilibrium for a unit vector game
for which the best response polytope is the dual of a cyclic polytope is
polynomial-time reducible
\todo{do we have to prove ``polynomial''?}
to the problem \anothergale, where

\begin{fctproblem}
{\anothergale}
{A labeling $l:[n]\to[d]$, where $d$ is even and $d<n$.
A Gale string $s\in G(d,n)$, completely labeled by $l$.}
{A Gale string $s'\in G(d,n)$, completely labeled by $l$,
such that $s' \neq s$.}
\end{fctproblem}

\end{theorem}
