\subsection{Cyclic Polytopes and Gale Strings}\label{gs-ssect}

We have now built a correspondence between labeled best response polytopes
and bimatrix games. We now consider a special case of games, where the
polytope in theorem \ref{unit-vector-thm} can be represented as a
combinatorial structure. We first give the definition of these
particular polytopes, called {\em cyclic polytopes}, then we define the
combinatorial structure that we will use to study them, the
{\em Gale strings}.

A {\em cyclic polytope} $P$ in dimension~$d$ with $n$ vertices is the
convex hull of $n$ distinct points $\mu_d (t_j)$ on the {\em moment curve}
$\mu_d :\ t\mapsto (t,t^2,\ldots,t^d)\T$ for $j\in [n]$ such that
$t_1 < t_2 < \cdots < t_n$.

\todo[inline]{example(s) with graphics!}

A {\em bitstring} is a string of labels that are either $0$s or $1$s.
Formally: given an integer $k$ and a set $S$, we can represent the function
$f_s:[k]\to S$ as the string $s = s(1)s(2)\cdots s(k)$; we have a bitstring
in the case where $S=\{0,1\}$. A maximal substring of consecutive
$1$'s in a bitstring is called a {\em run}.

A {\em Gale string of length $n$ and dimension $d$} is a bitstring of length
$n$, denoted as $s\in G(d,n)$, such that

\begin{enumerate}
\item exactly $d$ bits in $s$ are $1$ and
\item ({\em Gale evenness condition})
\begin{equation}
01^k0\text{ is a substring of }s\quad
{\Longrightarrow}\quad
k\text{ is even.}
\end{equation}
\end{enumerate}

The Gale evenness condition characterises Gale strings in $G(d,n)$ as
the bitstrings of length~$n$ with exactly~$d$ elements equal to $1$,
such that {\em interior} runs (that is, runs bounded on both sides by~$0$s)
must be of even length. In general, this condition allows Gale strings to
start or end with an odd-length run; but when $d$ is even this means that
$s$ starts with an odd run if and only if it ends with an odd run.
We can then consider the Gale strings in $G(d,n)$ with even $d$ as a ``loop''
obtained by ``glueing together'' the extremes of the string to form an even
run. Formally, we can see the indices of a Gale string $s\in G(d,n)$ with
$d$ even as equivalence classes modulo $n$, identifying $s(i+n)=s(i)$.
This also shows that the set of Gale strings of even dimension is invariant
under a cyclic shift of the strings.

\begin{example}\label{gs-example}
For instance, as a case where $d$ is even, we have

\begin{align*}
G(4,6) = \{ & \1\1\1\100, \\
            & \1\1\100\1, \\
            & \1\100\1\1, \\
            & \100\1\1\1, \\
            & 00\1\1\1\1, \\
            & 0\1\1\1\10, \\
            & \1\10\1\10, \\
            & \10\1\10\1, \\
            & 0\1\10\1\1 \}
\end{align*}

The strings $111100$, $111001$, $110011$, $100111$, $001111$ and $011110$ are
equivalent under a cyclic shift (if considering the strings as ``loops'', the
$1$'s are all consecutive), as are the strings $110110$, $101101$ and
$011011$ (if considering the strings as ``loops'', the even runs of $1$'s are
two couples separated by a single $0$).
\end{example}


\begin{example}
As a case where $d$ is odd, we consider

\begin{align*}
G(3,5) = \{ & \1\1\100, \\
            & \10\1\10, \\
            & \100\1\1, \\
            & \1\100\1, \\
            & 0\1\10\1, \\
            & 00\1\1\1 \}
\end{align*}

Note how $01011$ is a cyclic shift of $10110$, but it is not a Gale string.
\end{example}

The relation between cyclic polytopes and Gale strings is given by the
following theorem by Gale \cite{gale}.

\begin{theorem}[\cite{gale}]\label{cp-gs-gale}
For any positive integers $d,n$ let $P$ be the cyclic polytope in dimension
$d$ with $n$ vertices. Then the facets of $P$ are encoded by $G(d,n)$;
that is,

\begin{align}
& F\text{ is a facet of }P \nonumber & \\
& \Longleftrightarrow & \\
& F = \conv\{\mu(t_i)\mid i\in 1(s)\}\text{ for some }s\in G(d,n) & \nonumber
\end{align}

\end{theorem}

\todo[inline]{sketch of pf if not too long and it uses relevant techniques

example of cyclic polytope + equivalent gale string - pg 35 JM has nice one}

From this point forward, we will assume that $d$ is even.

\todo[inline]{give something to generalise to odd case}

The labeling of a cyclic polytope has a straigtforward equivalent in
its representation as a Gale string. A string $s$ is
{\em completely labeled} for some labeling function $l:[n]\to[d]$ if
$\{ \bar{l}\in [d] | s(i)=\bar{l}\text{ for some }i\in [n] \}=[d]$.
If $s\in G(d,n)$, this implies that for every $\bar{l}\in [d]$ there is
exactly one $s(i)$ such that $s(i) = 1$ and $l(i) = \bar{l}$,
since there are exactly $d$ positions such that $s(i) = 1$.

\begin{example}
Given the string of labels $l=123432$, there are four associated completely
labeled Gale strings: $111100$, $110110$, $100111$ and $101101$.\\

\begin{tabular}{c @{ } c @{ } c @{ } c @{ } c @{ } c @{ } c}
{\bf 1} & {\bf 2} & {\bf 3} & {\bf 4} & {\bf 3} & {\bf 2} \\
\hline
\1 & \1 & \1 & \1 & \tdot & \tdot \\
\1 & \1 & \tdot & \1 & \1 & \tdot \\
\1 & \tdot & \tdot & \1 & \1 & \1 \\
\1 & \tdot & \1 & \1 & \tdot & \1
\end{tabular}

\end{example}

Sometimes there aren't any completely labeled Gale strings that are
associated with a given labeling.

\begin{example}
For $l = 121314$, there are no completely labeled Gale strings.

The labels $l(i)=2,3,4$ appear only once in $l$, as $l(2),l(4),l(6)$
respectively; therefore we must have $s(2)=s(4)=s(6)=1$. For every other
$i\in [n]$ we have $l(i)=1$, so we have $l(i)=1$ for exactly one $i=1,3,5$.
The candidate strings are then \1\1 0\1 0\1, 0\1\1\1 0\1, 0\1 0\1\1\1; but
none of these satisfies the Gale evenness condition.
\end{example}

\todo[inline]{from here: labeled cyclic polytopes and their correspondence
to labeled GS}

% Essentially, this holds because any set $S\subset [n]$
% the moment curve defines a unique hyperplane which is crossed
% (and not just touched) by the moment curve; if the bitstring
% $s$ that encodes $F$ as $1(s)$ has a substring $01^k0$
For this cyclic polytope $P$, a labeling $l:[n]\to[d]$ can
be understood as a label $l(j)$ for each vertex $\mu(t_j)$
for $j\in [n]$.
A completely labeled Gale string $s$ therefore represents a
facet $F$ of $P$ that is completely labeled.

\todo[inline]{graphics of labeled cyclic polytope}


Special games are obtained by using cyclic polytopes in
Theorem~\ref{unit-vector-thm}, suitably affinely transformed with
a completely labeled facet $F_0$.
When $Q$ is a cyclic polytope in dimension $d$ with $d+n$
vertices, then the string of labels $l(1)\cdots l(n)$ in
Theorem~\ref{unit-vector-thm} defines a labeling $l':[d+n]\to [d]$
where $l'(i)=i$ for $i\in [d]$ and
$l'(d+j)=l(j)$ for $j\in [n]$.
In other words, the string of labels $l(1)\cdots l(n)$ is
just prefixed with the string $1\,2\cdots d$ to give $l'$.
Then $l'$ has a trivial completely labeled Gale string
$1^d0^n$ which defines the facet $F_0$.
Then the problem \anothergale\ defines exactly the problem of finding a Nash
equilibrium of the unit vector game $(I,B)$.
Note again that $B$ is here not a general matrix (which would
define a general game) but obtained from the last $n$ of
$d+n$ vertices of a cyclic polytope in dimension~$d$.

\begin{fctproblem}
{\anothergale}
{A labeling $l:[n]\to[d]$, where $d$ is even and $d<n$, and an associated
completely labeled Gale string $s$ in $G(d,n)$.}
{A completely labeled Gale string $s'$ in $G(d,n)$ associated with $l$, such
that $s' \neq s$.}
\end{fctproblem}
