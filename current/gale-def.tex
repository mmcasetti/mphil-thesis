\subsection{Cyclic Polytopes and Gale Strings}\label{gs-ssect}

A special case of games is obtained by taking a particular case of best
response polytope in theorem \ref{t-unitv}.

A {\em cyclic polytope} $P$ in dimension~$d$ with $n$ vertices is the
convex hull of distinct points $\mu(t_j)$, where $j\in [n]$ and $\mu$ is the
{\em moment curve}
\[
\mu\colon t\mapsto(t,t^2,\ldots,t^d)^\top
\]

Restricting the study of best response polytopes to the case of cyclic
polytopes gives an interesting case, since cyclic polytopes can be
represented as a combinatorial structure, called {\em Gale strings}.
These are a case of {\em bitstrings}, that is a string of $0$'s and $1$'s.

Formally: given an integer $k$ and a set $S$, we can represent the function
$f_s:[k]\to S$ as the string $s = s(1)s(2)\cdots s(k)$. In the case where
$S=\{0,1\}$ we call $s$ a bitstring.

A maximal substring of consecutive $1$'s in a bitstring is called a {\em run}.

We denote with $G(d,n)$ the set of all {\em Gale strings of length $n$ and
dimension $d$}, defined as the set of all bitstrings $s$ of length $n$
such that  {\em Gale string} is a

\begin{enumerate}
\item exactly $d$ bits in $s$ are $1$ and
\item $s$ fulfills the {\em Gale evenness condition}:
\[
01^k0\hbox{ is a substring of }s\quad{\Rightarrow}\quad k\hbox{ is even.}
\]
\end{enumerate}

The Gale evenness condition characterises Gale strings in $G(d,n)$ as
the bitstrings of length~$n$ with exactly~$d$ elements equal to $1$,
such that {\em interior} runs (that is, runs bounded on both sides by~$0$s)
must be of even length. In general, this condition allows Gale strings to
start or end with an odd-length run. When $d$ is even, on the other hand,
$s$ starts with an odd run if and only if it ends with an odd run.
We can then consider the Gale strings in $G(d,n)$ with even $d$ as a ``loop''
obtained by ``glueing together'' the extremes of the string to form an even
run; more formally, we can see the indices of the string as equivalence
classes modulo $n$, so that we identify $s(i+n)=s(i)$. This also implies
that the set of Gale strings of even dimension is therefore invariant
under a cyclic shift of the strings.

\begin{example}\label{gs-example}
We consider $G(4,6)$. We have
\begin{align*}
G(4,6) = \{ & 111100, \\
        & 111001, \\
        & 110011, \\
        & 100111, \\
        & 001111, \\
        & 011110, \\
        & 110110, \\
        & 101101, \\
        & 011011\}
\end{align*}

The strings $111100$, $111001$, $110011$, $100111$, $001111$ and $011110$ are
equivalent under a cyclic shift (if considering the strings as loops, the
$1$'s are all consecutive), as are the strings $110110$, $101101$ and
$011011$ (if considering the strings as loops, the even runs of $1$'s are
two couples separated by a single $0$).
\end{example}

\todo[inline]{here to end subsect: polytopes - edit all anyway}

The relation between cyclic polytopes and Gale strings is given by the
following theorem by Gale \cite{gale-cyclicpoly}.

\begin{theorem}[\cite{gale-cyclicpoly}]\label{cp-gs-gale}
For any positive integer $n$, assume that $t_1 < t_2 < \cdots < t_n$ and let
P be the cyclic polytope obtained by taking $t_j$, where $j \in [n]$, in
definition \ref{cyclic-polytope}.

Then the facets of $P$ are encoded by $G(d,n)$; that is, $F$ is a facet of
$P$ if and only if
\[
F = \conv\{\mu(t_i)\mid i\in 1(s)\} \qquad \hbox{ for some }s\in G(d,n)
\]
\end{theorem}

\todo[inline]{sketch of pf if not too long and it uses relevant techniques}

\todo[inline]{graphics of cyclic polytope - parallel to gale string}

From this point forward, we will assume that $d$ is even.

\todo[inline]{give something to generalise to odd case}

\subsection{Labeling and the Problem \anothergale}

Given a set $G$ of bitstrings of length $n$ and a parameter $d$, a
{\em labeling} is a function $l:[n]\to[d]$. A string $s$ in $G(d,n)$ is
{\em completely labeled} if $l(\mathnormal{1}(s))=[d]$. Any $l(i)\in [d]$
is called a {\em label}

If $s \in G(d,n)$ is completely labeled for the labeling $l:[n]\to[d]$, then
for each label $l(i)$ there is a bit $s(i)=1$. We therefore have exactly
$d$ positions $i$ for which $s(i)=1$; hence, $|l(\mathnormal{1}(s))|=d$.

\begin{example}
Given the string of labels $l=123432$, there are four associated completely
labeled Gale strings: $111100$, $110110$, $100111$ and $101101$.\\

\begin{center}
{\renewcommand{\tabcolsep}{2ex}
\begin{tabular}{|c|c|c|c|}
\hline
\textbf{1234}32 &
\textbf{12}3\textbf{43}2 &
\textbf{1}23\textbf{432} &
\textbf{1}2\textbf{34}3\textbf{2} \\
\textbf{1111}00 &
\textbf{11}0\textbf{11}0 &
\textbf{1}00\textbf{111} &
\textbf{1}0\textbf{11}0\textbf{1} \\
\hline
\end{tabular}
}\\
\end{center}

\end{example}

Sometimes there aren't any completely labeled Gale strings that are
associated with a given labeling.

\begin{example}
For $l = 121314$, there are no completely labeled Gale strings.
\end{example}

\todo[inline]{here to end subsect: polytopes}
\todo[inline]{graphics of labeled cyclic polytope}

% Essentially, this holds because any set $S\subset [n]$
% the moment curve defines a unique hyperplane which is crossed
% (and not just touched) by the moment curve; if the bitstring
% $s$ that encodes $F$ as $1(s)$ has a substring $01^k0$
For this cyclic polytope $P$, a labeling $l:[n]\to[d]$ can
be understood as a label $l(j)$ for each vertex $\mu(t_j)$
for $j\in [n]$.
A completely labeled Gale string $s$ therefore represents a
facet $F$ of $P$ that is completely labeled.

Special games are obtained by using cyclic polytopes in
Theorem~\ref{t-unitv}, suitably affinely transformed with
a completely labeled facet $F_0$.
When $Q$ is a cyclic polytope in dimension $d$ with $d+n$
vertices, then the string of labels $l(1)\cdots l(n)$ in
Theorem~\ref{t-unitv} defines a labeling $l':[d+n]\to [d]$
where $l'(i)=i$ for $i\in [d]$ and
$l'(d+j)=l(j)$ for $j\in [n]$.
In other words, the string of labels $l(1)\cdots l(n)$ is
just prefixed with the string $1\,2\cdots d$ to give $l'$.
Then $l'$ has a trivial completely labeled Gale string
$1^d0^n$ which defines the facet $F_0$.
Then the problem \anothergale\ defines exactly the problem of finding a Nash
equilibrium of the unit vector game $(I,B)$.
Note again that $B$ is here not a general matrix (which would
define a general game) but obtained from the last $n$ of
$d+n$ vertices of a cyclic polytope in dimension~$d$.

\begin{fctproblem}
{\anothergale}
{A labeling $l:[n]\to[d]$, where $d$ is even and $d<n$, and an associated
completely labeled Gale string $s$ in $G(d,n)$.}
{A completely labeled Gale string $s'$ in $G(d,n)$ associated with $l$, such
that $s' \neq s$.}
\end{fctproblem}
