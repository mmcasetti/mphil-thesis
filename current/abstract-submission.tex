\documentclass{article}

\usepackage{amsmath}
\usepackage[latin1]{inputenc}

\linespread{1.3}

\thispagestyle{empty}

\hyphenation{Edmonds}

\begin{document}

\noindent Marta Maria Casetti -- student number 200614154

\section*{Complexity of the Gale String Problem \\
for Equilibrium Computation in Games}

This thesis presents a report on original research, published
as joint work with Merschen and von Stengel in {\em Electronic Notes
in Discrete Mathematics} (2010). Our result shows a polynomial time
algorithm to solve two problems related to labeled Gale strings, a
combinatorial structure consisting a string of labels and a bitstring
satisfying certain conditions.

Gale strings can be used in the representation of a particular class
of games that Savani and von Stengel (2006) used as an example of
exponential running time for the classical Lemke-Howson algorithm to
find a Nash equilibrium of a bimatrix game (1964). It was conjectured
that solving these games via the Lemke-Howson algorithm was complete
in the class PPAD (Proof by Parity Argument, Directed version). A
major motivation for the definition of this class by Papadimitriou
(1994) was, in turn, to capture the pivoting technique of many
results related to the Nash equilibrium, including the Lemke-Howson
algorithm.

Our result, on the contrary, sets apart this class of games as a case
for which there is a polynomial-time algorithm to find a Nash
equilibrium. Since Daskalakis, Goldberg and Papaditrimiou (2005) and
Chen and Deng (2009) proved the PPAD-completeness of finding a Nash
equilibrium in general normal-form games, we have a special class of
games, unless PPAD = P.

Our proof exploits two results. The first one is the representation
of the Nash equilibria of these games as a string of labels and a
bitstring, as seen in Savani and von Stengel (2006). The second one
is the polynomial-time solvability of the problem of finding a perfect
matching in a graph, solved by Edmonds (1965).

Further results by Merschen (2012) and V\'{e}gh and von Stengel (2014)
will be mentioned.

An appendix relates an amendment to the proof of the PPAD-completeness
result by Daskalakis, Goldberg and Papaditrimiou (2005).

\end{document}
