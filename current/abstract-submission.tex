\documentclass{article}

\usepackage{amsmath}
\usepackage[latin1]{inputenc}

\thispagestyle{empty}

\hyphenation{Edmonds}

\begin{document}

\title{The Gale String Problem for Equilibrium Computation in Games}

This thesis presents a report on original research, published as conjoint work with Merschen and von Stengel in ENDM (2010). Our result shows a polynomial time algorithm to find a Nash equilibrium for a particular class of games, which was previously used by Savani and von Stengel (2006) as an example of exponential time for the classical Lemke-Howson algorithm for bimatrix games (1964).

It was conjectured that solving these games via the Lemke-Howson algorithm was complete in the class $\mathbf{PPAD}$ (Proof by Parity Argument, Directed version). A major motivation for the definition of this class by Papadimitriou (1994) was, in turn, to capture the pivoting technique of many results related to the Nash equilibrium, including the Lemke-Howson algorithm. A $\mathbf{PPAD}$-completness proof of the games we consider would have provided a traceable proof of the Daskalakis, Goldberg and Papaditrimiou (2005) and Chen and Deng (2009) results about the $\mathbf{PPAD}$-completeness of every normal form game. Our result of polynomial-time solvability, on the other hand, indicates the existence of a special class of games, unless $\mathbf{PPAD} = \mathbf{P}$.

Our proof exploits two results. The first one is the representation of the Nash equilibria of these games as a string of labels and an associated string of $0$s and $1$s satisfying some conditions, called \textit{Gale conditions}, as seen in Savani and von Stengel (2006). The second one is the polynomial-time solvability of the problem of finding a perfect matching in a graph, solved by Edmonds (1965).

Further results by Merschen (2012) and V\'{e}gh and von Stengel (2014) stemmed from our theorem. These proofs analysed the pivoting technique of the Lemke-Howson algorithm in the framework of the \textit{Euler complexes} introduced by Edmonds (2005), and solved the open problem of the \textit{sign} of the equilibrium found in polynomial time.

\end{document}
