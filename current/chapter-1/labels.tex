\section{Bimatrix Games and Labels}

In the rest of this thesis we will focus on two-player normal-form games.
For sake of readability, we will use feminine pronouns when referring to
player 1 and masculine pronouns when referring to player 2.

Two-player normal-form games are
also called {\em bimatrix games}, since they can be characterized by the
$m \times n$ payoff matrices $A$ and $B$, where $a_{ij}$ and $b_{ij}$ are
the payoffs of player 1 and 2 when she plays her $i$th pure strategy
and he plays his $j$th pure strategy.
We will assume that $(A,B)$ are non-negative, and that $A$ and $B\T$ have
no zero column. This can be easily obtained without loss of generality via
an affine transformation that will not affect the equilibria of the game.

The Nash equilibria of bimatrix games can be analysed from a combinatorial
point of view using {\em labels}. This method is due to Shapley
\cite{shapley}, in a study building on ideas introduced in a
paper by Lemke and Howson \cite{lh}.


Let $(A,B)$ be bimatrix game. The mixed-strategy simplices of player 1 and 2
are, respectively

\begin{equation}
X = \{ x\in\reals^m | x\geq\0,\ \1\T x = 1 \},\quad
Y = \{ y\in\reals^n | y\geq\0,\ \1\T y = 1 \}
\end{equation}

A {\em labeling} of the game is then given as follows:

\begin{enumerate}
\item the $m$ pure strategies of player 1 are identified by $1,\ldots,m$;
\item the $n$ pure strategies of player 2 are identified by $m+1,\ldots,m+n$;
\item each mixed strategy $x\in X$ of player 1 has
    \begin{itemize}
    \item label $i$ for each $i\in [m]$ such that $x_i = 0$, that is if in
    $x$ player 1 does not play her $i$th pure strategy;
    \item label $m + j$ for each $j\in [n]$ such that the $j$th pure strategy
    of player 2 is a best response to $x$;
    \end{itemize}
\item each mixed strategy $y\in Y$ of player 2 has
    \begin{itemize}
    \item label $m + j$ for each $j\in [n]$ such that $y_j = 0$, that is if in
    $y$ player 2 does not play his $j$th pure strategy;
    \item label $i$ for each $i\in [m]$ such that the $i$th pure strategy
    of player 1 is a best response to $y$;
    \end{itemize}
\end{enumerate}

A strategy profile $(x,y)\in X\times Y$ is {\em completely labeled} if every
label $1,\ldots,m+n$ is a label of either $x$ or $y$. We have the following
theorem (Theorem 1 in \cite{shapley}):

\begin{theorem}\label{comp-label-bimatrix-thm}
Let $(x,y)\in X\times Y$; then $(x,y)$ is a Nash equilibrium of the bimatrix
game $(A,B)$ if and only if $(x,y)$ is completely labeled.

\begin{proof}
The mixed strategy $x\in X$ has label $m + j$ for some $j\in [n]$ if and
only if the $j$th pure strategy of player 2 is a best response to $x$; this,
in turn, is a necessary and sufficient condition for player 2 to play his
$j$th strategy at an equilibrium against $x$. Therefore, at an equilibrium
$(x,y)$ all labels $m + 1,\ldots,m + n$ will appear either as labels of
$x$ or of $y$. The analogous holds for the labels $i\in [n]$.
\end{proof}
\end{theorem}

An useful geometrical representation of labels can be given on the mixed
strategies simplices $X$ and $Y$. The outside of each simplex is
labeled according to the player's own pure strategies that are {\em not}
played; so, for instance, the outside of $X$ will have labels
$1,\ldots,n$. The interior of each simplex is subdivided in closed polyhedral
sets, called {\em best-response regions}. These are labeled according to
the other player's pure strategy that is a best response in that set;
so, for instance, the inside of $X$ will have labels $m + 1,\ldots,m + n$.

We give an example of this construction.

\begin{example}

\todo[inline]{page 3--4 of Savani, von Stengel, Unit Vector Games.

With graphics.}

\end{example}

We will now give a description of labeling on polytopes equivalent to
the construction based on best-response regions.

We begin by noticing that the best-response regions can be obtained as
projections on $X$ and $Y$ of the {\em best-response facets} of
the polyhedra

\begin{equation}\label{br-polyhedron}
\bar{P} = \{ (x,v)\in X\times\reals | B\T x\leq\1 v \},\quad
\bar{Q} = \{ (y,u)\in Y\times\reals | A y\leq\1 u \}.
\end{equation}

These facets in $\bar{P}$ are defined as the points $(x,v)\in X\times\reals$
such that $(B\T x)_j = v$. These points represent the strategies $x\in X$ of
player 1 that give exactly payoff $v$ to player 2 when he plays strategy $j$.
The projection of the facet defined by $(B\T x)_j = v$ to $X$ will have
label $j$. Analogously, in $\bar{Q}$, the facets are the points
$(y,u)\in Y\times\reals$ such that $a_i y = u$, and their projection to $Y$
will be the best-response region with label $i$.

\begin{example}

\todo[inline]{cont of ex above, page 4--5, image on page 5 left}

\end{example}

Given the assumptions on non-negativity of $A$ and $B\T$, we can give a
change coordinates to $x_i / v$ and $y_j / u$ and replace $\bar{P}$ and
$\bar{Q}$ with the {\em best-response polytopes}

\begin{equation}\label{br-polytopes}
P = \{ x\in\reals^m | x\geq\0,\ B\T x\leq\1 \},\quad
Q = \{ y\in\reals^n | y\geq\0,\ A y\leq\1 \},\quad
\end{equation}

Each one of these polytope is defined by half spaces corresponding to
either the player's own strategy that is not being played or the other
player's best response; each one of the facets of the polytope is labeled
by the strategy corresponding to the relative half-space.

This means that a point in $P$ has label $k$ if and only if either
$x_k = 0$ for $k\in \{ 1,\ldots,m \}$ or $(B\T x)_{k - m} = 0$ for
$k\in \{ m+1,\ldots,m+n \}$;
analogously, a point in $Q$ has label $k$ if and only if either
$y_{k - m} = 0$ for $k\in \{ m+1,\ldots,m+n \}$ or $(A y)_{k}$ for
$k\in \{ m+1,\ldots,m+n \}$. A point $(x,y)\in P\times Q$ is
{\em completely labeled} if every $k\in [m + n]$ is a label of $x$ or $y$.
Note that the point $(\0,\0)$ is completely labeled. Rescaling back to
$\bar{P}$ and $\bar{Q}$, all the non-zero completely labeled points give
exactly all the equilibria of $(A,B)$. In this construction, we will
call $(\0,\0)$ {\em artificial equilibrium}.

\begin{example}

\todo[inline]{ex in Savani, von Stengel, image on page 5 right}

\end{example}

A characterization of the completely labeled pairs in $P\times Q$ can be
given as follows.

\begin{proposition}\label{compl-orth-cond}
The pair $(x,y)\in P\times Q$ is completely labeled if and only if one of
the following condition holds:
\begin{itemize}
\item {\em (Complementarity condition)}

\begin{equation}
x_i = 0\text{ or }(Ay)_i = 1\text{ for all }i\in [m],\quad
y_j = 0\text{ or }(B\T x)_j\text{ for all }j\in [n]
\end{equation}

\item {\em (Orthogonality condition)}

\begin{equation}
x\T (\1 - Ay) = 0,\quad
y\T (\1 - B\T x) = 0
\end{equation}
\end{itemize}
\end{proposition}

Proposition \ref{compl-orth-cond} can be used to prove a useful property:
{\em symmetric games}, that is, games that have payoff matrix of the form
$(C,C\T)$ for some matrix $C$, can be used to study generic bimatrix games
without loss of generality. The result is due to Gale, Kuhn and Tucker
\cite{gale-kuhn-tucker} for zero-sum games; its extension to non-zero-sum
games is a folklore result.

\begin{proposition}\label{symmetric-eq-thm}
Let $(A,B)$ be a bimatrix game and $(x,y)$ be one of its Nash equilibria.
Then $(z,z)$, where $z=(x,y)$, is a Nash equilibrium of the symmetric game
$(C,C\T)$, where

\[
C = \left(
    \begin{array}{cc}
    0 & A \\
    B\T & 0
    \end{array}
    \right).
\]
\end{proposition}

The converse has been proved by McLennan and Tourky \cite{mclennan-tourky} in
their study of {\em imitiation games}, that is, bimatrix games of the form
$(I,B)$.

\begin{proposition}\label{imitation-thm}
The pair $(x,x)$ is a symmetric Nash equilibrium of the symmetric bimatrix
game $(C,C\T)$ if and only if there is some $y$ such that $(x,y)$ is a
Nash equilibrium of the imitation game $(I,C\T)$.
\end{proposition}

\begin{example}
Consider the symmetric game $(C,C\T)$, where $C\T = B$ in the previous
examples.

\todo[inline]{ex Savani, von Stengel, pg 8}

\end{example}

Savani and von Stengel \cite{uvg} extended the study of imitation games to
{\em unit vector games} $(U,B)$, where the columns of the matrix $U$ are
unit vectors. For these games, the use of labeling in polytopes to
characterize Nash equilibria of the game is given by the following
theorem, first proved in dual form by Balthasar \cite{balthasar}.

\begin{theorem}\cite{uvg}\label{unit-vector-thm}
Let $l:[n]\to [m]$, and let $(U,B)$ be the unit vector game where
$U=(e_{l(1)}\ \cdots\ e_{l(n)})$. Consider the polytopes $P^l$ and $Q^l$
where

\begin{equation}\label{p-l-unitv}
P^l = \{ x\in\reals^m | x\geq\0,\ B\T x\leq\1 \}
\end{equation}

\begin{equation}
Q^l = \{ y\in\reals^n | y\geq\0,\
\sum_{\substack{j\in N_i \\ i\in [m]}} y_j\leq 1 \}
\end{equation}

where $N_i = \{ j\in [n] | l(j)=i \}$ for $i\in [m]$.

Give a labeling $l_f$ of the facets of $P^l$ according to the
inequality defining it, as follows:

\begin{eqnarray}\label{facet-labeling-unitv}
x_i\geq 0\text{ has label }i\text{ for }i\in [m] \\
(B\T x)_j \leq 1\text{ has label }l(j)\text{ for }j\in [n]
\end{eqnarray}

Then $x\in P^l$ is a completely labeled point of $P^l\setminus\{\0\}$
if and only if there is some $y\in Q^l$ such that, after scaling,
the pair $(x,y)$ is a Nash equilibrium of $(U,B)$

\begin{proof}
Let $P,Q$ be the polytopes associated to the game $(U,B)$ as before.

Let $(x,y)\in P\times Q\setminus\{ \0,\0 \}$ be a Nash equilibrium of
$(U,B)$, therefore completely labeled in $[m + n]$.
Then, if $x_i=0$, then $x$ has label $i\in m$.
If $x_i > 0$ instead, then $y$ has label $i$, therefore $(Uy)_i = 1$,
therefore for some $j\in [n]$ we have $y_j > 0$ and $U_j = e_i$, so $l(j)=i$.
Since $y_j > 0$ and $(x,y)$ is completely labeled, $x\in P$ has label $m+j$,
that is, $(B\T x)_j = 1$, therefore $x\in P^l$ has label $l(j) = i$.
Hence, $x$ is a completely labeled point of $P^l$.

Conversely, let $x\in P^l\setminus \{ \0 \}$ be completely labeled.
If $x_i > 0$, then there is $j\in [m]$ such that $(B\T x) = j$ and
$l(j) = i$, that is, $j\in N_i$. For all $i$ such that $x_i >0 $,
define $y$ as follows: $y_h = 0$ for all $h\in N_i\setminus \{ j \}$,
$y_j = 1$. Then $(x,y)\in P\times Q$ is completely labeled.
\end{proof}
\end{theorem}

Theorem \ref{unit-vector-thm} gives a correspondence between the
completely labeled vertices of the polytope $P^l$ and the equilibria
of the unit vector game $(U,B)$, with an ``artificial''
equilibrium corresponding to the vertex $\0$.

The dual version of theorem \ref{unit-vector-thm}, given in
\cite{balthasar}, is constructed as follows.
We translate the polytope $P^l$ in theorem \ref{unit-vector-thm} to
$P = \{ x - \1\ |\ x\in P^l \}$. Multiplying all payoffs in $B$ by a
constant if necessary (an operation that does not change the game),
we can have \1 is in the interior of $P^l$ and \0 in the interior of $P$.
We have $x\in P$ if and only if $x + \1\geq\0$ and
$B\T(x + \1)=(x + \1)\T B\leq\1$;
that is, if and only if $-x_i\leq 1$ for $i\in [m]$ and
$x\T \frac{b_j}{1 - \1\T b_j} \leq 1$ for $j\in [n]$.
The polar of $P$ is then
\begin{equation}
P^\Delta = \conv(\{−e_i\ |\ i\in [m] \} \cup \{ \frac{b_j}{1 - \1\T b_j} \})
\end{equation}

\todo[inline]{$P$ is simple (why? because game nondegenerate!
show it before), so $P^\Delta$ is simplicial}

Since \0 is in the interior of $P$, we have that $P^{\Delta\Delta}=P$,
and the facets of $P^\Delta$ correspond to
the vertices of $P$ and vice versa. We can then label the vertices
of $P\Delta$ with the labels of the corresponding facets in $P^l$, so
the completely labeled facets of $P\Delta$ will correspond to the
completely labeled vertices of $P^l$.
In particular, the facet corresponding to \0 is
\begin{equation}
F_0 = \{ x\in P^\Delta\ |\ -\1\T x = 1 \} = \conv\{ e_i\ |\ i\in [m] \}.
\end{equation}
Theorem \ref{unit-vector-thm} then translates as follows.

\begin{theorem}\cite{balthasar}\label{unit-vector-dual-thm}
Let $Q$ be a labeled $m$-dimensional simplicial polytope with \0 in
its interior, with vertices $e_1,\ldots,e_m,c_1,\ldots,c_n$, so that
$F_0 = \conv\{ e_i\ |\ i\in [m] \}$ is a facet of $Q$.

Let $l:[n]\to [m]$, and let $(U,B)$ be the unit vector game with
$U=(e_{l(1)}\ \cdots\ e_{l(n)})$ and $B = (b_1\ \cdots\ b_n)$,
where $b_j = \frac{c_j}{1 + \1\T c_j}$ for $j\in [n]$.

Label the vertices of $Q$ as follows:
\begin{eqnarray}\label{vert-labeling-unitv}
l_v(-e_i)=i\text{ for }i\in [m] \\
l_v(c_j)=l(j)\text{ for }j\in [n]
\end{eqnarray}

Then a facet $F\neq F_0$ of $Q$ with normal vector $v$ is completely
labeled if and only if $(x,y)$ is a Nash equilibrium of $(U,B)$, where
$x = \frac{v + \1}{\1\T (v + \1)}$,
and $x_i = 0$ if and only if $−e_i\in F$ for $i\in [m]$.
Any $j$ so that $c_j$ is a vertex of $F$ represents a pure best reply to $x$;
the mixed strategy $y$ is the uniform distribution on the set of the pure best
replies to $x$.
\end{theorem}

In theorem \ref{unit-vector-dual-thm} we have a correspondence between
the completely labeled facets of the polytope $Q$
(the completely labeled vertices of $P^l$ in theorem \ref{unit-vector-thm})
and the equilibria of the unit vector game $(U,B)$,
with the ``artificial'' equilibrium corresponding to the
facet $F_0$ (the vertex $\0$ in theorem \ref{unit-vector-thm}).
This proves the following proposition.

\begin{proposition}\label{ne-cl-pbl}
The problem 2-{\sc Nash} for unit vector games is
polynomial-time reducible
\todo{do we have to prove ``polynomial''?}
to the problems {\sc Another Completely Labeled Vertex} and its dual
{\sc Another Completely Labeled Facet}, where

\begin{fctproblem}
{\sc Another Completely Labeled Vertex}
{A simple $m$-dimensional polytope $S$ with $m+n$ facets;
a labeling $l:[m+n]\to [n]$;
a facet $F_0$ of $S$, completely labeled by $l$.}
{A facet $F\neq F_0$ of $S$, completely labeled by $l$.}
\end{fctproblem}

\begin{fctproblem}
{\sc Another Completely Labeled Vertex}
{A simplicial $m$-dimensional polytope $S$ with $m+n$ vertices;
a labeling $l:[m+n]\to [n]$;
a vertex $v_0$ of $S$, completely labeled by $l$.}
{A vertex $v\neq v_0$ of $S$, completely labeled by $l$.}
\end{fctproblem}

\end{proposition}

\todo[inline]{
nondegeneracy; made nondegenerate by ``lexicographic'' perurbation
(what does it mean?);

nondegeneracy -> br polytope $P$ is simple -> $P^\Delta$ is simplicial

so: this goes before!

ex pg 9; odd no eq, mention homotopy method (find ref)
(tie with Nash, again?)
}
