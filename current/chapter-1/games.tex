\section{Normal Form Games and Nash Equilibria}

We now give the game-theoretic background that will be used in this thesis.
A {\em game}, as first defined by von Neumann in \cite{vn28}, is a model
of strategic interaction. A {\em finite normal form game} is
$\Gamma=(P,S=\times_{p\in P} S_p,u=\times_{p\in P}u^p)$
where both the set of {\em players} $P$ and the sets of
{\em pure strategies} $S_p$ (and therefore the set
of {\em pure strategy profiles} $S$) are finite. We will use the
notation $S_{-p}=\times_{q\neq p}S_p$.
The purpose of each player $p\in P$ is to maximize her {\em payoff function}
$u^p:S\to\reals$. In the following pages, by ``game'' we will always mean
``finite normal form game.'' If there are only two players, we will refer to
player 1 using feminine pronouns and to player 2 using masculine ones; such
games are called {\em bimatrix games} since they can be characterized by
the $m \times n$ payoff matrices $A$ and $B$, where $a_{ij}$ and $b_{ij}$ are
the payoffs of respectively player 1 and of player 2 when the former plays
her $i$th pure strategy and the latter plays his $j$th pure strategy.
A bimatrix game is {\em zero-sum} if $B=-A$, and {\em symmetric} if  $B=A\T$.

A {\em mixed strategy} of player $p$ is a probability
distribution on $S_P$; it can be described as a point
$x=(x^p_1,\ldots,x^p_{|S_p|})$ on the
$(|S_p|-1)$-dimensional {\em mixed strategy simplex} $\Delta_p$;
the {\em mixed strategy profiles} will be the simplicial polytope
$\Delta=\times_{p\in P}\Delta_p$.
We extend the payoff functions to $u^p:\Delta\to\reals$ by linearity.

A {\em Nash equilibrium} of a game is a strategy profile in which each
player cannot improve her expected payoff by unilaterally changing her
strategy. Such a strategy is called a {\em best response}; a strategy
that is not a best response is called {\em dominated}.
Formally: for $s\in S_{-p}$ let $x_s=\prod_{q\neq p} x^q_{s_q}$;
then a Nash equilibrium is a strategy profile
$x$ such that for every $p\in P$ and every $\sigma,\tau\in S_p$
\begin{equation}
\sum_{s\in S_{-p}} u^p(\sigma,s) x_s > \sum_{s\in S_{-p}} u^p(\tau,s) x_s
\quad
\Rightarrow
\quad
x_\tau^p = 0
\end{equation}

Note that applying an affine transformation to all the payoffs does not
change the Nash equilibria of the game. Note also that there might be more
than one equilibrium. The existence of a Nash equilibrium is guaranteed by
the fundamental theorem by Nash (\cite{nash}).

\begin{theorem}\label{nash-thm}
Every finite game in normal form has a Nash equilibrium.
\end{theorem}

We give a classic example of game: matching pennies.

\begin{example}
In the bimatrix game {\em matching pennies}, both players have payoff
zero unless they pay the same strategy. In this case,
player 1 (the {\em pursuer}) has payoff 1 and player 2 (the \emph{evader}),
has payoff -1.

\begin{center}
\def\mm#1{\makebox(0,0){\strut#1}}
\bimatrixgame{3mm}{2}{2}{{{\scriptsize pursuer}}}{{{\scriptsize evader}}}
{{{\scriptsize head}}{{\scriptsize tail}}}
{{\scriptsize head}{\scriptsize tail}}
{
\payoffpairs{1}{{1}0}{{-1}0}
\payoffpairs{2}{0{1}}{0{-1}}
}
\end{center}

At the unique equilibrium of the game, each player follows the uniform
distribution over their strategies.
\end{example}

Consider the problem $n$-{\sc Nash}, as follows.

\begin{fctproblem}
{$n$-Nash}
{A $n$-player game $\Gamma$.}
{A Nash equilibrium of $\Gamma$.}
\end{fctproblem}

By theorem \ref{nash-thm}, {\sc $n$-Nash} is a total function
problem; Megiddo and Papadimitriou (\cite{megiddo-papad}) proved that it is
in {\bf TFNP}. Daskalakis, Goldberg and Papadimitriou \cite{dgp} and Chen
and Deng \cite{cd} have proven its {\bf PPAD}-completeness, the former for
$n\geq 3$ and the latter for $n\geq 2$.

\begin{theorem}\label{nash-ppad-complete}
For $n\geq 2$, the problem {\sc $n$-Nash} is {\bf PPAD}-complete.
\end{theorem}
