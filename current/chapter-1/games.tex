\section{Normal Form Games and Nash Equilibria}

A {\em game}, first defined by von Neumann in \cite{vn28}, is a model
of strategic interaction. A {\em finite normal form game} is
$\Gamma=(P,S,u)$ where both $P$ and $S$ are finite.
The former is the set of {\em players}; $S=\times_{p\in P}$ is
the set of {\em pure strategy profiles} and $S_p$ is the set of
{\em pure strategies} of player $p$; we will use the
notation $S_{-p}=\times_{q\neq p}S_p$.
The purpose of each player $p\in P$ is to maximize their
{\em payoff function} $u^p:S\to\reals$, where $u=\times_{p\in P}u^p$.
In the following pages, by ``game'' we will always mean
``finite normal form game.'' If there are only two players, we will refer to
player 1 using feminine pronouns and to player 2 using masculine ones; such
games are called {\em bimatrix games} since they can be characterized by
the $m \times n$ payoff matrices $A$ and $B$, where $a_{ij}$ and $b_{ij}$ are
the payoffs of respectively player 1 and of player 2 when the former plays
her $i$th pure strategy and the latter plays his $j$th pure strategy.
A bimatrix game is {\em zero-sum} if $B=-A$, and {\em symmetric} if  $B=A\T$.

A {\em mixed strategy} of player $p$ is a probability
distribution on $S_p$; it can be described as a point on the
$(|S_p|-1)$-dimensional {\em mixed strategy simplex}
$\Delta_p=\{ x\in\reals^{|S_p|}\ \mid\ x\geq\0,\ \1\T x = 1 \}$.
The set of {\em mixed strategy profiles} is the simplicial polytope
$\Delta=\times_{p\in P}\Delta_p$;
we extend the payoff functions to $u^p:\Delta\to\reals$ linearly.

A {\em Nash equilibrium} of a game is a strategy profile in which each
player cannot improve their expected payoff by unilaterally changing their
strategy; such a strategy is called a {\em best response}.
Note that applying an affine transformation to all the payoffs does not
change the Nash equilibria of the game.

\begin{proposition}
\label{positive-p-br-thm}
A mixed strategy $x\in \Delta_p$ is a best response against some mixed
strategy profile $y\in \Delta_{-p}$ of the other players if and only if
every pure strategy $s_i\in S_p$ chosen with positive probability
in $x$ is a best response to $y$.
\end{proposition}

The existence of a Nash equilibrium is guaranteed by
the fundamental theorem by Nash (\cite{nash}). Note that there might be more
than one equilibrium.

\begin{theorem}{\rm (Nash \cite{nash})}\label{nash-thm}
Every finite game in normal form has a Nash equilibrium.
\end{theorem}

We give two classic examples of games: the prisoners' dilemma and a
coordination game.

\begin{example}
In the symmetric non zero-sum {\em prisoners' dilemma} of
Figure~\ref{prisoners-dilemma}, each player must
decide whether to ``help'' the other one or to ``betray'' them. If both
players help each other, they will get a small reward; if both betray, they
will pay a small penalty; if one betrays and the other cooperate the former
will get a large reward and the latter will pay a large penalty.

\begin{figure}[hbt]
\begin{center}
\def\mm#1{\makebox(0,0){\strut#1}}
\bimatrixgame{3mm}{2}{2}{1}{2}
{{\scriptsize betray}{\scriptsize help}}
{{\scriptsize betray}{\scriptsize help}}
{
\payoffpairs{1}{{1}{3}}{{1}{0}}
\payoffpairs{2}{{0}{2}}{{3}{2}}
}
\end{center}
\caption[The prisoners' dilemma]{The prisoners' dilemma.}
\label{prisoners-dilemma}
\end{figure}

The only equilibrium is the profile in which both players betray.
If player~2 betrays, the best response of player~1 is to betray, since
it gives her payoff 1 instead of 0; if player~2 helps, her payoff
for betraying is 3 and her payoff for helping is 2, so betraying is
again the best response. The same holds for player~2, so at the
equilibrium both players will betray.

Figure~\ref{coordination-game} shows a {\em coordination} game.
Both players drive on a mountain road; they lose if drive on the same side
of the road and win if they avoid each other, regardless of which side they
take.

\begin{figure}[hbt]
\begin{center}
\def\mm#1{\makebox(0,0){\strut#1}}
\bimatrixgame{3mm}{2}{2}{{{\small 1}}}{{{\small 2}}}
{{\scriptsize mountain}{\scriptsize valley}}
{{\scriptsize mountain}{\scriptsize valley}}
{
\payoffpairs{1}{{0}{1}}{{0}{1}}
\payoffpairs{2}{{1}{0}}{{1}{0}}
}
\end{center}
\caption[A coordination game]{A coordination game.}
\label{coordination-game}
\end{figure}

The pure strategy Nash equilibria are (mountain,valley) and
(valley,mountain); there is also a symmetric equilibrium in mixed strategies
at $((1/2,1/2),(1/2,1/2))$.
\end{example}
