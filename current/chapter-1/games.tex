\section{Normal Form Games and Nash Equilibria}

We now give the game-theoretic background that will be used in this thesis.
A {\em game}, as first defined by von Neumann in \cite{vn28}, is a model
of strategic interaction. A {\em finite normal form game} is
$\Gamma=(P,S=\times_{p\in P} S_p,u=\times_{p\in P}u^p)$
where both the set of {\em players} $P$ and the sets of
{\em pure strategies} $S_p$ (and therefore the set
of {\em pure strategy profiles} $S$) are finite. We will use the
notation $S_{-p}=\times_{q\neq p}S_p$.
The purpose of each player $p\in P$ is to maximize her {\em payoff function}
$u^p:S\to\reals$. In the following pages, by ``game'' we will always mean
``finite normal form game.'' If there are only two players, we will refer to
player 1 using feminine pronouns and to player 2 using masculine ones; such
games are called {\em bimatrix games} since they can be characterized by
the $m \times n$ payoff matrices $A$ and $B$, where $a_{ij}$ and $b_{ij}$ are
the payoffs of respectively player 1 and of player 2 when the former plays
her $i$th pure strategy and the latter plays his $j$th pure strategy.
A bimatrix game is {\em zero-sum} if $B=-A$, and {\em symmetric} if  $B=A\T$.

A {\em mixed strategy} of player $p$ is a probability
distribution on $S_P$; it can be described as a point
$x=(x^p_1,\ldots,x^p_{|S_p|})$ on the
$(|S_p|-1)$-dimensional {\em mixed strategy simplex}
$\Delta_p=\{ x\in\reals^{|S_p|}\ |\ x\geq\0,\ \1\T x = 1 \}$;
the set of {\em mixed strategy profiles} will be the simplicial polytope
$\Delta=\times_{p\in P}\Delta_p$.
We extend the payoff functions to $u^p:\Delta\to\reals$ by linearity.

A {\em Nash equilibrium} of a game is a strategy profile in which each
player cannot improve her expected payoff by unilaterally changing her
strategy; such a strategy is called a {\em best response}.
Note that applying an affine transformation to all the payoffs does not
change the Nash equilibria of the game.

\begin{proposition}
\label{positive-p-br-thm}
A mixed strategy $x\in \Delta_p$ is a best response against some mixed
strategy profile $y\in \Delta_{-p}$ of the other players if and only if
every pure strategy $s_i\in S_p$ chosen with positive probability
in $x$ is a best response to $y$.
\end{proposition}

The existence of a Nash equilibrium is guaranteed by
the fundamental theorem by Nash (\cite{nash}). Note that there might be more
than one equilibrium.

\begin{theorem}{\rm (Nash \cite{nash})}\label{nash-thm}
Every finite game in normal form has a Nash equilibrium.
\end{theorem}

We give three classic examples of games: matching pennies, the prisoners'
dilemma and a coordination game.

\begin{example}
Figure \ref{matching-pennies} shows the payoffs of the non-symmetric
zero-sum game {\em matching pennies}, with
payoff matrices $A=\binom{1\ 0}{0\ 1}$ and $B=-A$.

\begin{figure}[hbt]
\begin{center}
\def\mm#1{\makebox(0,0){\strut#1}}
\bimatrixgame{3mm}{2}{2}{1}{1}
{{$s^1_1$}{$s^1_2$}}
{{$s^2_1$}{$s^2_2$}}
{
\payoffpairs{1}{{1}{0}}{{-1}{0}}
\payoffpairs{2}{{0}{1}}{{0}{-1}}
}
\end{center}
\caption{Matching pennies.}
\label{matching-pennies}
\end{figure}

At the unique equilibrium of the game, each player follows the uniform
distribution over their strategies.

In the symmetric non zero-sum {\em prisoners' dilemma} of figure
\ref{prisoners-dilemma}, each player must
decide whether to ``help'' the other one or to ``betray'' them. If both
players help each other, they will get a small reward; if both betray, they
will pay a small penalty; if one betrays and the other cooperate the former
will get a large reward and the latter will pay a large penalty.

\begin{figure}[hbt]
\begin{center}
\def\mm#1{\makebox(0,0){\strut#1}}
\bimatrixgame{3mm}{2}{2}{1}{2}
{{\scriptsize betray}{\scriptsize help}}
{{\scriptsize betray}{\scriptsize help}}
{
\payoffpairs{1}{{1}{3}}{{1}{0}}
\payoffpairs{2}{{0}{2}}{{3}{2}}
}
\end{center}
\caption{The prisoners' dilemma.}
\label{prisoners-dilemma}
\end{figure}

The only equilibrium is the profile in which both players betray.
Assume that player 1 helps: then she must switch to betrayal, since she would
get $10$ instead of $5$ if player 2 helps and $-2$ instead of $-8$ if player 2
betrays. The same applies to player 2, so both players will betray.
The payoff matrices are $A=\binom{1\ 3}{0\ 2}$ and $B=A\T$.

Finally, in figure \ref{coordination-game}, a {\em coordination} game
with $A=B$:
both players drive on a mountain road; they lose if drive on the same side
of the road and win if they avoid each other, regardless of which side they
take.

\begin{figure}[hbt]
\begin{center}
\def\mm#1{\makebox(0,0){\strut#1}}
\bimatrixgame{3mm}{2}{2}{{{\small 1}}}{{{\small 2}}}
{{\scriptsize mountain}{\scriptsize valley}}
{{\scriptsize mountain}{\scriptsize valley}}
{
\payoffpairs{1}{{0}{1}}{{0}{1}}
\payoffpairs{2}{{1}{0}}{{1}{0}}
}
\end{center}
\caption{Coordination game.}
\label{coordination-game}
\end{figure}

Both (mountain,valley) and (valley,mountain) are Nash equilibria; in fact,
every mixed strategy where $x^1_1 = x^2_2$, where $x^p_i$ is the
probability that player $p$ assigns to strategy $s^p_i$, is a Nash
equilibrium.
\end{example}
