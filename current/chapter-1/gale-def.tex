\section{Cyclic Polytopes and Gale Strings}

In theorems \ref{unit-vector-thm} and \ref{unit-vector-dual-thm} we have
built a correspondence between labeled polytopes and unit vector games,
where Nash equilibria correspond to completely labeled vertices or facets.
We now focus on a particular kind of simplicial polytopes, called
{\em cyclic polytopes}, that can be represented as a combinatorial structure,
the {\em Gale strings}. We will first give the definition of cyclic polytope,
then of Gale string, then the theorem by Gale \cite{gale} about their
correspondence.

The {\em moment curve} in dimension $d$ is defined as
\begin{equation}
\mu_d:\reals\longrightarrow\reals^d,\quad
\mu_d:t\longmapsto (t,t^2,\ldots,t^d)\T .
\end{equation}
The {\em cyclic polytope} in dimension~$d$ with $n$ vertices, where $n>d$ is
\begin{equation}
C_d(n)=\conv\{ \mu_d(t_i)\text{ for }t_1<\ldots<t_n \}.
\end{equation}

Given $k\in\naturals$ and a set $S$, we can represent the function
$f:[k]\to S$ as the string $s=s(1)s(2)\cdots s(k)$; we have a {\em bitstring}
\todo{bitstring already def in section complexity?}
if $S=\{0,\1\}$. A maximal substring of consecutive
\1's in a bitstring is called a {\em run}; an {\em interior} run is
bounded on both sides by 0's. We will use the notation $\1^k$ for a
run of length $k$, and $0^k$ for a string of 0's of length $k$
A {\em Gale string of length $n$ and dimension $d$}, where $n>d$, is
a bitstring $s\in G(d,n)$ satisfying the following conditions:
\begin{enumerate}
\item exactly $d$ bits in $s$ are $\1$ and
\item ({\em Gale evenness condition})
\begin{equation}
0\1^k0\text{ is a substring of }s\quad
{\Longrightarrow}\quad
k\text{ is even.}
\end{equation}
\end{enumerate}

In general, the Gale evenness conditions allows for Gale strings that start
or end with an odd-length run; but if $d$ is even then $s$ can start with
an odd run if and only if it ends with an odd run.
We can then consider the Gale strings in $G(d,n)$ with even $d$ as a
the ``loops'' obtained by ``glueing together'' the extremes of the strings,
so that all runs on the loops are even.
Formally: we can see the indices of a Gale string $s\in G(d,n)$ with
$d$ even as equivalence classes modulo $n$, identifying $s(i+n)=s(i)$.
This also shows that the set of Gale strings of even dimension is invariant
under a cyclic shift of the strings.

\begin{example}\label{gs-example}
As an example of $d$ even, we have
\begin{align*}
G(4,6) = \{ & \1\1\1\100, \1\1\100\1, \1\100\1\1, \100\1\1\1, 00\1\1\1\1, \\
            & 0\1\1\1\10, \1\10\1\10, \10\1\10\1, 0\1\10\1\1 \}
\end{align*}

The strings $\1\1\1\100$, $\1\1\100\1$, $\1\100\1\1$, $\100\1\1\1$,
$00\1\1\1\1$ and $0\1\1\1\10$ are
equivalent under a cyclic shift (if considering the strings as ``loops'', the
$\1$'s are all consecutive), as are the strings $\1\10\1\10$, $\10\1\10\1$
and $0\1\10\1\1$ (if considering the strings as ``loops'', the even runs of
$\1$'s are two couples separated by a single $0$).

As an example for $d$ odd, we have
\[
G(3,5) = \{ \1\1\100, \10\1\10, \100\1\1, \1\100\1, 0\1\10\1, 00\1\1\1 \}
\]

Note how $0\10\1\1$ is a cyclic shift of $\10\1\10$, but it is not a Gale
string.
\end{example}

The relation between cyclic polytopes and Gale strings is given by the
following theorem by Gale \cite{gale}.

\begin{theorem}{\rm (Gale \cite{gale})}\label{cp-gs-thm}
For any positive integers $d,n$ with $n>d$\begin{align}
F \text{ is a} & \text{ facet of } C_d(n) \nonumber \\
& \Longleftrightarrow \nonumber \\
F = \conv\{ \mu(t_j)\ |\ s(j)=1 & \text{ for some }j\in[n]
\text{ and }s\in G(d,n) \}.
\end{align}

\end{theorem}

\todo[inline]{sketch of pf - see Ziegler - with drawing of moment curve +
hyperplane

CP simplicial

CP does not depend on the choice of $t_i$'s

\small
Essentially, this holds because any set $S\subset [n]$
the moment curve defines a unique hyperplane which is crossed
(and not just touched) by the moment curve; if the bitstring
$s$ that encodes $F$ as $1(s)$ has a substring $01^k0$
\normalsize

example of cyclic polytope + equivalent gale string (a simple one)}

From this point forward, we will assume that $d$ is even.
We will also assume that the labeling $l:[n]\to [d]$ is such
that $l(i)\neq l(i+1)$; this can be done without loss of generality,
given the following consideration.
Suppose that $l(i)=l(i + 1)$ for some index $i$, and let $s$ be a
completely labeled Gale string for $l$. Then only one of
$s(i)$ and $s(i+1)$ can be equal to \1 (note that it's possible that both
are 0s). So $s(i)s(i+1)$ will never be a run of even length that
``interferes'' with the Gale Evenness Condition, so we can ``simplify''
by identifying the indices $i$ and $i + 1$.

Theorem \ref{cp-gs-thm} gives a correspondence between Gale strings and
facets of cyclic polytopes; we have also seen that these polytopes are
simplicial.
On the other hand, theorem \ref{unit-vector-dual-thm} gives a correspondence
between completely labeled facets of a simplicial polytopes and Nash
equilibria of unit vector games. To exploit these connections, we now
give a definition of labeling for Gale strings that will allow us to
study the Nash equilibria of a unit vector game for which the best response
polytope is the dual of a cyclic polytope (recall that the polytope in
theorem \ref{unit-vector-thm} is the best response polytope, whereas
theorem \ref{unit-vector-dual-thm} describes its dual version). This might
seem a very specific case of bimatrix game; we will see in the next chapter
that it leads to very interesting results.
We therefore need a definition of ``completely labeled'' for Gale strings,
and a labeling $l_s$ for $G(d,n)$ such that $s\in G(d,n)$ is completely
labeled if and only if the corresponding facet in $C_d(n)$ is completely
labeled by $l_v$ as given in theorem \ref{unit-vector-dual-thm}.

We say that $s\in G(d,n)$ is a {\em completely labeled Gale string} if
for some labeling function $l_s:[n]\to[d]$ the set
$\{ i\in [n]\ |\ s(i)=\1 \}$ is completely labeled by $l_s$.
Since $s\in G(d,n)$ has exacty $d$ bits equal to \1, this means that for
each $j\in [d]$ there is exactly one
$i\in [n]$ such that $s(i)=\1$ and $l_s(i)=j$.
Note that it is not always possible to find a completely labeled Gale string.

\begin{example}
For $l = 121314$, there are no completely labeled Gale strings.

The labels $l(i)=2,3,4$ appear only once in $l$, as $l(2),l(4),l(6)$
respectively; therefore we must have $s(2)=s(4)=s(6)=1$. For every other
$i\in [n]$ we have $l(i)=1$, so we have $l(i)=1$ for exactly one $i=1,3,5$.
The candidate strings are then \1\10\10\1, 0\1\1\10\1, 0\10\1\1\1; but
none of these satisfies the Gale evenness condition.
\end{example}

Let $(U,B)$, where $U=(e_{l(1),\ldots,e_{l(d)}})$ for some labeling
$l:[n]\to [d]$, be a unit vector game for which the dual of the best
response polytope is a cyclic polytope
$Q=\conv\{ e_1,\ldots,e_d,c_1,\ldots,c_n \}$. Theorem
\ref{unit-vector-dual-thm} gives a labeling $l_v$ of the $d+n$ vertices of
$Q$ as in \ref{vert-labeling-unitv}:
\begin{eqnarray*}
l_v(-e_i)=i\text{ for }i\in [m]; \\
l_v(c_j)=l(j)\text{ for }j\in [n].
\end{eqnarray*}
Let the labeling $l_s:[d+n]\to [d]$ be defined as follows:
\begin{eqnarray}\label{gs-labeling-unitv}
l_s(i)=i\text{ for }i\in [d]; \nonumber \\
l_s(d+j)=l(j)\text{ for }j\in [n].
\end{eqnarray}
Then the Gale strings $s\in G(d,d+n)$ that are completely labeled for $l_s$
correspond exactly to the completely labeled facets of $Q$, with the
facet $F_0$ corresponding to the ``trivial'' completely labeled
string $\1^d 0$.

\begin{example}
Given the string of labels $l=123432$, there are four associated completely
labeled Gale strings: $\1\1\1\100$, $\1\10\1\10$, $\100\1\1\1$ and
$\10\1\10\1$.

\begin{tabular}{c @{ } c @{ } c @{ } c @{ } c @{ } c @{ } c}
{\bf 1} & {\bf 2} & {\bf 3} & {\bf 4} & {\bf 3} & {\bf 2} \\
\hline
\1 & \1 & \1 & \1 & \tdot & \tdot \\
\1 & \1 & \tdot & \1 & \1 & \tdot \\
\1 & \tdot & \tdot & \1 & \1 & \1 \\
\1 & \tdot & \1 & \1 & \tdot & \1
\end{tabular}

\todo[inline]{draw polytope}
\end{example}

From a computational point of view, we can define the problem \anothergale\
and {\sc Unit Vector Cyclic Nash} as follows:

\begin{fctproblem}
{\anothergale}
{A labeling $l:[n]\to[d]$, where $d$ is even and $d<n$.
A Gale string $s\in G(d,n)$, completely labeled by $l$.}
{A Gale string $s'\in G(d,n)$, completely labeled by $l$,
such that $s' \neq s$.}
\end{fctproblem}

\begin{fctproblem}
{Unit Vector Cyclic Nash}
{A unit vector game $\Gamma$ with dual cyclic best response polytope.}
{A Nash equilibrium of $\Gamma$.}
\end{fctproblem}

\todo[inline]{prove that encoding GS is poly}

Then, by proposition \ref{ne-cl-pbl}, we have the following theorem.

\begin{proposition}\label{ne-another-gale-pbl}
The problem {\sc Unit Vector Cyclic Nash} is polynomial-time reducible
to the problem \anothergale.
\end{proposition}
