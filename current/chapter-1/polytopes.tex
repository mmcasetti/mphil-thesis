\section{Preliminary definitions}

We denote the transpose of a matrix $A$ as $A\T$.
We consider vectors $u,v\in\reals^d$ as column vectors, so $u\T v$ is
their scalar product. A vector in $\reals^d$ for which all components
are $0$'s will be denoted as $\0$; a vectors for which all
components are $1$'s will be denoted as $\1$; the {\em $i$-th unit vector},
for which all the components are 0 except for the $i$-th component equal to
1, will be denoted as $e_i$.
An inequality of the form $u\geq v$, is intended to hold for every component.

An {\em affine combination} of points in an Euclidean space
$\{z_1,\ldots,z_m \}\subset \real^n$
is $\sum_{i=1}^m \lambda_i z_i$ where $\lambda_i\in\reals$ such that
$\sum_{i=1}^m \lambda_i = 1$;
the points $z_1,\ldots,z_m$ are {\em affinely independent} if none of them
is an affine combination of the others.
The {\em convex hull} of the points $z_1,\ldots,z_m$ is an affine
with $\lambda_i\geq 0$ for all $i\in [n]$; that is
$\conv \{z_1,\ldots,z_m \}=\{ \sum_i \lambda_i z_i\ |\ %
\lambda_i\geq 0,\ \sum_i \lambda_i=1$
A set of points $Z=\{z_1,\ldots,z_m \}$ is {\em convex} if $Z=\conv(Z)$;
a convex set has {\em dimension} $d$ if it has exactly $d + 1$
affinely independent points; in this case we say that $Z$ is a
{\em $d$-simplex}. The {\em standard $d$-simplex} is denoted
$\Delta_d=\conv\{ e_1,\ldots,e_{d+1} \}$.

In general, a {\em polytope} is the convex hull of a finite set of points
(not necessarily affinely independent)
$\{z_1,\ldots,z_m \}\subset\reals^n$; the
{\em dimension} of the polytope is the dimension of its convex hull.
A {polyhedron} is the intersection of finitely many closed halfspaces
$\{ x\in \reals^d\mid a^T x\leq a_0\}$; a bounded polyhedron is, again,
a polytope.

A {\em vertex} of a d-dimensional polytope $P=\conv(Z)$ is a
point $z\in Z$ such that $\conv(Z\setminus \{ z \})\neq P$;
an {\em edge} of P is a 1-dimensional line segment that has two vertices
as endpoints.
A {\em facet} of $P$ is the convex hull of a set of $d$ vertices
$F=\{ z_1,\ldots,z_d \}$ that lie on a hyperplane of the form
$\{ x\in \reals^d\mid a^T x=a_0\}$ so that $a^T u<a_0$ for all other vertices
$u$ of $P$; the vector $a$ (unique up to a scalar multiple) is called the
{\em normal vector} of the facet.

A $d$-dimensional {\em simplicial polytope} $P$ is the convex hull of a set
of at least $d+1$ points $v\in \reals^d$ such that no $d+1$ of them are on
a common hyperplane; this is equivalent to requiring that every facet of
$P$ is a $d$-simplex. A $d$-dimensional polytope $P$ is {\em simple}
if every point of $P$ lies on at most than $d$ facets; the points that
lie on exactly $d$ facets will be the vertices.

The {\em polar} of the polytope
$P=\{ x\in\reals^d\ |\ x\T c_i\leq 1,\ i\in [k] \}$ with $c_i\in\reals^d$
is $P^\Delta = \conv\{ c_i, i\in [k] \}$.
