This thesis centers on a problem in the field of
{\em algorithmic game theory}, which concerns the study of
strategic interactions from the point of view of computer science.
Our result consists in an algorithm to find a Nash equilibrium in
polynomial time for games in a special class, called Gale games.

Gale games can be represented through
a combinatorial structure called Gale strings (Gale \cite{gale}),
using a construction (Savani and von Stengel \cite{svs}) on
labeled polytopes derived by the game itself (Lemke and Howson \cite{lh},
Shapley \cite{shapley}). This connection was used to build games for which
the classic Lemke-Howson Algorithm (Lemke and Howson \cite{lh}) takes
exponential running time to find a Nash equilibrium (Savani and
von Stengel \cite{svs}).
The Lemke-Howson Algorithm gives a quite straightforward proof of
how finding a Nash equilibrium of any two-player game is a problem in the
class {\bf PPAD} (Proof By Parity Argument, Directed version;
Papadimitriou \cite{ppad}); its {\bf PPAD}-completeness has
been proven (Daskalakis, Goldberg
and Papadimitriou \cite{dgp}, Chen and Deng \cite{cd}), but the result
required the use of approximated $\epsilon$-Nash equilibria.
We conjectured that the combinatorial representation of Gale games could
be exploited to give an alternative and purely discrete proof of
{\bf PPAD}-completeness. Eventually, our
result turned out to point in the opposite direction: unless
\mbox{{\bf PPAD = P}}, Gale games are indeed a case apart, but because of their
tractability, not of their hardness.

The remainder of this chapter will cover some basic definitions and notation
that will be used throughout the thesis, concerning polytopes
(see Ziegler~\cite{ziegler} for details), basic game theory (see Osborne and
Rubinstein \cite{gth})
and computational complexity (see Papadimitriou \cite{papad-cc}). Chapter 2 will deal with the geometric and
combinatorial constructions leading to the definition of Gale games: in the
first section we will see Nash equilibria of 2-player games as facets or
vertices of polytopes built from the game itself; in the second section
we will define Gale games, for which these polytopes belong to a special
class, and we will translate the framework of the previous section to
the point of view of Gale strings.
In Chapter 3 we will tackle the computational complexity of the
issues arising from the previous chapter: after an introduction to the
classes related to proofs by parity argument, we will present different
versions of the Lemke-Howson algorithm; finally, in the last section, we
will present our original result, solving Gale games in polynomial time.
A last chapter will relate further results in the field and open problems.
