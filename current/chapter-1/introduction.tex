This thesis centers on a problem in the field of
{\em algorithmic game theory}, which concerns the study of
strategic interactions from the point of view of computer science.
In particular, we will focus on a class of games, called Gale games, and
we will give an algorithm to find their equilibrium in polynomial time.

Gale games can be represented through
a combinatorial structure called Gale strings (Gale \cite{gale}),
using a construction (Savani and von Stengel \cite{svs}) on
labeled polytopes derived by the game itself (Lemke and Howson \cite{lh},
Shapley \cite{shapley}). This connection was used to build games for which
the classic Lemke-Howson Algorithm (Lemke and Howson \cite{lh}) takes
exponential running time to find a Nash equilibrium (Savani and
von Stengel \cite{svs}).
On the other hand, considering the Lemke-Howson Algorithm from the point
of view of computational complexity gives a quite straightforward proof of
how finding a Nash equilibrium of any normal-form game is a problem in the
class {\bf PPAD} (Proof By Parity Argument, Directed version;
Papadimitriou \cite{ppad}); even more, it was one of
the motivations
that led to the definition of the class itself. After the proof of
the {\bf PPAD}-completeness of this problem ( Daskalakis, Goldberg
and Papadimitriou \cite{dgp}, Chen and Deng \cite{cd}),
we conjectured that the Gale games could be exploited to give an
alternative proof that circumvents the approximated $\epsilon$-Nash
equilibria of Daskalakis, Goldberg and Papadimitriou. Eventually, our
result turned out to point in the opposite direction: unless
{\bf PPAD = P}, Gale games are indeed a case apart, but because of their
tractability, not of their hardness.

The remainder of this chapter will cover some basic definitions and notation
that will be used throughout the thesis, concerning polytopes
(see Ziegler~\cite{ziegler} for further details), basic game theory
and computational complexity. Chapter 2 will deal with the geometric and
combinatorial constructions leading to the definition of Gale games: in the
first section we will see Nash equilibria of 2-player games as facets or
vertices of polytopes built from the game itself; in the second section
we will focus on Gale games, for which these polytopes belong to a special
class, translating the framework to the point of view of Gale strings.
In Chapter 3 we will tackle the computational complexity of the
issues arising from the previous chapter: after an introduction to the
classes related to proofs by parity argument, we will present different
versions of the Lemke-Howson algorithm; finally, in the last section, we
will present our original result, solving Gale games in polynomial time.
A last chapter will relate further results in the field and open problems.
