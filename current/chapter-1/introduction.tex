This thesis centers on a problem in the field of
{\em algorithmic game theory}, which concerns the study of
strategic interactions from the point of view of computer science.
Our result is an algorithm to find a Nash equilibrium in
polynomial time for games in a special class, called Gale games.

Gale games can be represented through
a combinatorial structure called Gale strings (Gale \cite{gale}),
using a construction (Savani and von Stengel \cite{svs}) on
labeled polytopes derived from the game itself (Lemke and Howson \cite{lh},
Shapley \cite{shapley}). This connection was used to
construct games for which
the classic Lemke-Howson Algorithm (Lemke and Howson \cite{lh}) takes
exponential running time to find a Nash equilibrium (Savani and
von Stengel \cite{svs}).
The Lemke-Howson Algorithm gives a quite straightforward proof of
how finding a Nash equilibrium of any two-player game is a problem in the
class {\bf PPAD} (Polynomial Parity Argument, Directed version;
Papadimitriou \cite{ppad}). The {\bf PPAD}-completeness of the problem has
been proven (Daskalakis, Goldberg
and Papadimitriou \cite{dgp}, Chen and Deng \cite{cd}), but the result
required the use of approximate $\epsilon$-Nash equilibria.
We conjectured that the combinatorial
representation of Gale games could
be exploited to give an alternative and purely discrete proof of
{\bf PPAD}-completeness. Eventually, our
result turned out to point in the opposite direction: unless
\mbox{{\bf PPAD = P}}, Gale games are indeed a case apart, but because of their
tractability, not because of their hardness.

The remainder of this chapter will cover some basic definitions and notation
that will be used throughout the thesis, concerning polytopes
(see Ziegler~\cite{ziegler} for details) in section \ref{polytopes-sect},
basic game theory (see Myerson \cite{gth}) in section \ref{gth-sect}, and
computational complexity (see Papadimitriou \cite{papad-cc}) in section
\ref{cc-sect}. Chapter \ref{def-chapt} will deal
with the geometric and combinatorial constructions leading to the definition
of Gale games. In section \ref{labels-sect} we will see the
representation of Nash equilibria of 2-player games as facets or vertices of
polytopes built from the game itself. In section \ref{gale-def-sect}
we will define Gale strings and give the proof of Gale's theorem,
showing the representation of cyclic polytopes as Gale strings.
Section \ref{gale-games-sect} will then translate the framework of
section \ref{labels-sect} to the point of view of Gale strings,
introducing Gale games.
In Chapter~\ref{main-chapt} we will tackle the computational complexity
of the issues arising from the previous chapter. After an introduction to the
complexity classes related to proofs by parity argument in section
\ref{ppad-sect}, we will present different versions of the Lemke-Howson
algorithm in section \ref{lh-sect}. Finally, in section \ref{main-sect},
we will present our original result, solving Gale games in
polynomial time. A last chapter will relate further results in the field and
open problems.
