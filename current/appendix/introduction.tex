
The topic of this thesis is a problem in the field of
{\em algorithmic game theory}, that is, the study of game-theoretic problems
from the point of view of computer science. In particular, we focus on the
computational complexity of a particular class of games. These

General refs for comp compl \cite{papad-cc}

General refs for geometry \cite{ziegler}


from SvS15

Savani and von Stengel (2006) showed that the LH algorithm may take exponentially many steps. Their construction uses “dual cyclic polytopes” which have a well-known vertex structure for any dimension and number of linear inequalities. Morris (1994) used similarly labeled dual cyclic polytopes where all “Lemke paths” are exponentially long. A Lemke path is related to the path computed by the LH algorithm, but is defined on a single polytope that does not have a product structure corresponding to a bimatrix game. The completely labeled vertex found by a Lemke path can be interpreted as a symmetric equilibrium of a symmetric bimatrix game. However, as in the example in Figure 4 below, such a symmetric game may also have nonsymmetric equilibria which here are easy to compute, so that the result by Morris (1994) seemed not suitable to describe games that are hard to solve with the LH algorithm.

The “imitiation games” defined by McLennan and Tourky (2010) changed this picture. In an imitation game, the payoff matrix of one of the players is the identity matrix. The mixed strategy of that player in any Nash equilibrium of the imitation game corresponds exactly to a symmetric equilibrium of the symmetric game defined by the payoff matrix of the other player. In that way, an algorithm that finds a Nash equilibrium of a bimatrix game can be used to find a symmetric Nash equilibrium of a symmetric game.

In one sense the two-polytope construction of Savani and von Stengel (2006) was overly com- plicated: the imitation games by McLennan and Tourky (2010) provide a simple and elegant way to turn the single-polytope construction of Morris (1994) into exponentially-long LH paths for bimatrix games. In another sense, the construction of Savani and von Stengel was not redun- dant, since it provided examples that are simultaneously bad for the LH algorithm and “support enumeration”, which is another natural and simple algorithm for finding equilibria. The support of a mixed strategy is the set of pure strategies that are played with positive probability. Given a pair of supports of equal size, the mixed strategy probabilities are found by equating all payoffs for the other player’s support, which then have to be compared with payoffs outside the support to establish the equilibrium property (see Dickhaut and Kaplan, 1991).


In this paper, we extend the idea of imitation games to games where one payoff matrix is arbi- trary and the other is a set of unit vectors. We call these unit vector games.

we use them to extend Morris’s con- struction to give bimatrix games that use only one dual cyclic polytope, rather than the two used by Savani and von Stengel, and for which both the LH algorithm and support enumeration are simultaneously bad. This result (Theorem 11) was first described by Savani (2006, Section 3.8).



summary of chapt 2:

We begin this section with the definition of almost complete labeling; we
then move on to the classic version of the Lemke-Howson algorithm
for the problem {\sc Another Completely Labeled Vertex}, as given
in the beautiful exposition by Shapley \cite{shapley}, and its dual
version for {\sc Another Completely Labeled Facet}. Finally, we present
the Lemke-Howson for Gale algorithm.
In the next session we will tackle the
issue of the computational complexity of these algorithms:
{\sc Another Completely Labeled Facet} and
{\sc Another Completely Labeled Vertex} are {\bf PPA},
{\sc Nash} is {\bf PPAD}, as first shown in Papadimitriou \cite{ppad};
furhermore, as shown by Morris \cite{morris} and by
Savani and von Stengel \cite{svs}, there are cases of exponential running
time. This had led us to conjecture that these problems could be exploited
for a proof of {\bf PPAD} completeness, also considering that finding
a completely labeled facet (or vertex, or the existence of a Nash
equilibrium) is {\bf NP} in the case of a general labeled polytope, as
proven by von Stengel \cite{vs-np-facet}
\todo{check proof, citation}
. In the last section we will finally present our original
\todo{original? main?}
result, that goes in the opposite direction: the problem \anothergale
can be solved in polynomial time, that is, it is a problem in {\bf TFP}.
Unit vector games with dual cyclic best response polytope present therefore
a case apart, as expected, but not because they are harder than others, but
because they are easier.
