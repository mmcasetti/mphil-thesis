
The topic of this thesis is a problem in the field of
{\em algorithmic game theory}, that is, the study of game-theoretic problems
from the point of view of computer science. In particular, we focus on the
computational complexity of a particular class of games. These

General refs for comp compl \cite{papad-cc}

General refs for geometry \cite{ziegler}


summary of chapt 2:

We begin this section with the definition of almost complete labeling; we
then move on to the classic version of the Lemke-Howson algorithm
for the problem {\sc Another Completely Labeled Vertex}, as given
in the beautiful exposition by Shapley \cite{shapley}, and its dual
version for {\sc Another Completely Labeled Facet}. Finally, we present
the Lemke-Howson for Gale algorithm.
In the next session we will tackle the
issue of the computational complexity of these algorithms:
{\sc Another Completely Labeled Facet} and
{\sc Another Completely Labeled Vertex} are {\bf PPA},
{\sc Nash} is {\bf PPAD}, as first shown in Papadimitriou \cite{ppad};
furhermore, as shown by Morris \cite{morris} and by
Savani and von Stengel \cite{svs}, there are cases of exponential running
time. This had led us to conjecture that these problems could be exploited
for a proof of {\bf PPAD} completeness, also considering that finding
a completely labeled facet (or vertex, or the existence of a Nash
equilibrium) is {\bf NP} in the case of a general labeled polytope, as
proven by von Stengel \cite{vs-np-facet}
\todo{check proof, citation}
. In the last section we will finally present our original
\todo{original? main?}
result, that goes in the opposite direction: the problem \anothergale
can be solved in polynomial time, that is, it is a problem in {\bf TFP}.
Unit vector games with dual cyclic best response polytope present therefore
a case apart, as expected, but not because they are harder than others, but
because they are easier.
