
Advantage of two dual cyclic polytopes as in SvS 06 over only one
(and without "prefix" in labels for gale string) as seen in SvS 15 is:
good example of exp on supports - see SvS 15

thanks to dual of theorem SvS-15 \ref{unit-paths}, when doing labeling
we can take the str of labels $l(n+j)\cdots l(n+m)$
instead of $l(1)\cdots l(n+m)$,
that is, we could cut the ``artificial'' first labels $12...n$.

%l_s(i)=i\text{ for }i\in [d] \\
%l_s(d+j)=l(j)\text{ for }j\in [n].

After all, in main we're studying ANOTHER GALE
in general, not nec starting from $12...n$; and we're interested in finding
{\em one} eq that's not the one we started from
(and is at other end of LPath, since index and so on),
{\em not all equilibria};
but the eq we started from is not nec the artificial one - actually,
if we go with this we can take any NE to start looking for another,
and we're sure to find a ``non-artificial'' one.
Note: if we were looking for all NE, LH doesn't work anyway - see ex by
Wilson in Shapley, where ``disconnected'' paths between equilibria.


\# P complexity of finding all (?)
