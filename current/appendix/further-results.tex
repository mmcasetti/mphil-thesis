Algorithm \ref{anothergale-alg} allows us to find a Nash equilibrium of a
Gale game in polynomial time starting from another equilibrium (usually the
artificial one), but it doesn't give any information on the relationship
of these equilibria as endpoints of the Lemke-Howson algorithm. In
particular, the issue of the sign of the equilibrium is left open.
Following the construction in Proposition \ref{lhg-works-ppad-thm},
we can reduce
the problem ``given a Nash equilibrium of a Gale game, find another one
of opposite sign'' to the problem {\sc Opposite Sign Gale String} of
Table \ref{opposite-gale}.
Analogously, it is straightforward to see that these problems
belongs to the class {\bf PPADS}, the equivalent of {\bf PPAD} where
the solution is restricted to sinks of the {\sc End Of The Line} graph;
Daskalakis, Goldberg and Papadimitriou \cite{dgp} conjecture that this class
is larger than {\bf PPAD}.

\begin{problem}
{Opposite Sign Gale String}
{A labeling $l:[n]\to [d]$ and a completely labeled Gale
string $s_0\in G(d,n)$.}
{A completely labeled Gale string $s\in G(d,n)$ such
that $\sign(s)=-\sign(s_0)$.}
\label{opposite-gale}
\end{problem}

A first polynomial-time algorithm for {\sc Opposite Sign Gale String} was
given by Merschen \cite{jm} for labelings for which the Gale graph of
Theorem \ref{gale-thm} is planar; a general result has since been given by
V\'{e}gh and von Stengel \cite{vvs}. The latter relies on the definition of
a general framework to deal with pivoting algorithms, called
{\em complementary pivoting with direction}, that is applied not only
to perfect matchings of a Gale graph, but to the
wider class of Euler complexes and their room partitions,
first introduced by Edmonds \cite{edm-oiks} and
Edmonds and Sanit\`{a} \cite{edmonds-sanita}.

A {\em $d$-dimensional Euler complex} $(V,M)$
(in the following: {\em $d$-oik})
over a finite set of nodes $V$ is a collection of
sets $M=\{ R\ \mid\ R\subset V,\ |R|=d \}$, called {\em rooms},
such that any set of $d-1$ nodes,
called {\em wall}, is contained in an even number of rooms;
a {\em room partitioning} of $(V,M)$ is $P\subset M$ such that
each vertex of $V$ is in exactly one room in $P$.
Edmonds and Sanit\'{a} used an ``exchange algorithm'' and a parity argument
to show that there is an even number of room partitions of $(V,M)$.
This can be applied to a family of two oiks (of possibly different dimension)
corresponding to the best response polytopes of two players in a bimatrix
game, with the room partitions corresponding to equilibria; the Lemke–Howson
Algorithm is then a special case of the exchange algorithm. Another special
case is an Euler graph: the edges are rooms, and perfect matchings are then
room partitions.
The connections between oiks and the topic of our study could appear
trivial; the issue of giving a sign to room partitions, unfortunately, is
much more complex than expected.

\begin{example}
Consider the octahedron with vertices labeled as in
Figure \ref{octa-poly-fig}, left; the endpoints of the Lemke paths
in the Dual Lemke-Howson Algorithm \ref{lh-dual-alg} when dropping
different labels is shown in
Figure \ref{octa-poly-fig}, right. Notice that this graph is bipartite:
this corresponds to a parity argument with sign similar to Proposition
\ref{lhg-works-ppad-thm}.

On the other hand, consider the same octahedron from the point of view of
room partitions, Figure \ref{octa-rp-fig}, left. The exchange algorithm
of Edmonds \cite{edm-oiks} from room partition $B$ to room partition $A$
dropping vertex $v=1$
is shown in Figure \ref{exchange-rp-1-fig} and
Figure \ref{exchange-rp-2-fig}. The graph describing the
endpoints of the exchange algorithm analogous to
Figure \ref{octa-poly-fig}, right, is shown in Figure
\ref{octa-rp-fig}, right, and it is not bipartite.
Giving a sign to the room partitions cannot be done simply through
Edmonds' exchange algorithm.

\begin{figure}[hp]
\strut\hfill
\includegraphics[height=35ex]{appendix/fig/octa-poly-facets-v2.pdf}%
\hfill
\includegraphics[height=35ex]{appendix/fig/octa-poly-paths-v2.pdf}%
\hfill\strut
\caption[Endpoints of the Dual Lemke-Howson Algorithm on an octahedron]
{Left: A labeled octahedron and its completely labeled facets.\\
Right: The endpoints of its Lemke paths.}
\label{octa-poly-fig}
\end{figure}

\begin{figure}[hp]
\strut\hfill
\includegraphics[height=35ex]{appendix/fig/octa-rp.pdf}%
\hfill
\includegraphics[height=35ex]{appendix/fig/octa-rp-paths.pdf}%
\hfill\strut
\caption[Endpoints of the Exchange Algorithm on an octahedron]
{Left: An octahedron and its room partitions.\\
Right: The endpoints of its exchange algorithm paths.}
\label{octa-rp-fig}
\end{figure}

\clearpage

\begin{figure}[p]
\strut\hfill
\includegraphics[height=45ex]{appendix/fig/octa-exchange-1.pdf}%
\hfill\strut
\caption[The Exchange Algorithm on an octahedron - first step]
{Drop vertex $1$ from the room partition $B$ (green), leave
room $B_1$ (light green), enter room $A_1$ (blue), picking up vertex $4$.
The new vertex is duplicate in the new room $A_1$ and in room $B_2$.}
\label{exchange-rp-1-fig}
\end{figure}

\begin{figure}[p]
\strut\hfill
\includegraphics[height=43ex]{appendix/fig/octa-exchange-2.pdf}%
\hfill\strut
\caption[The Exchange Algorithm on an octahedron - end]
{Drop vertex $4$, leave room $B_2$ (light green), enter
room $A_2$ (blue), picking up verte $1$, missing.}
\label{exchange-rp-2-fig}
\end{figure}

\clearpage


\end{example}
