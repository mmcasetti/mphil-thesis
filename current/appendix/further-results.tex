Algorithm \ref{anothergale-alg} allows us to find a Nash equilibrium of a
Gale game in polynomial time starting from another equilibrium (usually the
artificial one), but it doesn't give any information on the relationship
of these equilibria as endpoints of the Lemke-Howson algorithm. In
particular, the issue of the sign of the equilibrium is left open.
Following the construction in Proposition \ref{lhg-works-ppad-thm},
we can reduce
the problem ``given a Nash equilibrium of a Gale game, find another one
of opposite sign'' to the problem {\sc Opposite Sign Gale String} of
Table \ref{opposite-gale}.
Analogously, it is straightforward to see that these problems
belongs to the class {\bf PPADS}, see Daskalakis, Goldberg
and Papadimitriou \cite{dgp}; this is the equivalent of {\bf PPAD} where
the solution is restricted to sinks of the {\sc End Of The Line} graph.

\begin{problem}
{Opposite Sign Gale String}
{A labeling $l:[n]\to [d]$ and a completely labeled Gale
string $s_0\in G(d,n)$.}
{A completely labeled Gale string $s\in G(d,n)$ such
that $\sign(s)=-\sign(s_0)$.}
\label{opposite-gale}
\end{problem}

A first polynomial-time algorithm for {\sc Opposite Sign Gale String} of
Table \ref{opposite-gale} was given by Merschen \cite{jm} in the case of
labelings for which the Gale graph of
Theorem \ref{gale-thm} is planar; a general result has since been given by
V\'{e}gh and von Stengel \cite{vvs}. The latter relies on the definition of
a general framework to deal with pivoting algorithms, called
{\em Complementary Pivoting with Direction Algorithm}, that is defined not
only for perfect matchings of a Gale graph, but for the
wider class of room partitions of Euler complexes.

A {\em $d$-dimensional Euler complex} $(V,M)$
(in the following: {\em $d$-oik})
over a finite set of nodes $V$,
as first introduced by
Edmonds \cite{edm-oiks} and Edmonds and Sanit\`{a} \cite{edmonds-sanita},
is a collection of
sets $M=\{ R\ \mid\ R\subset V,\ |R|=d \}$, called {\em rooms},
such that any set of $d-1$ nodes,
called {\em wall}, is contained in an even number of rooms;
a {\em room partitioning} of $(V,M)$ is $P\subset M$ such that
each vertex of $V$ is in exactly one room in $P$.
Edmonds and Sanit\'{a} used an ``exchange algorithm'' and a parity argument
to show that there is an even number of room partitions of $(V,M)$.
This can be applied to a family of two oiks (of possibly different dimension)
corresponding to the best response polytopes of two players in a bimatrix
game, with the room partitions corresponding to equilibria; the Lemke-Howson
Algorithm is then a special case of the exchange algorithm. Another special
case is an Euler graph: the edges are rooms, and perfect matchings are then
room partitions.
The connections between oiks and the topic of our study could appear
trivial; but the issue of giving a sign to room partitions, unfortunately,
is much more complex than that.

\begin{example}
Consider the octahedron with vertices labeled as in
Figure \ref{octa-poly-facets-fig}; the endpoints of the Lemke paths
in the Dual Lemke-Howson Algorithm \ref{lh-dual-alg} when dropping
different labels is shown in
Figure \ref{octa-poly-paths-fig}. Notice that this graph is bipartite:
this corresponds to a parity argument with sign similar to Proposition
\ref{lhg-works-ppad-thm}.

\clearpage

\begin{figure}[p]
\strut\hfill
\includegraphics[width=70ex]{appendix/fig/octa-poly-facets-v2.pdf}%
\hfill\strut
\caption[A labeled octahedron]
{A labeled octahedron and its completely labeled facets.}
\label{octa-poly-facets-fig}
\end{figure}

\begin{figure}[p]
\strut\hfill
\includegraphics[width=60ex]{appendix/fig/octa-poly-paths-v2.pdf}%
\hfill\strut
\caption[Endpoints of the Dual Lemke-Howson Algorithm]
{The endpoints of the Lemke paths on the octahedron of
Figure \ref{octa-poly-facets-fig}.}
\label{octa-poly-paths-fig}
\end{figure}

\clearpage

On the other hand, consider the same octahedron from the point of view of
room partitions, Figure \ref{octa-rp-fig}, left. The exchange algorithm
of Edmonds \cite{edm-oiks} from room partition $B$ to room partition $A$
dropping vertex $v=1$
is shown in Figure \ref{exchange-rp-1-fig} and
Figure \ref{exchange-rp-2-fig}. The graph describing the
endpoints of the exchange algorithm, analogous to
Figure \ref{octa-poly-paths-fig}, is shown in Figure
\ref{octa-rp-fig}, right, and it is not bipartite.
Giving a sign to the room partitions cannot be done simply through
Edmonds' exchange algorithm.

\begin{figure}[hp]
\strut\hfill
\includegraphics[height=60ex]{appendix/fig/octa-exchange-1.pdf}%
\hfill\strut
\caption[The Exchange Algorithm on an octahedron - first step]
{Drop vertex $1$ from the room partition $B$ (green). Leave
room $B_1$ (light green); enter room $A_1$ (blue), picking up vertex $4$.
The new vertex is duplicate in the new room $A_1$ and in
room $B_2$ (dark green).}
\label{exchange-rp-1-fig}
\end{figure}

\begin{figure}[hp]
\strut\hfill
\includegraphics[height=75ex]{appendix/fig/octa-exchange-2.pdf}%
\hfill\strut
\caption[The Exchange Algorithm on an octahedron - end]
{Drop duplicate vertex $4$. Leave the old room $B_2$ (light green); enter
room $A_2$ (blue), picking up verte $1$. The new vertex is the missing
one; end.}
\label{exchange-rp-2-fig}
\end{figure}
\clearpage


\begin{figure}[p]
\strut\hfill
\includegraphics[width=70ex]{appendix/fig/octa-rp.pdf}%
\hfill\strut
\caption[Room partitions of an octahedron]
{An octahedron and its room partitions.}
\label{octa-rp-fig}
\end{figure}

\begin{figure}[p]
\strut\hfill
\includegraphics[width=60ex]{appendix/fig/octa-rp-paths.pdf}%
\hfill\strut
\caption[Endpoints of the Exchange Algorithm]
{The endpoints of its exchange algorithm paths.}
\label{octa-rp-paths-fig}
\end{figure}

\clearpage


\end{example}

V\'{e}gh and von Stengel proved that the endpoints of the
Complementary Pivoting with Direction Algorithm have opposite orientation,
as long as the room partitions are defined on an oriented oik;
the parity result follows as in the previous ones. Unfortunately, as
for the Lemke-Howson Algorithm, the
Complementary Pivoting with Direction Algorithm may take exponential time
in the general case.
Despite this, V\'{e}gh and von Stengel \cite{vvs} give a near-linear-time
algorithm that, given a perfect matching of an Euler graph, finds a perfect
matching of opposite sign; this allows to find a solution to
{\sc Opposite Sign Gale String} in polynomial time.

The wider issue of the complexity of the Lemke-Howson Algorithm has been
solved by Goldberg, Papadimitriou and Savani \cite{gps} as
{\bf PSPACE}-complete. This invites, once more, a further investigation of
exchange-like algorithms to build ``hard to solve'' games.
Since restricting to games that can be reduced to oiks of dimension $2$
seems to give games that are, after all, easily solvable, a possible
direction for research could be the study of games built on oiks of
dimension $3$ or higher; studying products of these could also be
interesting, although the use of products of polytopes to make the
solvability via enumeration support harder in Savani and
von Stengel \cite{svs} was circumvented together with the simpler
case of Gale games by our result.

Another result worth mentioning is found in Merschen \cite{jm}: finding the
number of equilibria of a Gale game is {\bf \#P}-complete.
Gilboa and Zemel \cite{gilboa-zemel} have proven that deciding the
uniqueness of Nash equilibria is a {\bf co-NP}-complete problem;
a proof from the point of view of completely labeled facets of a
polytope was given by von Stengel \cite{vs-noclf} with the result of
{\bf NP}-completeness of the problem to find a completely labeled facet
in a generic labeled polytope. Given these results, it could be interesting
to find a polynomial-time algorithm to decide the uniqueness of Nash
equilibria in Gale games, or to find yet another watershed for the
complexity of games in the more general framework described above.
