This thesis presents a report on original research, published as joint
work with Merschen and von Stengel in {\em Electronic Notes in Discrete
Mathematics}~\cite{main}. Our result shows a polynomial time
algorithm to solve two problems related to labeled Gale strings, a
combinatorial structure consisting a string of labels and a bitstring
satisfying certain conditions introduced by Gale \mbox{in \cite{gale}.}

Gale strings can be used in the representation of a particular class
of games that Savani and von Stengel~\cite{svs} used as an example
of exponential running time for the classical \mbox{Lemke-Howson}
algorithm to find a Nash equilibrium of a bimatrix game~\cite{lh}.
It was conjectured that solving these games via the Lemke-Howson
algorithm was complete in the class {\bf PPAD} (Proof by Parity
Argument, Directed version). A major motivation for the definition of
this class by Papadimitriou~\cite{ppad} was, in turn, to capture
the pivoting technique of many results related to the Nash equilibrium,
including the Lemke-Howson algorithm.

Our result, on the contrary, sets apart this class of games as a case
for which there is a polynomial-time algorithm to find a Nash equilibrium.
Since Daskalakis, Goldberg and Papaditrimiou~\cite{dgp} and
Chen and Deng~\cite{cd} proved the \mbox{{\bf PPAD}-completeness} of
finding a Nash equilibrium in general normal-form games, we have a special
class of games, unless \mbox{{\bf PPAD} = {\bf P}}.

Our proof exploits two results. The first one is the representation
of the Nash equilibria of these games as Gale strings, as seen in Savani
and von Stengel~\cite{svs}. The second one is the polynomial-time
solvability of the problem of finding a perfect matching in a graph, proven
by Edmonds~\cite{edm}.

Merschen~\cite{jm} and V\'{e}gh and von Stengel~\cite{vvs} expanded our
technique to prove further interesting results.

An appendix relates an amendment to the proof of the
\mbox{{\bf PPAD}-completeness} result by Daskalakis, Goldberg and
Papaditrimiou~\cite{dgp}.
