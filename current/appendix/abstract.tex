This thesis presents a report on original research, published as joint
work with Merschen and von Stengel in {\em Electronic Notes in Discrete
Mathematics}~\cite{main}. Our result shows a polynomial time
algorithm to solve two problems related to labeled Gale strings, a
combinatorial structure introduced by Gale in \cite{gale} that can be
used in the representation of a particular class of games.

These games were used by Savani and von Stengel~\cite{svs} as an
example of exponential running time for the classical
\mbox{Lemke-Howson} algorithm to find a Nash equilibrium of a
bimatrix game~\cite{lh}.
It was therefore conjectured that \mbox{solving} these games was a problem
complete in the class {\bf PPAD} (Proof by Parity
Argument, Directed version); a major motivation for the definition of
this class by Papadimitriou~\cite{ppad} was, in turn, to capture
the pivoting technique of many results related to the Nash equilibrium,
including the Lemke-Howson algorithm.

Our result, on the contrary, sets apart this class of games as a case
for which there a Nash equilibrium can be found in polynomial time.
Since Daskalakis, Goldberg and Papaditrimiou~\cite{dgp} and
Chen and Deng~\cite{cd} proved the \mbox{{\bf PPAD}-completeness} of
finding a Nash equilibrium in general normal-form games, we have a special
class of games, unless \mbox{{\bf PPAD} = {\bf P}}.

Our proof exploits two results: first of all, the representation
of the Nash equilibria of these games as Gale strings, as seen in Savani
and von
\linebreak[5]
Stengel~\cite{svs}~\cite{uvg}; then, we use the polynomial-time
solvability of the problem of finding a perfect matching in a graph, proven
by Edmonds~\cite{edm}.

Merschen~\cite{jm} and V\'{e}gh and von Stengel~\cite{vvs} expanded our
ideas to prove further results concerning the index of a Nash
equilibrium, as defined in
\linebreak[5]
Shapley~\cite{shapley}, in the framework of the oiks by
Edmonds~\cite{edm-oiks} and Edmonds and Sanit\`a~\cite{edm-sanita}.
