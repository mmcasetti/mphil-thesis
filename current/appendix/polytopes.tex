
We denote the transpose of a matrix $A$ as $A\T$.
We consider vectors $u,v\in\reals^d$ as column vectors, so $u\T v$ is
their scalar product. A vector in $\reals^d$ for which all components
are $0$'s will be denoted as $\0$; similarly, a vectors for which all
components are $1$'s will be denoted as $\1$.
The {\em unit vector} $e_i$ is the vector that has $i$-th component
\smash{${e_i}_i = 1$} and \smash{${e_i}_j=0$} for all other components.
When writing an inequality of the form $u\geq v$ (and analogous), we mean
that it holds for every component; that is, $u_i\geq v_i$ for all
$i\in [d]$.

An {\em affine combination} of points in an Euclidean space $z_1,\ldots,z_n$
is
\[
\sum_{i=1}^n \lambda_i z_i \quad \text{where }\lambda_i\in\reals
\text{ such that }\sum_{i=1}^n \lambda_i = 1
\]

The points $z_1,\ldots,z_n$ are {\em affinely independent} if none of them
is an affine combination of the others.

A {\em convex combination} of points $z_1,\ldots,z_n$ is an affine
combination where $\lambda_i\geq 0$ for all $i\in [n]$.
Note that such $\lambda_i$'s can be seen as a probability distribution over
the $z_i$'s.

A set of point $Z$ is {\em convex} if it is closed under forming convex
combinations, that is, if $\bar{z}=\sum_{i=1}^n \lambda_i z_i$,
where $z_i\in Z$, $\lambda_i\geq 0$ and $\sum_{i=1}^n \lambda_i = 1$,
then $\bar{z}\in Z$. A convex set has {\em dimension} $d$ if it has exactly
$d + 1$ affinely independent points.


\todo[inline]{convex hull (needed for def cyclic poly);

pow hyperplanes;

polyhedra, polytopes

simplex

simple and simplicial polytopes

polar:
$Q=\{ x\in\reals^d\ |\ x\T c_i\leq 1,\ i\in [k] \}$

with $c_i\in\reals^d$. Then the polar (Ziegler, 1995) of $Q$ is given by

$Q^\Delta = \conv\{ c_i, i\in [k] \}$

}



\todo[inline]{from here: notes - copy-paste}

A ($d$-dimensional) {\em simplicial polytope} $P$ is the convex hull of a set
of at least $d+1$ points $v$ in $\reals^d$ in general position, that is, no
$d+1$ of them are on a common hyperplane.

If a point $v$ cannot be omitted from these points without changing $P$ then
$v$ is called a {\em vertex} of $P$. A {\em facet} of $P$ is the convex hull
$\conv\,F$ of a set $F$ of $d$ vertices of $P$ that lie on a hyperplane
$\{ x\in \reals^d\mid a^T x=a_0\}$ so that $a^T u<a_0$ for all other vertices
$u$ of $P$; the vector $a$ (unique up to a scalar multiple) is called the
{\em normal vector} of the facet. We often identify the facet with its set of
vertices~$F$.
