%\documentclass[final,5p,times,twocolumn]{elsarticle}
\documentclass[preprint,12pt]{elsarticle}
\usepackage{mathptmx}
\usepackage{amsmath}
\usepackage{amssymb}
\usepackage{dsfont}
\usepackage{verbatim}
\usepackage[ruled,vlined, lined,linesnumbered]{algorithm2e}
\usepackage{graphicx}
\usepackage[all]{xy}

% \usepackage[notref,notcite]{showkeys} 

% \newtheorem{theorem}{Theorem}[section]
\newtheorem{theorem}{Theorem}
\newtheorem{lemma}[theorem]{Lemma}
\newtheorem{proposition}[theorem]{Proposition}
\newtheorem{corollary}[theorem]{Corollary}
\newtheorem{definition}[theorem]{Definition}

\newenvironment{proof}[1][Proof]{\begin{trivlist}
\item[\hskip \labelsep {\bfseries #1}]}{\end{trivlist}}
% \newenvironment{definition}[1][Definition]{\begin{trivlist}
% \item[\hskip \labelsep {\bfseries #1}]}{\end{trivlist}}
 \newenvironment{example}[1][Example]{\begin{trivlist}
 \item[\hskip \labelsep {\bfseries #1}]}{\end{trivlist}}
% \newenvironment{remark}[1][Remark]{\begin{trivlist}
% \item[\hskip \labelsep {\bfseries #1}]}{\end{trivlist}}

%% Use the option review to obtain double line spacing
%% \documentclass[authoryear,preprint,review,12pt]{elsarticle}

%% Use the options 1p,twocolumn; 3p; 3p,twocolumn; 5p; or 5p,twocolumn
%% for a journal layout:
%% \documentclass[final,1p,times]{elsarticle}
%% \documentclass[final,1p,times,twocolumn]{elsarticle}
%% \documentclass[final,3p,times]{elsarticle}
%% \documentclass[final,3p,times,twocolumn]{elsarticle}
%% \documentclass[final,5p,times]{elsarticle}
%% \documentclass[final,5p,times,twocolumn]{elsarticle}

%% if you use PostScript figures in your article
%% use the graphics package for simple commands
%% \usepackage{graphics}
%% or use the graphicx package for more complicated commands
%% \usepackage{graphicx}
%% or use the epsfig package if you prefer to use the old commands
%% \usepackage{epsfig}

%% The amssymb package provides various useful mathematical symbols
%\usepackage[bottom]{draftcopy}

%\include{pstricks}
%\include{cases}

%% The amsthm package provides extended theorem environments
%% \usepackage{amsthm}


%% The lineno packages adds line numbers. Start line numbering with
%% \begin{linenumbers}, end it with \end{linenumbers}. Or switch it on
%% for the whole article with \linenumbers.
%% \usepackage{lineno}


\newdimen\einr\einr2em
\def\abs#1{\par\hangafter=1\hangindent=\einr
  \noindent\hbox to\einr{\ignorespaces#1\hfill}\ignorespaces} 
\def\conv{{\rm conv}}
\def\0{{\bf0}}
\def\1{{\bf1}}
\def\T{^{\top}}
\def\rone{{\1\T}}
\def\reals{{\mathbb R}}
%% \parskip 1ex
\def\proof{\noindent{\em Proof.\enspace}}
\def\proofof#1{\noindent{\em Proof of #1.\enspace}}
\def\endproof{\hfill\strut\nobreak\hfill\tombstone\par\medbreak}
\def\tombstone{\hbox{\lower.4pt\vbox{\hrule\hbox{\vrule
  \kern7.6pt\vrule height7.6pt}\hrule}\kern.5pt}}

\begin{document}

\begin{frontmatter}

%% Title, authors and addresses

%% use the tnoteref command within \title for footnotes;
%% use the tnotetext command for theassociated footnote;
%% use the fnref command within \author or \address for footnotes;
%% use the fntext command for theassociated footnote;
%% use the corref command within \author for corresponding author footnotes;
%% use the cortext command for theassociated footnote;
%% use the ead command for the email address,
%% and the form \ead[url] for the home page:
%% \title{Title\tnoteref{label1}}
%% \tnotetext[label1]{}
%% \author{Name\corref{cor1}\fnref{label2}}
%% \ead{email address}
%% \ead[url]{home page}
%% \fntext[label2]{}
%% \cortext[cor1]{}
%% \address{Address\fnref{label3}}
%% \fntext[label3]{}

\title{Finding Gale Strings}

%% use optional labels to link authors explicitly to addresses:
%% \author[label1,label2]{}
%% \address[label1]{}
%% \address[label2]{}

\author{Marta M. Casetti, Julian Merschen, Bernhard von Stengel}

\address{Dept. of Mathematics, London School of Economics,
London WC2A 2AE, United Kingdom}
\ead{m.m.casetti@lse.ac.uk, j.merschen@lse.ac.uk, stengel@maths.lse.ac.uk}

\begin{abstract}
\small
The problem \textsc{2-Nash} of finding a Nash equilibrium of
a bimatrix game belongs to the complexity class PPAD.
This class comprises computational problems that
are known to have a solution by means of a path-following
argument.
For bimatrix games, this argument is provided by the
Lemke--Howson algorithm.
It has been shown that this algorithm is worst-case
exponential with the help of dual cyclic polytopes, where
the algorithm can be expressed combinatorially via labeled
bitstrings defined by the ``Gale evenness condition'' that
characterize the vertices of these polytopes.
We define the combinatorial problem \textsc{Another
completely labeled Gale string} whose solutions define the
Nash equilibria of games defined by cyclic polytopes,
including games where the Lemke--Howson algorithm takes
exponential time.
If this problem was PPAD-complete, this would imply
that \textsc{2-Nash} is PPAD-complete, in a much simpler way
than the currently known proofs, including the original
proof by Chen and Deng \cite{cd}.
However, we show that \textsc{Another completely labeled
Gale string} is solvable in polynomial time by a simple
reduction to \textsc{Perfect matching} in graphs, making it
unlikely to be PPAD-complete.
Although this result is negative, we hope that it stimulates
research into combinatorially defined problems that are
PPAD-complete and imply this property for \textsc{2-Nash}.  

\begin{keyword}
%% keywords here, in the form: keyword \sep keyword
Nash Equilibria\sep  Lemke-Howson \sep Gale Evenness Strings \sep Complexity \sep Perfect Matching \sep Polynomial Time Algorithm 
%% PACS codes here, in the form: \PACS code \sep code

%% MSC codes here, in the form: \MSC code \sep code
%% or \MSC[2008] code \sep code (2000 is the default)
\end{keyword}

\end{abstract}

\end{frontmatter}

%% \linenumbers

%% main text

%///////////////////////////////////////////Introduction///////////////////////////////////////////////////////////////////////////////////

\section{Labeled Gale strings}
\label{s-gale}

Let $[k]=\{1,\ldots,k\}$ for any positive integer $k$.
If $S$ is a set, we often consider a function $s:[k]\to S$
as a string $s(1)s(2)\cdots s(k)$ of $k$ elements of $S$. 
% and vice versa.
For $s:[k]\to S$ and a subset $A$ of $[k]$,
let $s(A)$ be the set $\{s(i)\mid i\in [k]\}$.
If $S=\{0,1\}$, we call $s$ a string of {\em bits}.
A bit string $s:[k]\to \{0,1\}$ can be considered as an
indicator function of a subset of $[k]$ that we denote by
$1(s)$, that is,
\[
1(s)=s^{-1}(1)=\{j\in [k]\mid s(j)=1\}.
\]

\begin{definition}
$G(d,n)$ is the set of all strings $s$ of $n$ bits so that 
exactly $d$ bits in $s$ are $1$
% (so the remaining $n-d$ bits in $s$ are~$0$)
and so that $s$ fulfills the {\em Gale evenness condition},
that is, whenever $01^k0$ is a substring of $s$, then $k$
is even.
An element of of $G(d,n)$ is also called a {\em Gale string}
of {\em dimension}~$d$ (and length $n$).
\end{definition}

For example, $G(4,6)$ consists of the nine strings
111100, 111001, 110110, 110011, 101101, 100111, 011110, 011011, 001111.

For a bit string $s$, a maximal substring of $s$ of
consecutive $1$'s is called a {\em run}.
A Gale string may only have interior runs (bounded on both
sides by a~0) of even length but may start or end with an
odd(-length) run.
%% When $d$ is even, $d=2k$, then it is easy to see that
%% \[
%% |G(2k,n)|=\begin{pmatrix} n-k\\ k\\ \end{pmatrix}+
%% \begin{pmatrix} n-k-1\\ k-1\\ \end{pmatrix}.
%% \]
If $d$ is even, then any $s$ in $G(d,n)$ that starts with an
odd run also ends with an odd run, and these two odd runs
may be ``glued together'' to form an even run.
This shows that the set of Gale strings of even dimension is
invariant under a cyclic shift of the strings.
We normally assume that $d$ is even.
%%%%%%%%%% used to define orientation
%% If $s\in G(d,n)$ and $s$ both starts and ends with a $1$,
%% then we consider the first (possibly odd) run of $s$ as a
%% continuation of the last run of~$s$.
%% If $d<n$, then $s$ has at least one~$0$, and uniquely splits
%% into up to $d/2$ runs.

Given a set $G$ of bit strings of length $n$ and a parameter
$d$, a {\em labeling\/} is a function $l:[n]\to[d]$.
Given a labeling, a string $s$ in $G$ is called 
{\em completely labeled\/} if $l(1(s))=[d]$, that is, if
every label in $[d]$ appears as $l(i)$ for at least one bit
$s(i)$ so that $s(i)=1$.
Clearly, if $s$ is completely labeled, then $s$ has
at least $d$ bits that are~$1$, and
if exactly $d$ bits in $s$ are~$1$, then every label
in $[d]$ occurs exactly once.

We consider the following decision problem.
\einr 5em

\noindent
\textsc{Completely labeled Gale string}
{\parskip0pt
\abs{\textbf{Input}:} 
% Positive integers $n,d$, where $d$ is even and $d<n$, and 
% a string of {\em labels} $l:[n]\to[d]$.
A labeling $l:[n]\to[d]$, where $d$ is even and $d<n$.
\abs{\textbf{Question}:}
Is there a Gale string $s$ in $G(d,n)$ that is
{\em completely labeled}?
\par}

For example, for the string of labels $l = 1123143$ (with
$d=4$) the completely labeled Gale strings are 0110011 and
0011110.
For $l = 123432$ they are 111100, 110110, 100111, and
101101.
For $l = 121314$, there are no completely labeled Gale strings.

The set $G(d,n)$ of Gale strings has a combinatorial
structure that allows the use of a ``parity argument'',
which we consider in detail later, to show the following
known property; it holds for odd $d$ as well but we assume
throughout that $d$ is even.

\begin{theorem}
\label{t-even}
For any labeling $l:[n]\to[d]$, where $d$ is even and $d<n$,
the number of completely labeled Gale strings in $G(d,n)$ is
even.  
\end{theorem}

Theorem~\ref{t-even} implies that if there is one completely
labeled Gale string, there is also a second one.  
The following function problem asks to compute a completely
labeled Gale string if one such string is already given.

\noindent
\textsc{Another completely labeled Gale string}
{\parskip0pt
\abs{\textbf{Input}:} 
% Positive integers $n,d$, where $d$ is even and $d<n$, and 
% a string of {\em labels} $l:[n]\to[d]$.
A labeling $l:[n]\to[d]$, where $d$ is even and $d<n$,
and a completely labeled Gale string $s$ in $G(d,n)$.
\abs{\textbf{Output}:}
A completely labeled Gale string $s'$ in
$G(d,n)$ where $s'\ne s$.
\par}

The main result of this paper is that both problems,
\textsc{Completely labeled Gale string} and
\textsc{Another completely labeled Gale string},
can be solved in polynomial time.
The proof uses a reduction to the following problem,
which was first shown to be solvable in polynomial time by
Edmonds~\cite{edm}. 

\noindent \textsc{Perfect Matching}
{\parskip0pt
\abs{\textbf{Input}:} Graph $G = (V,E)$.
\abs{\textbf{Question}:}
Is there a set $M\subseteq E$ of pairwise non-adjacent edges
so that every vertex $v \in V$ is incident to exactly one
edge in~$M$?
\par}

\begin{theorem}
\label{t-main}
The problems 
\textsc{Completely labeled Gale string} and
\textsc{Almost completely labeled Gale string}
can be solved in polynomial time.
\end{theorem}

\proof
We give a rather simple reduction to \textsc{Perfect Matching}.
Given the labeling $l:[n]\to[d]$, construct the 
(multi-)graph $G$ with vertex set $V=[d]$ and up to $n$
(possibly parallel) edges with endpoints $l(i),l(i+1)$ for
$i\in [n]$ whenever these endpoints are distinct (so $G$ has
no loops); here we let $n+1=1$ (``modulo~$n$'') so that
$n,n+1$ is to be understood as $n,1$.
Then a completely labeled Gale string $s$ in $G(d,n)$ splits
into a number of runs which are uniquely split into $d/2$
pairs $i,i+1$ so that the labels $l(i)$ and $l(i+1)$ are
distinct, and all labels $1,\ldots,n$ occur among them.  
So this defines a perfect matching for~$G$.

Conversely, a perfect matching $M$ of $G$ defines a Gale
string $s$ where $s(i)=s({i+1})=1$ if the edge that joins
$l(i)$ and $l(i+1)$ is in $M$ and $s(i)=0$ otherwise, so $s$ 
is completely labeled.
This shows how \textsc{Completely labeled Gale string}
reduces to \textsc{Perfect Matching}.
Finding a perfect matching, or deciding that $G$ has none,
can be done in polynomial time~\cite{edm}.

The reduction for \textsc{Another completely labeled Gale string}
is an extension of this.
Consider the given completely labeled Gale string $s$ and
the matching $M$ for it.
If $G$ has multiple edges between two nodes and one of them
is in $M$, simply replace that edge by a parallel edge to
obtain another completely labeled Gale string~$s'$.
Hence, we can assume that $M$ has no edges that have a
parallel edge.
Another completely labeled Gale string $s'$ exists by
Theorem~\ref{t-even}.
The corresponding matching $M'$ does not use at least one
edge in $M$.
Hence, at least one of the $d/2$ graphs $G$ which have one
of the edges of $M$ removed has a perfect matching $M'$,
which is a perfect matching of $G$, and which defines 
a completely labeled Gale string $s'$ different from~$s$.  
The search for $M'$ takes again polynomial time.
\endproof

% Theorem~\ref{t-main} has a simple proof which raises some
% interesting observations on its own.
The significance of Theorem~\ref{t-main} is to be understood
in the context of equilibrium computation for games, which
we discuss next.
The remainder of this paper contains only known results.

%% In addition, the multigraph $G$ in the proof of
%% Theorem~\ref{t-main} is of interest of its own:
%% It is a Euler graph (each node is the endpoint of an even
%% number of edges), which is also known as a 1-oik (see
%% Edmonds \cite{je07}), and its perfect matchings are known as
%% room partitions.
%% The set of such room partitions is even, shown by the
%% ``exchange algorithm'', also due to Edmonds \cite{je07}.
%% This provides an alternative proof of Theorem~\ref{t-even}.
%% However, the exchange algorithm for perfect matchings of Euler
%% graphs is different (and apparently faster) than the
%% Lemke--Howson algorithm for labeled Gale strings, which
%% can be exponential (a result due to Morris \cite{morris},
%% see below).

\section{Labeled polytopes and equilibria in games}
\label{s-cyclic}

For a matrix $A$ its transpose is $A\T$.
We treat vectors $u,v$ in $\reals^d$ as column vectors,
so $u\T v$ is their scalar product.
By $\0$ we denote a vector of all $0$'s, of suitable dimension,
by $\1$ a vector of all $1$'s.
A unit vector, which has a 1 in its $i$th component and 0
otherwise, is denoted by $e_i$.
Inequalities like $u\ge\0$ hold for all components.
For a set of points $S$ we denote its convex hull by
$\conv\,S$.

% A {\em hyperplane} in $\reals^d$ is a set
% $\{ x\in \reals^d\mid a\T x=a_0\}$ for some $a\in \reals^d$,
% $a\ne\0$, and $a_0\in \reals$.
% Given points in $\reals^d$ are said to be in {\em general
% position} if no $d+1$ of them are on a common hyperplane.

% A simplex in $\reals^d$ is the convex hull of up to $d+1$
% points in $\reals^d$ in general position.
% [Not quite right; need affine independence.] 

A ($d$-dimensional) {\em simplicial polytope} $P$ is the
convex hull of a set of at least $d+1$ points $v$ in
$\reals^d$ in general position, that is, no $d+1$ of them
are on a common hyperplane.
If $v$ cannot be omitted from these points without
changing $P$ then $v$ is called a {\em vertex} of $P$.
A {\em facet} of $P$ is the convex hull $\conv\,F$ of a
set $F$ of $d$ vertices of $P$ that lie on a hyperplane 
$\{ x\in \reals^d\mid a\T x=a_0\}$ so that $a\T u<a_0$ for
all other vertices $u$ of $P$; the vector $a$ (unique up to 
a scalar multiple) is called the {\em normal vector} of the
facet.
We often identify the facet with its set of vertices~$F$.

A {\em cyclic polytope\/} $P$ in dimension~$d$ with $n$
vertices is the convex hull of $n$ points $\mu(t_j)$ on the
{\em moment curve\/} $\mu\colon t\mapsto
(t,t^2,\ldots,t^d)^\top$ for $j\in [n]$.
Suppose that $t_1<t_2<\cdots< t_n$.
Then the facets of $P$ are encoded by $G(d,n)$, that is,
\[
F \hbox{ is a facet of }P
\quad
\Longleftrightarrow 
\quad
F = \conv\{\mu(t_i)\mid i\in 1(s)\}
\hbox { for some }s\in G(d,n),
\]
as shown by Gale \cite{gale}.
% Essentially, this holds because any set $S\subset [n]$
% the moment curve defines a unique hyperplane which is crossed
% (and not just touched) by the moment curve; if the bitstring
% $s$ that encodes $F$ as $1(s)$ has a substring $01^k0$ 
For this cyclic polytope $P$, a labeling $l:[n]\to[d]$ can
be understood as a label $l(j)$ for each vertex $\mu(t_j)$
for $j\in [n]$.
A completely labeled Gale string $s$ therefore represents a
facet $F$ of $P$ that is completely labeled.

The following theorem, due to Balthasar and von Stengel
\cite{B09,BvS10}, establishes a connection between general
labeled polytopes and equilibria of certain $d\times n$
bimatrix games $(U,B)$.

\begin{theorem}\label{t-unitv}
Consider a labeled  $d$-dimensional simplicial polytope $Q$ with $\0$ in
its interior, with vertices $-e_1,\ldots,-e_d,c_1,\ldots,c_n$,
% We need to assume that the c_i are pairwise distinct, otherwise
% a vertex can have several labels.
% We need to assume that c_i vertex of Q for the following reason:
% the definition of completely labeled facet is difficult if
% c_i is in a facet but not vertex of the facet.
so that % $F_0$ in {\rm(\ref{F0})}
$F_0=\conv\{-e_1,\ldots,-e_d\}$
is a facet of $Q$.
Let $-e_i$ have label $i$ for $i\in[d]$, and let $c_j$
have label $l(j)\in[d]$ for $j\in[n]$.
Let $(U,B)$ be the $d\times n$ bimatrix game with
$U=[e_{l(1)}\cdots\,e_{l(n)}]$ and $B=[b_1\,\cdots\,b_n]$,
where $b_j=c_j/(1+\rone c_j)$ for $j\in[n]$.
Then the completely labeled facets $F$ of $Q$, with the
exception of~$F_0$, are in one-to-one correspondence to the
Nash equilibria $(x,y)$ of the game $(U,B)$ as follows:
if $v$ is the normal vector of $F$, then
$x=(v+\1)/\rone (v+\1)$, 
and $x_i=0$ if and only if $-e_i\in F$ for $i\in[d]$;
any other label~$j$ of $F$, so that $c_j$ is a
vertex of~$F$, represents a pure best reply to~$x$.
The mixed strategy $y$ is the uniform distribution on
the set of pure best replies to~$x$.
\end{theorem}

In the preceding theorem, any simplicial polytope can take
the role of $Q$ as long as it has one completely labeled
facet~$F_0$.
Then an affine transformation, which does not change the
incidences of the facets of $Q$, can be used to map $F_0$ to
the negative unit vectors $-e_1,\ldots,-e_d$ as described,
with $Q$ if necessary expanded in the direction $\1$ so that
$\0$ is in its interior.

A $d\times n$ bimatrix game $(U,B)$ is a {\em unit vector game}
if all columns of $U$ are unit vectors.
For such a game $B$ with $B=[b_1\cdots b_n]$, the columns
$b_j$ for $j\in[n]$ can be obtained from $c_j$ as in
Theorem~\ref{t-unitv} if $b_j>\0$ and $\1\T b_j<1$.
This is always possible via a positive-affine transformation
of the payoffs in~$B$, which does not change the game.
The unit vectors $e_{l(j)}$ that constitute the columns of
$U$ define the labels of the vertices $c_j$.
The corresponding polytope with these vertices is simplicial
if the game $(U,B)$ is nondegenerate \cite{vS02}, which here
means that no mixed strategy $x$ of the row player has more
than $|\{i\in[d]\mid x_i>0\}|$ pure best replies.
Any game can be made nondegenerate by a suitable
``lexicographic'' perturbation of $B$, which can be
implemented symbolically.

Unit vector games encode arbitrary bimatrix games:
An $m\times n$ bimatrix game $(A,B)$ with (w.l.o.g.{})
positive payoff matrices $A,B$ can be symmetrized so 
that its Nash equilibria are in one-to-correspondence to the
symmetric equilibria of the $(m+n)\times(m+n)$
symmetric game $(C\T,C)$ where 
\[
\label{symmetrize}
C=\biggl(\begin{matrix} 0 & B\cr A^\top & 0\cr\end{matrix}\biggr).
\]
In turn, as shown by McLennan and Tourky \cite{mt}, 
the symmetric equilibria $(x,x)$ of any symmetric game
$(C\T,C)$ are in one-to-one correspondence to the Nash
equilibria $(x,y)$ of the ``imitation game'' $(I,C)$ where
$I$ is the identity matrix; the mixed strategy $y$ of the
second player is simply the uniform distribution on the
set $\{i\mid x_i>0\}$.
Clearly, $I$ is a matrix of unit vectors, so $(I,C)$ is a
special unit vector game.

Special games are obtained by using cyclic polytopes in
Theorem~\ref{t-unitv}, suitably affinely transformed with
a completely labeled facet $F_0$.
When $Q$ is a cyclic polytope in dimension $d$ with $d+n$
vertices, then the string of labels $l(1)\cdots l(n)$ in
Theorem~\ref{t-unitv} defines a labeling $l':[d+n]\to [d]$
where $l'(i)=i$ for $i\in [d]$ and
$l'(d+j)=l(j)$ for $j\in [n]$.
In other words, the string of labels $l(1)\cdots l(n)$ is
just prefixed with the string $1\,2\cdots d$ to give $l'$.
Then $l'$ has a trivial completely labeled Gale string
$1^d0^n$ which defines the facet $F_0$.
Then the problem \textsc{Another completely labeled Gale
String} defines exactly the problem of finding a Nash
equilibrium of the unit vector game $(I,B)$.
Note again that $B$ is here not a general matrix (which would
define a general game) but obtained from the last $n$ of
$d+n$ vertices of a cyclic polytope in dimension~$d$.

\section{Lemke--Howson and PPAD}
\label{s-lh}

The algorithm due to Lemke and Howson \cite{lh}, here 
called the LH algorithm, finds one Nash equilibrium of a
bimatrix game.
It can be translated to labeled simplicial polytopes as 
follows.
Start with a completely labeled facet (such as $F_0$ above).
Select one label $i$ that is allowed to be missing 
(or ``dropped'') and move from $F_0$ to the unique adjacent facet
that shares all vertices with $F_0$ except the one with label~$i$.
This is computationally implemented as a {\em pivoting} step
as in the simplex algorithm, which is a local transformation
of the current normal vector by using the other vertices not
on the current facet.
The newly obtained facet $F_1$, say, has a new vertex with a
label $j$; if $j=i$, then $F_1$ is completely labeled and
the algorithm stops.
Otherwise, take the vertex $v$ of $F_1$ that had label $j$ so 
far and move to the unique adjacent facet $F_2$ that has
all vertices of $F_1$ except $v$, and continue as before.
This defines a unique ``path'' of facets that must 
eventually terminate at a completely labeled
facet different from~$F_0$.
Applied to cyclic polytopes, this proves
Theorem~\ref{t-even}.

The result of Morris \cite{morris} implies that for suitably
labeled cyclic polytopes in dimension $d$ with $2d$
vertices, the described path can be exponentially long, for
{\em any\/} initially dropped label.
His labeling $l$ for $d=6$ is given by the string of labels
$123456\,645321$, for $d=8$ it is $12345678\,86745231$,
which shows the general pattern.
With the help of imitation games \cite{mt}, this defines
exponentially long LH paths for bimatrix games.
Savani and von Stengel \cite{svs} obtained this result
differently by considering payoff matrices that for both players
are defined via cyclic polytopes, rather than a matrix of
unit vectors for the row player as in Theorem~\ref{t-unitv}.

The problem \textsc{$n$-Nash} of computing Nash equilibrium of
an $n$-player game belongs to the complexity class PPAD
\cite{ppad}.
It comprises function problems that are known to have a
solution via a ``polynomial parity argument with
direction''.
For  \textsc{$2$-Nash}, this argument is provided by the
LH algorithm.
Formally, PPAD consists of problems that reduce to the
problem \textsc{End of the Line}, given by two
polynomial-sized Boolean circuits $\sigma$ and $\pi$ with
$k$ input and $k$ output bits.
This pair $\sigma,\pi$ defines an implicit digraph with
$k$-bit strings as vertices and arcs $(u,v)$ whenever
$\sigma(u)=v$ and $\pi(v)=u$.
If $\sigma(\pi(u))\ne u$ then $u$ is a source and if
$\pi(\sigma(v))\ne v$ then $v$ is a sink of this digraph.
It is assumed that $0^k$ is a source.
The sought output is any sink, or source other than $0^k$.
It exists because the digraph is a collection of directed
paths and cycles, with at least one path which starts
at~$0^k$.

Daskalakis, Goldberg and Papadimitriou \cite{dgp} and 
Chen and Deng \cite{cd}, respectively, have shown that
\textsc{3-Nash} and \textsc{2-Nash} are PPAD-complete.
As indicated in Section~\ref{s-cyclic}, the PPAD-completenes
of \textsc{2-Nash} due to \cite{cd} shows that the following 
problem is PPAD-complete:
Given a labeled polytope $Q$ as in Theorem~\ref{t-unitv}
with a completely labeled facet $F_0$, find another
completely labeled facet.
(If the games in \cite{cd} are degenerate then $Q$ is not
simplicial; this can be treated by a suitable extension
of the LH algorithm.)

The orientation of the path (the ``D'' in PPAD) can be
proved by a suitable orientation of the facets of the
polytope, via the determinant of their $d$ vertices in the
order of the labels \cite{lg,todd}. 

For the special cyclic polytopes, the LH algorithm can be
described very simply in terms of the Gale evenness strings,
see \cite{svs}.
The orientation can also be defined simpler via signs of
permutations rather than of determinants, which we omit for
reasons of space.

Another abstraction of the LH algorithm is provided by Euler
complexes or ``oiks'' introduced by Edmonds \cite{je07}.
A special case are abstract manifolds, defined by a family
of $d$-element sets called {\em rooms\/} so that any 
{\em wall}, obtained by removing one vertex from a room,
is the wall of exactly one other room.
Given a labeling (called coloring in \cite{je07}) of the
vertices, any manifold has an even number of completely
labeled rooms, in analogy to Theorem~\ref{t-even}.
If the manifold is orientable, the orientation argument
of Lemke and Grotzinger \cite{lg} applies; in particular,
the endpoints of the LH paths are rooms of opposite
orientation.

What is important in our context is that the manifold is
not defined as an explicit list of rooms but implicitly 
with rooms as facets of a simplicial polytope, given by its
vertices.
For the cyclic polytope with $n$ vertices in dimension $d$,
the rooms are even more simply specified as the sets $1(s)$
for $s\in G(d,n)$. 

Our Theorem~\ref{t-main} shows: even though cyclic polytopes
may give rise to exponentially long LH paths, the respective
computational problem of finding another completely labeled
facet is solvable in polynomial time.
Hence, Gale evenness strings are most likely too simple to
define a PPAD-complete problem.


\begin{thebibliography}{00}

\frenchspacing\parskip-1ex
\small

\bibitem{B09} 
A. V. Balthasar (2009),
Geometry and Equilibria in Bimatrix Games.
PhD Thesis, London School of Economics.  

\bibitem{BvS10} 
A. Balthasar, B. von Stengel (2010),
Index and uniqueness of symmetric equilibria.
In preparation.  

\bibitem{cd} X. Chen, X. Deng (2006).
Settling the complexity of two-player Nash equilibrium.
\emph{Proc. 47th FOCS}, pp. 261--272.

\bibitem{dgp} C. Daskalakis, P. W. Goldberg, C. H. Papadimitriou (2006).
The complexity of computing a Nash equilibrium.
\emph{Proc. Ann. 38th STOC}, pp. 71--78.

% \bibitem{dgp2} C. Daskalakis, P. W. Goldberg, C. H.  Papadimitriou (2009).
% The complexity of computing a Nash equilibrium.
% Commun.\ ACM 52, pp. 89--97.

\bibitem{edm} J. Edmonds (1965).
Paths, trees, and flowers. \emph{Canad. J. Math.} 17, pp. 449--467. 

\bibitem{je07}
J. Edmonds (2007). Euler complexes. Manuscript,  5 pp.

\bibitem{gale} D. Gale (1963),
Neighborly and cyclic polytopes.
In: Convexity, Proc. Symposia in Pure Math., Vol. 7, ed. V. Klee, American Math. Soc., Providence, Rhode Island, pp. 225--232.

% \bibitem{gz} I. Gilboa and E. Zemel. Nash and correlated equilibria: Some complexity considerations. Games and Economic Behavior, 1989.

% \bibitem{k} L. G. Khachiyan (1979). A polynomial algorithm in linear programming. \emph{Soviet Mathematics Doklady}, 20(1), pp. 191--194.

\bibitem{lg}
C. E. Lemke, S. J. Grotzinger (1976).
On generalizing Shapley's index theory to labelled
pseudomanifolds.
Math. Programming 10, 245--262.  

\bibitem{lh} C. E. Lemke, J. T. Howson, Jr. (1964).
Equilibrium points of bimatrix games.
\emph{J.  Soc. Indust. Appl. Mathematics} 12, pp.  413--423.

\bibitem{mt}
A. McLennan, R. Tourky (2009).
Simple complexity from imitation games.
\emph{Games and Economic Behavior},
in press, doi:10.1016/j.geb.2009.10.003 

\bibitem{morris} 
W. D. Morris Jr. (1994).
Lemke paths on simple polytopes.
\emph{Math. Oper. Res.} 19, pp. 780--789.

% \bibitem{nash} J. Nash (1951). Noncooperative games.  
% \emph{Ann. Math.} 54, pp. 289--295.

\bibitem{ppad} C. H. Papadimitriou (1994).
On the complexity of the parity argument and other inefficient proofs of existence.
\emph{J. Comput. System Sci.} 48, pp. 498--532.

\bibitem{svs} R. Savani, B. von Stengel (2006).
Hard-to-solve bimatrix games.
\emph{Econometrica} 74, pp. 397--429. 

% \bibitem{shapley} L. S. Shapley (1974), A Note on the Lemke-Howson Algorithm. Mathematical Programming Study, 1, pp. 175--189.

\bibitem{todd} M. J. Todd (1976), Orientation in
complementary pivot algorithms. 
\emph{Math. Oper. Res.} 1, pp. 54--66.

\bibitem{vS02}
B. von~Stengel (2002).
Computing equilibria for two-person games.
In: \emph{Handbook of Game Theory, Vol.~3,}
eds. R. J. Aumann and S. Hart, North-Holland, Amsterdam,
pp. 1723--1759.


% \bibitem{val} L. G. Valiant (1979), The complexity of computing the permanent. Theoretical Computer Science 8, pp. 89--201.

% \bibitem{vN}  J. von Neumann (1928). Zur Theorie der Gesellschaftspiele, \emph{Mathematische Annalen}, 100, pp. 295--320.

  
\end{thebibliography}
\end{document}
\endinput


